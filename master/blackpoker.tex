%% Generated by Sphinx.
\def\sphinxdocclass{jsbook}
\documentclass[letterpaper,10pt,dvipdfmx]{sphinxmanual}
\ifdefined\pdfpxdimen
   \let\sphinxpxdimen\pdfpxdimen\else\newdimen\sphinxpxdimen
\fi \sphinxpxdimen=.75bp\relax
\ifdefined\pdfimageresolution
    \pdfimageresolution= \numexpr \dimexpr1in\relax/\sphinxpxdimen\relax
\fi
%% let collapsible pdf bookmarks panel have high depth per default
\PassOptionsToPackage{bookmarksdepth=5}{hyperref}

\PassOptionsToPackage{warn}{textcomp}


\usepackage{cmap}
\usepackage[T1]{fontenc}
\usepackage{amsmath,amssymb,amstext}




\usepackage{tgtermes}
\usepackage{tgheros}
\renewcommand{\ttdefault}{txtt}




\usepackage[,numfigreset=1,mathnumfig]{sphinx}

\fvset{fontsize=auto}
\usepackage[dvipdfm]{geometry}


% Include hyperref last.
\usepackage{hyperref}
% Fix anchor placement for figures with captions.
\usepackage{hypcap}% it must be loaded after hyperref.
% Set up styles of URL: it should be placed after hyperref.
\urlstyle{same}
\usepackage{pxjahyper}


\usepackage{sphinxmessages}
\setcounter{tocdepth}{2}



\title{BlackPoker}
\date{2023年03月30日}
\release{}
\author{BlackPoker}
\newcommand{\sphinxlogo}{\vbox{}}
\renewcommand{\releasename}{}
\makeindex
\begin{document}

\pagestyle{empty}
\sphinxmaketitle
\pagestyle{plain}
\sphinxtableofcontents
\pagestyle{normal}
\phantomsection\label{\detokenize{index::doc}}
\index{execution@\spxentry{execution}!context@\spxentry{context}}\index{pair  : loop@\spxentry{pair  : loop}!statement@\spxentry{statement}}\index{module@\spxentry{module}!search path@\spxentry{search} \spxentry{path}}\index{search@\spxentry{search}!path, module@\spxentry{path}, \spxentry{module}}\index{path@\spxentry{path}!module search@\spxentry{module} \spxentry{search}}\ignorespaces 


\sphinxstepscope


\chapter{はじめに}
\label{\detokenize{init/init:id1}}\label{\detokenize{init/init::doc}}
\sphinxAtStartPar
この文章はトランプゲーム「BlackPoker」の全てのルールをまとめた文章です。

\sphinxAtStartPar
詳細なルールが記載されており、初心者の方は文章の量に圧倒されます。
ゲームをプレイする際に全てを熟読する必要はありませんが、
ルールについて深く知りたい、または新しいルールに触れたい方はぜひ熟読してください。


\section{ルールの構成}
\label{\detokenize{init/init:id2}}
\sphinxAtStartPar
ルールの構成は次のようになっています。

\noindent\sphinxincludegraphics{{plantuml-0b4f05a00976af29a919ccd57da91a1d511d7894}.pdf}


\section{ルール指針}
\label{\detokenize{init/init:id3}}
\sphinxAtStartPar
ルールを作成・修正するための指針を示します。


\subsection{誰とでも戦える \textasciitilde{}目指すは老若男女\textasciitilde{}}
\label{\detokenize{init/init:id4}}
\sphinxAtStartPar
ルールを知りトランプを持っていれば誰とでも遊べるゲームを目指します。


\subsection{個性が出せる \textasciitilde{}オリジナルトランプ・デッキ構築\textasciitilde{}}
\label{\detokenize{init/init:id5}}
\sphinxAtStartPar
さまざまなトランプが使え見た目で個性を出せるのはもちろんのこと、
デッキ構築の面でも自分のしたい戦い方が表現できることを目指します。


\subsection{短く終わる \textasciitilde{}1戦15分\textasciitilde{}}
\label{\detokenize{init/init:id6}}
\sphinxAtStartPar
時間をかけずさっと遊べることを目指します。


\subsection{ずっと使えるデッキ}
\label{\detokenize{init/init:id7}}
\sphinxAtStartPar
愛着のあるカードがずっと使えるようなルールとします。


\subsection{必要な物は最小限 \textasciitilde{}トランプのみ\textasciitilde{}}
\label{\detokenize{init/init:id8}}
\sphinxAtStartPar
用意するものはトランプのみ。それ以外の道具は必要ないルールとします。


\subsection{プレイング重視 \textasciitilde{}5:3:2=技:運:構築\textasciitilde{}}
\label{\detokenize{init/init:id9}}
\sphinxAtStartPar
運やデッキ構築より技量を重視したルールを目指します。


\subsection{ベースルールはトレーディングカードゲーム}
\label{\detokenize{init/init:id10}}
\sphinxAtStartPar
カードゲームプレイヤーが覚えやすいルールを目指します。


\subsection{カスタマイズ可能 \textasciitilde{}基本と拡張の分離\textasciitilde{}}
\label{\detokenize{init/init:id11}}
\sphinxAtStartPar
基本ルールと拡張ルールを分離し、大富豪のようにローカルルールが作成できることを目指します。


\subsection{ルールの更新 \textasciitilde{}飽き防止&不備改善\textasciitilde{}}
\label{\detokenize{init/init:id12}}
\sphinxAtStartPar
新たなルールを度々公開し、飽きを防止します。またルールに不備がある場合、随時改善します。


\subsection{相手のカードに触らない}
\label{\detokenize{init/init:id13}}
\sphinxAtStartPar
盗難防止とネット対戦対応に努めます。

\sphinxstepscope


\chapter{コアルール}
\label{\detokenize{core/core:corerule}}\label{\detokenize{core/core:id1}}\label{\detokenize{core/core::doc}}
\sphinxAtStartPar
コアルールは割込み処理が可能なターン制ゲームの開始から勝敗が決まるまでを定義します。


\section{ターン}
\label{\detokenize{core/core:id2}}
\sphinxAtStartPar
このルールを説明する上でターンとは持つことができるものとします。
ターンを持っているプレイヤーは先に行動することができます。
ターンを持っているプレイヤーをターンプレイヤーといいます。


\section{アクション}
\label{\detokenize{core/core:id3}}
\sphinxAtStartPar
アクションとは、プレイヤーの行動を示します。
ターン制のゲームでは、プレイヤーは様々な行動を行います。
チェスであればコマを進めたり、ババ抜きであれば隣の人からカードを引くなどがあります。
それらをアクションと定義します。


\section{チャンス}
\label{\detokenize{core/core:id4}}
\sphinxAtStartPar
アクションを起こすことができる機会をチャンスといいます。
チャンスを持っている間は何度でもアクションを起こすことができます。


\section{ステージ}
\label{\detokenize{core/core:id5}}
\sphinxAtStartPar
アクションの解決順を整理するために使う領域です。
後入れ先出し方式で最後に積まれたアクションから順に解決されていきます。


\section{アクションの定義項目}
\label{\detokenize{core/core:id6}}
\sphinxAtStartPar
アクション、チャンス、ステージについて簡単に説明しました。
これらの概念を用いて、アクションに定義する項目を説明します。

\sphinxAtStartPar
アクションは次の項目を定義する必要があります。
他の項目は具体的にアクションを定義する際に、ゲームに合わせて追加して下さい。
\begin{itemize}
\item {} 
\sphinxAtStartPar
効果(通常効果・即時効果)

\item {} 
\sphinxAtStartPar
タイミング

\end{itemize}


\subsection{効果}
\label{\detokenize{core/core:id7}}
\sphinxAtStartPar
効果とはアクションの解決時にプレイヤーが行う行動です。
効果の中には、通常効果と即時効果があります。
違いについては、図(\hyperref[\detokenize{core/core:coreflow-2}]{Fig.\@ \ref{\detokenize{core/core:coreflow-2}}})を説明する際に分岐条件として登場します。


\subsection{タイミング}
\label{\detokenize{core/core:timing}}\label{\detokenize{core/core:id8}}
\sphinxAtStartPar
タイミングとは、アクションを起こすことができる時を示します。
タイミングには「メイン」と「クイック」の2種類あります。
\begin{description}
\sphinxlineitem{メイン}
\sphinxAtStartPar
ターンプレイヤーかつステージが空の時に起こせるアクションです。

\sphinxAtStartPar
条件をまとめると次のようになります。
\begin{itemize}
\item {} 
\sphinxAtStartPar
チャンスを持っている

\item {} 
\sphinxAtStartPar
自分のターン

\item {} 
\sphinxAtStartPar
ステージが空

\end{itemize}

\sphinxlineitem{クイック}
\sphinxAtStartPar
いつでも起こせるため、アクションをステージに積み重ねることができます。

\sphinxAtStartPar
条件をまとめると次のようになります。
\begin{itemize}
\item {} 
\sphinxAtStartPar
チャンスを持っている

\end{itemize}

\end{description}


\subsubsection{エンドアクションの定義}
\label{\detokenize{core/core:id9}}
\sphinxAtStartPar
定義するアクションの中で最低1つは
ターンを別のプレイヤーにわたす効果を定義してください。
そうしないと、ターンが別のプレイヤーに渡らす、ゲームが進行しなくなります。


\subsubsection{アクションのコントローラー}
\label{\detokenize{core/core:id10}}
\sphinxAtStartPar
アクションを起こしたプレイヤーをそのアクションのコントローラーと呼びます。
効果はこのコントローラー視点で解釈されることになります。


\section{コンポーネント}
\label{\detokenize{core/core:component}}\label{\detokenize{core/core:id11}}
\sphinxAtStartPar
ゲームにてプレイヤーが保有する駒やカードのことをコンポーネントと定義します。
コンポーネントは次の項目を持っています。
\begin{description}
\sphinxlineitem{オーナー}
\sphinxAtStartPar
コンポーネントの所有者を示します。大体のトランプゲームではトランプを1セットしか用いないため無視されますが、TCGのデッキなど個人所有のものを用いるゲームでは必要な項目となります。

\sphinxlineitem{コントローラー}
\sphinxAtStartPar
現在そのコンポーネントを操作しているプレイヤーを示します。オーナーとコントローラーは基本同じプレイヤーが設定されますが、コントロールを奪うアクションがある場合、オーナーとコントローラーは異なります。

\end{description}

\begin{sphinxadmonition}{note}{注釈:}
\sphinxAtStartPar
コンポーネントとアクションのコントローラー

\sphinxAtStartPar
コントローラーは制御している人という意味になるため、コンポーネントとアクションのコントローラー制御する対象が異なることになります。
コンポーネントとアクションの属性を次の図に示します。アクションにはオーナーがいない点が異なります。

\begin{figure}[H]
\centering
\capstart

\noindent\sphinxincludegraphics{{plantuml-97eb2fb6ceaeae89223169fc6f937aee84be4fe1}.pdf}
\caption{コンポーネントとアクションの属性}\label{\detokenize{core/core:id44}}\end{figure}
\end{sphinxadmonition}


\section{能力}
\label{\detokenize{core/core:id12}}
\sphinxAtStartPar
能力とはアクションの効果とは異なる概念で、アクションを起こすことができたり、 アクションを誘発したりすることがでる力です。

\sphinxAtStartPar
能力を持つことができるのは、プレイヤーの他に駒やカードなどのゲームに登場するコンポーネントも持つことができます。
(\hyperref[\detokenize{core/core:ability-image}]{Fig.\@ \ref{\detokenize{core/core:ability-image}}})

\begin{figure}[htbp]
\centering
\capstart

\noindent\sphinxincludegraphics{{plantuml-976f45327f77d4f4ee1da59bff1cd262ba402df4}.pdf}
\caption{能力のイメージ}\label{\detokenize{core/core:id45}}\label{\detokenize{core/core:ability-image}}\end{figure}

\sphinxAtStartPar
能力には、次の種類があります。
\begin{description}
\sphinxlineitem{常在型能力}
\sphinxAtStartPar
能力が有効である場合、継続的に発揮される能力

\sphinxlineitem{誘発型能力}
\sphinxAtStartPar
能力が有効である間に何かの契機でアクションを起こす能力

\end{description}

\sphinxAtStartPar
概ねのゲームでは、
ターン終了や駒をすすめるなどのアクションが定義されています。
そして、そのアクションを起こせる能力(常在型能力)を
プレイヤーは保持しています。


\section{コアフロー}
\label{\detokenize{core/core:coreflowsec}}\label{\detokenize{core/core:id13}}
\sphinxAtStartPar
この図にゲームの開始から勝敗が決まるまでの流れが集約されいます。(\hyperref[\detokenize{core/core:coreflow-2}]{Fig.\@ \ref{\detokenize{core/core:coreflow-2}}})

\begin{figure}[htbp]
\centering
\capstart

\noindent\sphinxincludegraphics{{plantuml-6ffd6bbf3c42408b4d7754731293c179bd76e166}.pdf}
\caption{コアフロー}\label{\detokenize{core/core:id46}}\label{\detokenize{core/core:coreflow-2}}\end{figure}


\subsection{{[}1{]}ゲーム開始}
\label{\detokenize{core/core:gamestart}}\label{\detokenize{core/core:id14}}
\sphinxAtStartPar
先攻を決め、ゲームを始める準備を行います。


\subsection{{[}2{]}ターンプレイヤーにチャンスを移動}
\label{\detokenize{core/core:id15}}
\sphinxAtStartPar
ターンを持っているプレイヤーにチャンスを移動します。


\subsection{{[}3{]}ステージが空か?}
\label{\detokenize{core/core:id16}}
\sphinxAtStartPar
ステージにアクションが存在していないか判定します。


\subsection{{[}4{]}パス名簿リセット}
\label{\detokenize{core/core:id17}}
\sphinxAtStartPar
パスしたプレイヤーを記録するパス名簿をリセットします。


\subsection{{[}5{]}アクションを起こす}
\label{\detokenize{core/core:id18}}
\sphinxAtStartPar
アクションを起こしこれからプレイヤーが行うことを宣言します。
ゲームによってアクションの起こし方は異なります。BlackPokerではアクション名を言い、コストの支払や対象を指定しアクションを起こします。
一方ババ抜きでは、隣のプレイヤーからカードを引く際に宣言せず暗黙にアクションが起きている場合もあります。


\subsection{{[}6{]}即時効果か?}
\label{\detokenize{core/core:id19}}
\sphinxAtStartPar
起こしたアクションが即時効果か通常効果か判定します。


\subsection{{[}7{]}アクションの解決}
\label{\detokenize{core/core:actresolve}}\label{\detokenize{core/core:id20}}
\sphinxAtStartPar
アクションの効果に定義されている内容を実行します。
その他にコンポーネントを捨て山に移動するなどゲームによって決まった処理があれば行います。
アクションの解決の中でも効果に定義されている内容を実行することのみを指す場合「効果を発揮する」と言います。


\subsection{{[}8{]}勝敗判定}
\label{\detokenize{core/core:winlose}}\label{\detokenize{core/core:id21}}
\sphinxAtStartPar
ゲームの勝敗を判定します。決着した場合ゲームが終了します。判定の方法はゲームにより異なります。


\subsection{{[}9{]}ステージに追加}
\label{\detokenize{core/core:id22}}
\sphinxAtStartPar
ステージというアクションを貯めておける領域に追加します。


\subsection{{[}10{]}誘発チェック}
\label{\detokenize{core/core:id23}}
\sphinxAtStartPar
ここに至るまでに誘発したアクションがないかチェックします。誘発した場合、効果を解決するかスタックに追加します。詳しいフローは \hyperref[\detokenize{core/core:trigger-check}]{\ref{\detokenize{core/core:trigger-check}} \nameref{\detokenize{core/core:trigger-check}}} を参照してください。


\subsection{{[}11{]}アクションを起こすか?}
\label{\detokenize{core/core:id24}}
\sphinxAtStartPar
チャンスを持っているプレイヤーはアクションを起こすかを判断します。


\subsection{{[}12{]}パス名簿に登録}
\label{\detokenize{core/core:id25}}
\sphinxAtStartPar
パスしたプレイヤーを記録するパス名簿に登録します。同じプレイヤー名は2回登録されません。


\subsection{{[}13{]}パス名簿の件数=プレイヤー数か?}
\label{\detokenize{core/core:id26}}
\sphinxAtStartPar
パス名簿の件数がゲームに参加しているプレイヤーの数と一致しているか判定します。


\subsection{{[}14{]}ステージから取出し}
\label{\detokenize{core/core:id27}}
\sphinxAtStartPar
最後にステージに追加されたアクションをステージから取出します。


\subsection{{[}15{]}チャンス移動}
\label{\detokenize{core/core:id28}}
\sphinxAtStartPar
チャンスを持っているプレイヤーからチャンスを持っていないプレイヤーにチャンスを移動します。
チャンスを移動するルールはゲームによって異なります。


\subsection{誘発チェック}
\label{\detokenize{core/core:trigger-check}}\label{\detokenize{core/core:id29}}
\sphinxAtStartPar
能力の中でも誘発型能力は、なにかをきっかけにしてアクションが起きる条件が定義されています。
誘発する条件は「〜の場合」、「〜時」などで記載されており、誘発するアクションは「〜を誘発する」と記載されています。

\sphinxAtStartPar
誘発チェックでは、誘発したアクションの効果を解決もしくは、ステージに追加します。
誘発したアクションのコントローラーは起因となった誘発型能力を持ったコンポーネントのコントローラーになります。
誘発チェックは次の図のように行います。(\hyperref[\detokenize{core/core:trigger-flow}]{Fig.\@ \ref{\detokenize{core/core:trigger-flow}}})

\begin{figure}[htbp]
\centering
\capstart

\noindent\sphinxincludegraphics{{plantuml-8539752d4b269fd5d2cb708edac3b5defd3e41f1}.pdf}
\caption{誘発チェック}\label{\detokenize{core/core:id47}}\label{\detokenize{core/core:trigger-flow}}\end{figure}


\subsubsection{{[}10\sphinxhyphen{}1{]}誘発したアクションをプレイヤー毎の誘発即時リストと誘発通常リストに追加}
\label{\detokenize{core/core:trigger-act-gather}}\label{\detokenize{core/core:id30}}
\sphinxAtStartPar
全てのプレイヤー、コンポーネントが持っている誘発型能力を確認します。
誘発したアクションをコントローラーのプレイヤー毎に即時効果と通常効果に分け、
プレイヤー毎の誘発即時リスト、誘発通常リストに追加します。


\subsubsection{{[}10\sphinxhyphen{}2{]}誘発即時リスト、誘発通常リスト全体の件数判定}
\label{\detokenize{core/core:id31}}
\sphinxAtStartPar
プレイヤー毎の誘発即時リスト、誘発通常リストの合計件数を判定します。


\subsubsection{{[}10\sphinxhyphen{}3{]}プレイヤー毎に誘発即時リストの即時効果のアクションを解決}
\label{\detokenize{core/core:id32}}
\sphinxAtStartPar
プレイヤー毎に誘発即時リストの即時効果のアクションを解決を行います。
順番はターンプレイヤーからターンが回る順にプレイヤー毎に行います。


\subsubsection{{[}10\sphinxhyphen{}4{]}誘発即時リストから即時効果のアクションを1つ取り出す}
\label{\detokenize{core/core:id33}}
\sphinxAtStartPar
順番のプレイヤーは、 プレイヤー毎の誘発即時リストから1つ即時効果のアクションを取り出します。
取り出すアクションは任意に選択できます。


\subsubsection{{[}10\sphinxhyphen{}5{]}即時効果のアクションを解決}
\label{\detokenize{core/core:id34}}
\sphinxAtStartPar
アクションの効果を解決します。
詳しくは \hyperref[\detokenize{core/core:actresolve}]{\ref{\detokenize{core/core:actresolve}} \nameref{\detokenize{core/core:actresolve}}} 参照。


\subsubsection{{[}10\sphinxhyphen{}6{]}勝敗判定}
\label{\detokenize{core/core:id35}}
\sphinxAtStartPar
勝敗を判定します。
詳しくは \hyperref[\detokenize{core/core:winlose}]{\ref{\detokenize{core/core:winlose}} \nameref{\detokenize{core/core:winlose}}} 参照。


\subsubsection{{[}10\sphinxhyphen{}7{]}誘発したアクションをプレイヤー毎の誘発即時リスト、誘発通常リストに追加}
\label{\detokenize{core/core:id36}}
\sphinxAtStartPar
詳しくは \hyperref[\detokenize{core/core:trigger-act-gather}]{\ref{\detokenize{core/core:trigger-act-gather}} \nameref{\detokenize{core/core:trigger-act-gather}}} 参照。


\subsubsection{{[}10\sphinxhyphen{}8{]}誘発即時リストの件数が0件でなけば繰り返す}
\label{\detokenize{core/core:id37}}
\sphinxAtStartPar
順番のプレイヤーの誘発即時リストに未解決の即時効果がある場合、
即時効果の解決を繰返します。


\subsubsection{{[}10\sphinxhyphen{}9{]}全ての誘発即時リストの件数が0件でなければ繰り返す}
\label{\detokenize{core/core:id38}}
\sphinxAtStartPar
プレイヤー毎の誘発即時リストに未解決のアクションがある場合、
再びプレイヤー毎に誘発即時リストの即時効果の解決を繰返します。


\subsubsection{{[}10\sphinxhyphen{}10{]}プレイヤー毎に誘発通常リストのアクションをステージに追加}
\label{\detokenize{core/core:id39}}
\sphinxAtStartPar
プレイヤー毎に誘発通常リストのアクションをステージに追加します。
順番はターンプレイヤーからターンが回る順にプレイヤー毎に行います。


\subsubsection{{[}10\sphinxhyphen{}11{]}通常効果のアクションを任意の順でステージに追加}
\label{\detokenize{core/core:id40}}
\sphinxAtStartPar
順番のプレイヤーは、 プレイヤー毎の誘発通常リストからアクションを任意の順でステージに追加します。


\subsubsection{{[}10\sphinxhyphen{}12{]}誘発したアクションをプレイヤー毎に誘発即時リストと誘発通常リストにまとめる}
\label{\detokenize{core/core:id41}}
\sphinxAtStartPar
詳しくは \hyperref[\detokenize{core/core:trigger-act-gather}]{\ref{\detokenize{core/core:trigger-act-gather}} \nameref{\detokenize{core/core:trigger-act-gather}}} 参照。


\subsubsection{{[}10\sphinxhyphen{}13{]}誘発通常リストにアクションがあれば繰り返す}
\label{\detokenize{core/core:id42}}
\sphinxAtStartPar
プレイヤー毎の誘発通常リストにアクションがある場合、
順番を次のプレイヤーに渡し、プレイヤー毎に誘発通常リストのアクションをステージに追加します。


\section{まとめ}
\label{\detokenize{core/core:id43}}
\sphinxAtStartPar
コアルールについて説明しました。
すでにあるターン制のゲームからアクションを洗い出し、能力を整理することで割込処理を可能としゲームの新しい遊び方が見つけられます。
また、新しく作成するゲームに関してもコアルールを意識して作成することで、ルール追加がしやすいゲームが考えやすいと思います。

\sphinxstepscope


\chapter{共通ルール}
\label{\detokenize{common/common:id1}}\label{\detokenize{common/common::doc}}
\sphinxAtStartPar
この章では、カードの配置などコアルールで定義されていない内容を定義します。


\section{プレイ人数}
\label{\detokenize{common/common:id2}}
\sphinxAtStartPar
フォーマット、対戦レギュレーションに定義されていない場合、2人です。
プレイする際に確認してください。


\section{用意するもの}
\label{\detokenize{common/common:id3}}\begin{itemize}
\item {} 
\sphinxAtStartPar
1人1セットのトランプが必要です。

\item {} 
\sphinxAtStartPar
覚えていない場合、フォーマットに応じてアクションリスト、エクストラリストがあると便利です。

\end{itemize}


\section{使用できるトランプ}
\label{\detokenize{common/common:id4}}
\sphinxAtStartPar
BlackPokerでは次の条件を満たしたトランプを使うことができます。
一般的なトランプなら満たす条件となっています。
\begin{quote}
\begin{itemize}
\item {} 
\sphinxAtStartPar
スートと数字が分かる

\item {} 
\sphinxAtStartPar
スートの{\normalsize $\spadesuit$} {\normalsize $\heartsuit$} {\normalsize $\diamondsuit$} {\normalsize $\clubsuit$} が判断できる

\item {} 
\sphinxAtStartPar
数字のA\sphinxhyphen{}K(1\sphinxhyphen{}13)が判断できる

\item {} 
\sphinxAtStartPar
スートと数字の組合せが重複していない

\item {} 
\sphinxAtStartPar
裏から表がわからない

\item {} 
\sphinxAtStartPar
縦向き横向きが判断できる

\item {} 
\sphinxAtStartPar
Jokerは2枚まで入れられる

\item {} 
\sphinxAtStartPar
54枚無くてもよい

\end{itemize}

\sphinxAtStartPar
対戦レギュレーションにより使用できるトランプの枚数など異なる場合があるため、
対戦する際に、対戦レギュレーション、フォーマットを確認してください。
\end{quote}


\section{トランプの数字}
\label{\detokenize{common/common:id5}}
\sphinxAtStartPar
ゲーム全体を通してトランプの数字は次のような数値として扱います。(\hyperref[\detokenize{common/common:cardrank}]{Table \ref{\detokenize{common/common:cardrank}}})


\begin{savenotes}\sphinxattablestart
\centering
\sphinxcapstartof{table}
\sphinxthecaptionisattop
\sphinxcaption{トランプの数字}\label{\detokenize{common/common:id48}}\label{\detokenize{common/common:cardrank}}
\sphinxaftertopcaption
\begin{tabulary}{\linewidth}[t]{|T|T|}
\hline
\sphinxstyletheadfamily 
\sphinxAtStartPar
カード
&\sphinxstyletheadfamily 
\sphinxAtStartPar
数字
\\
\hline
\sphinxAtStartPar
A
&
\sphinxAtStartPar
1
\\
\hline
\sphinxAtStartPar
2〜10
&
\sphinxAtStartPar
表記どおり
\\
\hline
\sphinxAtStartPar
J
&
\sphinxAtStartPar
11
\\
\hline
\sphinxAtStartPar
Q
&
\sphinxAtStartPar
12
\\
\hline
\sphinxAtStartPar
K
&
\sphinxAtStartPar
13
\\
\hline
\sphinxAtStartPar
Joker
&
\sphinxAtStartPar
0
\\
\hline
\end{tabulary}
\par
\sphinxattableend\end{savenotes}


\section{カードの配置}
\label{\detokenize{common/common:id6}}
\sphinxAtStartPar
カードの配置には次のような場所があります。(\hyperref[\detokenize{common/common:field-ex}]{Fig.\@ \ref{\detokenize{common/common:field-ex}}})

\begin{figure}[htbp]
\centering
\capstart

\noindent\sphinxincludegraphics{{field-ex}.pdf}
\caption{プレイ中のカードの配置}\label{\detokenize{common/common:id49}}\label{\detokenize{common/common:field-ex}}\end{figure}
\begin{description}
\sphinxlineitem{デッキ}
\sphinxAtStartPar
山札。ゲームを始める時に自分のトランプを裏向きに置く場所です。
ダメージを受けるとデッキの一番上から墓地にカードを移します。

\sphinxlineitem{墓地}
\sphinxAtStartPar
捨て札置き場。ダメージを受けた時などに表向きでカードを重ねて置きます。

\sphinxlineitem{場}
\sphinxAtStartPar
兵士や防壁などのキャラクターを置きます。

\sphinxlineitem{手札}
\sphinxAtStartPar
デッキから引いたカードを持っておく場所です。相手から見えないようにしましょう。

\sphinxlineitem{切札}
\sphinxAtStartPar
能力が割り当てられたカードを置きます。エクストラフォーマットのみで使用します。
エクストラのルールについては、 \hyperref[\detokenize{common/common:extra}]{\ref{\detokenize{common/common:extra}} \nameref{\detokenize{common/common:extra}}} で説明します。

\end{description}


\section{勝利条件}
\label{\detokenize{common/common:id7}}
\sphinxAtStartPar
プレイヤーは順に対戦相手に対し攻撃を行い、ダメージを与え先に相手のデッキを0枚にした方が勝ちです。ダメージは1点につき1枚デッキが減ります。


\section{ダメージ}
\label{\detokenize{common/common:id8}}
\sphinxAtStartPar
プレイヤーがダメージを受けた場合、デッキの一番上から受けた点数分墓地にカードを表向きで移動します。移動する際は、カードの表を対戦相手に見せる必要はありません。


\section{キャラクター}
\label{\detokenize{common/common:id9}}
\sphinxAtStartPar
キャラクターとは、場に存在する兵士や防壁のことを指します。
コアルールのコンポーネントにあたります。

\sphinxAtStartPar
キャラクターは1枚のカードで1体を表すこともあれば、
複数枚で1体を表すこともあります。(\hyperref[\detokenize{common/common:character}]{Fig.\@ \ref{\detokenize{common/common:character}}})

\begin{figure}[htbp]
\centering
\capstart

\noindent\sphinxincludegraphics{{character}.pdf}
\caption{キャラクターの例}\label{\detokenize{common/common:id50}}\label{\detokenize{common/common:character}}\end{figure}


\subsection{キャラクターのもつ項目}
\label{\detokenize{common/common:id10}}
\sphinxAtStartPar
キャラクターのもつ項目について説明します。
凡例のキャラクター「一般兵」を見てみましょう。(\hyperref[\detokenize{common/common:character-sample}]{Fig.\@ \ref{\detokenize{common/common:character-sample}}})

\begin{figure}[htbp]
\centering
\capstart

\noindent\sphinxincludegraphics{{character-sample}.pdf}
\caption{一般兵}\label{\detokenize{common/common:id51}}\label{\detokenize{common/common:character-sample}}\end{figure}
\begin{description}
\sphinxlineitem{キャラクター名}
\sphinxAtStartPar
キャラクターの名称を示します。

\sphinxlineitem{タイプ}
\sphinxAtStartPar
キャラクターのタイプを示します。タイプは兵士と防壁の2種類が存在します。

\sphinxlineitem{キーカード}
\sphinxAtStartPar
キャラクターを示すカードが記載されています。複数のカードで1体のキャラクターを示す場合もあります。

\sphinxlineitem{能力}
\sphinxAtStartPar
キャラクターが持っている能力を記載しています。

\end{description}


\subsection{キャラクターの数字}
\label{\detokenize{common/common:id11}}
\sphinxAtStartPar
トランプの数字は、キャラクターの強さを示します。
基本はカードに記載された数字を示しますが、魔法などのアクションを使うことで
加算したり減算されたりします。


\subsection{キャラクターの注意点}
\label{\detokenize{common/common:id12}}

\subsubsection{複数枚で1体となるキャラクターが防壁になったら?}
\label{\detokenize{common/common:id13}}
\sphinxAtStartPar
アクションの効果で兵士を防壁にすることがあります。
防壁は1枚で1体のキャラクターであるため、
複数枚からなるキャラクターが防壁となった場合、
複数体の防壁となります。

\sphinxAtStartPar
なお、複数枚からなるキャラクターが
墓地や手札に移った場合、
1体のキャラクターとして
扱うため複数枚合わせて移します。
チャージ状態、ドライブ状態となった場合も同様に1体のキャラクター
として扱います。


\subsection{チャージとドライブ}
\label{\detokenize{common/common:id14}}
\sphinxAtStartPar
キャラクターには、チャージ状態とドライブ状態が存在します。
チャージ状態は未使用状態を示し、ドライブ状態は使用済み状態を示しています。
また、キャラクターを横向きにすることを「ドライブ」、縦向きにすることを「チャージ」と言います。(\hyperref[\detokenize{common/common:chargedrive}]{Fig.\@ \ref{\detokenize{common/common:chargedrive}}})

\begin{figure}[htbp]
\centering
\capstart

\noindent\sphinxincludegraphics{{charge&drive}.pdf}
\caption{チャージとドライブ}\label{\detokenize{common/common:id52}}\label{\detokenize{common/common:chargedrive}}\end{figure}


\section{ゲームの始め方}
\label{\detokenize{common/common:id15}}\begin{quote}

\sphinxAtStartPar
次の手順でゲームを始めます。
\begin{enumerate}
\sphinxsetlistlabels{\arabic}{enumi}{enumii}{}{.}%
\item {} 
\sphinxAtStartPar
デッキをよく切る。

\item {} 
\sphinxAtStartPar
デッキより7枚引き手札にする。

\item {} 
\sphinxAtStartPar
両者デッキの一番上を表にする。

\item {} 
\sphinxAtStartPar
大きい数字のプレイヤーが先攻。数字については、 \hyperref[\detokenize{common/common:cardrank}]{Table \ref{\detokenize{common/common:cardrank}}} 参照。

\item {} 
\sphinxAtStartPar
数字が同じ場合、さらにデッキの一番上を表にし同様のルールで比べる。

\item {} 
\sphinxAtStartPar
表にしたカードを墓地へ移す。

\item {} 
\sphinxAtStartPar
先攻プレイヤーはデッキより1枚引き手札に加える。

\item {} 
\sphinxAtStartPar
先攻プレイヤーがターンとチャンスをもちゲームを開始する。

\end{enumerate}
\end{quote}

\sphinxAtStartPar
この行動が \hyperref[\detokenize{core/core:gamestart}]{\ref{\detokenize{core/core:gamestart}} \nameref{\detokenize{core/core:gamestart}}} に該当します。
この後はコアフローに準じアクションを起こしてゲームを進行します。

\sphinxAtStartPar
ゲーム内で起こせるアクションは対戦レギュレーション、フォーマットより異なります。
対戦前に確認してください。


\section{アクション}
\label{\detokenize{common/common:id16}}

\subsection{アクションが持つ項目}
\label{\detokenize{common/common:id17}}
\sphinxAtStartPar
アクションが持つ項目について説明します。
凡例の「サンプル」アクションを見てみましょう。(\hyperref[\detokenize{common/common:action-sample}]{Fig.\@ \ref{\detokenize{common/common:action-sample}}})

\begin{figure}[htbp]
\centering
\capstart

\noindent\sphinxincludegraphics{{action-sample}.pdf}
\caption{サンプルアクション}\label{\detokenize{common/common:id53}}\label{\detokenize{common/common:action-sample}}\end{figure}
\begin{description}
\sphinxlineitem{アクション名}
\sphinxAtStartPar
アクションの名称を示します。

\sphinxlineitem{キーカード}
\sphinxAtStartPar
アクションの核となるカードを示します。
キーカードは★を使って表記します。
凡例の場合、手札からコストとは別に{\normalsize $\heartsuit$} A〜10に該当するカードを1枚
キーカードとして使用します。

\sphinxlineitem{特記事項}
\sphinxAtStartPar
特記事項は※を使って表記し、その他の項目では書き表せない条件を示します。

\sphinxlineitem{対象}
\sphinxAtStartPar
効果の対象を示します。

\sphinxlineitem{即時効果/通常効果}
\sphinxAtStartPar
効果の内容を示します。

\sphinxlineitem{コスト}
\sphinxAtStartPar
アクションを起こすのに必要な対価です。
コストは$を使って表記し、コストの支払いはアクションを起こすプレイヤーが行います。コストの種類は \hyperref[\detokenize{common/common:cost}]{\ref{\detokenize{common/common:cost}} \nameref{\detokenize{common/common:cost}}} で説明します。

\sphinxlineitem{タイミング}
\sphinxAtStartPar
アクションを起こせる時を示します。
タイミングはコアルール \hyperref[\detokenize{core/core:timing}]{\ref{\detokenize{core/core:timing}} \nameref{\detokenize{core/core:timing}}} を参照してください。

\sphinxlineitem{タイプ}
\sphinxAtStartPar
アクションの種類を表します。アクション名の後に括弧書きで記載します。

\end{description}


\subsubsection{記載されていないアクションの項目}
\label{\detokenize{common/common:id18}}
\sphinxAtStartPar
アクションによっては記載されていない項目もあります。
記載されていない項目は無視して構いません。
たとえばコスト項目がなければコストを支払う必要はありません。


\subsection{コストの種類}
\label{\detokenize{common/common:cost}}\label{\detokenize{common/common:id19}}
\sphinxAtStartPar
アクションによって支払うコストが異なります。
コストには次の種類があり、それぞれ支払い方が異なります。(\hyperref[\detokenize{common/common:table-cost}]{Table \ref{\detokenize{common/common:table-cost}}})


\begin{savenotes}\sphinxattablestart
\centering
\sphinxcapstartof{table}
\sphinxthecaptionisattop
\sphinxcaption{コストの種類}\label{\detokenize{common/common:id54}}\label{\detokenize{common/common:table-cost}}
\sphinxaftertopcaption
\begin{tabulary}{\linewidth}[t]{|T|T|}
\hline
\sphinxstyletheadfamily 
\sphinxAtStartPar
表記(名称)
&\sphinxstyletheadfamily 
\sphinxAtStartPar
対価
\\
\hline
\sphinxAtStartPar
B (Bulwark)
&
\sphinxAtStartPar
防壁をドライブする
\\
\hline
\sphinxAtStartPar
L (Life)
&
\sphinxAtStartPar
1点ダメージを受ける
\\
\hline
\sphinxAtStartPar
D (Discard)
&
\sphinxAtStartPar
手札を1枚捨てる
\\
\hline
\sphinxAtStartPar
S (Sacrifice)
&
\sphinxAtStartPar
キャラクター1体を墓地に移す
\\
\hline
\end{tabulary}
\par
\sphinxattableend\end{savenotes}

\sphinxAtStartPar
たとえばコストが \sphinxstylestrong{「\$BL」} の場合、自分の場にいるチャージ状態の防壁を1体ドライブし、1点ダメージを受けることでコストが支払われたことになります。


\subsection{アクションの起こし方}
\label{\detokenize{common/common:id20}}
\sphinxAtStartPar
次の手順でアクションを起こします。
\begin{enumerate}
\sphinxsetlistlabels{\arabic}{enumi}{enumii}{}{.}%
\item {} 
\sphinxAtStartPar
起こすアクションを対戦相手に伝える。

\item {} 
\sphinxAtStartPar
アクションに応じたコストを支払う。

\item {} 
\sphinxAtStartPar
必要なら手札からキーカードを出す。

\item {} 
\sphinxAtStartPar
対象の指定が必要な場合、対象を指定する。

\end{enumerate}

\sphinxAtStartPar
「サンプル」アクションを起こす例を見てみましょう。(\hyperref[\detokenize{common/common:action-sample2}]{Fig.\@ \ref{\detokenize{common/common:action-sample2}}})

\begin{figure}[htbp]
\centering
\capstart

\noindent\sphinxincludegraphics{{action-sample2}.pdf}
\caption{アクションを起こす例}\label{\detokenize{common/common:id55}}\label{\detokenize{common/common:action-sample2}}\end{figure}


\subsubsection{アクションを起こすときの注意点}
\label{\detokenize{common/common:id21}}

\paragraph{対象を指定しないでアクションを起こせるか?}
\label{\detokenize{common/common:id22}}
\sphinxAtStartPar
「サンプル」アクションのように対象を指定するアクションがあります。
「対象」項目がある場合、記載された条件を満たした対象を指定できなければ、
そのアクションを起こすことはできません。


\paragraph{アクションを対象とするアクションは自身を対象にできるか?}
\label{\detokenize{common/common:id23}}
\sphinxAtStartPar
アクションは、自分自身を対象とすることはできません。
そのため、「カウンター」アクションのようにアクションを対象とするアクションは
自身を対象とすることはできません。


\subsection{アクションの解決}
\label{\detokenize{common/common:id24}}
\sphinxAtStartPar
\hyperref[\detokenize{core/core:coreflowsec}]{\ref{\detokenize{core/core:coreflowsec}} \nameref{\detokenize{core/core:coreflowsec}}} の
\hyperref[\detokenize{core/core:actresolve}]{\ref{\detokenize{core/core:actresolve}} \nameref{\detokenize{core/core:actresolve}}} に行うことを順に示します。


\subsubsection{対象条件を確認}
\label{\detokenize{common/common:id25}}
\sphinxAtStartPar
対象を指定するアクションが効果を発揮しようとした時に
対象が存在していない場合、効果を発揮する対象を失うため効果が発揮されず
アクションが解決されます。

\sphinxAtStartPar
たとえば兵士に対して「アップ」アクションを起こし、対応して「ダウン」
アクションを起こされました。
「ダウン」の方が先に解決されるため、「アップ」を解決する時には
兵士が墓地に移っていたとします。その場合、「アップ」アクションは効果を発揮せず解決されます。


\subsubsection{効果を発揮}
\label{\detokenize{common/common:id26}}
\sphinxAtStartPar
アクションの効果に定義されている内容を実行します。
効果の中に実行不可能な部分がある場合、可能な部分のみ実行します。

\sphinxAtStartPar
たとえば、デッキの枚数が残1枚の時に5点のダメージを受けたとします。
デッキは1枚しかないので5点ダメージを受けることはできませんが、
1点までなら受けることが可能なため、
この場合1点のダメージを受けることになります。


\subsubsection{キーカードを墓地に移す}
\label{\detokenize{common/common:keycard-gy}}\label{\detokenize{common/common:id27}}
\sphinxAtStartPar
効果を発揮した後、そのアクションをステージから取り除き、キーカードを墓地に移します。
ただし効果によってキーカードを場に出した場合や手札に戻した場合、
そのカードを移す先が明確になっているため、墓地には移しません。


\subsection{勝敗判定}
\label{\detokenize{common/common:id28}}
\sphinxAtStartPar
\hyperref[\detokenize{core/core:winlose}]{\ref{\detokenize{core/core:winlose}} \nameref{\detokenize{core/core:winlose}}} で確認する内容は次になります。

\sphinxAtStartPar
デッキを確認し0枚の場合そのプレイヤーは敗北となります。両プレイヤーのデッキが0枚の場合、引き分けとなります。


\subsection{その他補足事項}
\label{\detokenize{common/common:id29}}

\subsubsection{防壁の置き方}
\label{\detokenize{common/common:id30}}
\sphinxAtStartPar
防壁を場に出すときは次のルールにしたがって場に出して下さい。(\hyperref[\detokenize{common/common:set-bulwork}]{Fig.\@ \ref{\detokenize{common/common:set-bulwork}}})
\begin{itemize}
\item {} 
\sphinxAtStartPar
防壁を置く時はデッキ側に詰めて置いて下さい。

\item {} 
\sphinxAtStartPar
防壁の左右の入れ替えは行わないでください。

\end{itemize}

\begin{figure}[htbp]
\centering
\capstart

\noindent\sphinxincludegraphics{{set-bulwork}.pdf}
\caption{防壁の置き方}\label{\detokenize{common/common:id56}}\label{\detokenize{common/common:set-bulwork}}\end{figure}


\subsubsection{1ターンに1回制限}
\label{\detokenize{common/common:id31}}
\sphinxAtStartPar
特記事項に「プレイヤーは1ターンに1回しかこのアクションを起こすことができない。」と記載されているアクションは、
ターンを持っているプレイヤーが変わるまでの間に1回しか起こす
ことができません。

\sphinxAtStartPar
ターンを持っているプレイヤーが変わればまた起こすことができます。


\subsubsection{直接起こせないアクション}
\label{\detokenize{common/common:id32}}
\sphinxAtStartPar
特記事項に「プレイヤーはこのアクションを直接起こすことが出来ない。」
と記載されているアクションは、
プレイヤーがチャンスを持っていても
アクションを起こすことができません。
また、この特記事項が記載されたアクションが何らかの起因で起きても、プレイヤーが起こした訳ではないためパスは自動的に発生せず、チャンスは移りません。


\section{エクストラ}
\label{\detokenize{common/common:extra}}\label{\detokenize{common/common:id33}}
\sphinxAtStartPar
エクストラではアクションに加え切札の能力を使うことができます。
使用できるアクション、切札は対戦レギュレーションを確認してください。


\subsection{切札}
\label{\detokenize{common/common:id34}}
\sphinxAtStartPar
切札とは、切札領域に置かれたカードを示します。
具体的な切札の置き場所については、 \hyperref[\detokenize{common/common:field-ex}]{Fig.\@ \ref{\detokenize{common/common:field-ex}}} を参照して下さい。
切札には各々能力が割り当てられており、表にするとその能力が有効になります。
切札を操作するアクションは、「エクストラリスト」を参照して下さい。


\subsection{バージョン}
\label{\detokenize{common/common:id35}}
\sphinxAtStartPar
エクストラには、バージョンが存在します。
対戦を開始する前に対戦相手とバージョンの確認をしましょう。


\subsection{版数との関係}
\label{\detokenize{common/common:id36}}
\sphinxAtStartPar
版数毎に使える切札の種類が異なります。
たとえば、第一版、第二版ではエクストラで遊ぶことはできません。
第三版以降は、次版が出るまでの間に公開された切札であれば
使用することができます。(\hyperref[\detokenize{common/common:ver-ex}]{Table \ref{\detokenize{common/common:ver-ex}}})


\begin{savenotes}\sphinxattablestart
\centering
\sphinxcapstartof{table}
\sphinxthecaptionisattop
\sphinxcaption{版数とエクストラのバージョン}\label{\detokenize{common/common:id57}}\label{\detokenize{common/common:ver-ex}}
\sphinxaftertopcaption
\begin{tabulary}{\linewidth}[t]{|T|T|}
\hline
\sphinxstyletheadfamily 
\sphinxAtStartPar
版数
&\sphinxstyletheadfamily 
\sphinxAtStartPar
エクストラのバージョン
\\
\hline
\sphinxAtStartPar
第一版
&
\sphinxAtStartPar
−
\\
\hline
\sphinxAtStartPar
第二版
&
\sphinxAtStartPar
−
\\
\hline
\sphinxAtStartPar
第三版
&
\sphinxAtStartPar
ex3.4.0 〜 ex3.10.0
\\
\hline
\sphinxAtStartPar
第四版
&
\sphinxAtStartPar
ex4.14.0 〜 ex4.22.0
\\
\hline
\sphinxAtStartPar
第五版
&
\sphinxAtStartPar
ex5.22.0 〜
\\
\hline
\end{tabulary}
\par
\sphinxattableend\end{savenotes}


\subsection{ゲームのはじめ方}
\label{\detokenize{common/common:extra-start}}\label{\detokenize{common/common:id37}}
\sphinxAtStartPar
エクストラでは、切札を置いてからゲームを始めます。
切札を置くルールは次のようになっています。(\hyperref[\detokenize{common/common:trump}]{Fig.\@ \ref{\detokenize{common/common:trump}}})
\begin{itemize}
\item {} 
\sphinxAtStartPar
対戦前に裏向きで2枚まで切札を置くことができる。

\item {} 
\sphinxAtStartPar
切札はデッキと角度を変えて交わるようにデッキの下に置く。

\item {} 
\sphinxAtStartPar
切札を表にするときはスートと数字が見えるようにし、対応する能力の名称を言う。

\item {} 
\sphinxAtStartPar
デッキが0枚になった場合、切札が残っていても敗北する。

\item {} 
\sphinxAtStartPar
能力が割り当てられていないカードも切札とすることができるが、表になっても能力が有効にならない。

\end{itemize}

\begin{figure}[htbp]
\centering
\capstart

\noindent\sphinxincludegraphics{{trump}.pdf}
\caption{切札の置き方}\label{\detokenize{common/common:id58}}\label{\detokenize{common/common:trump}}\end{figure}

\sphinxAtStartPar
これ以降は、通常のゲームの始め方と同様です。


\subsection{切札の能力}
\label{\detokenize{common/common:id38}}
\sphinxAtStartPar
エクストラでは切札を使って能力を得ることができます。
切札1枚1枚に異なった能力が割り当てられており、
表にすることで能力が有効になります。
割り当てられている能力については、「エクストラリスト」を参照して下さい。


\subsubsection{能力を有効にする}
\label{\detokenize{common/common:id39}}
\sphinxAtStartPar
切札に割り当てられた能力は
「オープン」アクションを起こし表にすることで有効になります。(\hyperref[\detokenize{common/common:trump-open}]{Fig.\@ \ref{\detokenize{common/common:trump-open}}})
「オープン」アクションの詳細は、 \hyperref[\detokenize{appendix/appendix:extralist}]{\ref{\detokenize{appendix/appendix:extralist}} \nameref{\detokenize{appendix/appendix:extralist}}} を参照して下さい。
切札が表でいる限り、
その切札の能力は持続的に有効になります。
また切札を表にする時は、
対戦相手に有効となった能力が分かるように、
能力の名称を言いスートと数字が見えるようにしましょう。

\begin{figure}[htbp]
\centering
\capstart

\noindent\sphinxincludegraphics{{trump-open}.pdf}
\caption{切札を表にする例}\label{\detokenize{common/common:id59}}\label{\detokenize{common/common:trump-open}}\end{figure}


\subsubsection{能力を無効する}
\label{\detokenize{common/common:id40}}
\sphinxAtStartPar
切札は裏向きもしくは、
墓地に移されると能力が無効になります。
切札を無効化するためには、「クローズ」アクションを用い
切札を裏向きにするか、
「切札破壊」アクションを用いて切札を破壊しましょう。
「クローズ」アクション、
「切札破壊」アクションの詳細は、 \hyperref[\detokenize{appendix/appendix:extralist}]{\ref{\detokenize{appendix/appendix:extralist}} \nameref{\detokenize{appendix/appendix:extralist}}} を参照して下さい。


\subsection{エクストラ注意事項}
\label{\detokenize{common/common:id41}}

\subsubsection{1ターンに1回制限のアクションについて}
\label{\detokenize{common/common:id42}}
\sphinxAtStartPar
切札がもたらすアクションの中には「プレイヤーは1ターンに1回しかこのアクションを起こすことができない。」
と特記事項に記載されているものがあります。
このアクションは1ターンに1回しか起こすことができないため、
切札が無効化され再度オープンし有効となっても、そのターンを通して1回しか起こすことができません。


\section{その他のルール}
\label{\detokenize{common/common:id43}}
\sphinxAtStartPar
この章では、
公開・非公開情報やシャッフルの仕方といった
細かな決まりごとを説明します。


\subsection{公開・非公開情報}
\label{\detokenize{common/common:id44}}
\sphinxAtStartPar
配置されているカードには、アクションの効果
を使わなくても中身や枚数を知れるものがあります。
知れる度合いには次の種類があります。
\begin{description}
\sphinxlineitem{完全公開}
\sphinxAtStartPar
全てのプレイヤーが知ることができ、
聞かれたプレイヤーは正しく答える必要がある

\sphinxlineitem{個人公開}
\sphinxAtStartPar
デッキの持ち主のみ知ることができる

\sphinxlineitem{非公開}
\sphinxAtStartPar
全てのプレイヤーは知ることができない

\end{description}

\sphinxAtStartPar
完全公開の情報であれば、ゲーム中いつでも対戦相手に聞くことができます。
各カードの配置と公開・非公開の度合いは次のとおりです。
\begin{description}
\sphinxlineitem{デッキ}
\begin{DUlineblock}{0em}
\item[] 完全公開:10枚未満のデッキ枚数
\item[] 個人公開:デッキの枚数
\item[] 非公開:デッキの中身
\end{DUlineblock}

\sphinxlineitem{墓地}
\begin{DUlineblock}{0em}
\item[] 完全公開:墓地の一番上のカード
\item[] 個人公開:墓地の中身
\item[] 非公開:なし
\end{DUlineblock}

\sphinxlineitem{場}
\begin{DUlineblock}{0em}
\item[] 完全公開:表裏を変えずに見えるカード
\item[] 個人公開:伏せてあるカード
\item[] 非公開:なし
\end{DUlineblock}

\sphinxlineitem{手札}
\begin{DUlineblock}{0em}
\item[] 完全公開:手札の枚数
\item[] 個人公開:手札の中身
\item[] 非公開:なし
\end{DUlineblock}

\sphinxlineitem{切札}
\begin{DUlineblock}{0em}
\item[] 完全公開:表裏を変えずに見えるカード
\item[] 個人公開:伏せてあるカード
\item[] 非公開:なし
\end{DUlineblock}

\end{description}


\subsubsection{残りのデッキ枚数を聞かれたらどうしたらいいの?}
\label{\detokenize{common/common:id45}}
\sphinxAtStartPar
対戦相手から残りのデッキ枚数を聞かれた場合、自分のデッキの枚数を上から10枚まで数え、相手に数えたカードの枚数が分かるように裏向きで見せます。
10枚未満であれば枚数を答え、10枚以上の場合「10枚以上です」と答えて下さい。
10枚以上の場合、正確な枚数を答える必要はありません。


\subsubsection{墓地の一番上のカードはいつ決まるのか?}
\label{\detokenize{common/common:id46}}
\sphinxAtStartPar
カードを墓地に移す際に移すカードの中から1枚を公開してください。
すでに墓地にあるカードを改めて公開しないでください。


\subsection{デッキのシャッフルについて}
\label{\detokenize{common/common:id47}}
\sphinxAtStartPar
BlackPokerでは
コンセプトの1つに”相手のカードに触らない”があるため、
対戦相手にデッキのシャッフルをお願いする必要はありません。

\sphinxAtStartPar
ただシャッフルしてほしいのであれば、お願いしても構いません。
逆に、対戦相手があまりシャッフルしていない場合は、
さらにシャッフルをお願いすることができます。

\sphinxstepscope


\chapter{フォーマット}
\label{\detokenize{format/format:id1}}\label{\detokenize{format/format::doc}}

\section{フォーマットとは}
\label{\detokenize{format/format:id2}}
\sphinxAtStartPar
BlackPokerにはいくつかのフォーマットがあり、
フォーマットによりゲーム内でできる行動が異なります。
同じトランプでもフォーマットを変えることで様々な遊び方をすることができます。

\sphinxAtStartPar
BlackPokerはアクションという行動を起こし、兵士などのキャラクターを出してターンを進めていくゲームです。
アクション、キャラクターの種類は、ライト < スタンダード < プロ < マスター < エクストラの順に増えていきます。
カードゲームでいうところのカードの種類が増えていくイメージです。
覚える量が多いほど難易度が高いため、初心者はライトから始めることをお勧めします。

\sphinxAtStartPar
アクションはアクションリストに記載されており、
フォーマットによって参照するアクションリストが異なります。


\subsection{対戦レギュレーションとの違い}
\label{\detokenize{format/format:id3}}
\sphinxAtStartPar
フォーマットと対戦レギュレーションの違いは、
フォーマットはゲーム内でできるアクション等のできる行動を定義している
のに対して、対戦レギュレーションはフォーマットを前提としてそれを加工している位置づけになります。

\sphinxAtStartPar
フォーマット名と対戦レギュレーション名が等しい対戦レギュレーションは、
フォーマットのルールが無加工で遊べます。


\subsection{共通ルールとの関係}
\label{\detokenize{format/format:id4}}
\sphinxAtStartPar
フォーマットは共通ルールを参照しており、
共通ルールにBlackPokerの主なルールが定義されています。

\sphinxAtStartPar
フォーマット、対戦レギュレーション、共通ルールの関係は次の図のようになります。

\noindent\sphinxincludegraphics{{plantuml-c4b165770c9149e49e8be64e90ba8bbbafdb51b5}.pdf}


\section{種類}
\label{\detokenize{format/format:id5}}
\sphinxAtStartPar
フォーマットは次の種類があります。


\begin{savenotes}\sphinxattablestart
\centering
\begin{tabulary}{\linewidth}[t]{|T|T|T|T|T|}
\hline
\sphinxstyletheadfamily 
\sphinxAtStartPar
フォーマット名
&\sphinxstyletheadfamily 
\sphinxAtStartPar
難易度
&\sphinxstyletheadfamily 
\sphinxAtStartPar
アクション数
&\sphinxstyletheadfamily 
\sphinxAtStartPar
キャラクター数
&\sphinxstyletheadfamily 
\sphinxAtStartPar
切札有無
\\
\hline
\sphinxAtStartPar
ライト
&
\sphinxAtStartPar
★
&
\sphinxAtStartPar
17
&
\sphinxAtStartPar
5
&
\sphinxAtStartPar
無
\\
\hline
\sphinxAtStartPar
スタンダード
&
\sphinxAtStartPar
★★
&
\sphinxAtStartPar
24
&
\sphinxAtStartPar
6
&
\sphinxAtStartPar
無
\\
\hline
\sphinxAtStartPar
プロ
&
\sphinxAtStartPar
★★★
&
\sphinxAtStartPar
29
&
\sphinxAtStartPar
6
&
\sphinxAtStartPar
無
\\
\hline
\sphinxAtStartPar
マスター
&
\sphinxAtStartPar
★★★★
&
\sphinxAtStartPar
35
&
\sphinxAtStartPar
6
&
\sphinxAtStartPar
無
\\
\hline
\sphinxAtStartPar
エクストラ
&
\sphinxAtStartPar
★★★★★
&
\sphinxAtStartPar
39〜
&
\sphinxAtStartPar
6〜
&
\sphinxAtStartPar
有
\\
\hline
\end{tabulary}
\par
\sphinxattableend\end{savenotes}

\sphinxAtStartPar
アクション数、キャラクター数の増加にともない覚える数が増えるため、難易度が上がります。


\section{定義項目}
\label{\detokenize{format/format:id6}}
\sphinxAtStartPar
フォーマットには次の項目が定義されています。
\begin{description}
\sphinxlineitem{アクションリスト}
\sphinxAtStartPar
起こせるアクション、キャラクターのリスト

\sphinxlineitem{エクストラリスト}
\sphinxAtStartPar
切札のリスト

\sphinxlineitem{事前準備}
\sphinxAtStartPar
対戦前に行う事項

\sphinxlineitem{その他事項}
\sphinxAtStartPar
上記項目で説明できないルール

\end{description}


\section{フォーマット定義}
\label{\detokenize{format/format:id7}}
\sphinxAtStartPar
公式として次のフォーマットを定義しています。

\sphinxstepscope


\subsection{ライト}
\label{\detokenize{format/lite:format-lite}}\label{\detokenize{format/lite:id1}}\label{\detokenize{format/lite::doc}}

\subsubsection{アクションリスト}
\label{\detokenize{format/lite:id2}}\begin{itemize}
\item {} 
\sphinxAtStartPar
\hyperref[\detokenize{appendix/appendix:actionlist-lite}]{\ref{\detokenize{appendix/appendix:actionlist-lite}} \nameref{\detokenize{appendix/appendix:actionlist-lite}}}

\end{itemize}


\subsubsection{エクストラリスト}
\label{\detokenize{format/lite:id3}}\begin{itemize}
\item {} 
\sphinxAtStartPar
なし

\end{itemize}


\subsubsection{事前準備}
\label{\detokenize{format/lite:id4}}\begin{itemize}
\item {} 
\sphinxAtStartPar
なし

\end{itemize}


\subsubsection{その他事項}
\label{\detokenize{format/lite:id5}}\begin{itemize}
\item {} 
\sphinxAtStartPar
なし

\end{itemize}

\sphinxstepscope


\subsection{スタンダード}
\label{\detokenize{format/standard:format-standard}}\label{\detokenize{format/standard:id1}}\label{\detokenize{format/standard::doc}}

\subsubsection{アクションリスト}
\label{\detokenize{format/standard:id2}}\begin{itemize}
\item {} 
\sphinxAtStartPar
\hyperref[\detokenize{appendix/appendix:actionlist-std}]{\ref{\detokenize{appendix/appendix:actionlist-std}} \nameref{\detokenize{appendix/appendix:actionlist-std}}}

\end{itemize}


\subsubsection{エクストラリスト}
\label{\detokenize{format/standard:id3}}\begin{itemize}
\item {} 
\sphinxAtStartPar
なし

\end{itemize}


\subsubsection{事前準備}
\label{\detokenize{format/standard:id4}}\begin{itemize}
\item {} 
\sphinxAtStartPar
なし

\end{itemize}


\subsubsection{その他事項}
\label{\detokenize{format/standard:id5}}\begin{itemize}
\item {} 
\sphinxAtStartPar
なし

\end{itemize}

\sphinxstepscope


\subsection{プロ}
\label{\detokenize{format/pro:format-pro}}\label{\detokenize{format/pro:id1}}\label{\detokenize{format/pro::doc}}

\subsubsection{アクションリスト}
\label{\detokenize{format/pro:id2}}\begin{itemize}
\item {} 
\sphinxAtStartPar
\hyperref[\detokenize{appendix/appendix:actionlist-pro}]{\ref{\detokenize{appendix/appendix:actionlist-pro}} \nameref{\detokenize{appendix/appendix:actionlist-pro}}}

\end{itemize}


\subsubsection{エクストラリスト}
\label{\detokenize{format/pro:id3}}\begin{itemize}
\item {} 
\sphinxAtStartPar
なし

\end{itemize}


\subsubsection{事前準備}
\label{\detokenize{format/pro:id4}}\begin{itemize}
\item {} 
\sphinxAtStartPar
なし

\end{itemize}


\subsubsection{その他事項}
\label{\detokenize{format/pro:id5}}\begin{itemize}
\item {} 
\sphinxAtStartPar
なし

\end{itemize}

\sphinxstepscope


\subsection{マスター}
\label{\detokenize{format/master:format-master}}\label{\detokenize{format/master:id1}}\label{\detokenize{format/master::doc}}

\subsubsection{アクションリスト}
\label{\detokenize{format/master:id2}}\begin{itemize}
\item {} 
\sphinxAtStartPar
\hyperref[\detokenize{appendix/appendix:actionlist-master}]{\ref{\detokenize{appendix/appendix:actionlist-master}} \nameref{\detokenize{appendix/appendix:actionlist-master}}}

\end{itemize}


\subsubsection{エクストラリスト}
\label{\detokenize{format/master:id3}}\begin{itemize}
\item {} 
\sphinxAtStartPar
なし

\end{itemize}


\subsubsection{事前準備}
\label{\detokenize{format/master:id4}}\begin{itemize}
\item {} 
\sphinxAtStartPar
なし

\end{itemize}


\subsubsection{その他事項}
\label{\detokenize{format/master:id5}}\begin{itemize}
\item {} 
\sphinxAtStartPar
なし

\end{itemize}

\sphinxstepscope


\subsection{エクストラ}
\label{\detokenize{format/extra:format-extra}}\label{\detokenize{format/extra:id1}}\label{\detokenize{format/extra::doc}}

\subsubsection{アクションリスト}
\label{\detokenize{format/extra:id2}}\begin{itemize}
\item {} 
\sphinxAtStartPar
\hyperref[\detokenize{appendix/appendix:actionlist-master}]{\ref{\detokenize{appendix/appendix:actionlist-master}} \nameref{\detokenize{appendix/appendix:actionlist-master}}}

\end{itemize}


\subsubsection{エクストラリスト}
\label{\detokenize{format/extra:id3}}\begin{itemize}
\item {} 
\sphinxAtStartPar
\hyperref[\detokenize{appendix/appendix:extralist}]{\ref{\detokenize{appendix/appendix:extralist}} \nameref{\detokenize{appendix/appendix:extralist}}}

\end{itemize}


\subsubsection{事前準備}
\label{\detokenize{format/extra:id4}}\begin{itemize}
\item {} 
\sphinxAtStartPar
切札を置く。詳細は、 \hyperref[\detokenize{common/common:extra-start}]{\ref{\detokenize{common/common:extra-start}} \nameref{\detokenize{common/common:extra-start}}} 参照。

\end{itemize}


\subsubsection{その他事項}
\label{\detokenize{format/extra:id5}}\begin{itemize}
\item {} 
\sphinxAtStartPar
なし

\end{itemize}

\sphinxstepscope


\chapter{対戦レギュレーション}
\label{\detokenize{match-regulations/match-regulations:id1}}\label{\detokenize{match-regulations/match-regulations::doc}}
\sphinxAtStartPar
対戦レギュレーションとは、
BlackPokerで対戦する前にプレイヤー間で決定する
規則のことです。

\sphinxAtStartPar
BlackPokerはトランプだけで遊べるため、
対戦する前にプレイヤー間でルールのすり合わせをする必要があります。


\section{定義項目}
\label{\detokenize{match-regulations/match-regulations:id2}}
\sphinxAtStartPar
各対戦レギュレーションには次の項目が定義されています。
\begin{description}
\sphinxlineitem{フォーマット}
\sphinxAtStartPar
使用するフォーマット。詳しくは {\hyperref[\detokenize{format/format::doc}]{\sphinxcrossref{\DUrole{doc}{フォーマット}}}} 参照

\sphinxlineitem{デッキ条件}
\sphinxAtStartPar
対戦に使用するデッキの条件

\sphinxlineitem{対戦前準備事項}
\sphinxAtStartPar
切札の選定など対戦前に行う事項

\sphinxlineitem{その他制約事項}
\sphinxAtStartPar
上記項目で説明できない制約事項

\end{description}


\section{レギュレーション定義}
\label{\detokenize{match-regulations/match-regulations:id3}}
\sphinxAtStartPar
公式として次の対戦レギュレーションを定義しています。

\sphinxstepscope


\subsection{ライト}
\label{\detokenize{match-regulations/lite:id1}}\label{\detokenize{match-regulations/lite::doc}}

\subsubsection{フォーマット}
\label{\detokenize{match-regulations/lite:id2}}
\sphinxAtStartPar
\hyperref[\detokenize{format/lite:format-lite}]{\ref{\detokenize{format/lite:format-lite}} \nameref{\detokenize{format/lite:format-lite}}}


\subsubsection{デッキ条件}
\label{\detokenize{match-regulations/lite:id3}}
\sphinxAtStartPar
次のカードの中から54までカードを選びデッキとする


\begin{savenotes}\sphinxattablestart
\centering
\begin{tabulary}{\linewidth}[t]{|T|T|}
\hline

\sphinxAtStartPar
{\normalsize $\spadesuit$} 
&
\sphinxAtStartPar
A〜K
\\
\hline
\sphinxAtStartPar
{\normalsize $\heartsuit$} 
&
\sphinxAtStartPar
A〜K
\\
\hline
\sphinxAtStartPar
{\normalsize $\diamondsuit$} 
&
\sphinxAtStartPar
A〜K
\\
\hline
\sphinxAtStartPar
{\normalsize $\clubsuit$} 
&
\sphinxAtStartPar
A〜K
\\
\hline\sphinxstartmulticolumn{2}%
\begin{varwidth}[t]{\sphinxcolwidth{2}{2}}
\sphinxAtStartPar
Joker x2
\par
\vskip-\baselineskip\vbox{\hbox{\strut}}\end{varwidth}%
\sphinxstopmulticolumn
\\
\hline
\end{tabulary}
\par
\sphinxattableend\end{savenotes}


\subsubsection{対戦前準備事項}
\label{\detokenize{match-regulations/lite:id4}}\begin{itemize}
\item {} 
\sphinxAtStartPar
なし

\end{itemize}


\subsubsection{その他制限事項}
\label{\detokenize{match-regulations/lite:id5}}\begin{itemize}
\item {} 
\sphinxAtStartPar
なし

\end{itemize}

\sphinxstepscope


\subsection{ライト40}
\label{\detokenize{match-regulations/lite40:id1}}\label{\detokenize{match-regulations/lite40::doc}}

\subsubsection{フォーマット}
\label{\detokenize{match-regulations/lite40:id2}}
\sphinxAtStartPar
\hyperref[\detokenize{format/lite:format-lite}]{\ref{\detokenize{format/lite:format-lite}} \nameref{\detokenize{format/lite:format-lite}}}


\subsubsection{デッキ条件}
\label{\detokenize{match-regulations/lite40:id3}}
\sphinxAtStartPar
次のカードの中から40枚までカードを選びデッキとする


\begin{savenotes}\sphinxattablestart
\centering
\begin{tabulary}{\linewidth}[t]{|T|T|}
\hline

\sphinxAtStartPar
{\normalsize $\spadesuit$} 
&
\sphinxAtStartPar
A〜K
\\
\hline
\sphinxAtStartPar
{\normalsize $\heartsuit$} 
&
\sphinxAtStartPar
A〜K
\\
\hline
\sphinxAtStartPar
{\normalsize $\diamondsuit$} 
&
\sphinxAtStartPar
A〜K
\\
\hline
\sphinxAtStartPar
{\normalsize $\clubsuit$} 
&
\sphinxAtStartPar
A〜K
\\
\hline\sphinxstartmulticolumn{2}%
\begin{varwidth}[t]{\sphinxcolwidth{2}{2}}
\sphinxAtStartPar
Joker x2
\par
\vskip-\baselineskip\vbox{\hbox{\strut}}\end{varwidth}%
\sphinxstopmulticolumn
\\
\hline
\end{tabulary}
\par
\sphinxattableend\end{savenotes}


\subsubsection{対戦前準備事項}
\label{\detokenize{match-regulations/lite40:id4}}\begin{itemize}
\item {} 
\sphinxAtStartPar
なし

\end{itemize}


\subsubsection{その他制限事項}
\label{\detokenize{match-regulations/lite40:id5}}\begin{itemize}
\item {} 
\sphinxAtStartPar
なし

\end{itemize}

\sphinxstepscope


\subsection{ライトランダムハーフ}
\label{\detokenize{match-regulations/lite_randomhalf:id1}}\label{\detokenize{match-regulations/lite_randomhalf::doc}}

\subsubsection{フォーマット}
\label{\detokenize{match-regulations/lite_randomhalf:id2}}
\sphinxAtStartPar
\hyperref[\detokenize{format/lite:format-lite}]{\ref{\detokenize{format/lite:format-lite}} \nameref{\detokenize{format/lite:format-lite}}}


\subsubsection{デッキ条件}
\label{\detokenize{match-regulations/lite_randomhalf:id3}}
\sphinxAtStartPar
次のカードの中からランダムに27枚ずつ選び2つのデッキとする


\begin{savenotes}\sphinxattablestart
\centering
\begin{tabulary}{\linewidth}[t]{|T|T|}
\hline

\sphinxAtStartPar
{\normalsize $\spadesuit$} 
&
\sphinxAtStartPar
A〜K
\\
\hline
\sphinxAtStartPar
{\normalsize $\heartsuit$} 
&
\sphinxAtStartPar
A〜K
\\
\hline
\sphinxAtStartPar
{\normalsize $\diamondsuit$} 
&
\sphinxAtStartPar
A〜K
\\
\hline
\sphinxAtStartPar
{\normalsize $\clubsuit$} 
&
\sphinxAtStartPar
A〜K
\\
\hline\sphinxstartmulticolumn{2}%
\begin{varwidth}[t]{\sphinxcolwidth{2}{2}}
\sphinxAtStartPar
Joker x2
\par
\vskip-\baselineskip\vbox{\hbox{\strut}}\end{varwidth}%
\sphinxstopmulticolumn
\\
\hline
\end{tabulary}
\par
\sphinxattableend\end{savenotes}


\subsubsection{対戦前準備事項}
\label{\detokenize{match-regulations/lite_randomhalf:id4}}\begin{itemize}
\item {} 
\sphinxAtStartPar
デッキ条件に該当するようにデッキを2つ用意する。

\item {} 
\sphinxAtStartPar
試合毎にいずれか1つのデッキを使用する。

\item {} 
\sphinxAtStartPar
試合に使用しないデッキは重ねて左前方に横向きに置く。勝利した試合に使用したデッキであれば表向き、そうでなければ裏向きとする。

\item {} 
\sphinxAtStartPar
デッキは非公開である。つまり、カードのオーナーも見ることができない。ただし、試合に使用したデッキは試合の終了後に一時的に個人公開となり、次の試合の開始前に非公開となる。

\end{itemize}


\subsubsection{その他制限事項}
\label{\detokenize{match-regulations/lite_randomhalf:id5}}\begin{itemize}
\item {} 
\sphinxAtStartPar
2試合以上を行って勝者を決める。この対戦をマッチと言う。マッチの勝利条件は2試合に勝つことである。

\item {} 
\sphinxAtStartPar
試合毎に使用するデッキを選択する。ただし、勝利した試合に使用したデッキはそれ以降使用できない。

\end{itemize}

\sphinxstepscope


\subsection{スタンダード}
\label{\detokenize{match-regulations/standard:id1}}\label{\detokenize{match-regulations/standard::doc}}

\subsubsection{フォーマット}
\label{\detokenize{match-regulations/standard:id2}}
\sphinxAtStartPar
\hyperref[\detokenize{format/standard:format-standard}]{\ref{\detokenize{format/standard:format-standard}} \nameref{\detokenize{format/standard:format-standard}}}


\subsubsection{デッキ条件}
\label{\detokenize{match-regulations/standard:id3}}
\sphinxAtStartPar
次のカードの中から54までカードを選びデッキとする


\begin{savenotes}\sphinxattablestart
\centering
\begin{tabulary}{\linewidth}[t]{|T|T|}
\hline

\sphinxAtStartPar
{\normalsize $\spadesuit$} 
&
\sphinxAtStartPar
A〜K
\\
\hline
\sphinxAtStartPar
{\normalsize $\heartsuit$} 
&
\sphinxAtStartPar
A〜K
\\
\hline
\sphinxAtStartPar
{\normalsize $\diamondsuit$} 
&
\sphinxAtStartPar
A〜K
\\
\hline
\sphinxAtStartPar
{\normalsize $\clubsuit$} 
&
\sphinxAtStartPar
A〜K
\\
\hline\sphinxstartmulticolumn{2}%
\begin{varwidth}[t]{\sphinxcolwidth{2}{2}}
\sphinxAtStartPar
Joker x2
\par
\vskip-\baselineskip\vbox{\hbox{\strut}}\end{varwidth}%
\sphinxstopmulticolumn
\\
\hline
\end{tabulary}
\par
\sphinxattableend\end{savenotes}


\subsubsection{対戦前準備事項}
\label{\detokenize{match-regulations/standard:id4}}\begin{itemize}
\item {} 
\sphinxAtStartPar
なし

\end{itemize}


\subsubsection{その他制限事項}
\label{\detokenize{match-regulations/standard:id5}}\begin{itemize}
\item {} 
\sphinxAtStartPar
なし

\end{itemize}

\sphinxstepscope


\subsection{スタンダード40}
\label{\detokenize{match-regulations/standard40:id1}}\label{\detokenize{match-regulations/standard40::doc}}

\subsubsection{フォーマット}
\label{\detokenize{match-regulations/standard40:id2}}
\sphinxAtStartPar
\hyperref[\detokenize{format/standard:format-standard}]{\ref{\detokenize{format/standard:format-standard}} \nameref{\detokenize{format/standard:format-standard}}}


\subsubsection{デッキ条件}
\label{\detokenize{match-regulations/standard40:id3}}
\sphinxAtStartPar
次のカードの中から40枚までカードを選びデッキとする


\begin{savenotes}\sphinxattablestart
\centering
\begin{tabulary}{\linewidth}[t]{|T|T|}
\hline

\sphinxAtStartPar
{\normalsize $\spadesuit$} 
&
\sphinxAtStartPar
A〜K
\\
\hline
\sphinxAtStartPar
{\normalsize $\heartsuit$} 
&
\sphinxAtStartPar
A〜K
\\
\hline
\sphinxAtStartPar
{\normalsize $\diamondsuit$} 
&
\sphinxAtStartPar
A〜K
\\
\hline
\sphinxAtStartPar
{\normalsize $\clubsuit$} 
&
\sphinxAtStartPar
A〜K
\\
\hline\sphinxstartmulticolumn{2}%
\begin{varwidth}[t]{\sphinxcolwidth{2}{2}}
\sphinxAtStartPar
Joker x2
\par
\vskip-\baselineskip\vbox{\hbox{\strut}}\end{varwidth}%
\sphinxstopmulticolumn
\\
\hline
\end{tabulary}
\par
\sphinxattableend\end{savenotes}


\subsubsection{対戦前準備事項}
\label{\detokenize{match-regulations/standard40:id4}}\begin{itemize}
\item {} 
\sphinxAtStartPar
なし

\end{itemize}


\subsubsection{その他制限事項}
\label{\detokenize{match-regulations/standard40:id5}}\begin{itemize}
\item {} 
\sphinxAtStartPar
なし

\end{itemize}

\sphinxstepscope


\subsection{スタンダードランダムハーフ}
\label{\detokenize{match-regulations/standard_randomhalf:id1}}\label{\detokenize{match-regulations/standard_randomhalf::doc}}

\subsubsection{フォーマット}
\label{\detokenize{match-regulations/standard_randomhalf:id2}}
\sphinxAtStartPar
\hyperref[\detokenize{format/standard:format-standard}]{\ref{\detokenize{format/standard:format-standard}} \nameref{\detokenize{format/standard:format-standard}}}


\subsubsection{デッキ条件}
\label{\detokenize{match-regulations/standard_randomhalf:id3}}
\sphinxAtStartPar
次のカードの中からランダムに27枚ずつ選び2つのデッキとする


\begin{savenotes}\sphinxattablestart
\centering
\begin{tabulary}{\linewidth}[t]{|T|T|}
\hline

\sphinxAtStartPar
{\normalsize $\spadesuit$} 
&
\sphinxAtStartPar
A〜K
\\
\hline
\sphinxAtStartPar
{\normalsize $\heartsuit$} 
&
\sphinxAtStartPar
A〜K
\\
\hline
\sphinxAtStartPar
{\normalsize $\diamondsuit$} 
&
\sphinxAtStartPar
A〜K
\\
\hline
\sphinxAtStartPar
{\normalsize $\clubsuit$} 
&
\sphinxAtStartPar
A〜K
\\
\hline\sphinxstartmulticolumn{2}%
\begin{varwidth}[t]{\sphinxcolwidth{2}{2}}
\sphinxAtStartPar
Joker x2
\par
\vskip-\baselineskip\vbox{\hbox{\strut}}\end{varwidth}%
\sphinxstopmulticolumn
\\
\hline
\end{tabulary}
\par
\sphinxattableend\end{savenotes}


\subsubsection{対戦前準備事項}
\label{\detokenize{match-regulations/standard_randomhalf:id4}}\begin{itemize}
\item {} 
\sphinxAtStartPar
デッキ条件に該当するようにデッキを2つ用意する。

\item {} 
\sphinxAtStartPar
試合毎にいずれか1つのデッキを使用する。

\item {} 
\sphinxAtStartPar
試合に使用しないデッキは重ねて左前方に横向きに置く。勝利した試合に使用したデッキであれば表向き、そうでなければ裏向きとする。

\item {} 
\sphinxAtStartPar
デッキは非公開である。つまり、カードのオーナーも見ることができない。ただし、試合に使用したデッキは試合の終了後に一時的に個人公開となり、次の試合の開始前に非公開となる。

\end{itemize}


\subsubsection{その他制限事項}
\label{\detokenize{match-regulations/standard_randomhalf:id5}}\begin{itemize}
\item {} 
\sphinxAtStartPar
2試合以上を行って勝者を決める。この対戦をマッチと言う。マッチの勝利条件は2試合に勝つことである。

\item {} 
\sphinxAtStartPar
試合毎に使用するデッキを選択する。ただし、勝利した試合に使用したデッキはそれ以降使用できない。

\end{itemize}

\sphinxstepscope


\subsection{プロ}
\label{\detokenize{match-regulations/pro:id1}}\label{\detokenize{match-regulations/pro::doc}}

\subsubsection{フォーマット}
\label{\detokenize{match-regulations/pro:id2}}
\sphinxAtStartPar
\hyperref[\detokenize{format/pro:format-pro}]{\ref{\detokenize{format/pro:format-pro}} \nameref{\detokenize{format/pro:format-pro}}}


\subsubsection{デッキ条件}
\label{\detokenize{match-regulations/pro:id3}}
\sphinxAtStartPar
次のカードの中から54までカードを選びデッキとする


\begin{savenotes}\sphinxattablestart
\centering
\begin{tabulary}{\linewidth}[t]{|T|T|}
\hline

\sphinxAtStartPar
{\normalsize $\spadesuit$} 
&
\sphinxAtStartPar
A〜K
\\
\hline
\sphinxAtStartPar
{\normalsize $\heartsuit$} 
&
\sphinxAtStartPar
A〜K
\\
\hline
\sphinxAtStartPar
{\normalsize $\diamondsuit$} 
&
\sphinxAtStartPar
A〜K
\\
\hline
\sphinxAtStartPar
{\normalsize $\clubsuit$} 
&
\sphinxAtStartPar
A〜K
\\
\hline\sphinxstartmulticolumn{2}%
\begin{varwidth}[t]{\sphinxcolwidth{2}{2}}
\sphinxAtStartPar
Joker x2
\par
\vskip-\baselineskip\vbox{\hbox{\strut}}\end{varwidth}%
\sphinxstopmulticolumn
\\
\hline
\end{tabulary}
\par
\sphinxattableend\end{savenotes}


\subsubsection{対戦前準備事項}
\label{\detokenize{match-regulations/pro:id4}}\begin{itemize}
\item {} 
\sphinxAtStartPar
なし

\end{itemize}


\subsubsection{その他制限事項}
\label{\detokenize{match-regulations/pro:id5}}\begin{itemize}
\item {} 
\sphinxAtStartPar
なし

\end{itemize}

\sphinxstepscope


\subsection{プロ40}
\label{\detokenize{match-regulations/pro40:id1}}\label{\detokenize{match-regulations/pro40::doc}}

\subsubsection{フォーマット}
\label{\detokenize{match-regulations/pro40:id2}}
\sphinxAtStartPar
\hyperref[\detokenize{format/pro:format-pro}]{\ref{\detokenize{format/pro:format-pro}} \nameref{\detokenize{format/pro:format-pro}}}


\subsubsection{デッキ条件}
\label{\detokenize{match-regulations/pro40:id3}}
\sphinxAtStartPar
次のカードの中から40枚までカードを選びデッキとする


\begin{savenotes}\sphinxattablestart
\centering
\begin{tabulary}{\linewidth}[t]{|T|T|}
\hline

\sphinxAtStartPar
{\normalsize $\spadesuit$} 
&
\sphinxAtStartPar
A〜K
\\
\hline
\sphinxAtStartPar
{\normalsize $\heartsuit$} 
&
\sphinxAtStartPar
A〜K
\\
\hline
\sphinxAtStartPar
{\normalsize $\diamondsuit$} 
&
\sphinxAtStartPar
A〜K
\\
\hline
\sphinxAtStartPar
{\normalsize $\clubsuit$} 
&
\sphinxAtStartPar
A〜K
\\
\hline\sphinxstartmulticolumn{2}%
\begin{varwidth}[t]{\sphinxcolwidth{2}{2}}
\sphinxAtStartPar
Joker x2
\par
\vskip-\baselineskip\vbox{\hbox{\strut}}\end{varwidth}%
\sphinxstopmulticolumn
\\
\hline
\end{tabulary}
\par
\sphinxattableend\end{savenotes}


\subsubsection{対戦前準備事項}
\label{\detokenize{match-regulations/pro40:id4}}\begin{itemize}
\item {} 
\sphinxAtStartPar
なし

\end{itemize}


\subsubsection{その他制限事項}
\label{\detokenize{match-regulations/pro40:id5}}\begin{itemize}
\item {} 
\sphinxAtStartPar
なし

\end{itemize}

\sphinxstepscope


\subsection{プロランダムハーフ}
\label{\detokenize{match-regulations/pro_randomhalf:id1}}\label{\detokenize{match-regulations/pro_randomhalf::doc}}

\subsubsection{フォーマット}
\label{\detokenize{match-regulations/pro_randomhalf:id2}}
\sphinxAtStartPar
\hyperref[\detokenize{format/pro:format-pro}]{\ref{\detokenize{format/pro:format-pro}} \nameref{\detokenize{format/pro:format-pro}}}


\subsubsection{デッキ条件}
\label{\detokenize{match-regulations/pro_randomhalf:id3}}
\sphinxAtStartPar
次のカードの中からランダムに27枚ずつ選び2つのデッキとする


\begin{savenotes}\sphinxattablestart
\centering
\begin{tabulary}{\linewidth}[t]{|T|T|}
\hline

\sphinxAtStartPar
{\normalsize $\spadesuit$} 
&
\sphinxAtStartPar
A〜K
\\
\hline
\sphinxAtStartPar
{\normalsize $\heartsuit$} 
&
\sphinxAtStartPar
A〜K
\\
\hline
\sphinxAtStartPar
{\normalsize $\diamondsuit$} 
&
\sphinxAtStartPar
A〜K
\\
\hline
\sphinxAtStartPar
{\normalsize $\clubsuit$} 
&
\sphinxAtStartPar
A〜K
\\
\hline\sphinxstartmulticolumn{2}%
\begin{varwidth}[t]{\sphinxcolwidth{2}{2}}
\sphinxAtStartPar
Joker x2
\par
\vskip-\baselineskip\vbox{\hbox{\strut}}\end{varwidth}%
\sphinxstopmulticolumn
\\
\hline
\end{tabulary}
\par
\sphinxattableend\end{savenotes}


\subsubsection{対戦前準備事項}
\label{\detokenize{match-regulations/pro_randomhalf:id4}}\begin{itemize}
\item {} 
\sphinxAtStartPar
デッキ条件に該当するようにデッキを2つ用意する。

\item {} 
\sphinxAtStartPar
試合毎にいずれか1つのデッキを使用する。

\item {} 
\sphinxAtStartPar
試合に使用しないデッキは重ねて左前方に横向きに置く。勝利した試合に使用したデッキであれば表向き、そうでなければ裏向きとする。

\item {} 
\sphinxAtStartPar
デッキは非公開である。つまり、カードのオーナーも見ることができない。ただし、試合に使用したデッキは試合の終了後に一時的に個人公開となり、次の試合の開始前に非公開となる。

\end{itemize}


\subsubsection{その他制限事項}
\label{\detokenize{match-regulations/pro_randomhalf:id5}}\begin{itemize}
\item {} 
\sphinxAtStartPar
2試合以上を行って勝者を決める。この対戦をマッチと言う。マッチの勝利条件は2試合に勝つことである。

\item {} 
\sphinxAtStartPar
試合毎に使用するデッキを選択する。ただし、勝利した試合に使用したデッキはそれ以降使用できない。

\end{itemize}

\sphinxstepscope


\subsection{マスター}
\label{\detokenize{match-regulations/master:id1}}\label{\detokenize{match-regulations/master::doc}}

\subsubsection{フォーマット}
\label{\detokenize{match-regulations/master:id2}}
\sphinxAtStartPar
\hyperref[\detokenize{format/master:format-master}]{\ref{\detokenize{format/master:format-master}} \nameref{\detokenize{format/master:format-master}}}


\subsubsection{デッキ条件}
\label{\detokenize{match-regulations/master:id3}}
\sphinxAtStartPar
次のカードの中から54までカードを選びデッキとする


\begin{savenotes}\sphinxattablestart
\centering
\begin{tabulary}{\linewidth}[t]{|T|T|}
\hline

\sphinxAtStartPar
{\normalsize $\spadesuit$} 
&
\sphinxAtStartPar
A〜K
\\
\hline
\sphinxAtStartPar
{\normalsize $\heartsuit$} 
&
\sphinxAtStartPar
A〜K
\\
\hline
\sphinxAtStartPar
{\normalsize $\diamondsuit$} 
&
\sphinxAtStartPar
A〜K
\\
\hline
\sphinxAtStartPar
{\normalsize $\clubsuit$} 
&
\sphinxAtStartPar
A〜K
\\
\hline\sphinxstartmulticolumn{2}%
\begin{varwidth}[t]{\sphinxcolwidth{2}{2}}
\sphinxAtStartPar
Joker x2
\par
\vskip-\baselineskip\vbox{\hbox{\strut}}\end{varwidth}%
\sphinxstopmulticolumn
\\
\hline
\end{tabulary}
\par
\sphinxattableend\end{savenotes}


\subsubsection{対戦前準備事項}
\label{\detokenize{match-regulations/master:id4}}\begin{itemize}
\item {} 
\sphinxAtStartPar
なし

\end{itemize}


\subsubsection{その他制限事項}
\label{\detokenize{match-regulations/master:id5}}\begin{itemize}
\item {} 
\sphinxAtStartPar
なし

\end{itemize}

\sphinxstepscope


\subsection{マスター40}
\label{\detokenize{match-regulations/master40:id1}}\label{\detokenize{match-regulations/master40::doc}}

\subsubsection{フォーマット}
\label{\detokenize{match-regulations/master40:id2}}
\sphinxAtStartPar
\hyperref[\detokenize{format/master:format-master}]{\ref{\detokenize{format/master:format-master}} \nameref{\detokenize{format/master:format-master}}}


\subsubsection{デッキ条件}
\label{\detokenize{match-regulations/master40:id3}}
\sphinxAtStartPar
次のカードの中から40枚までカードを選びデッキとする


\begin{savenotes}\sphinxattablestart
\centering
\begin{tabulary}{\linewidth}[t]{|T|T|}
\hline

\sphinxAtStartPar
{\normalsize $\spadesuit$} 
&
\sphinxAtStartPar
A〜K
\\
\hline
\sphinxAtStartPar
{\normalsize $\heartsuit$} 
&
\sphinxAtStartPar
A〜K
\\
\hline
\sphinxAtStartPar
{\normalsize $\diamondsuit$} 
&
\sphinxAtStartPar
A〜K
\\
\hline
\sphinxAtStartPar
{\normalsize $\clubsuit$} 
&
\sphinxAtStartPar
A〜K
\\
\hline\sphinxstartmulticolumn{2}%
\begin{varwidth}[t]{\sphinxcolwidth{2}{2}}
\sphinxAtStartPar
Joker x2
\par
\vskip-\baselineskip\vbox{\hbox{\strut}}\end{varwidth}%
\sphinxstopmulticolumn
\\
\hline
\end{tabulary}
\par
\sphinxattableend\end{savenotes}


\subsubsection{対戦前準備事項}
\label{\detokenize{match-regulations/master40:id4}}\begin{itemize}
\item {} 
\sphinxAtStartPar
なし

\end{itemize}


\subsubsection{その他制限事項}
\label{\detokenize{match-regulations/master40:id5}}\begin{itemize}
\item {} 
\sphinxAtStartPar
なし

\end{itemize}

\sphinxstepscope


\subsection{マスターランダムハーフ}
\label{\detokenize{match-regulations/master_randomhalf:id1}}\label{\detokenize{match-regulations/master_randomhalf::doc}}

\subsubsection{フォーマット}
\label{\detokenize{match-regulations/master_randomhalf:id2}}
\sphinxAtStartPar
\hyperref[\detokenize{format/master:format-master}]{\ref{\detokenize{format/master:format-master}} \nameref{\detokenize{format/master:format-master}}}


\subsubsection{デッキ条件}
\label{\detokenize{match-regulations/master_randomhalf:id3}}
\sphinxAtStartPar
次のカードの中からランダムに27枚ずつ選び2つのデッキとする


\begin{savenotes}\sphinxattablestart
\centering
\begin{tabulary}{\linewidth}[t]{|T|T|}
\hline

\sphinxAtStartPar
{\normalsize $\spadesuit$} 
&
\sphinxAtStartPar
A〜K
\\
\hline
\sphinxAtStartPar
{\normalsize $\heartsuit$} 
&
\sphinxAtStartPar
A〜K
\\
\hline
\sphinxAtStartPar
{\normalsize $\diamondsuit$} 
&
\sphinxAtStartPar
A〜K
\\
\hline
\sphinxAtStartPar
{\normalsize $\clubsuit$} 
&
\sphinxAtStartPar
A〜K
\\
\hline\sphinxstartmulticolumn{2}%
\begin{varwidth}[t]{\sphinxcolwidth{2}{2}}
\sphinxAtStartPar
Joker x2
\par
\vskip-\baselineskip\vbox{\hbox{\strut}}\end{varwidth}%
\sphinxstopmulticolumn
\\
\hline
\end{tabulary}
\par
\sphinxattableend\end{savenotes}


\subsubsection{対戦前準備事項}
\label{\detokenize{match-regulations/master_randomhalf:id4}}\begin{itemize}
\item {} 
\sphinxAtStartPar
デッキ条件に該当するようにデッキを2つ用意する。

\item {} 
\sphinxAtStartPar
試合毎にいずれか1つのデッキを使用する。

\item {} 
\sphinxAtStartPar
試合に使用しないデッキは重ねて左前方に横向きに置く。勝利した試合に使用したデッキであれば表向き、そうでなければ裏向きとする。

\item {} 
\sphinxAtStartPar
デッキは非公開である。つまり、カードのオーナーも見ることができない。ただし、試合に使用したデッキは試合の終了後に一時的に個人公開となり、次の試合の開始前に非公開となる。

\end{itemize}


\subsubsection{その他制限事項}
\label{\detokenize{match-regulations/master_randomhalf:id5}}\begin{itemize}
\item {} 
\sphinxAtStartPar
2試合以上を行って勝者を決める。この対戦をマッチと言う。マッチの勝利条件は2試合に勝つことである。

\item {} 
\sphinxAtStartPar
試合毎に使用するデッキを選択する。ただし、勝利した試合に使用したデッキはそれ以降使用できない。

\end{itemize}

\sphinxstepscope


\subsection{エクストラ}
\label{\detokenize{match-regulations/extra:id1}}\label{\detokenize{match-regulations/extra::doc}}

\subsubsection{フォーマット}
\label{\detokenize{match-regulations/extra:id2}}
\sphinxAtStartPar
\hyperref[\detokenize{format/extra:format-extra}]{\ref{\detokenize{format/extra:format-extra}} \nameref{\detokenize{format/extra:format-extra}}}


\subsubsection{デッキ条件}
\label{\detokenize{match-regulations/extra:id3}}
\sphinxAtStartPar
次のカードの中から54までカードを選びデッキとする


\begin{savenotes}\sphinxattablestart
\centering
\begin{tabulary}{\linewidth}[t]{|T|T|}
\hline

\sphinxAtStartPar
{\normalsize $\spadesuit$} 
&
\sphinxAtStartPar
A〜K
\\
\hline
\sphinxAtStartPar
{\normalsize $\heartsuit$} 
&
\sphinxAtStartPar
A〜K
\\
\hline
\sphinxAtStartPar
{\normalsize $\diamondsuit$} 
&
\sphinxAtStartPar
A〜K
\\
\hline
\sphinxAtStartPar
{\normalsize $\clubsuit$} 
&
\sphinxAtStartPar
A〜K
\\
\hline\sphinxstartmulticolumn{2}%
\begin{varwidth}[t]{\sphinxcolwidth{2}{2}}
\sphinxAtStartPar
Joker x2
\par
\vskip-\baselineskip\vbox{\hbox{\strut}}\end{varwidth}%
\sphinxstopmulticolumn
\\
\hline
\end{tabulary}
\par
\sphinxattableend\end{savenotes}


\subsubsection{対戦前準備事項}
\label{\detokenize{match-regulations/extra:id4}}\begin{itemize}
\item {} 
\sphinxAtStartPar
なし

\end{itemize}


\subsubsection{その他制限事項}
\label{\detokenize{match-regulations/extra:id5}}\begin{itemize}
\item {} 
\sphinxAtStartPar
なし

\end{itemize}

\sphinxstepscope


\chapter{付録}
\label{\detokenize{appendix/appendix:id1}}\label{\detokenize{appendix/appendix::doc}}

\section{PDF版ルール}
\label{\detokenize{appendix/appendix:pdf}}
\sphinxAtStartPar
\sphinxurl{https://blackpoker.github.io/BlackPoker/master/blackpoker.pdf}


\section{アクションリスト}
\label{\detokenize{appendix/appendix:id2}}

\subsection{ライト}
\label{\detokenize{appendix/appendix:actionlist-lite}}\label{\detokenize{appendix/appendix:id3}}\begin{quote}
\begin{description}
\sphinxlineitem{URL}
\sphinxAtStartPar
\sphinxurl{https://blackpoker.github.io/BlackPoker/master/actionlist/html/v6-lite.html}

\sphinxlineitem{PDF}
\sphinxAtStartPar
\sphinxurl{https://blackpoker.github.io/BlackPoker/master/actionlist/pdf/blackpoker-v6-lite.pdf}
\sphinxurl{https://blackpoker.github.io/BlackPoker/master/actionlist/pdf/blackpoker-v6-lite-2up.pdf}

\end{description}
\end{quote}


\subsection{スタンダード}
\label{\detokenize{appendix/appendix:actionlist-std}}\label{\detokenize{appendix/appendix:id4}}\begin{quote}
\begin{description}
\sphinxlineitem{URL}
\sphinxAtStartPar
\sphinxurl{https://blackpoker.github.io/BlackPoker/master/actionlist/html/v6-std.html}

\sphinxlineitem{PDF}
\sphinxAtStartPar
\sphinxurl{https://blackpoker.github.io/BlackPoker/master/actionlist/pdf/blackpoker-v6-std.pdf}
\sphinxurl{https://blackpoker.github.io/BlackPoker/master/actionlist/pdf/blackpoker-v6-std-2up.pdf}

\end{description}
\end{quote}


\subsection{プロ}
\label{\detokenize{appendix/appendix:actionlist-pro}}\label{\detokenize{appendix/appendix:id5}}\begin{quote}
\begin{description}
\sphinxlineitem{URL}
\sphinxAtStartPar
\sphinxurl{https://blackpoker.github.io/BlackPoker/master/actionlist/html/v6-pro.html}

\sphinxlineitem{PDF}
\sphinxAtStartPar
\sphinxurl{https://blackpoker.github.io/BlackPoker/master/actionlist/pdf/blackpoker-v6-pro.pdf}
\sphinxurl{https://blackpoker.github.io/BlackPoker/master/actionlist/pdf/blackpoker-v6-pro-2up.pdf}

\end{description}
\end{quote}


\subsection{マスター}
\label{\detokenize{appendix/appendix:actionlist-master}}\label{\detokenize{appendix/appendix:id6}}\begin{quote}
\begin{description}
\sphinxlineitem{URL}
\sphinxAtStartPar
\sphinxurl{https://blackpoker.github.io/BlackPoker/master/actionlist/html/v6-mast.html}

\sphinxlineitem{PDF}
\sphinxAtStartPar
\sphinxurl{https://blackpoker.github.io/BlackPoker/master/actionlist/pdf/blackpoker-v6-mast.pdf}
\sphinxurl{https://blackpoker.github.io/BlackPoker/master/actionlist/pdf/blackpoker-v6-mast-2up.pdf}

\end{description}
\end{quote}


\section{エクストラリスト}
\label{\detokenize{appendix/appendix:extralist}}\label{\detokenize{appendix/appendix:id7}}\begin{quote}
\begin{description}
\sphinxlineitem{URL}
\sphinxAtStartPar
\sphinxurl{https://blackpoker.github.io/BlackPoker/master/actionlist/html/v6-ex.html}

\sphinxlineitem{PDF}
\sphinxAtStartPar
\sphinxurl{https://blackpoker.github.io/BlackPoker/master/actionlist/pdf/blackpoker-v6-extra.pdf}
\sphinxurl{https://blackpoker.github.io/BlackPoker/master/actionlist/pdf/blackpoker-v6-extra-2up.pdf}

\end{description}
\end{quote}



\renewcommand{\indexname}{索引}
\printindex
\end{document}