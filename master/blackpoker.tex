%% Generated by Sphinx.
\def\sphinxdocclass{jsbook}
\documentclass[letterpaper,10pt,dvipdfmx]{sphinxmanual}
\ifdefined\pdfpxdimen
   \let\sphinxpxdimen\pdfpxdimen\else\newdimen\sphinxpxdimen
\fi \sphinxpxdimen=.75bp\relax
\ifdefined\pdfimageresolution
    \pdfimageresolution= \numexpr \dimexpr1in\relax/\sphinxpxdimen\relax
\fi
%% let collapsible pdf bookmarks panel have high depth per default
\PassOptionsToPackage{bookmarksdepth=5}{hyperref}


\PassOptionsToPackage{warn}{textcomp}


\usepackage{cmap}
\usepackage[T1]{fontenc}
\usepackage{amsmath,amssymb,amstext}




\usepackage{tgtermes}
\usepackage{tgheros}
\renewcommand{\ttdefault}{txtt}




\usepackage[,numfigreset=1,mathnumfig]{sphinx}

\fvset{fontsize=auto}
\usepackage[dvipdfm]{geometry}


% Include hyperref last.
\usepackage{hyperref}
% Fix anchor placement for figures with captions.
\usepackage{hypcap}% it must be loaded after hyperref.
% Set up styles of URL: it should be placed after hyperref.
\urlstyle{same}
\usepackage{pxjahyper}


\usepackage{sphinxmessages}
\setcounter{tocdepth}{1}


\makeatletter
\renewcommand\sphinxlineitem[2]{%
  % safe test of whether #2 is \sphinxlineitem
  \sphinx@gobto@sphinxlineitem#2\@gobbletwo\sphinxlineitem\unless
  \iftrue
    % Accumulate successive terms until actual definition or sub-list is found
    \spx@lineitemlabel\expandafter{\the\spx@lineitemlabel\strut#1\\}%
  \else
    % Issue the \item command with possibly multi-line contents
    \item[\kern\labelwidth\kern-\itemindent\kern-\leftmargin
          {\parbox[t]{1.4\linewidth}{% <- ここで幅を調整
          \raggedright
          \the\spx@lineitemlabel% Accumulated terms before this one, CR separated
          \strut#1}}% No \par token allowed here, but the \parbox will insert one tacitly at end
          \kern-\labelsep]%
    \spx@lineitemlabel{}%
    % This causes the label to be typeset (filling up the line), clearing up
    % things in case a nested list follows.
    \leavevmode
  \fi #2%
}
\makeatother


\title{BlackPoker}
\date{2025年03月27日}
\release{2025/3/x}
\author{BlackPoker}
\newcommand{\sphinxlogo}{\vbox{}}
\renewcommand{\releasename}{リリース}
\makeindex
\begin{document}

\pagestyle{empty}
\sphinxmaketitle
\pagestyle{plain}
\sphinxtableofcontents
\pagestyle{normal}
\phantomsection\label{\detokenize{index::doc}}


\sphinxAtStartPar
release: 2025/3/x

\sphinxstepscope


\chapter{はじめに}
\label{\detokenize{init/init:init-rst}}\label{\detokenize{init/init:id1}}\label{\detokenize{init/init::doc}}
\sphinxAtStartPar
この文章はトランプゲーム「BlackPoker」の全てのルールをまとめた文章です。

\sphinxAtStartPar
詳細なルールが記載されており、初心者の方は文章の量に圧倒されます。
ゲームをプレイする際に全てを熟読する必要はありませんが、
ルールについて深く知りたい、または新しいルールに触れたい方はぜひ熟読してください。


\section{BlackPokerとは}
\label{\detokenize{init/init:blackpoker}}
\sphinxAtStartPar
当サークルが考案した1人1セットのトランプを使ったターン制の
トレーディングカードゲームのようなトランプゲームです。
自分だけのオリジナルトランプを使って友達と遊べます。
プレイしている時のイメージは図のような感じです。(\hyperref[\detokenize{init/init:play-image}]{Fig.\@ \ref{\detokenize{init/init:play-image}}})

\begin{figure}[htbp]
\centering
\capstart

\noindent\sphinxincludegraphics{{play-image}.pdf}
\caption{プレイ風景}\label{\detokenize{init/init:id7}}\label{\detokenize{init/init:play-image}}\end{figure}


\section{きっかけ}
\label{\detokenize{init/init:id2}}
\begin{sphinxVerbatim}[commandchars=\\\{\}]
「また昔みたいに友達とカードゲームがしたい」
\end{sphinxVerbatim}

\sphinxAtStartPar
けど、仕事に追われ時間もなく、お金もかけられないのでカード資産も抱えられない。
昔のデッキを引っ張りだしてもカードパワーが違って平等に遊べないし、コミニティが狭い。
そんな悩みを解消しようと誰でも持っているトランプでカードゲーム風のルールを作ろうと思いました。


\section{ゲームのストーリー}
\label{\detokenize{init/init:id3}}
\sphinxAtStartPar
あなたのトランプはただのトランプではありません。
かつて大地を支配していた偉大な国々や強固な絆で結ばれた組織の魂が宿る遺産です。

\sphinxAtStartPar
これらのカードには、世界の歴史を彩る多彩な物語が織り込まれており、その物語たちはまるで時間を超えたタペストリーのように絡み合っています。

\sphinxAtStartPar
プレイヤーは歴史の再現者、リアニメーターとしてデッキに宿る過去の栄光を再現し、その国や組織がいかに素晴らしかったかを
戦いを通して対戦相手に示してください。

\sphinxAtStartPar
え?そんな国知らない?でしたらあなたのカードから創造してみてはいかがでしょうか。


\section{読み方}
\label{\detokenize{init/init:id4}}
\sphinxAtStartPar
各章は次のことを説明しています。
\begin{description}
\sphinxlineitem{\hyperref[\detokenize{init/init:init-rst}]{\ref{\detokenize{init/init:init-rst}} \nameref{\detokenize{init/init:init-rst}}}}
\sphinxAtStartPar
ルールの指針や全体像を説明

\sphinxlineitem{\hyperref[\detokenize{common/common:common-rst}]{\ref{\detokenize{common/common:common-rst}} \nameref{\detokenize{common/common:common-rst}}}}
\sphinxAtStartPar
ゲームの開始方法など全体的なゲームの流れを説明

\sphinxlineitem{\hyperref[\detokenize{match-regulations/match-regulations:match-regulations-rst}]{\ref{\detokenize{match-regulations/match-regulations:match-regulations-rst}} \nameref{\detokenize{match-regulations/match-regulations:match-regulations-rst}}}}
\sphinxAtStartPar
ゲーム開始時に決定する規則について説明

\sphinxlineitem{\hyperref[\detokenize{format/format:format-rst}]{\ref{\detokenize{format/format:format-rst}} \nameref{\detokenize{format/format:format-rst}}}}
\sphinxAtStartPar
対戦レギュレーションを決める際に選択できるフォーマットを紹介

\sphinxlineitem{\hyperref[\detokenize{frame/frame:frame-rst}]{\ref{\detokenize{frame/frame:frame-rst}} \nameref{\detokenize{frame/frame:frame-rst}}}}
\sphinxAtStartPar
対戦レギュレーションを決める際に選択できるフレームを紹介

\sphinxlineitem{\hyperref[\detokenize{core/core:core-rst}]{\ref{\detokenize{core/core:core-rst}} \nameref{\detokenize{core/core:core-rst}}}}
\sphinxAtStartPar
BlackPokerのコアであるターン制ゲームのルールを説明

\end{description}


\subsection{用途別読み方}
\label{\detokenize{init/init:id5}}
\sphinxAtStartPar
文章が複雑であるため、用途に合わせて読むことをおすすめします。
\begin{description}
\sphinxlineitem{ゲームの流れを理解したい方}
\sphinxAtStartPar
\hyperref[\detokenize{common/common:common-rst}]{\ref{\detokenize{common/common:common-rst}} \nameref{\detokenize{common/common:common-rst}}} から順に読んでください。 \hyperref[\detokenize{core/core:core-rst}]{\ref{\detokenize{core/core:core-rst}} \nameref{\detokenize{core/core:core-rst}}} は飛ばしても大丈夫です。

\sphinxlineitem{アクションの解決順で悩んでいる方}
\sphinxAtStartPar
\hyperref[\detokenize{core/core:core-rst}]{\ref{\detokenize{core/core:core-rst}} \nameref{\detokenize{core/core:core-rst}}} を熟読してください。

\end{description}


\section{ルール指針}
\label{\detokenize{init/init:id6}}
\sphinxAtStartPar
ルールを作成・修正するための指針を示します。
\begin{description}
\sphinxlineitem{\sphinxstylestrong{誰とでも戦える \textasciitilde{}目指すは老若男女\textasciitilde{}}}
\sphinxAtStartPar
ルールを知りトランプを持っていれば誰とでも遊べるゲームを目指します。

\sphinxlineitem{\sphinxstylestrong{個性が出せる \textasciitilde{}オリジナルトランプ・デッキ構築\textasciitilde{}}}
\sphinxAtStartPar
さまざまなトランプが使え見た目で個性を出せるのはもちろんのこと、
デッキ構築の面でも自分のしたい戦い方が表現できることを目指します。

\sphinxlineitem{\sphinxstylestrong{短く終わる \textasciitilde{}1戦15分\textasciitilde{}}}
\sphinxAtStartPar
時間をかけずさっと遊べることを目指します。

\sphinxlineitem{\sphinxstylestrong{ずっと使えるデッキ}}
\sphinxAtStartPar
愛着のあるカードがずっと使えるようなルールとします。

\sphinxlineitem{\sphinxstylestrong{必要な物は最小限 \textasciitilde{}トランプのみ\textasciitilde{}}}
\sphinxAtStartPar
用意するものはトランプのみ。それ以外の道具は必要ないルールとします。

\sphinxlineitem{\sphinxstylestrong{プレイング重視 \textasciitilde{}5:3:2=技:運:構築\textasciitilde{}}}
\sphinxAtStartPar
運やデッキ構築より技量を重視したルールを目指します。

\sphinxlineitem{\sphinxstylestrong{ベースルールはトレーディングカードゲーム}}
\sphinxAtStartPar
カードゲームプレイヤーが覚えやすいルールを目指します。

\sphinxlineitem{\sphinxstylestrong{カスタマイズ可能 \textasciitilde{}基本と拡張の分離\textasciitilde{}}}
\sphinxAtStartPar
基本ルールと拡張ルールを分離し、大富豪のようにローカルルールが作成できることを目指します。

\sphinxlineitem{\sphinxstylestrong{ルールの更新 \textasciitilde{}飽き防止&不備改善\textasciitilde{}}}
\sphinxAtStartPar
新たなルールを度々公開し、飽きを防止します。またルールに不備がある場合、随時改善します。

\sphinxlineitem{\sphinxstylestrong{相手のカードに触らない}}
\sphinxAtStartPar
盗難防止とネット対戦対応に努めます。

\end{description}

\sphinxstepscope


\chapter{共通ルール}
\label{\detokenize{common/common:common-rst}}\label{\detokenize{common/common:id1}}\label{\detokenize{common/common::doc}}
\sphinxAtStartPar
BlackPokerには様々な対戦方法があります。
個別の違いは対戦レギュレーションで定義されています。

\sphinxAtStartPar
この章では、BlackPokerの共通的なルールを説明します。


\section{プレイ人数}
\label{\detokenize{common/common:id2}}
\sphinxAtStartPar
2人です。


\section{用意するもの}
\label{\detokenize{common/common:id3}}\begin{itemize}
\item {} 
\sphinxAtStartPar
1人1セットのトランプが必要です。

\item {} 
\sphinxAtStartPar
覚えていない場合、フォーマットに応じてアクションリスト、エクストラリストがあると便利です。

\end{itemize}


\section{使用できるトランプ}
\label{\detokenize{common/common:id4}}
\sphinxAtStartPar
BlackPokerでは次の条件を満たしたトランプを使うことができます。
一般的なトランプなら満たす条件となっています。
\begin{quote}
\begin{itemize}
\item {} 
\sphinxAtStartPar
スートと数字が分かる

\item {} 
\sphinxAtStartPar
スートの{\normalsize $\spadesuit$} {\normalsize $\heartsuit$} {\normalsize $\diamondsuit$} {\normalsize $\clubsuit$} が判断できる

\item {} 
\sphinxAtStartPar
数字のA\sphinxhyphen{}K(1\sphinxhyphen{}13)が判断できる

\item {} 
\sphinxAtStartPar
スートと数字の組合せが重複していない

\item {} 
\sphinxAtStartPar
裏から表がわからない

\item {} 
\sphinxAtStartPar
縦向き横向きが判断できる

\item {} 
\sphinxAtStartPar
Jokerは2枚まで入れられる

\item {} 
\sphinxAtStartPar
54枚無くてもよい

\end{itemize}

\sphinxAtStartPar
対戦レギュレーションにより使用できるトランプの枚数など異なる場合があるため、
対戦する際に、対戦レギュレーションを確認してください。
\end{quote}


\section{トランプの数字}
\label{\detokenize{common/common:id5}}
\sphinxAtStartPar
ゲーム全体を通してトランプの数字は次のような数値として扱います。(\hyperref[\detokenize{common/common:cardrank}]{Table \ref{\detokenize{common/common:cardrank}}})


\begin{savenotes}\sphinxattablestart
\sphinxthistablewithglobalstyle
\centering
\sphinxcapstartof{table}
\sphinxthecaptionisattop
\sphinxcaption{トランプの数字}\label{\detokenize{common/common:id46}}\label{\detokenize{common/common:cardrank}}
\sphinxaftertopcaption
\begin{tabulary}{\linewidth}[t]{|T|T|}
\sphinxtoprule
\sphinxstyletheadfamily 
\sphinxAtStartPar
カード
&\sphinxstyletheadfamily 
\sphinxAtStartPar
数字
\\
\sphinxmidrule
\sphinxtableatstartofbodyhook
\sphinxAtStartPar
A
&
\sphinxAtStartPar
1
\\
\sphinxhline
\sphinxAtStartPar
2〜10
&
\sphinxAtStartPar
表記どおり
\\
\sphinxhline
\sphinxAtStartPar
J
&
\sphinxAtStartPar
11
\\
\sphinxhline
\sphinxAtStartPar
Q
&
\sphinxAtStartPar
12
\\
\sphinxhline
\sphinxAtStartPar
K
&
\sphinxAtStartPar
13
\\
\sphinxhline
\sphinxAtStartPar
Joker
&
\sphinxAtStartPar
0
\\
\sphinxbottomrule
\end{tabulary}
\sphinxtableafterendhook\par
\sphinxattableend\end{savenotes}


\section{カードの配置}
\label{\detokenize{common/common:id6}}
\sphinxAtStartPar
カードの配置には次のような場所があります。(\hyperref[\detokenize{common/common:field}]{Fig.\@ \ref{\detokenize{common/common:field}}})

\begin{figure}[htbp]
\centering
\capstart

\noindent\sphinxincludegraphics{{field}.pdf}
\caption{プレイ中のカードの配置}\label{\detokenize{common/common:id47}}\label{\detokenize{common/common:field}}\end{figure}

\index{ライフ@\spxentry{ライフ}}\ignorespaces \begin{description}
\sphinxlineitem{ライフ}
\sphinxAtStartPar
山札。ゲームを始める時に自分のトランプを裏向きに置く場所です。
ダメージを受けるとライフの一番上から墓地にカードを移します。

\end{description}

\index{ぼ\textbar{}墓地@\spxentry{ぼ\textbar{}墓地}}\ignorespaces \begin{description}
\sphinxlineitem{墓地}
\sphinxAtStartPar
捨て札置き場。ダメージを受けた時などに表向きでカードを重ねて置きます。

\end{description}

\index{ば\textbar{}場@\spxentry{ば\textbar{}場}}\ignorespaces \begin{description}
\sphinxlineitem{場}
\sphinxAtStartPar
兵士や防壁などのキャラクターを置きます。

\end{description}

\index{て\textbar{}手札@\spxentry{て\textbar{}手札}}\ignorespaces \begin{description}
\sphinxlineitem{手札}
\sphinxAtStartPar
ライフから引いたカードを持っておく場所です。相手から見えないようにしましょう。

\end{description}

\index{き\textbar{}切札(場所)@\spxentry{き\textbar{}切札(場所)}}\ignorespaces \begin{description}
\sphinxlineitem{フォグ}
\sphinxAtStartPar
このターンのみ影響を与えるカードを置きます。

\end{description}


\subsection{デッキとライフ}
\label{\detokenize{common/common:id7}}
\sphinxAtStartPar
対戦レギュレーションなどでデッキという表現が出てきます。

\index{デッキ@\spxentry{デッキ}}\ignorespaces \begin{description}
\sphinxlineitem{デッキ}
\sphinxAtStartPar
ゲーム開始前にゲームで使用するカードの束(カード構成)

\end{description}

\sphinxAtStartPar
ゲームの始め方を経てデッキはライフとなります。詳細は \hyperref[\detokenize{common/common:common-gamestart}]{\ref{\detokenize{common/common:common-gamestart}} \nameref{\detokenize{common/common:common-gamestart}}} で説明します。


\section{勝利条件}
\label{\detokenize{common/common:id8}}
\sphinxAtStartPar
プレイヤーは順に対戦相手に対し攻撃を行い、ダメージを与え先に相手のライフを0枚にした方が勝ちです。ダメージは1点につき、ライフが1枚減ります。

\index{ダメージ@\spxentry{ダメージ}}\ignorespaces 

\section{ダメージ}
\label{\detokenize{common/common:index-6}}\label{\detokenize{common/common:id9}}
\sphinxAtStartPar
プレイヤーがダメージを受けた場合、ライフの一番上から受けた点数分墓地にカードを表向きで移動します。移動する際は、カードの表を対戦相手に見せる必要はありません。

\index{キャラクター@\spxentry{キャラクター}}\ignorespaces 

\section{キャラクター}
\label{\detokenize{common/common:index-7}}\label{\detokenize{common/common:id10}}
\sphinxAtStartPar
キャラクターとは、場に存在する兵士や防壁のことを指します。
コアルールのコンポーネントにあたります。

\sphinxAtStartPar
キャラクターは1枚のカードで1体を表すこともあれば、
複数枚で1体を表すこともあります。(\hyperref[\detokenize{common/common:character}]{Fig.\@ \ref{\detokenize{common/common:character}}})

\begin{figure}[htbp]
\centering
\capstart

\noindent\sphinxincludegraphics{{character}.pdf}
\caption{キャラクターの例}\label{\detokenize{common/common:id48}}\label{\detokenize{common/common:character}}\end{figure}


\subsection{キャラクターのもつ項目}
\label{\detokenize{common/common:id11}}
\sphinxAtStartPar
キャラクターのもつ項目について説明します。

\index{キャラクター名@\spxentry{キャラクター名}}\ignorespaces \begin{description}
\sphinxlineitem{キャラクター名}
\sphinxAtStartPar
キャラクターの名称を示します。

\end{description}

\index{タイプ(キャラクター)@\spxentry{タイプ(キャラクター)}}\ignorespaces \begin{description}
\sphinxlineitem{タイプ}
\sphinxAtStartPar
キャラクターのタイプを示します。タイプは兵士と防壁の2種類が存在します。

\end{description}

\index{キーカード@\spxentry{キーカード}}\ignorespaces \begin{description}
\sphinxlineitem{キーカード}
\sphinxAtStartPar
キャラクターを示すカードが記載されています。複数のカードで1体のキャラクターを示す場合もあります。

\end{description}

\index{ラベル@\spxentry{ラベル}}\ignorespaces \begin{description}
\sphinxlineitem{ラベル}
\sphinxAtStartPar
キャラクターもつ属性を示します。「速攻」や「アタッカー」など様々なラベルがあります。

\end{description}

\index{サイズ@\spxentry{サイズ}}\ignorespaces \begin{description}
\sphinxlineitem{サイズ}
\sphinxAtStartPar
兵士の持つ大きさを示します。

\end{description}

\index{の\textbar{}能力(キャラクター)@\spxentry{の\textbar{}能力(キャラクター)}}\ignorespaces \begin{description}
\sphinxlineitem{能力}
\sphinxAtStartPar
キャラクターが持っている能力を記載しています。

\end{description}


\subsection{キャラクターのサイズ}
\label{\detokenize{common/common:id12}}
\sphinxAtStartPar
トランプの数字は、キャラクターの強さを示します。
基本はカードに記載された数字に準じますが、魔法などのアクションを使うことで
加算や減算されることがあります。


\subsection{キャラクターの注意点}
\label{\detokenize{common/common:id13}}

\subsubsection{複数枚で1体となるキャラクターが防壁になったら?}
\label{\detokenize{common/common:id14}}
\sphinxAtStartPar
アクションの効果で兵士を防壁にすることがあります。
防壁は1枚で1体のキャラクターであるため、
複数枚からなるキャラクターが防壁となった場合、
複数体の防壁となります。

\sphinxAtStartPar
なお、複数枚からなるキャラクターが
墓地や手札に移った場合、
1体のキャラクターとして
扱うため複数枚合わせて移します。
チャージ状態、ドライブ状態となった場合も同様に1体のキャラクター
として扱います。

\index{チャージ@\spxentry{チャージ}}\index{ドライブ@\spxentry{ドライブ}}\ignorespaces 

\subsection{チャージとドライブ}
\label{\detokenize{common/common:index-14}}\label{\detokenize{common/common:id15}}
\sphinxAtStartPar
キャラクターには、チャージ状態とドライブ状態が存在します。
チャージ状態は未使用状態を示し、ドライブ状態は使用済み状態を示しています。
また、キャラクターを横向きにすることを「ドライブ」、縦向きにすることを「チャージ」と言います。(\hyperref[\detokenize{common/common:chargedrive}]{Fig.\@ \ref{\detokenize{common/common:chargedrive}}})

\begin{figure}[htbp]
\centering
\capstart

\noindent\sphinxincludegraphics{{charge&drive}.pdf}
\caption{チャージとドライブ}\label{\detokenize{common/common:id49}}\label{\detokenize{common/common:chargedrive}}\end{figure}


\section{ゲームの始め方}
\label{\detokenize{common/common:common-gamestart}}\label{\detokenize{common/common:id16}}
\sphinxAtStartPar
デッキをよく切り、次の手順でゲームを始めます。


\subsection{配置準備}
\label{\detokenize{common/common:common-gamestart-field}}\label{\detokenize{common/common:id17}}\begin{enumerate}
\sphinxsetlistlabels{\arabic}{enumi}{enumii}{}{.}%
\item {} 
\sphinxAtStartPar
デッキをよく切る。

\item {} 
\sphinxAtStartPar
デッキより7枚引き手札にする。

\item {} 
\sphinxAtStartPar
デッキをライフの場所に置き、ライフとする。

\end{enumerate}


\subsection{先攻決定}
\label{\detokenize{common/common:common-gamestart-first}}\label{\detokenize{common/common:id18}}\begin{enumerate}
\sphinxsetlistlabels{\arabic}{enumi}{enumii}{}{.}%
\item {} 
\sphinxAtStartPar
両者ライフの一番上を表にする。

\item {} 
\sphinxAtStartPar
大きい数字のプレイヤーが先攻。数字については、 \hyperref[\detokenize{common/common:cardrank}]{Table \ref{\detokenize{common/common:cardrank}}} 参照。

\item {} 
\sphinxAtStartPar
数字が同じ場合、さらにライフの一番上を表にし同様のルールで比べる。

\item {} 
\sphinxAtStartPar
表にしたカードを墓地へ移す。

\end{enumerate}


\subsection{ゲーム開始}
\label{\detokenize{common/common:common-gamestart-start}}\label{\detokenize{common/common:id19}}\begin{enumerate}
\sphinxsetlistlabels{\arabic}{enumi}{enumii}{}{.}%
\item {} 
\sphinxAtStartPar
先攻プレイヤーはライフより1枚引き手札に加える。

\item {} 
\sphinxAtStartPar
先攻プレイヤーがターンとチャンスをもちゲームを開始する。

\end{enumerate}

\sphinxAtStartPar
ゲームの始め方は対戦レギュレーションによって異なることがあります。
対戦前に確認してください。

\index{アクション@\spxentry{アクション}}\ignorespaces 

\section{アクション}
\label{\detokenize{common/common:index-15}}\label{\detokenize{common/common:id20}}
\sphinxAtStartPar
BlackPokerは割込み可能なターン制ゲームです。

\sphinxAtStartPar
例えば次の状況をイメージしてください。
\begin{itemize}
\item {} 
\sphinxAtStartPar
\sphinxstylestrong{Aくん}:「この兵士アップします。」

\item {} 
\sphinxAtStartPar
\sphinxstylestrong{Bさん}:「その前にこの兵士ダウンします。」

\item {} 
\sphinxAtStartPar
\sphinxstylestrong{Aくん}:「じゃあそのダウンをカウンターします。」

\item {} 
\sphinxAtStartPar
\sphinxstylestrong{Bさん}:「それをさらにカウンターします。」

\item {} 
\sphinxAtStartPar
\sphinxstylestrong{Aくん}:「・・・(泣)」

\item {} 
\sphinxAtStartPar
\sphinxstylestrong{Bさん}:「(どやっ!)」

\end{itemize}

\sphinxAtStartPar
このやり取りの中で「アップします」や「ダウンします」などの1行1行がアクションになります。

\sphinxAtStartPar
割込み可能なターン制ゲームは、見方を変えると“許可制のゲーム”とも表現できます。

\sphinxAtStartPar
このアクションを実行したいとルールシステムに要求(リクエスト)し、相手に許可を得てリクエストが実行されます。もちろん相手はリクエストに対して割り込んでリクエストすることもできます。

\sphinxAtStartPar
アクションには、プレイヤーのすべての行動を定義しており、従来のTCGでいう「魔法」や「ターン制御」が含まれています。

\sphinxAtStartPar
参考: \hyperref[\detokenize{core/core:core-rst}]{\ref{\detokenize{core/core:core-rst}} \nameref{\detokenize{core/core:core-rst}}}


\subsection{アクションが持つ項目}
\label{\detokenize{common/common:id21}}
\sphinxAtStartPar
アクションが持つ項目について説明します。

\index{アクション名@\spxentry{アクション名}}\ignorespaces \begin{description}
\sphinxlineitem{アクション名}
\sphinxAtStartPar
アクションの名称を示します。

\end{description}

\index{タイプ(アクション)@\spxentry{タイプ(アクション)}}\ignorespaces \begin{description}
\sphinxlineitem{タイプ}
\sphinxAtStartPar
アクションの種類を表します。アクション名の後に括弧書きで記載します。

\end{description}

\index{トリガー@\spxentry{トリガー}}\ignorespaces \begin{description}
\sphinxlineitem{トリガー}
\sphinxAtStartPar
アクションには自分で起こせるアクションと誘発するアクションがあります。
トリガー項目では「直接」か「誘発」が設定されています。

\sphinxAtStartPar
参考: \hyperref[\detokenize{core/core:trigger-core}]{\ref{\detokenize{core/core:trigger-core}} \nameref{\detokenize{core/core:trigger-core}}}

\end{description}

\index{スピード@\spxentry{スピード}}\ignorespaces \begin{description}
\sphinxlineitem{スピード}
\sphinxAtStartPar
アクションはすぐに効果が解決されるものとそうでないものがあります。
スピード項目では「即時」か「通常」が設定されています。

\sphinxAtStartPar
参考: \hyperref[\detokenize{core/core:speed-core}]{\ref{\detokenize{core/core:speed-core}} \nameref{\detokenize{core/core:speed-core}}}

\end{description}

\index{タイミング@\spxentry{タイミング}}\ignorespaces \begin{description}
\sphinxlineitem{タイミング}
\sphinxAtStartPar
アクションは起こせるタイミングが2種類あります。「メイン」は自分のターンかつステージが空のときに起こせます。
「クイック」はいつでも起こすことができます。

\sphinxAtStartPar
参考: \hyperref[\detokenize{core/core:timing}]{\ref{\detokenize{core/core:timing}} \nameref{\detokenize{core/core:timing}}}

\end{description}

\index{キーカード(アクション)@\spxentry{キーカード(アクション)}}\ignorespaces \begin{description}
\sphinxlineitem{キーカード}
\sphinxAtStartPar
アクションの核となるカードを示します。
キーカードは★を使って表記します。
凡例の場合、手札からコストとは別に{\normalsize $\heartsuit$} A〜10に該当するカードを1枚
キーカードとして使用します。

\end{description}

\index{コスト@\spxentry{コスト}}\ignorespaces \begin{description}
\sphinxlineitem{コスト}
\sphinxAtStartPar
アクションを起こすのに必要な対価です。
コストは$を使って表記し、コストの支払いはアクションを起こすプレイヤーが行います。コストの種類は \hyperref[\detokenize{common/common:cost}]{\ref{\detokenize{common/common:cost}} \nameref{\detokenize{common/common:cost}}} で説明します。

\end{description}

\index{た\textbar{}対象@\spxentry{た\textbar{}対象}}\ignorespaces \begin{description}
\sphinxlineitem{対象}
\sphinxAtStartPar
効果の対象を示します。

\end{description}

\index{と\textbar{}特記事項@\spxentry{と\textbar{}特記事項}}\ignorespaces \begin{description}
\sphinxlineitem{特記事項}
\sphinxAtStartPar
特記事項は※を使って表記し、その他の項目では書き表せない条件を示します。

\end{description}

\index{つ\textbar{}通常効果@\spxentry{つ\textbar{}通常効果}!そ\textbar{}即時効果@\spxentry{そ\textbar{}即時効果}}\index{そ\textbar{}即時効果@\spxentry{そ\textbar{}即時効果}!つ\textbar{}通常効果@\spxentry{つ\textbar{}通常効果}}\ignorespaces \begin{description}
\sphinxlineitem{効果}
\sphinxAtStartPar
効果の内容を示します。

\end{description}

\begin{sphinxadmonition}{note}{注釈:}
\sphinxAtStartPar
トリガー,スピード,タイミングの表記

\sphinxAtStartPar
トリガー,スピード,タイミングは@を使って次のように表記されます。

\sphinxAtStartPar
@{[}トリガー{]}\sphinxhyphen{}{[}スピード{]}\sphinxhyphen{}{[}タイミング{]}

\sphinxAtStartPar
例えば次のようになります。

\sphinxAtStartPar
@誘発\sphinxhyphen{}即時\sphinxhyphen{}クイック
\end{sphinxadmonition}


\subsubsection{記載されていないアクションの項目}
\label{\detokenize{common/common:id22}}
\sphinxAtStartPar
アクションによっては記載されていない項目もあります。
記載されていない項目は無視して構いません。
たとえばコスト項目がなければコストを支払う必要はありません。


\subsection{コストの種類}
\label{\detokenize{common/common:cost}}\label{\detokenize{common/common:id23}}
\sphinxAtStartPar
アクションによって支払うコストが異なります。
コストには次の種類があり、それぞれ支払い方が異なります。(\hyperref[\detokenize{common/common:table-cost}]{Table \ref{\detokenize{common/common:table-cost}}})


\begin{savenotes}\sphinxattablestart
\sphinxthistablewithglobalstyle
\centering
\sphinxcapstartof{table}
\sphinxthecaptionisattop
\sphinxcaption{コストの種類}\label{\detokenize{common/common:id50}}\label{\detokenize{common/common:table-cost}}
\sphinxaftertopcaption
\begin{tabulary}{\linewidth}[t]{|T|T|}
\sphinxtoprule
\sphinxstyletheadfamily 
\sphinxAtStartPar
表記(名称)
&\sphinxstyletheadfamily 
\sphinxAtStartPar
対価
\\
\sphinxmidrule
\sphinxtableatstartofbodyhook
\sphinxAtStartPar
B (Bulwark)
&
\sphinxAtStartPar
防壁をドライブする
\\
\sphinxhline
\sphinxAtStartPar
L (Life)
&
\sphinxAtStartPar
1点ダメージを受ける
\\
\sphinxhline
\sphinxAtStartPar
D (Discard)
&
\sphinxAtStartPar
手札を1枚捨てる
\\
\sphinxhline
\sphinxAtStartPar
S (Sacrifice)
&
\sphinxAtStartPar
キャラクター1体を墓地に移す
\\
\sphinxbottomrule
\end{tabulary}
\sphinxtableafterendhook\par
\sphinxattableend\end{savenotes}

\sphinxAtStartPar
たとえばコストが \sphinxstylestrong{「\$BL」} の場合、自分の場にいるチャージ状態の防壁を1体ドライブし、1点ダメージを受けることでコストが支払われたことになります。


\subsection{アクションの起こし方(リクエスト)}
\label{\detokenize{common/common:id24}}
\sphinxAtStartPar
BlackPokerは実行したいアクションを要求(リクエスト)し、進める形式のゲームです。

\sphinxAtStartPar
アクションを要求することを「アクションを起こす」または「アクションをリクエストする」といいます。

\sphinxAtStartPar
次の手順でアクションをリクエストすることができます。
\begin{enumerate}
\sphinxsetlistlabels{\arabic}{enumi}{enumii}{}{.}%
\item {} 
\sphinxAtStartPar
起こすアクションを対戦相手に伝える。

\item {} 
\sphinxAtStartPar
アクションに応じたコストを支払う。

\item {} 
\sphinxAtStartPar
必要なら手札からキーカードを出す。

\item {} 
\sphinxAtStartPar
対象の指定が必要な場合、対象を指定する。

\end{enumerate}


\subsubsection{アクションを起こすときの注意点}
\label{\detokenize{common/common:id25}}

\paragraph{対象を指定しないでアクションを起こせるか?}
\label{\detokenize{common/common:id26}}
\sphinxAtStartPar
「対象」項目がある場合、記載された条件を満たした対象を指定できなければ、
そのアクションを起こすことはできません。


\paragraph{アクションを対象とするアクションは自身を対象にできるか?}
\label{\detokenize{common/common:id27}}
\sphinxAtStartPar
アクションは、自分自身を対象とすることはできません。
そのため、「カウンター」アクションのようにアクションを対象とするアクションは
自身を対象とすることはできません。


\subsection{アクションの解決}
\label{\detokenize{common/common:id28}}
\sphinxAtStartPar
リクエストされたアクションを実行済みにすることを「アクションを解決する」といいます。

\sphinxAtStartPar
実際にアクションが解決される流れを見ていきましょう。

\sphinxAtStartPar
\sphinxstylestrong{用語}

\sphinxAtStartPar
登場する用語を説明します。
\begin{description}
\sphinxlineitem{\sphinxstylestrong{チャンス}}
\sphinxAtStartPar
アクションをリクエストする(ステージに積む)権利(他TCGの優先権)

\sphinxlineitem{\sphinxstylestrong{ターン}}
\sphinxAtStartPar
手番を示す印

\sphinxlineitem{\sphinxstylestrong{ステージ}}
\sphinxAtStartPar
アクションが蓄積される場所(他TCGのスタック)

\end{description}
\begin{enumerate}
\sphinxsetlistlabels{\arabic}{enumi}{enumii}{}{.}%
\item {} 
\sphinxAtStartPar
プレイヤーAがアクションを積む

\end{enumerate}

\begin{figure}[htbp]
\centering
\capstart

\noindent\sphinxincludegraphics{{action-request1}.pdf}
\caption{アクションの解決1}\label{\detokenize{common/common:id51}}\label{\detokenize{common/common:action-request1-image}}\end{figure}
\begin{enumerate}
\sphinxsetlistlabels{\arabic}{enumi}{enumii}{}{.}%
\setcounter{enumi}{1}
\item {} 
\sphinxAtStartPar
プレイヤーBがアクションを積む

\end{enumerate}

\begin{figure}[htbp]
\centering
\capstart

\noindent\sphinxincludegraphics{{action-request2}.pdf}
\caption{アクションの解決2}\label{\detokenize{common/common:id52}}\label{\detokenize{common/common:action-request2-image}}\end{figure}
\begin{enumerate}
\sphinxsetlistlabels{\arabic}{enumi}{enumii}{}{.}%
\setcounter{enumi}{2}
\item {} 
\sphinxAtStartPar
アクションを実行(解決)

\end{enumerate}

\begin{figure}[htbp]
\centering
\capstart

\noindent\sphinxincludegraphics{{action-request3}.pdf}
\caption{アクションの解決3}\label{\detokenize{common/common:id53}}\label{\detokenize{common/common:action-request3-image}}\end{figure}

\sphinxAtStartPar
大まかな流れは図の通りとなります。

\sphinxAtStartPar
更に厳密な処理は \hyperref[\detokenize{core/core:coreflowsec}]{\ref{\detokenize{core/core:coreflowsec}} \nameref{\detokenize{core/core:coreflowsec}}} を参照してください。

\sphinxAtStartPar
次に細かな部分を説明します。

\sphinxAtStartPar
補足ですが次に説明する部分は、 \hyperref[\detokenize{core/core:coreflowsec}]{\ref{\detokenize{core/core:coreflowsec}} \nameref{\detokenize{core/core:coreflowsec}}} の {\hyperref[\detokenize{core/core:actresolve}]{\sphinxcrossref{\DUrole{std,std-ref}{{[}8{]} リクエストの解決}}}} で順に行われます。


\subsubsection{対象条件を確認}
\label{\detokenize{common/common:id29}}
\sphinxAtStartPar
対象を指定するアクションが効果を発揮しようとした時に次の条件に該当する場合、効果を発揮する対象を失うため効果が発揮されず
アクションが解決されます。
\begin{itemize}
\item {} 
\sphinxAtStartPar
対象が存在していない場合

\item {} 
\sphinxAtStartPar
対象が分裂した場合

\end{itemize}

\sphinxAtStartPar
たとえば兵士に対して「アップ」アクションを起こし、対応して「ダウン」
アクションを起こされました。
「ダウン」の方が先に解決されるため、「アップ」を解決する時には
兵士が墓地に移っていたとします。その場合、「アップ」アクションは効果を発揮せず解決されます。

\sphinxAtStartPar
「リバース」による対象分裂も同様です。
たとえば装備兵に対して「ツイスト」アクションを起こし、対応して「リバース」アクションを起こしたとします。
この場合、「リバース」が先に解決され、装備兵が分裂します。
その場合、「ツイスト」は対象を失いアクションの効果を発揮せず解決されます。


\subsubsection{効果を発揮}
\label{\detokenize{common/common:id30}}
\sphinxAtStartPar
リクエストが解決する際に、アクションの効果に定義されている内容を実行します。
効果の中に実行不可能な部分がある場合、可能な部分のみ実行します。

\sphinxAtStartPar
たとえば、ライフの枚数が残1枚の時に5点のダメージを受けたとします。
ライフは1枚しかないので5点ダメージを受けることはできませんが、
1点までなら受けることが可能なため、
この場合1点のダメージを受けることになります。


\subsubsection{解決は墓地移動までを含む}
\label{\detokenize{common/common:id31}}
\sphinxAtStartPar
「リクエストを解決する」という文言には、キーカードを墓地に移動し終えるまでが含まれています。

\sphinxAtStartPar
「リクエストを解決する」をまとめると次のようになります。
\begin{enumerate}
\sphinxsetlistlabels{\arabic}{enumi}{enumii}{}{.}%
\item {} 
\sphinxAtStartPar
ステージの一番上にあるリクエストを特定する

\item {} 
\sphinxAtStartPar
リクエストの対象が正しいか確認する

\item {} 
\sphinxAtStartPar
正しい場合、リクエストされたアクションの効果を可能な限り実行する

\item {} 
\sphinxAtStartPar
リクエストをステージから取り除く

\item {} 
\sphinxAtStartPar
キーカードがある場合、墓地に移す

\end{enumerate}


\subsubsection{ステージ上で効果を発揮}
\label{\detokenize{common/common:id32}}
\sphinxAtStartPar
アクションの効果を実行する際にリクエストはまだステージ上にあります。

\sphinxAtStartPar
効果の実行が完了した後、ステージ上から取り除かれます。


\subsubsection{キーカードを墓地に移す}
\label{\detokenize{common/common:keycard-gy}}\label{\detokenize{common/common:id33}}
\sphinxAtStartPar
効果を発揮した後、そのアクションをステージから取り除き、キーカードを墓地に移します。
ただし効果によってキーカードを場に出した場合や手札に戻した場合、
そのカードを移す先が明確になっているため、墓地には移しません。


\subsection{勝敗判定}
\label{\detokenize{common/common:id34}}
\sphinxAtStartPar
アクションを解決するたびに勝敗判定が行われます。

\sphinxAtStartPar
勝敗はライフを確認し0枚の場合そのプレイヤーは敗北となります。

\sphinxAtStartPar
勝敗判定はターンプレイヤーから行われます。
もし、両プレイヤーのライフが0枚の場合、ターンプレイヤーの負けとなります。

\sphinxAtStartPar
補足ですがこの勝敗判定は、 \hyperref[\detokenize{core/core:coreflowsec}]{\ref{\detokenize{core/core:coreflowsec}} \nameref{\detokenize{core/core:coreflowsec}}} の {\hyperref[\detokenize{core/core:winlose}]{\sphinxcrossref{\DUrole{std,std-ref}{{[}9{]} 勝敗判定}}}} で確認されます。


\subsection{誘発チェック}
\label{\detokenize{common/common:id35}}
\sphinxAtStartPar
アクションが解決された際に、
アクションの誘発条件に該当するとアクションが誘発されることがあります。

\sphinxAtStartPar
誘発とは、自動的にアクションがリクエストされることです。

\sphinxAtStartPar
BlackPokerの基本的なルールでは、次の2つの誘発パターンがほとんどです。

\sphinxAtStartPar
それ以外の場合は、\hyperref[\detokenize{core/core:coreflowsec}]{\ref{\detokenize{core/core:coreflowsec}} \nameref{\detokenize{core/core:coreflowsec}}} を参照してください。


\subsubsection{パターン1: フェイズ系}
\label{\detokenize{common/common:id36}}
\sphinxAtStartPar
アタックアクションでは対象のアクションが解決すると次のアクションが誘発します。
\begin{enumerate}
\sphinxsetlistlabels{\arabic}{enumi}{enumii}{}{.}%
\item {} 
\sphinxAtStartPar
アタック

\item {} 
\sphinxAtStartPar
ブロック

\item {} 
\sphinxAtStartPar
ダメージ判定

\end{enumerate}

\sphinxAtStartPar
アタックアクションが解決すると、ブロックアクションが誘発します。

\sphinxAtStartPar
ブロックアクションが解決するとダメージ判定アクションが誘発します。

\sphinxAtStartPar
BlackPokerでは一般的なTCGではフェイズとして扱われるものも全てアクションとして定義されています。

\sphinxAtStartPar
同様に、エンド、チャージ、ドローアクションも定義されています。


\subsubsection{パターン2: 世代交代}
\label{\detokenize{common/common:id37}}
\sphinxAtStartPar
BlackPokerでは、Joker,A,J,Q,Kのカードが場から墓地に移った場合、世代交代というアクションが誘発します。

\sphinxAtStartPar
効果の内容はアクションリストを参照してください。

\sphinxAtStartPar
\hyperref[\detokenize{auto/actionlist:act-nextgeneration}]{\ref{\detokenize{auto/actionlist:act-nextgeneration}} \nameref{\detokenize{auto/actionlist:act-nextgeneration}}}


\section{その他のルール}
\label{\detokenize{common/common:id38}}
\sphinxAtStartPar
この章では、
公開レベルやシャッフルの仕方といった
細かな決まりごとを説明します。


\subsection{公開レベル}
\label{\detokenize{common/common:id39}}
\sphinxAtStartPar
配置されているカードには、アクションの効果
を使わなくても中身や枚数を知れるものがあります。
知れる度合いには次の種類があります。
\begin{description}
\sphinxlineitem{完全公開}
\sphinxAtStartPar
全てのプレイヤーが知ることができ、
聞かれたプレイヤーは正しく答える必要がある

\sphinxlineitem{個人公開}
\sphinxAtStartPar
ライフの持ち主のみ知ることができる

\sphinxlineitem{非公開}
\sphinxAtStartPar
全てのプレイヤーは知ることができない

\end{description}

\sphinxAtStartPar
完全公開の情報であれば、ゲーム中いつでも対戦相手に聞くことができます。
各カードの配置と公開・非公開の度合いは次のとおりです。
\begin{description}
\sphinxlineitem{ライフ}
\begin{DUlineblock}{0em}
\item[] 完全公開:10枚未満のライフ枚数
\item[] 個人公開:ライフの枚数
\item[] 非公開:ライフの中身
\end{DUlineblock}

\sphinxlineitem{墓地}
\begin{DUlineblock}{0em}
\item[] 完全公開:墓地の一番上のカード
\item[] 個人公開:墓地の中身
\item[] 非公開:なし
\end{DUlineblock}

\sphinxlineitem{場}
\begin{DUlineblock}{0em}
\item[] 完全公開:表裏を変えずに見えるカード
\item[] 個人公開:伏せてあるカード
\item[] 非公開:なし
\end{DUlineblock}

\sphinxlineitem{手札}
\begin{DUlineblock}{0em}
\item[] 完全公開:手札の枚数
\item[] 個人公開:手札の中身
\item[] 非公開:なし
\end{DUlineblock}

\sphinxlineitem{フォグ}
\begin{DUlineblock}{0em}
\item[] 完全公開:表裏を変えずに見えるカード
\item[] 個人公開:伏せてあるカード
\item[] 非公開:なし
\end{DUlineblock}

\end{description}


\subsubsection{残りライフを聞かれたらどうしたらいいの?}
\label{\detokenize{common/common:id40}}
\sphinxAtStartPar
対戦相手から残りのライフを聞かれた場合、自分のライフの枚数を10枚まで数えます。
10枚未満であれば枚数を答え、10枚以上の場合「10枚以上です」と答えて下さい。
10枚以上の場合、正確な枚数を答える必要はありません。


\subsubsection{墓地の一番上のカードはいつ決まるのか?}
\label{\detokenize{common/common:id41}}
\sphinxAtStartPar
カードを墓地に移す際に移すカードの中から1枚を公開してください。
すでに墓地にあるカードを改めて公開しないでください。


\subsection{デッキのシャッフルについて}
\label{\detokenize{common/common:id42}}
\sphinxAtStartPar
BlackPokerでは
コンセプトの一つに「相手のカードに触らない」があるため、
対戦相手にデッキのシャッフルをお願いする必要はありません。

\sphinxAtStartPar
ただし、シャッフルしてほしい場合は、対戦相手にお願いしても構いません。
逆に、対戦相手があまりシャッフルしていない場合は、
さらにシャッフルをお願いできます。


\subsection{防壁の置き方}
\label{\detokenize{common/common:id43}}
\sphinxAtStartPar
防壁を場に出す際は、次のルールに従って配置してください。(\hyperref[\detokenize{common/common:set-bulwork}]{Fig.\@ \ref{\detokenize{common/common:set-bulwork}}})
\begin{itemize}
\item {} 
\sphinxAtStartPar
防壁を置く際は、ライフ側に寄せて配置してください。

\item {} 
\sphinxAtStartPar
防壁の左右の入れ替えは行わないでください。

\end{itemize}

\begin{figure}[htbp]
\centering
\capstart

\noindent\sphinxincludegraphics{{set-bulwork}.pdf}
\caption{防壁の置き方}\label{\detokenize{common/common:id54}}\label{\detokenize{common/common:set-bulwork}}\end{figure}


\subsection{フォグの置き方}
\label{\detokenize{common/common:id44}}
\sphinxAtStartPar
フォグにアップなどのカードを置く場合は、次のルールに従って配置してください。(\hyperref[\detokenize{common/common:set-fog}]{Fig.\@ \ref{\detokenize{common/common:set-fog}}})
\begin{itemize}
\item {} 
\sphinxAtStartPar
対象の向きにカードを傾けて置いてください。

\item {} 
\sphinxAtStartPar
ダウンなど対戦相手のカードを対象とする場合も同様に置いてください。

\item {} 
\sphinxAtStartPar
フォースなど対象を取らない場合は、対象がないため、傾ける必要はありません。

\item {} 
\sphinxAtStartPar
フォグのカードは、場のカードと重ならないように間隔を空けて配置してください。

\end{itemize}

\begin{figure}[htbp]
\centering
\capstart

\noindent\sphinxincludegraphics{{set-fog}.pdf}
\caption{フォグの置き方}\label{\detokenize{common/common:id55}}\label{\detokenize{common/common:set-fog}}\end{figure}


\subsection{ステージの置き方}
\label{\detokenize{common/common:id45}}
\sphinxAtStartPar
ステージ上にあるリクエストのキーカードは、場のカードと区別できるように、次のルールに従って配置してください。(\hyperref[\detokenize{common/common:set-stage}]{Fig.\@ \ref{\detokenize{common/common:set-stage}}})
\begin{itemize}
\item {} 
\sphinxAtStartPar
ステージ上にあるリクエストのキーカードは傾けて置いてください。

\item {} 
\sphinxAtStartPar
フォグと区別できるように置いてください。

\item {} 
\sphinxAtStartPar
対象がある場合は、その方向に傾けることを推奨します。

\end{itemize}

\begin{figure}[htbp]
\centering
\capstart

\noindent\sphinxincludegraphics{{set-stage}.pdf}
\caption{ステージの置き方}\label{\detokenize{common/common:id56}}\label{\detokenize{common/common:set-stage}}\end{figure}

\sphinxstepscope


\chapter{対戦レギュレーション}
\label{\detokenize{match-regulations/match-regulations:match-regulations-rst}}\label{\detokenize{match-regulations/match-regulations:id1}}\label{\detokenize{match-regulations/match-regulations::doc}}
\index{た\textbar{}対戦レギュレーション@\spxentry{た\textbar{}対戦レギュレーション}}\ignorespaces 

\section{対戦レギュレーションとは}
\label{\detokenize{match-regulations/match-regulations:index-0}}\label{\detokenize{match-regulations/match-regulations:id2}}
\sphinxAtStartPar
対戦レギュレーションとは、
BlackPokerで対戦する前にプレイヤー間で決定する
規則のことです。

\sphinxAtStartPar
BlackPokerはトランプだけで遊べるため、
対戦する前にプレイヤー間でルールのすり合わせをする必要があります。


\section{定義項目}
\label{\detokenize{match-regulations/match-regulations:id3}}
\sphinxAtStartPar
対戦レギュレーションは次の各項目を決めることで決定します。

\index{フォーマット(対戦レギュレーション)@\spxentry{フォーマット(対戦レギュレーション)}}\ignorespaces \begin{description}
\sphinxlineitem{フォーマット}
\sphinxAtStartPar
使用するフォーマット。

\sphinxAtStartPar
フォーマットとは、使用できるアクションなどをまとめたセット。
いくつか種類があり、プレイヤーの熟練度に合わせて選ぶことができる。

\sphinxAtStartPar
詳しくは \hyperref[\detokenize{format/format:format-rst}]{\ref{\detokenize{format/format:format-rst}} \nameref{\detokenize{format/format:format-rst}}} 参照

\end{description}

\index{フレーム(対戦レギュレーション)@\spxentry{フレーム(対戦レギュレーション)}}\ignorespaces \begin{description}
\sphinxlineitem{フレーム}
\sphinxAtStartPar
使用するフレーム。

\sphinxAtStartPar
フレームとは、フォーマットとは異なる角度から使えるアクションや領域をまとめたセット。
フォーマットは熟練度で選び、フレームは遊び方で選ぶイメージです。

\sphinxAtStartPar
詳しくは \hyperref[\detokenize{frame/frame:frame-rst}]{\ref{\detokenize{frame/frame:frame-rst}} \nameref{\detokenize{frame/frame:frame-rst}}} 参照

\end{description}


\subsection{対戦レギュレーションの表記}
\label{\detokenize{match-regulations/match-regulations:id4}}
\sphinxAtStartPar
対戦レギュレーションは次のように表記します。

\begin{sphinxVerbatim}[commandchars=\\\{\}]
\PYG{n}{フォーマット}\PYG{o}{+}\PYG{n}{フレーム}
\end{sphinxVerbatim}

\sphinxAtStartPar
例えば、次のように表記します。

\begin{sphinxVerbatim}[commandchars=\\\{\}]
\PYG{n}{ライト}\PYG{o}{+}\PYG{n}{エントリー}
\end{sphinxVerbatim}


\section{対戦レギュレーションの決め方}
\label{\detokenize{match-regulations/match-regulations:id5}}
\sphinxAtStartPar
対戦レギュレーションを決定する手順を記載します。

\sphinxAtStartPar
公式では対応していない組み合わせが存在するため、手順に従って対戦レギュレーションを決定してください。
\begin{enumerate}
\sphinxsetlistlabels{\arabic}{enumi}{enumii}{}{.}%
\item {} 
\sphinxAtStartPar
\sphinxstylestrong{フォーマットの決定}

\sphinxAtStartPar
「ライト」「スタンダード」などフォーマットを決めます。

\sphinxAtStartPar
フォーマットが変わると覚えるアクションなどの量が大きく変わるので、プレイヤー同士の熟練度をもとに決定しましょう。

\item {} 
\sphinxAtStartPar
\sphinxstylestrong{フレームの決定}

\sphinxAtStartPar
手順1で選択したフォーマットをもとに次のフレーム対応一覧より、フレームを決めます。

\sphinxAtStartPar
「◯」と表記されている組み合わせが選択出来ます。(\hyperref[\detokenize{match-regulations/match-regulations:frame-format}]{Table \ref{\detokenize{match-regulations/match-regulations:frame-format}}})


\begin{savenotes}\sphinxattablestart
\sphinxthistablewithglobalstyle
\centering
\sphinxcapstartof{table}
\sphinxthecaptionisattop
\sphinxcaption{フレーム対応一覧}\label{\detokenize{match-regulations/match-regulations:id6}}\label{\detokenize{match-regulations/match-regulations:frame-format}}
\sphinxaftertopcaption
\begin{tabulary}{\linewidth}[t]{|T|T|T|T|T|}
\sphinxtoprule
\sphinxtableatstartofbodyhook
\sphinxAtStartPar
【フレーム】
&
\sphinxAtStartPar
ライト
&
\sphinxAtStartPar
スタンダード
&
\sphinxAtStartPar
プロ
&
\sphinxAtStartPar
マスター
\\
\sphinxhline
\sphinxAtStartPar
エントリー20
&
\sphinxAtStartPar
◯
&
\sphinxAtStartPar
◯
&
\sphinxAtStartPar
x
&
\sphinxAtStartPar
x
\\
\sphinxhline
\sphinxAtStartPar
パック
&
\sphinxAtStartPar
◯
&
\sphinxAtStartPar
◯
&
\sphinxAtStartPar
◯
&
\sphinxAtStartPar
◯
\\
\sphinxhline
\sphinxAtStartPar
レアパック
&
\sphinxAtStartPar
x
&
\sphinxAtStartPar
◯
&
\sphinxAtStartPar
◯
&
\sphinxAtStartPar
◯
\\
\sphinxhline
\sphinxAtStartPar
ストラテジー
&
\sphinxAtStartPar
x
&
\sphinxAtStartPar
x
&
\sphinxAtStartPar
◯
&
\sphinxAtStartPar
◯
\\
\sphinxbottomrule
\end{tabulary}
\sphinxtableafterendhook\par
\sphinxattableend\end{savenotes}

\end{enumerate}

\sphinxstepscope


\chapter{フォーマット}
\label{\detokenize{format/format:format-rst}}\label{\detokenize{format/format:id1}}\label{\detokenize{format/format::doc}}
\index{フォーマット@\spxentry{フォーマット}}\ignorespaces 

\section{フォーマットとは}
\label{\detokenize{format/format:index-0}}\label{\detokenize{format/format:id2}}
\sphinxAtStartPar
BlackPokerにはいくつかのフォーマットがあり、
フォーマットによりゲーム内でできる行動が異なります。
同じトランプでもフォーマットを変えることで様々な遊び方ができます。


\section{定義項目}
\label{\detokenize{format/format:id3}}
\sphinxAtStartPar
フォーマットには次の項目が定義されています。
\begin{description}
\sphinxlineitem{アクションリスト}
\sphinxAtStartPar
リクエスト可能なアクションの一覧

\sphinxlineitem{キャラクターリスト}
\sphinxAtStartPar
場に登場可能なキャラクターの一覧

\sphinxlineitem{フォグリスト}
\sphinxAtStartPar
アクションなどによってフォグに置かれるコンポーネントの一覧

\end{description}


\section{フォーマット定義}
\label{\detokenize{format/format:id4}}
\sphinxAtStartPar
公式として次のフォーマットを定義しています。
\begin{description}
\sphinxlineitem{ライト}
\sphinxAtStartPar
初心者向けフォーマット。覚えるアクションを最小限にしています。

\sphinxlineitem{スタンダード}
\sphinxAtStartPar
中級者向けフォーマット。ライトより幅広い戦略が楽しめます。

\sphinxlineitem{プロ}
\sphinxAtStartPar
上級者向けフォーマット。高度なアクションが多数含まれています。

\sphinxlineitem{マスター}
\sphinxAtStartPar
達人向けフォーマット。全てのアクションが含まれています。

\end{description}

\sphinxAtStartPar
各フォーマットの詳細は次の公式フォーマットを参照してください。

\sphinxstepscope


\subsection{アクションリスト}
\label{\detokenize{auto/actionlist:act-act}}\label{\detokenize{auto/actionlist:id1}}\label{\detokenize{auto/actionlist::doc}}
\begin{sphinxShadowBox}
\sphinxstyletopictitle{目次}
\begin{itemize}
\item {} 
\sphinxAtStartPar
\phantomsection\label{\detokenize{auto/actionlist:id59}}{\hyperref[\detokenize{auto/actionlist:id3}]{\sphinxcrossref{基本}}}
\begin{itemize}
\item {} 
\sphinxAtStartPar
\phantomsection\label{\detokenize{auto/actionlist:id60}}{\hyperref[\detokenize{auto/actionlist:act-end}]{\sphinxcrossref{エンド (基本)}}}

\item {} 
\sphinxAtStartPar
\phantomsection\label{\detokenize{auto/actionlist:id61}}{\hyperref[\detokenize{auto/actionlist:act-charge}]{\sphinxcrossref{チャージ (基本)}}}

\item {} 
\sphinxAtStartPar
\phantomsection\label{\detokenize{auto/actionlist:id62}}{\hyperref[\detokenize{auto/actionlist:act-draw}]{\sphinxcrossref{ドロー (基本)}}}

\item {} 
\sphinxAtStartPar
\phantomsection\label{\detokenize{auto/actionlist:id63}}{\hyperref[\detokenize{auto/actionlist:act-attack}]{\sphinxcrossref{アタック (基本)}}}

\item {} 
\sphinxAtStartPar
\phantomsection\label{\detokenize{auto/actionlist:id64}}{\hyperref[\detokenize{auto/actionlist:act-block}]{\sphinxcrossref{ブロック (基本)}}}

\item {} 
\sphinxAtStartPar
\phantomsection\label{\detokenize{auto/actionlist:id65}}{\hyperref[\detokenize{auto/actionlist:act-damagejudge}]{\sphinxcrossref{ダメージ判定 (基本)}}}

\item {} 
\sphinxAtStartPar
\phantomsection\label{\detokenize{auto/actionlist:id66}}{\hyperref[\detokenize{auto/actionlist:act-nextgeneration}]{\sphinxcrossref{世代交代 (基本)}}}

\end{itemize}

\item {} 
\sphinxAtStartPar
\phantomsection\label{\detokenize{auto/actionlist:id67}}{\hyperref[\detokenize{auto/actionlist:id11}]{\sphinxcrossref{召喚}}}
\begin{itemize}
\item {} 
\sphinxAtStartPar
\phantomsection\label{\detokenize{auto/actionlist:id68}}{\hyperref[\detokenize{auto/actionlist:act-setbulwark}]{\sphinxcrossref{防壁設置 (召喚)}}}

\item {} 
\sphinxAtStartPar
\phantomsection\label{\detokenize{auto/actionlist:id69}}{\hyperref[\detokenize{auto/actionlist:act-summonssoldier}]{\sphinxcrossref{兵士召喚 (召喚)}}}

\item {} 
\sphinxAtStartPar
\phantomsection\label{\detokenize{auto/actionlist:id70}}{\hyperref[\detokenize{auto/actionlist:act-summonshero}]{\sphinxcrossref{英雄召喚 (召喚)}}}

\item {} 
\sphinxAtStartPar
\phantomsection\label{\detokenize{auto/actionlist:id71}}{\hyperref[\detokenize{auto/actionlist:act-summonsace}]{\sphinxcrossref{エース召喚 (召喚)}}}

\item {} 
\sphinxAtStartPar
\phantomsection\label{\detokenize{auto/actionlist:id72}}{\hyperref[\detokenize{auto/actionlist:act-quicksummonsace}]{\sphinxcrossref{クイック召喚 (召喚)}}}

\item {} 
\sphinxAtStartPar
\phantomsection\label{\detokenize{auto/actionlist:id73}}{\hyperref[\detokenize{auto/actionlist:act-summonsmagic}]{\sphinxcrossref{魔術士召喚 (召喚)}}}

\item {} 
\sphinxAtStartPar
\phantomsection\label{\detokenize{auto/actionlist:id74}}{\hyperref[\detokenize{auto/actionlist:act-mountsoldier}]{\sphinxcrossref{装備 (召喚)}}}

\end{itemize}

\item {} 
\sphinxAtStartPar
\phantomsection\label{\detokenize{auto/actionlist:id75}}{\hyperref[\detokenize{auto/actionlist:id19}]{\sphinxcrossref{速攻魔法}}}
\begin{itemize}
\item {} 
\sphinxAtStartPar
\phantomsection\label{\detokenize{auto/actionlist:id76}}{\hyperref[\detokenize{auto/actionlist:act-up}]{\sphinxcrossref{アップ (速攻魔法)}}}

\item {} 
\sphinxAtStartPar
\phantomsection\label{\detokenize{auto/actionlist:id77}}{\hyperref[\detokenize{auto/actionlist:act-down}]{\sphinxcrossref{ダウン (速攻魔法)}}}

\item {} 
\sphinxAtStartPar
\phantomsection\label{\detokenize{auto/actionlist:id78}}{\hyperref[\detokenize{auto/actionlist:act-twist}]{\sphinxcrossref{ツイスト (速攻魔法)}}}

\item {} 
\sphinxAtStartPar
\phantomsection\label{\detokenize{auto/actionlist:id79}}{\hyperref[\detokenize{auto/actionlist:act-counter}]{\sphinxcrossref{カウンター (速攻魔法)}}}

\item {} 
\sphinxAtStartPar
\phantomsection\label{\detokenize{auto/actionlist:id80}}{\hyperref[\detokenize{auto/actionlist:act-reunion}]{\sphinxcrossref{再会 (速攻魔法)}}}

\item {} 
\sphinxAtStartPar
\phantomsection\label{\detokenize{auto/actionlist:id81}}{\hyperref[\detokenize{auto/actionlist:act-kill}]{\sphinxcrossref{キル (速攻魔法)}}}

\item {} 
\sphinxAtStartPar
\phantomsection\label{\detokenize{auto/actionlist:id82}}{\hyperref[\detokenize{auto/actionlist:act-truce}]{\sphinxcrossref{停戦 (速攻魔法)}}}

\item {} 
\sphinxAtStartPar
\phantomsection\label{\detokenize{auto/actionlist:id83}}{\hyperref[\detokenize{auto/actionlist:act-changetarget}]{\sphinxcrossref{対象変更 (速攻魔法)}}}

\item {} 
\sphinxAtStartPar
\phantomsection\label{\detokenize{auto/actionlist:id84}}{\hyperref[\detokenize{auto/actionlist:act-search}]{\sphinxcrossref{サーチ (速攻魔法)}}}

\item {} 
\sphinxAtStartPar
\phantomsection\label{\detokenize{auto/actionlist:id85}}{\hyperref[\detokenize{auto/actionlist:bj}]{\sphinxcrossref{B・J (速攻魔法)}}}

\item {} 
\sphinxAtStartPar
\phantomsection\label{\detokenize{auto/actionlist:id86}}{\hyperref[\detokenize{auto/actionlist:rsf}]{\sphinxcrossref{R・S・F (速攻魔法)}}}

\item {} 
\sphinxAtStartPar
\phantomsection\label{\detokenize{auto/actionlist:id87}}{\hyperref[\detokenize{auto/actionlist:act-reverse}]{\sphinxcrossref{リバース (速攻魔法)}}}

\item {} 
\sphinxAtStartPar
\phantomsection\label{\detokenize{auto/actionlist:id88}}{\hyperref[\detokenize{auto/actionlist:act-unsummons}]{\sphinxcrossref{帰還 (速攻魔法)}}}

\end{itemize}

\item {} 
\sphinxAtStartPar
\phantomsection\label{\detokenize{auto/actionlist:id89}}{\hyperref[\detokenize{auto/actionlist:id31}]{\sphinxcrossref{通常魔法}}}
\begin{itemize}
\item {} 
\sphinxAtStartPar
\phantomsection\label{\detokenize{auto/actionlist:id90}}{\hyperref[\detokenize{auto/actionlist:act-destroybulwark}]{\sphinxcrossref{防壁破壊 (通常魔法)}}}

\item {} 
\sphinxAtStartPar
\phantomsection\label{\detokenize{auto/actionlist:id91}}{\hyperref[\detokenize{auto/actionlist:act-throwing}]{\sphinxcrossref{投擲 (通常魔法)}}}

\item {} 
\sphinxAtStartPar
\phantomsection\label{\detokenize{auto/actionlist:id92}}{\hyperref[\detokenize{auto/actionlist:act-deathlance}]{\sphinxcrossref{死の槍 (通常魔法)}}}

\item {} 
\sphinxAtStartPar
\phantomsection\label{\detokenize{auto/actionlist:id93}}{\hyperref[\detokenize{auto/actionlist:act-addbulwark}]{\sphinxcrossref{防壁補充 (通常魔法)}}}

\item {} 
\sphinxAtStartPar
\phantomsection\label{\detokenize{auto/actionlist:id94}}{\hyperref[\detokenize{auto/actionlist:act-reanimate}]{\sphinxcrossref{リアニメイト (通常魔法)}}}

\item {} 
\sphinxAtStartPar
\phantomsection\label{\detokenize{auto/actionlist:id95}}{\hyperref[\detokenize{auto/actionlist:act-handeth}]{\sphinxcrossref{ハンデス (通常魔法)}}}

\item {} 
\sphinxAtStartPar
\phantomsection\label{\detokenize{auto/actionlist:id96}}{\hyperref[\detokenize{auto/actionlist:act-force}]{\sphinxcrossref{フォース (通常魔法)}}}

\item {} 
\sphinxAtStartPar
\phantomsection\label{\detokenize{auto/actionlist:id97}}{\hyperref[\detokenize{auto/actionlist:act-swordrain}]{\sphinxcrossref{剣の雨 (通常魔法)}}}

\item {} 
\sphinxAtStartPar
\phantomsection\label{\detokenize{auto/actionlist:id98}}{\hyperref[\detokenize{auto/actionlist:act-recruit}]{\sphinxcrossref{徴募 (通常魔法)}}}

\item {} 
\sphinxAtStartPar
\phantomsection\label{\detokenize{auto/actionlist:id99}}{\hyperref[\detokenize{auto/actionlist:act-surprise}]{\sphinxcrossref{奇襲 (通常魔法)}}}

\end{itemize}

\end{itemize}
\end{sphinxShadowBox}


\subsubsection{基本}
\label{\detokenize{auto/actionlist:id3}}

\paragraph{エンド (基本)}
\label{\detokenize{auto/actionlist:act-end}}\label{\detokenize{auto/actionlist:id4}}
\sphinxAtStartPar
\sphinxstylestrong{フォーマット:}


\begin{savenotes}\sphinxattablestart
\sphinxthistablewithglobalstyle
\centering
\begin{tabular}[t]{|\X{1}{4}|\X{1}{4}|\X{1}{4}|\X{1}{4}|}
\sphinxtoprule
\sphinxstyletheadfamily 
\sphinxAtStartPar
ライト
&\sphinxstyletheadfamily 
\sphinxAtStartPar
スタンダード
&\sphinxstyletheadfamily 
\sphinxAtStartPar
プロ
&\sphinxstyletheadfamily 
\sphinxAtStartPar
マスター
\\
\sphinxmidrule
\sphinxtableatstartofbodyhook
\sphinxAtStartPar
◯
&
\sphinxAtStartPar
◯
&
\sphinxAtStartPar
◯
&
\sphinxAtStartPar
◯
\\
\sphinxbottomrule
\end{tabular}
\sphinxtableafterendhook\par
\sphinxattableend\end{savenotes}

\sphinxAtStartPar
\sphinxstylestrong{トリガー:} 直接

\sphinxAtStartPar
\sphinxstylestrong{スピード:} 通常

\sphinxAtStartPar
\sphinxstylestrong{タイミング:} メイン

\sphinxAtStartPar
\sphinxstylestrong{効果:}
\begin{enumerate}
\sphinxsetlistlabels{\arabic}{enumi}{enumii}{}{.}%
\item {} 
\sphinxAtStartPar
手札が7枚を越えた場合、7枚になるよう手札を捨てる。

\item {} 
\sphinxAtStartPar
自分のフォグにあるカードを全て墓地に移す。

\item {} 
\sphinxAtStartPar
自分のターンを終了し、対戦相手にターンを渡す。

\end{enumerate}

\sphinxAtStartPar
\sphinxstylestrong{導入バージョン:}  0.2

\sphinxAtStartPar
\sphinxstylestrong{更新バージョン:}  8.0


\bigskip\hrule\bigskip



\paragraph{チャージ (基本)}
\label{\detokenize{auto/actionlist:act-charge}}\label{\detokenize{auto/actionlist:id5}}
\sphinxAtStartPar
\sphinxstylestrong{フォーマット:}


\begin{savenotes}\sphinxattablestart
\sphinxthistablewithglobalstyle
\centering
\begin{tabular}[t]{|\X{1}{4}|\X{1}{4}|\X{1}{4}|\X{1}{4}|}
\sphinxtoprule
\sphinxstyletheadfamily 
\sphinxAtStartPar
ライト
&\sphinxstyletheadfamily 
\sphinxAtStartPar
スタンダード
&\sphinxstyletheadfamily 
\sphinxAtStartPar
プロ
&\sphinxstyletheadfamily 
\sphinxAtStartPar
マスター
\\
\sphinxmidrule
\sphinxtableatstartofbodyhook
\sphinxAtStartPar
◯
&
\sphinxAtStartPar
◯
&
\sphinxAtStartPar
◯
&
\sphinxAtStartPar
◯
\\
\sphinxbottomrule
\end{tabular}
\sphinxtableafterendhook\par
\sphinxattableend\end{savenotes}

\sphinxAtStartPar
\sphinxstylestrong{トリガー:} 誘発

\sphinxAtStartPar
\sphinxstylestrong{スピード:} 即時

\sphinxAtStartPar
\sphinxstylestrong{タイミング:} メイン

\sphinxAtStartPar
\sphinxstylestrong{誘発条件:}

\sphinxAtStartPar
ターンプレイヤーかつ、エンドアクションが解決した時に誘発する。

\sphinxAtStartPar
\sphinxstylestrong{効果:}

\sphinxAtStartPar
ターンを持っているプレイヤーの場にいるキャラクターを全てチャージ状態にする。

\sphinxAtStartPar
\sphinxstylestrong{導入バージョン:}  4.0

\sphinxAtStartPar
\sphinxstylestrong{更新バージョン:}  8.0


\bigskip\hrule\bigskip



\paragraph{ドロー (基本)}
\label{\detokenize{auto/actionlist:act-draw}}\label{\detokenize{auto/actionlist:id6}}
\sphinxAtStartPar
\sphinxstylestrong{フォーマット:}


\begin{savenotes}\sphinxattablestart
\sphinxthistablewithglobalstyle
\centering
\begin{tabular}[t]{|\X{1}{4}|\X{1}{4}|\X{1}{4}|\X{1}{4}|}
\sphinxtoprule
\sphinxstyletheadfamily 
\sphinxAtStartPar
ライト
&\sphinxstyletheadfamily 
\sphinxAtStartPar
スタンダード
&\sphinxstyletheadfamily 
\sphinxAtStartPar
プロ
&\sphinxstyletheadfamily 
\sphinxAtStartPar
マスター
\\
\sphinxmidrule
\sphinxtableatstartofbodyhook
\sphinxAtStartPar
◯
&
\sphinxAtStartPar
◯
&
\sphinxAtStartPar
◯
&
\sphinxAtStartPar
◯
\\
\sphinxbottomrule
\end{tabular}
\sphinxtableafterendhook\par
\sphinxattableend\end{savenotes}

\sphinxAtStartPar
\sphinxstylestrong{トリガー:} 誘発

\sphinxAtStartPar
\sphinxstylestrong{スピード:} 通常

\sphinxAtStartPar
\sphinxstylestrong{タイミング:} メイン

\sphinxAtStartPar
\sphinxstylestrong{誘発条件:}

\sphinxAtStartPar
ターンプレイヤーかつ、チャージアクションが解決した時に誘発する。

\sphinxAtStartPar
\sphinxstylestrong{効果:}
\begin{enumerate}
\sphinxsetlistlabels{\arabic}{enumi}{enumii}{}{.}%
\item {} 
\sphinxAtStartPar
ライフの一番上からカードを1枚引き、手札に加える。

\item {} 
\sphinxAtStartPar
必要であれば、更にライフの一番上からカードを1枚引き、手札に加える。

\end{enumerate}

\sphinxAtStartPar
\sphinxstylestrong{導入バージョン:}  4.0

\sphinxAtStartPar
\sphinxstylestrong{更新バージョン:}  8.0


\bigskip\hrule\bigskip



\paragraph{アタック (基本)}
\label{\detokenize{auto/actionlist:act-attack}}\label{\detokenize{auto/actionlist:id7}}
\sphinxAtStartPar
\sphinxstylestrong{フォーマット:}


\begin{savenotes}\sphinxattablestart
\sphinxthistablewithglobalstyle
\centering
\begin{tabular}[t]{|\X{1}{4}|\X{1}{4}|\X{1}{4}|\X{1}{4}|}
\sphinxtoprule
\sphinxstyletheadfamily 
\sphinxAtStartPar
ライト
&\sphinxstyletheadfamily 
\sphinxAtStartPar
スタンダード
&\sphinxstyletheadfamily 
\sphinxAtStartPar
プロ
&\sphinxstyletheadfamily 
\sphinxAtStartPar
マスター
\\
\sphinxmidrule
\sphinxtableatstartofbodyhook
\sphinxAtStartPar
◯
&
\sphinxAtStartPar
◯
&
\sphinxAtStartPar
◯
&
\sphinxAtStartPar
◯
\\
\sphinxbottomrule
\end{tabular}
\sphinxtableafterendhook\par
\sphinxattableend\end{savenotes}

\sphinxAtStartPar
\sphinxstylestrong{トリガー:} 直接

\sphinxAtStartPar
\sphinxstylestrong{スピード:} 通常

\sphinxAtStartPar
\sphinxstylestrong{タイミング:} メイン

\sphinxAtStartPar
\sphinxstylestrong{起動条件:}

\sphinxAtStartPar
プレイヤーは、1ターンに1回までこのアクションを起こすことができる。

\sphinxAtStartPar
\sphinxstylestrong{効果:}

\sphinxAtStartPar
対戦相手を攻撃するアタッカーをドライブして指定する。

\sphinxAtStartPar
・アタッカーはラベルにアタッカーを持っているキャラクターを指定できる。

\sphinxAtStartPar
・アタッカーは複数指定可能。ドライブ状態の兵士は指定できない。

\sphinxAtStartPar
・このターン場に出たキャラクターは指定できない、ただしラベルに速攻があるキャラクターは指定できる。

\sphinxAtStartPar
\sphinxstylestrong{導入バージョン:}  0.2

\sphinxAtStartPar
\sphinxstylestrong{更新バージョン:}  8.0


\bigskip\hrule\bigskip



\paragraph{ブロック (基本)}
\label{\detokenize{auto/actionlist:act-block}}\label{\detokenize{auto/actionlist:id8}}
\sphinxAtStartPar
\sphinxstylestrong{フォーマット:}


\begin{savenotes}\sphinxattablestart
\sphinxthistablewithglobalstyle
\centering
\begin{tabular}[t]{|\X{1}{4}|\X{1}{4}|\X{1}{4}|\X{1}{4}|}
\sphinxtoprule
\sphinxstyletheadfamily 
\sphinxAtStartPar
ライト
&\sphinxstyletheadfamily 
\sphinxAtStartPar
スタンダード
&\sphinxstyletheadfamily 
\sphinxAtStartPar
プロ
&\sphinxstyletheadfamily 
\sphinxAtStartPar
マスター
\\
\sphinxmidrule
\sphinxtableatstartofbodyhook
\sphinxAtStartPar
◯
&
\sphinxAtStartPar
◯
&
\sphinxAtStartPar
◯
&
\sphinxAtStartPar
◯
\\
\sphinxbottomrule
\end{tabular}
\sphinxtableafterendhook\par
\sphinxattableend\end{savenotes}

\sphinxAtStartPar
\sphinxstylestrong{トリガー:} 誘発

\sphinxAtStartPar
\sphinxstylestrong{スピード:} 通常

\sphinxAtStartPar
\sphinxstylestrong{タイミング:} メイン

\sphinxAtStartPar
\sphinxstylestrong{誘発条件:}

\sphinxAtStartPar
ターンプレイヤーかつ、アタックアクションが解決した時にアタッカーが1体以上いる場合、誘発する。

\sphinxAtStartPar
\sphinxstylestrong{効果:}

\sphinxAtStartPar
対戦相手はアタックアクションにて指定されたアタッカー毎にそれをブロックするキャラクター(ブロッカー)を指定する。

\sphinxAtStartPar
・ブロッカーはラベルにブロッカーを持っているキャラクターを指定できる。

\sphinxAtStartPar
・兵士でブロックする場合、1アタッカーに対して複数の兵士を指定できる。ドライブ状態のキャラクターは指定できない。

\sphinxAtStartPar
\sphinxstylestrong{導入バージョン:}  0.2

\sphinxAtStartPar
\sphinxstylestrong{更新バージョン:}  8.0


\bigskip\hrule\bigskip



\paragraph{ダメージ判定 (基本)}
\label{\detokenize{auto/actionlist:act-damagejudge}}\label{\detokenize{auto/actionlist:id9}}
\sphinxAtStartPar
\sphinxstylestrong{フォーマット:}


\begin{savenotes}\sphinxattablestart
\sphinxthistablewithglobalstyle
\centering
\begin{tabular}[t]{|\X{1}{4}|\X{1}{4}|\X{1}{4}|\X{1}{4}|}
\sphinxtoprule
\sphinxstyletheadfamily 
\sphinxAtStartPar
ライト
&\sphinxstyletheadfamily 
\sphinxAtStartPar
スタンダード
&\sphinxstyletheadfamily 
\sphinxAtStartPar
プロ
&\sphinxstyletheadfamily 
\sphinxAtStartPar
マスター
\\
\sphinxmidrule
\sphinxtableatstartofbodyhook
\sphinxAtStartPar
◯
&
\sphinxAtStartPar
◯
&
\sphinxAtStartPar
◯
&
\sphinxAtStartPar
◯
\\
\sphinxbottomrule
\end{tabular}
\sphinxtableafterendhook\par
\sphinxattableend\end{savenotes}

\sphinxAtStartPar
\sphinxstylestrong{トリガー:} 誘発

\sphinxAtStartPar
\sphinxstylestrong{スピード:} 通常

\sphinxAtStartPar
\sphinxstylestrong{タイミング:} メイン

\sphinxAtStartPar
\sphinxstylestrong{誘発条件:}

\sphinxAtStartPar
ターンプレイヤーかつ、ブロックアクションが解決した時に誘発する。

\sphinxAtStartPar
\sphinxstylestrong{効果:}

\sphinxAtStartPar
アタッカーとブロッカーを比較する
\begin{enumerate}
\sphinxsetlistlabels{\arabic}{enumi}{enumii}{}{.}%
\item {} 
\sphinxAtStartPar
兵士(アタッカー)と兵士(ブロッカー)の場合、アタッカーとブロッカーで数字を比較し、少ない方を墓地に移す。同じ場合は両方を墓地に移動する。1アタッカーに対して複数ブロッカーがいる場合、ブロッカーの合計数字と比較する。

\item {} 
\sphinxAtStartPar
兵士(アタッカー)と防壁(ブロッカー)の場合、次を行う。
\begin{enumerate}
\sphinxsetlistlabels{\arabic}{enumii}{enumiii}{}{.}%
\item {} 
\sphinxAtStartPar
防壁を表にし、防壁が次の条件に当てはまる場合、アタッカーを墓地に移す。
\begin{enumerate}
\sphinxsetlistlabels{\Alph}{enumiii}{enumiv}{}{.}%
\item {} 
\sphinxAtStartPar
防壁がJokerの場合

\item {} 
\sphinxAtStartPar
防壁のカードに記載されている数字と同じ数字がアタッカーのカードに含まれている場合

\end{enumerate}

\item {} 
\sphinxAtStartPar
防壁を墓地に移す。

\end{enumerate}

\item {} 
\sphinxAtStartPar
アタッカーをブロックするブロッカーが場に存在しない場合、アタッカーの数字だけ対戦相手にダメージを与える。

\end{enumerate}

\sphinxAtStartPar
\sphinxstylestrong{導入バージョン:}  0.2

\sphinxAtStartPar
\sphinxstylestrong{更新バージョン:}  8.0


\bigskip\hrule\bigskip



\paragraph{世代交代 (基本)}
\label{\detokenize{auto/actionlist:act-nextgeneration}}\label{\detokenize{auto/actionlist:id10}}
\sphinxAtStartPar
\sphinxstylestrong{フォーマット:}


\begin{savenotes}\sphinxattablestart
\sphinxthistablewithglobalstyle
\centering
\begin{tabular}[t]{|\X{1}{4}|\X{1}{4}|\X{1}{4}|\X{1}{4}|}
\sphinxtoprule
\sphinxstyletheadfamily 
\sphinxAtStartPar
ライト
&\sphinxstyletheadfamily 
\sphinxAtStartPar
スタンダード
&\sphinxstyletheadfamily 
\sphinxAtStartPar
プロ
&\sphinxstyletheadfamily 
\sphinxAtStartPar
マスター
\\
\sphinxmidrule
\sphinxtableatstartofbodyhook
\sphinxAtStartPar
◯
&
\sphinxAtStartPar
◯
&
\sphinxAtStartPar
◯
&
\sphinxAtStartPar
◯
\\
\sphinxbottomrule
\end{tabular}
\sphinxtableafterendhook\par
\sphinxattableend\end{savenotes}

\sphinxAtStartPar
\sphinxstylestrong{トリガー:} 誘発

\sphinxAtStartPar
\sphinxstylestrong{スピード:} 即時

\sphinxAtStartPar
\sphinxstylestrong{タイミング:} クイック

\sphinxAtStartPar
\sphinxstylestrong{誘発条件:}

\sphinxAtStartPar
Joker,A,J,Q,Kのカードがあなたの場から墓地に行くたびに1枚につき1回誘発する。

\sphinxAtStartPar
\sphinxstylestrong{効果:}

\sphinxAtStartPar
ライフの一番上からJoker,A,J,Q,Kのいずれかが出るまで墓地にカードを移動し、出たら手札に加える。

\sphinxAtStartPar
\sphinxstylestrong{導入バージョン:}  1.0

\sphinxAtStartPar
\sphinxstylestrong{更新バージョン:}  8.0


\subsubsection{召喚}
\label{\detokenize{auto/actionlist:id11}}

\paragraph{防壁設置 (召喚)}
\label{\detokenize{auto/actionlist:act-setbulwark}}\label{\detokenize{auto/actionlist:id12}}
\sphinxAtStartPar
\sphinxstylestrong{フォーマット:}


\begin{savenotes}\sphinxattablestart
\sphinxthistablewithglobalstyle
\centering
\begin{tabular}[t]{|\X{1}{4}|\X{1}{4}|\X{1}{4}|\X{1}{4}|}
\sphinxtoprule
\sphinxstyletheadfamily 
\sphinxAtStartPar
ライト
&\sphinxstyletheadfamily 
\sphinxAtStartPar
スタンダード
&\sphinxstyletheadfamily 
\sphinxAtStartPar
プロ
&\sphinxstyletheadfamily 
\sphinxAtStartPar
マスター
\\
\sphinxmidrule
\sphinxtableatstartofbodyhook
\sphinxAtStartPar
◯
&
\sphinxAtStartPar
◯
&
\sphinxAtStartPar
◯
&
\sphinxAtStartPar
◯
\\
\sphinxbottomrule
\end{tabular}
\sphinxtableafterendhook\par
\sphinxattableend\end{savenotes}

\sphinxAtStartPar
\sphinxstylestrong{トリガー:} 直接

\sphinxAtStartPar
\sphinxstylestrong{スピード:} 即時

\sphinxAtStartPar
\sphinxstylestrong{タイミング:} メイン

\sphinxAtStartPar
\sphinxstylestrong{コスト:} L

\sphinxAtStartPar
\sphinxstylestrong{起動条件:}

\sphinxAtStartPar
プレイヤーは、1ターンに1回までこのアクションを起こすことができる。

\sphinxAtStartPar
\sphinxstylestrong{効果:}

\sphinxAtStartPar
手札からカード1枚を防壁として裏向きかつチャージ状態で場に出す。防壁の置き方は「防壁の置き方」参照。

\sphinxAtStartPar
\sphinxstylestrong{ストーリー:}

\sphinxAtStartPar
どの国もいきなり攻撃はしません。自国を守るため、防衛拠点(防壁:bulwark)を作り、徐々に相手に攻め入ります。

\sphinxAtStartPar
\sphinxstylestrong{フレーバー:}

\sphinxAtStartPar
\sphinxstyleemphasis{「まずは、防壁を出す。話はそれからだ。」}

\sphinxAtStartPar
\textasciitilde{} 再現の師より \textasciitilde{}
\sphinxstylestrong{導入バージョン:}  0.1

\sphinxAtStartPar
\sphinxstylestrong{更新バージョン:}  8.0


\bigskip\hrule\bigskip



\paragraph{兵士召喚 (召喚)}
\label{\detokenize{auto/actionlist:act-summonssoldier}}\label{\detokenize{auto/actionlist:id13}}
\sphinxAtStartPar
\sphinxstylestrong{フォーマット:}


\begin{savenotes}\sphinxattablestart
\sphinxthistablewithglobalstyle
\centering
\begin{tabular}[t]{|\X{1}{4}|\X{1}{4}|\X{1}{4}|\X{1}{4}|}
\sphinxtoprule
\sphinxstyletheadfamily 
\sphinxAtStartPar
ライト
&\sphinxstyletheadfamily 
\sphinxAtStartPar
スタンダード
&\sphinxstyletheadfamily 
\sphinxAtStartPar
プロ
&\sphinxstyletheadfamily 
\sphinxAtStartPar
マスター
\\
\sphinxmidrule
\sphinxtableatstartofbodyhook
\sphinxAtStartPar
◯
&
\sphinxAtStartPar
◯
&
\sphinxAtStartPar
◯
&
\sphinxAtStartPar
◯
\\
\sphinxbottomrule
\end{tabular}
\sphinxtableafterendhook\par
\sphinxattableend\end{savenotes}

\sphinxAtStartPar
\sphinxstylestrong{トリガー:} 直接

\sphinxAtStartPar
\sphinxstylestrong{スピード:} 通常

\sphinxAtStartPar
\sphinxstylestrong{タイミング:} メイン

\sphinxAtStartPar
\sphinxstylestrong{コスト:} BL

\sphinxAtStartPar
\sphinxstylestrong{キーカード:} 2〜10

\sphinxAtStartPar
\sphinxstylestrong{効果:}

\sphinxAtStartPar
キーカードを一般兵として表向きかつチャージ状態で場に出す。

\sphinxAtStartPar
\sphinxstylestrong{ストーリー:}

\sphinxAtStartPar
兵士をいきない戦場に出すことはできません。防衛拠点に受け入れてもらい、休息をとってから戦場に出ます。当然民の支援も必要です。

\sphinxAtStartPar
\sphinxstylestrong{フレーバー:}

\sphinxAtStartPar
\sphinxstyleemphasis{「防壁の次は兵士を出します。これが戦術の基本です。」}

\sphinxAtStartPar
\textasciitilde{} 再現者の心得第二章より \textasciitilde{}
\sphinxstylestrong{導入バージョン:}  3.0

\sphinxAtStartPar
\sphinxstylestrong{更新バージョン:}  8.0


\bigskip\hrule\bigskip



\paragraph{英雄召喚 (召喚)}
\label{\detokenize{auto/actionlist:act-summonshero}}\label{\detokenize{auto/actionlist:id14}}
\sphinxAtStartPar
\sphinxstylestrong{フォーマット:}


\begin{savenotes}\sphinxattablestart
\sphinxthistablewithglobalstyle
\centering
\begin{tabular}[t]{|\X{1}{4}|\X{1}{4}|\X{1}{4}|\X{1}{4}|}
\sphinxtoprule
\sphinxstyletheadfamily 
\sphinxAtStartPar
ライト
&\sphinxstyletheadfamily 
\sphinxAtStartPar
スタンダード
&\sphinxstyletheadfamily 
\sphinxAtStartPar
プロ
&\sphinxstyletheadfamily 
\sphinxAtStartPar
マスター
\\
\sphinxmidrule
\sphinxtableatstartofbodyhook
\sphinxAtStartPar
◯
&
\sphinxAtStartPar
◯
&
\sphinxAtStartPar
◯
&
\sphinxAtStartPar
◯
\\
\sphinxbottomrule
\end{tabular}
\sphinxtableafterendhook\par
\sphinxattableend\end{savenotes}

\sphinxAtStartPar
\sphinxstylestrong{トリガー:} 直接

\sphinxAtStartPar
\sphinxstylestrong{スピード:} 通常

\sphinxAtStartPar
\sphinxstylestrong{タイミング:} メイン

\sphinxAtStartPar
\sphinxstylestrong{コスト:} BBL

\sphinxAtStartPar
\sphinxstylestrong{キーカード:} J〜K

\sphinxAtStartPar
\sphinxstylestrong{効果:}

\sphinxAtStartPar
キーカードを英雄として表向きかつチャージ状態で場に出す。

\sphinxAtStartPar
\sphinxstylestrong{ストーリー:}

\sphinxAtStartPar
英雄の召喚には一般兵よりコストがかかります。防壁による厳重な警護。十分な休息。そして民の支援です。

\sphinxAtStartPar
\sphinxstylestrong{フレーバー:}

\sphinxAtStartPar
\sphinxstyleemphasis{「実力はわからんが一応英雄様の出陣じゃ。粗相のないようしっかりやれよ。」}

\sphinxAtStartPar
\textasciitilde{} 古参兵より \textasciitilde{}
\sphinxstylestrong{導入バージョン:}  3.0

\sphinxAtStartPar
\sphinxstylestrong{更新バージョン:}  8.0


\bigskip\hrule\bigskip



\paragraph{エース召喚 (召喚)}
\label{\detokenize{auto/actionlist:act-summonsace}}\label{\detokenize{auto/actionlist:id15}}
\sphinxAtStartPar
\sphinxstylestrong{フォーマット:}


\begin{savenotes}\sphinxattablestart
\sphinxthistablewithglobalstyle
\centering
\begin{tabular}[t]{|\X{1}{4}|\X{1}{4}|\X{1}{4}|\X{1}{4}|}
\sphinxtoprule
\sphinxstyletheadfamily 
\sphinxAtStartPar
ライト
&\sphinxstyletheadfamily 
\sphinxAtStartPar
スタンダード
&\sphinxstyletheadfamily 
\sphinxAtStartPar
プロ
&\sphinxstyletheadfamily 
\sphinxAtStartPar
マスター
\\
\sphinxmidrule
\sphinxtableatstartofbodyhook
\sphinxAtStartPar
◯
&
\sphinxAtStartPar
◯
&
\sphinxAtStartPar
◯
&
\sphinxAtStartPar
◯
\\
\sphinxbottomrule
\end{tabular}
\sphinxtableafterendhook\par
\sphinxattableend\end{savenotes}

\sphinxAtStartPar
\sphinxstylestrong{トリガー:} 直接

\sphinxAtStartPar
\sphinxstylestrong{スピード:} 通常

\sphinxAtStartPar
\sphinxstylestrong{タイミング:} メイン

\sphinxAtStartPar
\sphinxstylestrong{コスト:} L

\sphinxAtStartPar
\sphinxstylestrong{キーカード:} A

\sphinxAtStartPar
\sphinxstylestrong{効果:}

\sphinxAtStartPar
キーカードをエースとして表向きかつチャージ状態で場に出す。

\sphinxAtStartPar
\sphinxstylestrong{ストーリー:}

\sphinxAtStartPar
エースは素早く行動を起こします。いきなり戦場に登場し切り込んで行きます。

\sphinxAtStartPar
\sphinxstylestrong{フレーバー:}

\sphinxAtStartPar
\sphinxstyleemphasis{「あいつ!何も持たずに飛び出して行きやがった!」}

\sphinxAtStartPar
\textasciitilde{} 親方より \textasciitilde{}
\sphinxstylestrong{導入バージョン:}  3.0

\sphinxAtStartPar
\sphinxstylestrong{更新バージョン:}  8.0


\bigskip\hrule\bigskip



\paragraph{クイック召喚 (召喚)}
\label{\detokenize{auto/actionlist:act-quicksummonsace}}\label{\detokenize{auto/actionlist:id16}}
\sphinxAtStartPar
\sphinxstylestrong{フォーマット:}


\begin{savenotes}\sphinxattablestart
\sphinxthistablewithglobalstyle
\centering
\begin{tabular}[t]{|\X{1}{4}|\X{1}{4}|\X{1}{4}|\X{1}{4}|}
\sphinxtoprule
\sphinxstyletheadfamily 
\sphinxAtStartPar
ライト
&\sphinxstyletheadfamily 
\sphinxAtStartPar
スタンダード
&\sphinxstyletheadfamily 
\sphinxAtStartPar
プロ
&\sphinxstyletheadfamily 
\sphinxAtStartPar
マスター
\\
\sphinxmidrule
\sphinxtableatstartofbodyhook
\sphinxAtStartPar
x
&
\sphinxAtStartPar
x
&
\sphinxAtStartPar
◯
&
\sphinxAtStartPar
◯
\\
\sphinxbottomrule
\end{tabular}
\sphinxtableafterendhook\par
\sphinxattableend\end{savenotes}

\sphinxAtStartPar
\sphinxstylestrong{トリガー:} 直接

\sphinxAtStartPar
\sphinxstylestrong{スピード:} 通常

\sphinxAtStartPar
\sphinxstylestrong{タイミング:} クイック

\sphinxAtStartPar
\sphinxstylestrong{コスト:} D

\sphinxAtStartPar
\sphinxstylestrong{キーカード:} A

\sphinxAtStartPar
\sphinxstylestrong{起動条件:}

\sphinxAtStartPar
ターンを持っていない時しかこのアクションを起こすことができない。

\sphinxAtStartPar
\sphinxstylestrong{効果:}

\sphinxAtStartPar
キーカードをエースとして表向きかつチャージ状態で場に出す。

\sphinxAtStartPar
もしくは、キーカードを防壁として裏向きかつチャージ状態で場に出す。

\sphinxAtStartPar
防壁の置き方は「防壁の置き方」参照。

\sphinxAtStartPar
\sphinxstylestrong{導入バージョン:}  6.0

\sphinxAtStartPar
\sphinxstylestrong{更新バージョン:}  8.0


\bigskip\hrule\bigskip



\paragraph{魔術士召喚 (召喚)}
\label{\detokenize{auto/actionlist:act-summonsmagic}}\label{\detokenize{auto/actionlist:id17}}
\sphinxAtStartPar
\sphinxstylestrong{フォーマット:}


\begin{savenotes}\sphinxattablestart
\sphinxthistablewithglobalstyle
\centering
\begin{tabular}[t]{|\X{1}{4}|\X{1}{4}|\X{1}{4}|\X{1}{4}|}
\sphinxtoprule
\sphinxstyletheadfamily 
\sphinxAtStartPar
ライト
&\sphinxstyletheadfamily 
\sphinxAtStartPar
スタンダード
&\sphinxstyletheadfamily 
\sphinxAtStartPar
プロ
&\sphinxstyletheadfamily 
\sphinxAtStartPar
マスター
\\
\sphinxmidrule
\sphinxtableatstartofbodyhook
\sphinxAtStartPar
x
&
\sphinxAtStartPar
◯
&
\sphinxAtStartPar
◯
&
\sphinxAtStartPar
◯
\\
\sphinxbottomrule
\end{tabular}
\sphinxtableafterendhook\par
\sphinxattableend\end{savenotes}

\sphinxAtStartPar
\sphinxstylestrong{トリガー:} 直接

\sphinxAtStartPar
\sphinxstylestrong{スピード:} 通常

\sphinxAtStartPar
\sphinxstylestrong{タイミング:} メイン

\sphinxAtStartPar
\sphinxstylestrong{コスト:} BD

\sphinxAtStartPar
\sphinxstylestrong{キーカード:} Joker

\sphinxAtStartPar
\sphinxstylestrong{効果:}

\sphinxAtStartPar
キーカードを魔術士として表向きかつチャージ状態で場に出す。魔術士の能力は、キャラクターリスト参照。

\sphinxAtStartPar
\sphinxstylestrong{ストーリー:}

\sphinxAtStartPar
魔法使いは戦場では魔術士として戦いに臨みます。
コストとして民の支援より手元の予定を狂わせることを要求するあたり捻くれ者であることが伺えます。

\sphinxAtStartPar
\sphinxstylestrong{フレーバー:}

\sphinxAtStartPar
\sphinxstyleemphasis{「場に出れば強いんだよなぁ。場に出れば。」}

\sphinxAtStartPar
\textasciitilde{} 再現者見習いより \textasciitilde{}
\sphinxstylestrong{導入バージョン:}  3.0

\sphinxAtStartPar
\sphinxstylestrong{更新バージョン:}  8.0


\bigskip\hrule\bigskip



\paragraph{装備 (召喚)}
\label{\detokenize{auto/actionlist:act-mountsoldier}}\label{\detokenize{auto/actionlist:id18}}
\sphinxAtStartPar
\sphinxstylestrong{フォーマット:}


\begin{savenotes}\sphinxattablestart
\sphinxthistablewithglobalstyle
\centering
\begin{tabular}[t]{|\X{1}{4}|\X{1}{4}|\X{1}{4}|\X{1}{4}|}
\sphinxtoprule
\sphinxstyletheadfamily 
\sphinxAtStartPar
ライト
&\sphinxstyletheadfamily 
\sphinxAtStartPar
スタンダード
&\sphinxstyletheadfamily 
\sphinxAtStartPar
プロ
&\sphinxstyletheadfamily 
\sphinxAtStartPar
マスター
\\
\sphinxmidrule
\sphinxtableatstartofbodyhook
\sphinxAtStartPar
◯
&
\sphinxAtStartPar
◯
&
\sphinxAtStartPar
◯
&
\sphinxAtStartPar
◯
\\
\sphinxbottomrule
\end{tabular}
\sphinxtableafterendhook\par
\sphinxattableend\end{savenotes}

\sphinxAtStartPar
\sphinxstylestrong{トリガー:} 直接

\sphinxAtStartPar
\sphinxstylestrong{スピード:} 通常

\sphinxAtStartPar
\sphinxstylestrong{タイミング:} メイン

\sphinxAtStartPar
\sphinxstylestrong{コスト:} BL

\sphinxAtStartPar
\sphinxstylestrong{キーカード:} A〜K

\sphinxAtStartPar
\sphinxstylestrong{対象:}

\sphinxAtStartPar
自分の場にいるキーカードと同じスートの兵士1体を対象とする。Jokerは対象にできない。

\sphinxAtStartPar
\sphinxstylestrong{効果:}

\sphinxAtStartPar
対象とした兵士の上にキーカードを置き装備兵とする。装備兵の能力は、キャラクターリスト参照。

\sphinxAtStartPar
\sphinxstylestrong{ストーリー:}

\sphinxAtStartPar
兵士を強化することができます。強化した分強くなるのですが、弱点も増えるので要注意です。

\sphinxAtStartPar
\sphinxstylestrong{フレーバー:}

\sphinxAtStartPar
\sphinxstyleemphasis{「こんなにゴテゴテしてなんになるつもりだ?結局一人だろ?」}

\sphinxAtStartPar
\textasciitilde{} ライフよりの防壁より \textasciitilde{}
\sphinxstylestrong{導入バージョン:}  0.1

\sphinxAtStartPar
\sphinxstylestrong{更新バージョン:}  8.0


\subsubsection{速攻魔法}
\label{\detokenize{auto/actionlist:id19}}

\paragraph{アップ (速攻魔法)}
\label{\detokenize{auto/actionlist:act-up}}\label{\detokenize{auto/actionlist:id20}}
\sphinxAtStartPar
\sphinxstylestrong{フォーマット:}


\begin{savenotes}\sphinxattablestart
\sphinxthistablewithglobalstyle
\centering
\begin{tabular}[t]{|\X{1}{4}|\X{1}{4}|\X{1}{4}|\X{1}{4}|}
\sphinxtoprule
\sphinxstyletheadfamily 
\sphinxAtStartPar
ライト
&\sphinxstyletheadfamily 
\sphinxAtStartPar
スタンダード
&\sphinxstyletheadfamily 
\sphinxAtStartPar
プロ
&\sphinxstyletheadfamily 
\sphinxAtStartPar
マスター
\\
\sphinxmidrule
\sphinxtableatstartofbodyhook
\sphinxAtStartPar
◯
&
\sphinxAtStartPar
◯
&
\sphinxAtStartPar
◯
&
\sphinxAtStartPar
◯
\\
\sphinxbottomrule
\end{tabular}
\sphinxtableafterendhook\par
\sphinxattableend\end{savenotes}

\sphinxAtStartPar
\sphinxstylestrong{トリガー:} 直接

\sphinxAtStartPar
\sphinxstylestrong{スピード:} 通常

\sphinxAtStartPar
\sphinxstylestrong{タイミング:} クイック

\sphinxAtStartPar
\sphinxstylestrong{コスト:} D

\sphinxAtStartPar
\sphinxstylestrong{キーカード:} {\normalsize $\heartsuit$} A〜10

\sphinxAtStartPar
\sphinxstylestrong{対象:}

\sphinxAtStartPar
兵士1体を対象とする。

\sphinxAtStartPar
\sphinxstylestrong{効果:}

\sphinxAtStartPar
対象とした兵士のサイズは、このターンが終わるまでキーカードの数字分加算される。その印としてフォグにキーカードを置き、アップとする。

\sphinxAtStartPar
\sphinxstylestrong{導入バージョン:}  0.1

\sphinxAtStartPar
\sphinxstylestrong{更新バージョン:}  8.0


\bigskip\hrule\bigskip



\paragraph{ダウン (速攻魔法)}
\label{\detokenize{auto/actionlist:act-down}}\label{\detokenize{auto/actionlist:id21}}
\sphinxAtStartPar
\sphinxstylestrong{フォーマット:}


\begin{savenotes}\sphinxattablestart
\sphinxthistablewithglobalstyle
\centering
\begin{tabular}[t]{|\X{1}{4}|\X{1}{4}|\X{1}{4}|\X{1}{4}|}
\sphinxtoprule
\sphinxstyletheadfamily 
\sphinxAtStartPar
ライト
&\sphinxstyletheadfamily 
\sphinxAtStartPar
スタンダード
&\sphinxstyletheadfamily 
\sphinxAtStartPar
プロ
&\sphinxstyletheadfamily 
\sphinxAtStartPar
マスター
\\
\sphinxmidrule
\sphinxtableatstartofbodyhook
\sphinxAtStartPar
◯
&
\sphinxAtStartPar
◯
&
\sphinxAtStartPar
◯
&
\sphinxAtStartPar
◯
\\
\sphinxbottomrule
\end{tabular}
\sphinxtableafterendhook\par
\sphinxattableend\end{savenotes}

\sphinxAtStartPar
\sphinxstylestrong{トリガー:} 直接

\sphinxAtStartPar
\sphinxstylestrong{スピード:} 通常

\sphinxAtStartPar
\sphinxstylestrong{タイミング:} クイック

\sphinxAtStartPar
\sphinxstylestrong{コスト:} D

\sphinxAtStartPar
\sphinxstylestrong{キーカード:} {\normalsize $\spadesuit$} A〜10

\sphinxAtStartPar
\sphinxstylestrong{対象:}

\sphinxAtStartPar
兵士1体を対象とする。

\sphinxAtStartPar
\sphinxstylestrong{効果:}

\sphinxAtStartPar
対象とした兵士のサイズは、このターンが終わるまでキーカードの数字分減算される。

\sphinxAtStartPar
もし対象のサイズが0以下となった場合、対象を墓地に移す。

\sphinxAtStartPar
もし対象のサイズが0以下とならなかった場合、その印としてフォグにキーカードを置き、ダウンとする。

\sphinxAtStartPar
\sphinxstylestrong{導入バージョン:}  0.1

\sphinxAtStartPar
\sphinxstylestrong{更新バージョン:}  8.0


\bigskip\hrule\bigskip



\paragraph{ツイスト (速攻魔法)}
\label{\detokenize{auto/actionlist:act-twist}}\label{\detokenize{auto/actionlist:id22}}
\sphinxAtStartPar
\sphinxstylestrong{フォーマット:}


\begin{savenotes}\sphinxattablestart
\sphinxthistablewithglobalstyle
\centering
\begin{tabular}[t]{|\X{1}{4}|\X{1}{4}|\X{1}{4}|\X{1}{4}|}
\sphinxtoprule
\sphinxstyletheadfamily 
\sphinxAtStartPar
ライト
&\sphinxstyletheadfamily 
\sphinxAtStartPar
スタンダード
&\sphinxstyletheadfamily 
\sphinxAtStartPar
プロ
&\sphinxstyletheadfamily 
\sphinxAtStartPar
マスター
\\
\sphinxmidrule
\sphinxtableatstartofbodyhook
\sphinxAtStartPar
◯
&
\sphinxAtStartPar
◯
&
\sphinxAtStartPar
◯
&
\sphinxAtStartPar
◯
\\
\sphinxbottomrule
\end{tabular}
\sphinxtableafterendhook\par
\sphinxattableend\end{savenotes}

\sphinxAtStartPar
\sphinxstylestrong{トリガー:} 直接

\sphinxAtStartPar
\sphinxstylestrong{スピード:} 通常

\sphinxAtStartPar
\sphinxstylestrong{タイミング:} クイック

\sphinxAtStartPar
\sphinxstylestrong{コスト:} D

\sphinxAtStartPar
\sphinxstylestrong{キーカード:} {\normalsize $\diamondsuit$} A〜10

\sphinxAtStartPar
\sphinxstylestrong{対象:}

\sphinxAtStartPar
キャラクター1体を対象とする。

\sphinxAtStartPar
\sphinxstylestrong{効果:}

\sphinxAtStartPar
対象のキャラクターをドライブ状態またはチャージ状態にする。

\sphinxAtStartPar
\sphinxstylestrong{導入バージョン:}  0.2

\sphinxAtStartPar
\sphinxstylestrong{更新バージョン:}  8.0


\bigskip\hrule\bigskip



\paragraph{カウンター (速攻魔法)}
\label{\detokenize{auto/actionlist:act-counter}}\label{\detokenize{auto/actionlist:id23}}
\sphinxAtStartPar
\sphinxstylestrong{フォーマット:}


\begin{savenotes}\sphinxattablestart
\sphinxthistablewithglobalstyle
\centering
\begin{tabular}[t]{|\X{1}{4}|\X{1}{4}|\X{1}{4}|\X{1}{4}|}
\sphinxtoprule
\sphinxstyletheadfamily 
\sphinxAtStartPar
ライト
&\sphinxstyletheadfamily 
\sphinxAtStartPar
スタンダード
&\sphinxstyletheadfamily 
\sphinxAtStartPar
プロ
&\sphinxstyletheadfamily 
\sphinxAtStartPar
マスター
\\
\sphinxmidrule
\sphinxtableatstartofbodyhook
\sphinxAtStartPar
◯
&
\sphinxAtStartPar
◯
&
\sphinxAtStartPar
◯
&
\sphinxAtStartPar
◯
\\
\sphinxbottomrule
\end{tabular}
\sphinxtableafterendhook\par
\sphinxattableend\end{savenotes}

\sphinxAtStartPar
\sphinxstylestrong{トリガー:} 直接

\sphinxAtStartPar
\sphinxstylestrong{スピード:} 通常

\sphinxAtStartPar
\sphinxstylestrong{タイミング:} クイック

\sphinxAtStartPar
\sphinxstylestrong{コスト:} D

\sphinxAtStartPar
\sphinxstylestrong{キーカード:} {\normalsize $\clubsuit$} A〜10

\sphinxAtStartPar
\sphinxstylestrong{対象:}

\sphinxAtStartPar
キーカードが1枚または2枚のアクションを対象とする。

\sphinxAtStartPar
\sphinxstylestrong{効果:}

\sphinxAtStartPar
次のいずれかの場合、対象のアクションを無効にする。その場合、対象アクションをステージから取り除き、対象アクションのキーカードを墓地に移す。

\sphinxAtStartPar
・対象アクションのキーカードが1枚かつこのアクションのキーカードの数字が対象アクションのキーカードの数字以上

\sphinxAtStartPar
・対象アクションのキーカードが2枚

\sphinxAtStartPar
\sphinxstylestrong{導入バージョン:}  0.1

\sphinxAtStartPar
\sphinxstylestrong{更新バージョン:}  8.0


\bigskip\hrule\bigskip



\paragraph{再会 (速攻魔法)}
\label{\detokenize{auto/actionlist:act-reunion}}\label{\detokenize{auto/actionlist:id24}}
\sphinxAtStartPar
\sphinxstylestrong{フォーマット:}


\begin{savenotes}\sphinxattablestart
\sphinxthistablewithglobalstyle
\centering
\begin{tabular}[t]{|\X{1}{4}|\X{1}{4}|\X{1}{4}|\X{1}{4}|}
\sphinxtoprule
\sphinxstyletheadfamily 
\sphinxAtStartPar
ライト
&\sphinxstyletheadfamily 
\sphinxAtStartPar
スタンダード
&\sphinxstyletheadfamily 
\sphinxAtStartPar
プロ
&\sphinxstyletheadfamily 
\sphinxAtStartPar
マスター
\\
\sphinxmidrule
\sphinxtableatstartofbodyhook
\sphinxAtStartPar
x
&
\sphinxAtStartPar
x
&
\sphinxAtStartPar
◯
&
\sphinxAtStartPar
◯
\\
\sphinxbottomrule
\end{tabular}
\sphinxtableafterendhook\par
\sphinxattableend\end{savenotes}

\sphinxAtStartPar
\sphinxstylestrong{トリガー:} 直接

\sphinxAtStartPar
\sphinxstylestrong{スピード:} 通常

\sphinxAtStartPar
\sphinxstylestrong{タイミング:} クイック

\sphinxAtStartPar
\sphinxstylestrong{キーカード:} {\normalsize $\heartsuit$} A〜10 を 2枚

\sphinxAtStartPar
\sphinxstylestrong{効果:}

\sphinxAtStartPar
自分の墓地からカードを1枚選び対戦相手に見せ手札に加える。

\sphinxAtStartPar
\sphinxstylestrong{ストーリー:}

\sphinxAtStartPar
親愛な仲間との再会を強く強く願った。どんな手段も厭わないその願いが今叶った。

\sphinxAtStartPar
\sphinxstylestrong{フレーバー:}

\sphinxAtStartPar
\sphinxstyleemphasis{「ねぇねぇ知ってた。この後ヒロインの願いが届いて死んだはずの主人公が登場して、二人で敵を全員やっつけちゃうの!ありがちよねぇ。」}

\sphinxAtStartPar
\textasciitilde{} ネタバレ娘より \textasciitilde{}
\sphinxstylestrong{導入バージョン:}  1.0

\sphinxAtStartPar
\sphinxstylestrong{更新バージョン:}  8.0


\bigskip\hrule\bigskip



\paragraph{キル (速攻魔法)}
\label{\detokenize{auto/actionlist:act-kill}}\label{\detokenize{auto/actionlist:id25}}
\sphinxAtStartPar
\sphinxstylestrong{フォーマット:}


\begin{savenotes}\sphinxattablestart
\sphinxthistablewithglobalstyle
\centering
\begin{tabular}[t]{|\X{1}{4}|\X{1}{4}|\X{1}{4}|\X{1}{4}|}
\sphinxtoprule
\sphinxstyletheadfamily 
\sphinxAtStartPar
ライト
&\sphinxstyletheadfamily 
\sphinxAtStartPar
スタンダード
&\sphinxstyletheadfamily 
\sphinxAtStartPar
プロ
&\sphinxstyletheadfamily 
\sphinxAtStartPar
マスター
\\
\sphinxmidrule
\sphinxtableatstartofbodyhook
\sphinxAtStartPar
x
&
\sphinxAtStartPar
x
&
\sphinxAtStartPar
◯
&
\sphinxAtStartPar
◯
\\
\sphinxbottomrule
\end{tabular}
\sphinxtableafterendhook\par
\sphinxattableend\end{savenotes}

\sphinxAtStartPar
\sphinxstylestrong{トリガー:} 直接

\sphinxAtStartPar
\sphinxstylestrong{スピード:} 通常

\sphinxAtStartPar
\sphinxstylestrong{タイミング:} クイック

\sphinxAtStartPar
\sphinxstylestrong{キーカード:} {\normalsize $\spadesuit$} A〜10 を 2枚

\sphinxAtStartPar
\sphinxstylestrong{対象:}

\sphinxAtStartPar
兵士1体を対象とする。

\sphinxAtStartPar
\sphinxstylestrong{効果:}

\sphinxAtStartPar
対象とした兵士を墓地に移す。

\sphinxAtStartPar
\sphinxstylestrong{導入バージョン:}  1.0

\sphinxAtStartPar
\sphinxstylestrong{更新バージョン:}  8.0


\bigskip\hrule\bigskip



\paragraph{停戦 (速攻魔法)}
\label{\detokenize{auto/actionlist:act-truce}}\label{\detokenize{auto/actionlist:id26}}
\sphinxAtStartPar
\sphinxstylestrong{フォーマット:}


\begin{savenotes}\sphinxattablestart
\sphinxthistablewithglobalstyle
\centering
\begin{tabular}[t]{|\X{1}{4}|\X{1}{4}|\X{1}{4}|\X{1}{4}|}
\sphinxtoprule
\sphinxstyletheadfamily 
\sphinxAtStartPar
ライト
&\sphinxstyletheadfamily 
\sphinxAtStartPar
スタンダード
&\sphinxstyletheadfamily 
\sphinxAtStartPar
プロ
&\sphinxstyletheadfamily 
\sphinxAtStartPar
マスター
\\
\sphinxmidrule
\sphinxtableatstartofbodyhook
\sphinxAtStartPar
x
&
\sphinxAtStartPar
x
&
\sphinxAtStartPar
◯
&
\sphinxAtStartPar
◯
\\
\sphinxbottomrule
\end{tabular}
\sphinxtableafterendhook\par
\sphinxattableend\end{savenotes}

\sphinxAtStartPar
\sphinxstylestrong{トリガー:} 直接

\sphinxAtStartPar
\sphinxstylestrong{スピード:} 通常

\sphinxAtStartPar
\sphinxstylestrong{タイミング:} クイック

\sphinxAtStartPar
\sphinxstylestrong{キーカード:} {\normalsize $\diamondsuit$} A〜10 を 2枚

\sphinxAtStartPar
\sphinxstylestrong{対象:}

\sphinxAtStartPar
ダメージ判定アクションを対象とする。

\sphinxAtStartPar
\sphinxstylestrong{効果:}

\sphinxAtStartPar
対象のアクションを無効にする。対象アクションをステージから取り除き、対象アクションのキーカードを墓地に移す。

\sphinxAtStartPar
\sphinxstylestrong{導入バージョン:}  1.0

\sphinxAtStartPar
\sphinxstylestrong{更新バージョン:}  8.0


\bigskip\hrule\bigskip



\paragraph{対象変更 (速攻魔法)}
\label{\detokenize{auto/actionlist:act-changetarget}}\label{\detokenize{auto/actionlist:id27}}
\sphinxAtStartPar
\sphinxstylestrong{フォーマット:}


\begin{savenotes}\sphinxattablestart
\sphinxthistablewithglobalstyle
\centering
\begin{tabular}[t]{|\X{1}{4}|\X{1}{4}|\X{1}{4}|\X{1}{4}|}
\sphinxtoprule
\sphinxstyletheadfamily 
\sphinxAtStartPar
ライト
&\sphinxstyletheadfamily 
\sphinxAtStartPar
スタンダード
&\sphinxstyletheadfamily 
\sphinxAtStartPar
プロ
&\sphinxstyletheadfamily 
\sphinxAtStartPar
マスター
\\
\sphinxmidrule
\sphinxtableatstartofbodyhook
\sphinxAtStartPar
x
&
\sphinxAtStartPar
x
&
\sphinxAtStartPar
◯
&
\sphinxAtStartPar
◯
\\
\sphinxbottomrule
\end{tabular}
\sphinxtableafterendhook\par
\sphinxattableend\end{savenotes}

\sphinxAtStartPar
\sphinxstylestrong{トリガー:} 直接

\sphinxAtStartPar
\sphinxstylestrong{スピード:} 通常

\sphinxAtStartPar
\sphinxstylestrong{タイミング:} クイック

\sphinxAtStartPar
\sphinxstylestrong{キーカード:} {\normalsize $\clubsuit$} A〜10 を 2枚

\sphinxAtStartPar
\sphinxstylestrong{対象:}

\sphinxAtStartPar
対象が指定されているアクションを対象とする。

\sphinxAtStartPar
\sphinxstylestrong{効果:}

\sphinxAtStartPar
対象のアクションで指定されている対象をそのアクションが指定できる範囲で変更する。アクションを対象とするアクションの対象を対象変更に変更することは可能、アクションの対象にできないものへの対象の変更は不可能とする。

\sphinxAtStartPar
\sphinxstylestrong{導入バージョン:}  1.0

\sphinxAtStartPar
\sphinxstylestrong{更新バージョン:}  8.0


\bigskip\hrule\bigskip



\paragraph{サーチ (速攻魔法)}
\label{\detokenize{auto/actionlist:act-search}}\label{\detokenize{auto/actionlist:id28}}
\sphinxAtStartPar
\sphinxstylestrong{フォーマット:}


\begin{savenotes}\sphinxattablestart
\sphinxthistablewithglobalstyle
\centering
\begin{tabular}[t]{|\X{1}{4}|\X{1}{4}|\X{1}{4}|\X{1}{4}|}
\sphinxtoprule
\sphinxstyletheadfamily 
\sphinxAtStartPar
ライト
&\sphinxstyletheadfamily 
\sphinxAtStartPar
スタンダード
&\sphinxstyletheadfamily 
\sphinxAtStartPar
プロ
&\sphinxstyletheadfamily 
\sphinxAtStartPar
マスター
\\
\sphinxmidrule
\sphinxtableatstartofbodyhook
\sphinxAtStartPar
◯
&
\sphinxAtStartPar
◯
&
\sphinxAtStartPar
◯
&
\sphinxAtStartPar
◯
\\
\sphinxbottomrule
\end{tabular}
\sphinxtableafterendhook\par
\sphinxattableend\end{savenotes}

\sphinxAtStartPar
\sphinxstylestrong{トリガー:} 直接

\sphinxAtStartPar
\sphinxstylestrong{スピード:} 即時

\sphinxAtStartPar
\sphinxstylestrong{タイミング:} クイック

\sphinxAtStartPar
\sphinxstylestrong{キーカード:} Joker

\sphinxAtStartPar
\sphinxstylestrong{効果:}

\sphinxAtStartPar
ライフから好きなカードを1枚選び対戦相手に見せ手札に加える。その後ライフを切りなおす。

\sphinxAtStartPar
\sphinxstylestrong{導入バージョン:}  0.1

\sphinxAtStartPar
\sphinxstylestrong{更新バージョン:}  8.0


\bigskip\hrule\bigskip



\paragraph{B・J (速攻魔法)}
\label{\detokenize{auto/actionlist:bj}}\label{\detokenize{auto/actionlist:act-bj}}
\sphinxAtStartPar
\sphinxstylestrong{フォーマット:}


\begin{savenotes}\sphinxattablestart
\sphinxthistablewithglobalstyle
\centering
\begin{tabular}[t]{|\X{1}{4}|\X{1}{4}|\X{1}{4}|\X{1}{4}|}
\sphinxtoprule
\sphinxstyletheadfamily 
\sphinxAtStartPar
ライト
&\sphinxstyletheadfamily 
\sphinxAtStartPar
スタンダード
&\sphinxstyletheadfamily 
\sphinxAtStartPar
プロ
&\sphinxstyletheadfamily 
\sphinxAtStartPar
マスター
\\
\sphinxmidrule
\sphinxtableatstartofbodyhook
\sphinxAtStartPar
x
&
\sphinxAtStartPar
x
&
\sphinxAtStartPar
x
&
\sphinxAtStartPar
◯
\\
\sphinxbottomrule
\end{tabular}
\sphinxtableafterendhook\par
\sphinxattableend\end{savenotes}

\sphinxAtStartPar
\sphinxstylestrong{トリガー:} 直接

\sphinxAtStartPar
\sphinxstylestrong{スピード:} 通常

\sphinxAtStartPar
\sphinxstylestrong{タイミング:} クイック

\sphinxAtStartPar
\sphinxstylestrong{コスト:} SS

\sphinxAtStartPar
\sphinxstylestrong{キーカード:} 同じスートの A と J

\sphinxAtStartPar
\sphinxstylestrong{効果:}

\sphinxAtStartPar
ライフから好きなカードを2枚まで選び対戦相手に見せ手札に加える。その後ライフを切りなおす。

\sphinxAtStartPar
\sphinxstylestrong{ストーリー:}

\sphinxAtStartPar
若きジャックはエースの力をかり革命的な行動を起こした。自国の膿を排除し新しい風を送り込んだ。

\sphinxAtStartPar
\sphinxstylestrong{フレーバー:}

\sphinxAtStartPar
\sphinxstyleemphasis{「悪しき習慣を捨て去り、新しい風を吹き込む」}

\sphinxAtStartPar
\textasciitilde{} 革命者より \textasciitilde{}
\sphinxstylestrong{導入バージョン:}  1.0

\sphinxAtStartPar
\sphinxstylestrong{更新バージョン:}  8.0


\bigskip\hrule\bigskip



\paragraph{R・S・F (速攻魔法)}
\label{\detokenize{auto/actionlist:rsf}}\label{\detokenize{auto/actionlist:act-rsf}}
\sphinxAtStartPar
\sphinxstylestrong{フォーマット:}


\begin{savenotes}\sphinxattablestart
\sphinxthistablewithglobalstyle
\centering
\begin{tabular}[t]{|\X{1}{4}|\X{1}{4}|\X{1}{4}|\X{1}{4}|}
\sphinxtoprule
\sphinxstyletheadfamily 
\sphinxAtStartPar
ライト
&\sphinxstyletheadfamily 
\sphinxAtStartPar
スタンダード
&\sphinxstyletheadfamily 
\sphinxAtStartPar
プロ
&\sphinxstyletheadfamily 
\sphinxAtStartPar
マスター
\\
\sphinxmidrule
\sphinxtableatstartofbodyhook
\sphinxAtStartPar
x
&
\sphinxAtStartPar
x
&
\sphinxAtStartPar
x
&
\sphinxAtStartPar
◯
\\
\sphinxbottomrule
\end{tabular}
\sphinxtableafterendhook\par
\sphinxattableend\end{savenotes}

\sphinxAtStartPar
\sphinxstylestrong{トリガー:} 直接

\sphinxAtStartPar
\sphinxstylestrong{スピード:} 通常

\sphinxAtStartPar
\sphinxstylestrong{タイミング:} クイック

\sphinxAtStartPar
\sphinxstylestrong{コスト:} BB

\sphinxAtStartPar
\sphinxstylestrong{キーカード:} 同じスートのA,10〜K を 5枚

\sphinxAtStartPar
\sphinxstylestrong{特記事項:} ※このアクションはカウンターアクションの対象にならない。

\sphinxAtStartPar
\sphinxstylestrong{対象:}

\sphinxAtStartPar
プレイヤー1人を対象とする。

\sphinxAtStartPar
\sphinxstylestrong{効果:}

\sphinxAtStartPar
対象のプレイヤーに40点のダメージを与える。

\sphinxAtStartPar
\sphinxstylestrong{導入バージョン:}  1.0

\sphinxAtStartPar
\sphinxstylestrong{更新バージョン:}  8.0


\bigskip\hrule\bigskip



\paragraph{リバース (速攻魔法)}
\label{\detokenize{auto/actionlist:act-reverse}}\label{\detokenize{auto/actionlist:id29}}
\sphinxAtStartPar
\sphinxstylestrong{フォーマット:}


\begin{savenotes}\sphinxattablestart
\sphinxthistablewithglobalstyle
\centering
\begin{tabular}[t]{|\X{1}{4}|\X{1}{4}|\X{1}{4}|\X{1}{4}|}
\sphinxtoprule
\sphinxstyletheadfamily 
\sphinxAtStartPar
ライト
&\sphinxstyletheadfamily 
\sphinxAtStartPar
スタンダード
&\sphinxstyletheadfamily 
\sphinxAtStartPar
プロ
&\sphinxstyletheadfamily 
\sphinxAtStartPar
マスター
\\
\sphinxmidrule
\sphinxtableatstartofbodyhook
\sphinxAtStartPar
x
&
\sphinxAtStartPar
x
&
\sphinxAtStartPar
◯
&
\sphinxAtStartPar
◯
\\
\sphinxbottomrule
\end{tabular}
\sphinxtableafterendhook\par
\sphinxattableend\end{savenotes}

\sphinxAtStartPar
\sphinxstylestrong{トリガー:} 直接

\sphinxAtStartPar
\sphinxstylestrong{スピード:} 通常

\sphinxAtStartPar
\sphinxstylestrong{タイミング:} クイック

\sphinxAtStartPar
\sphinxstylestrong{キーカード:} 同じ数字を2枚

\sphinxAtStartPar
\sphinxstylestrong{対象:}

\sphinxAtStartPar
キャラクター1体を対象とする。

\sphinxAtStartPar
\sphinxstylestrong{効果:}
\begin{enumerate}
\sphinxsetlistlabels{\arabic}{enumi}{enumii}{}{.}%
\item {} 
\sphinxAtStartPar
必要であれば、対象のキャラクターをチャージ状態またはドライブ状態にする。

\item {} 
\sphinxAtStartPar
対象が兵士の場合、兵士を防壁にする。兵士の時に受けた効果、能力は無くなる。兵士が複数のカードから成る場合、1枚ずつ防壁にする。防壁の置き方は「防壁の置き方」参照

\item {} 
\sphinxAtStartPar
対象が防壁の場合、防壁を兵士にする。防壁の時に受けた効果、能力は無くなる。このターンに場に出た防壁を兵士にする場合、その兵士はこのターンに出た兵士と同様に扱う。

\item {} 
\sphinxAtStartPar
対象がアタッカーもしくは、ブロッカーの場合、それを解除する。

\end{enumerate}

\sphinxAtStartPar
\sphinxstylestrong{導入バージョン:}  0.1

\sphinxAtStartPar
\sphinxstylestrong{更新バージョン:}  8.0


\bigskip\hrule\bigskip



\paragraph{帰還 (速攻魔法)}
\label{\detokenize{auto/actionlist:act-unsummons}}\label{\detokenize{auto/actionlist:id30}}
\sphinxAtStartPar
\sphinxstylestrong{フォーマット:}


\begin{savenotes}\sphinxattablestart
\sphinxthistablewithglobalstyle
\centering
\begin{tabular}[t]{|\X{1}{4}|\X{1}{4}|\X{1}{4}|\X{1}{4}|}
\sphinxtoprule
\sphinxstyletheadfamily 
\sphinxAtStartPar
ライト
&\sphinxstyletheadfamily 
\sphinxAtStartPar
スタンダード
&\sphinxstyletheadfamily 
\sphinxAtStartPar
プロ
&\sphinxstyletheadfamily 
\sphinxAtStartPar
マスター
\\
\sphinxmidrule
\sphinxtableatstartofbodyhook
\sphinxAtStartPar
x
&
\sphinxAtStartPar
◯
&
\sphinxAtStartPar
◯
&
\sphinxAtStartPar
◯
\\
\sphinxbottomrule
\end{tabular}
\sphinxtableafterendhook\par
\sphinxattableend\end{savenotes}

\sphinxAtStartPar
\sphinxstylestrong{トリガー:} 直接

\sphinxAtStartPar
\sphinxstylestrong{スピード:} 通常

\sphinxAtStartPar
\sphinxstylestrong{タイミング:} クイック

\sphinxAtStartPar
\sphinxstylestrong{コスト:} B

\sphinxAtStartPar
\sphinxstylestrong{キーカード:} 同じスートを2枚

\sphinxAtStartPar
\sphinxstylestrong{対象:}

\sphinxAtStartPar
自分の場のキャラクター1体を対象とする。

\sphinxAtStartPar
\sphinxstylestrong{効果:}
\begin{enumerate}
\sphinxsetlistlabels{\arabic}{enumi}{enumii}{}{.}%
\item {} 
\sphinxAtStartPar
対象のキャラクターがチャージ状態の場合、対象のキャラクターを手札に戻す。

\item {} 
\sphinxAtStartPar
キーカードを手札に戻す。

\end{enumerate}

\sphinxAtStartPar
\sphinxstylestrong{導入バージョン:}  1.0

\sphinxAtStartPar
\sphinxstylestrong{更新バージョン:}  8.0


\subsubsection{通常魔法}
\label{\detokenize{auto/actionlist:id31}}

\paragraph{防壁破壊 (通常魔法)}
\label{\detokenize{auto/actionlist:act-destroybulwark}}\label{\detokenize{auto/actionlist:id32}}
\sphinxAtStartPar
\sphinxstylestrong{フォーマット:}


\begin{savenotes}\sphinxattablestart
\sphinxthistablewithglobalstyle
\centering
\begin{tabular}[t]{|\X{1}{4}|\X{1}{4}|\X{1}{4}|\X{1}{4}|}
\sphinxtoprule
\sphinxstyletheadfamily 
\sphinxAtStartPar
ライト
&\sphinxstyletheadfamily 
\sphinxAtStartPar
スタンダード
&\sphinxstyletheadfamily 
\sphinxAtStartPar
プロ
&\sphinxstyletheadfamily 
\sphinxAtStartPar
マスター
\\
\sphinxmidrule
\sphinxtableatstartofbodyhook
\sphinxAtStartPar
◯
&
\sphinxAtStartPar
◯
&
\sphinxAtStartPar
◯
&
\sphinxAtStartPar
◯
\\
\sphinxbottomrule
\end{tabular}
\sphinxtableafterendhook\par
\sphinxattableend\end{savenotes}

\sphinxAtStartPar
\sphinxstylestrong{トリガー:} 直接

\sphinxAtStartPar
\sphinxstylestrong{スピード:} 通常

\sphinxAtStartPar
\sphinxstylestrong{タイミング:} メイン

\sphinxAtStartPar
\sphinxstylestrong{キーカード:} {\normalsize $\heartsuit$} A〜K と {\normalsize $\diamondsuit$} A〜K

\sphinxAtStartPar
\sphinxstylestrong{対象:}

\sphinxAtStartPar
防壁1体を対象とする。

\sphinxAtStartPar
\sphinxstylestrong{効果:}

\sphinxAtStartPar
対象の防壁を場から墓地に移す。

\sphinxAtStartPar
\sphinxstylestrong{ストーリー:}

\sphinxAtStartPar
防壁はとても後ろ向きで心が弱い。愛{\normalsize $\heartsuit$} と金{\normalsize $\diamondsuit$} に目移りして簡単に籠絡してしまうことも。

\sphinxAtStartPar
\sphinxstylestrong{フレーバー:}

\sphinxAtStartPar
\sphinxstyleemphasis{「あれ?いつの間にかなくなってるよ?」}

\sphinxAtStartPar
\textasciitilde{} かくれんぼしていた子供より \textasciitilde{}
\sphinxstylestrong{導入バージョン:}  0.2

\sphinxAtStartPar
\sphinxstylestrong{更新バージョン:}  8.0


\bigskip\hrule\bigskip



\paragraph{投擲 (通常魔法)}
\label{\detokenize{auto/actionlist:act-throwing}}\label{\detokenize{auto/actionlist:id33}}
\sphinxAtStartPar
\sphinxstylestrong{フォーマット:}


\begin{savenotes}\sphinxattablestart
\sphinxthistablewithglobalstyle
\centering
\begin{tabular}[t]{|\X{1}{4}|\X{1}{4}|\X{1}{4}|\X{1}{4}|}
\sphinxtoprule
\sphinxstyletheadfamily 
\sphinxAtStartPar
ライト
&\sphinxstyletheadfamily 
\sphinxAtStartPar
スタンダード
&\sphinxstyletheadfamily 
\sphinxAtStartPar
プロ
&\sphinxstyletheadfamily 
\sphinxAtStartPar
マスター
\\
\sphinxmidrule
\sphinxtableatstartofbodyhook
\sphinxAtStartPar
◯
&
\sphinxAtStartPar
◯
&
\sphinxAtStartPar
◯
&
\sphinxAtStartPar
◯
\\
\sphinxbottomrule
\end{tabular}
\sphinxtableafterendhook\par
\sphinxattableend\end{savenotes}

\sphinxAtStartPar
\sphinxstylestrong{トリガー:} 直接

\sphinxAtStartPar
\sphinxstylestrong{スピード:} 通常

\sphinxAtStartPar
\sphinxstylestrong{タイミング:} メイン

\sphinxAtStartPar
\sphinxstylestrong{キーカード:} {\normalsize $\spadesuit$} A〜K と {\normalsize $\clubsuit$} A〜K

\sphinxAtStartPar
\sphinxstylestrong{対象:}

\sphinxAtStartPar
対戦相手1人を対象とする。

\sphinxAtStartPar
\sphinxstylestrong{効果:}

\sphinxAtStartPar
対象の対戦相手にX点のダメージを与える。Xはキーカードの{\normalsize $\spadesuit$} カードの数字に等しい。

\sphinxAtStartPar
\sphinxstylestrong{ストーリー:}

\sphinxAtStartPar
知恵{\normalsize $\clubsuit$} と殺意{\normalsize $\spadesuit$} をもってすれば兵士を出さずとも飛び道具で相手にダメージを負わすことができるでしょう。とにかく投げればいいのです。

\sphinxAtStartPar
\sphinxstylestrong{フレーバー:}

\sphinxAtStartPar
\sphinxstyleemphasis{「この戦法は戦い方を一変させた。なぜなら兵がいなくても勝てるからだ。」}

\sphinxAtStartPar
\textasciitilde{} 再現者の心得第十一章より \textasciitilde{}
\sphinxstylestrong{導入バージョン:}  0.1

\sphinxAtStartPar
\sphinxstylestrong{更新バージョン:}  8.0


\bigskip\hrule\bigskip



\paragraph{死の槍 (通常魔法)}
\label{\detokenize{auto/actionlist:act-deathlance}}\label{\detokenize{auto/actionlist:id34}}
\sphinxAtStartPar
\sphinxstylestrong{フォーマット:}


\begin{savenotes}\sphinxattablestart
\sphinxthistablewithglobalstyle
\centering
\begin{tabular}[t]{|\X{1}{4}|\X{1}{4}|\X{1}{4}|\X{1}{4}|}
\sphinxtoprule
\sphinxstyletheadfamily 
\sphinxAtStartPar
ライト
&\sphinxstyletheadfamily 
\sphinxAtStartPar
スタンダード
&\sphinxstyletheadfamily 
\sphinxAtStartPar
プロ
&\sphinxstyletheadfamily 
\sphinxAtStartPar
マスター
\\
\sphinxmidrule
\sphinxtableatstartofbodyhook
\sphinxAtStartPar
x
&
\sphinxAtStartPar
◯
&
\sphinxAtStartPar
◯
&
\sphinxAtStartPar
◯
\\
\sphinxbottomrule
\end{tabular}
\sphinxtableafterendhook\par
\sphinxattableend\end{savenotes}

\sphinxAtStartPar
\sphinxstylestrong{トリガー:} 直接

\sphinxAtStartPar
\sphinxstylestrong{スピード:} 通常

\sphinxAtStartPar
\sphinxstylestrong{タイミング:} メイン

\sphinxAtStartPar
\sphinxstylestrong{キーカード:} {\normalsize $\spadesuit$} A〜K と {\normalsize $\diamondsuit$} A〜K

\sphinxAtStartPar
\sphinxstylestrong{対象:}

\sphinxAtStartPar
兵士1体を対象とする。

\sphinxAtStartPar
\sphinxstylestrong{効果:}

\sphinxAtStartPar
対象の兵士の数字が0以外かつ、キーカードの{\normalsize $\diamondsuit$} カードの数字で割切れる場合、次を行う。
\begin{enumerate}
\sphinxsetlistlabels{\arabic}{enumi}{enumii}{}{.}%
\item {} 
\sphinxAtStartPar
対象の兵士をオーナーのライフの一番上に裏向きで移す。兵士が複数のカードから成る場合、任意の順でライフの一番上に裏向きで移す。

\item {} 
\sphinxAtStartPar
対象の兵士のオーナーにX点のダメージを与える。Xはキーカードの{\normalsize $\spadesuit$} カードの数字に等しい。

\end{enumerate}

\sphinxAtStartPar
\sphinxstylestrong{ストーリー:}

\sphinxAtStartPar
死{\normalsize $\spadesuit$} の槍{\normalsize $\diamondsuit$} 。そのままだな。

\sphinxAtStartPar
\sphinxstylestrong{フレーバー:}

\sphinxAtStartPar
\sphinxstyleemphasis{「この槍は非常に厄介な槍だ。刺さったら急に自国に戻り力尽きるまで暴れ回る。迷惑にも程がある。」}

\sphinxAtStartPar
\textasciitilde{} 再現者の心得第十二章より \textasciitilde{}
\sphinxstylestrong{導入バージョン:}  2.0

\sphinxAtStartPar
\sphinxstylestrong{更新バージョン:}  3.0,8.0


\bigskip\hrule\bigskip



\paragraph{防壁補充 (通常魔法)}
\label{\detokenize{auto/actionlist:act-addbulwark}}\label{\detokenize{auto/actionlist:id35}}
\sphinxAtStartPar
\sphinxstylestrong{フォーマット:}


\begin{savenotes}\sphinxattablestart
\sphinxthistablewithglobalstyle
\centering
\begin{tabular}[t]{|\X{1}{4}|\X{1}{4}|\X{1}{4}|\X{1}{4}|}
\sphinxtoprule
\sphinxstyletheadfamily 
\sphinxAtStartPar
ライト
&\sphinxstyletheadfamily 
\sphinxAtStartPar
スタンダード
&\sphinxstyletheadfamily 
\sphinxAtStartPar
プロ
&\sphinxstyletheadfamily 
\sphinxAtStartPar
マスター
\\
\sphinxmidrule
\sphinxtableatstartofbodyhook
\sphinxAtStartPar
x
&
\sphinxAtStartPar
◯
&
\sphinxAtStartPar
◯
&
\sphinxAtStartPar
◯
\\
\sphinxbottomrule
\end{tabular}
\sphinxtableafterendhook\par
\sphinxattableend\end{savenotes}

\sphinxAtStartPar
\sphinxstylestrong{トリガー:} 直接

\sphinxAtStartPar
\sphinxstylestrong{スピード:} 通常

\sphinxAtStartPar
\sphinxstylestrong{タイミング:} メイン

\sphinxAtStartPar
\sphinxstylestrong{キーカード:} {\normalsize $\heartsuit$} A〜K と {\normalsize $\clubsuit$} A〜K

\sphinxAtStartPar
\sphinxstylestrong{効果:}

\sphinxAtStartPar
自分のライフの一番上から1枚を防壁として裏向きかつチャージ状態で場に出す。もしくは、自分のライフの一番上から2枚を防壁として裏向きかつドライブ状態で場に出す。防壁の能力はキャラクターリスト参照。

\sphinxAtStartPar
防壁の置き方は「防壁の置き方」参照

\sphinxAtStartPar
\sphinxstylestrong{ストーリー:}

\sphinxAtStartPar
愛する人{\normalsize $\heartsuit$} を守るため知恵{\normalsize $\clubsuit$} を絞った結果守りを固めることにした。

\sphinxAtStartPar
\sphinxstylestrong{フレーバー:}

\sphinxAtStartPar
\sphinxstyleemphasis{「守れ!守れ!ひたすら守れ!防御は最大の防御?なり!」}

\sphinxAtStartPar
\textasciitilde{} 無能な指揮官より \textasciitilde{}
\sphinxstylestrong{導入バージョン:}  3.0

\sphinxAtStartPar
\sphinxstylestrong{更新バージョン:}  8.0


\bigskip\hrule\bigskip



\paragraph{リアニメイト (通常魔法)}
\label{\detokenize{auto/actionlist:act-reanimate}}\label{\detokenize{auto/actionlist:id36}}
\sphinxAtStartPar
\sphinxstylestrong{フォーマット:}


\begin{savenotes}\sphinxattablestart
\sphinxthistablewithglobalstyle
\centering
\begin{tabular}[t]{|\X{1}{4}|\X{1}{4}|\X{1}{4}|\X{1}{4}|}
\sphinxtoprule
\sphinxstyletheadfamily 
\sphinxAtStartPar
ライト
&\sphinxstyletheadfamily 
\sphinxAtStartPar
スタンダード
&\sphinxstyletheadfamily 
\sphinxAtStartPar
プロ
&\sphinxstyletheadfamily 
\sphinxAtStartPar
マスター
\\
\sphinxmidrule
\sphinxtableatstartofbodyhook
\sphinxAtStartPar
x
&
\sphinxAtStartPar
◯
&
\sphinxAtStartPar
◯
&
\sphinxAtStartPar
◯
\\
\sphinxbottomrule
\end{tabular}
\sphinxtableafterendhook\par
\sphinxattableend\end{savenotes}

\sphinxAtStartPar
\sphinxstylestrong{トリガー:} 直接

\sphinxAtStartPar
\sphinxstylestrong{スピード:} 通常

\sphinxAtStartPar
\sphinxstylestrong{タイミング:} メイン

\sphinxAtStartPar
\sphinxstylestrong{キーカード:} {\normalsize $\spadesuit$} A〜K と {\normalsize $\heartsuit$} A〜K

\sphinxAtStartPar
\sphinxstylestrong{対象:}

\sphinxAtStartPar
自分の場のキャラクター1体を対象とする。

\sphinxAtStartPar
\sphinxstylestrong{効果:}

\sphinxAtStartPar
自分の墓地にあるカード1枚を選ぶ。対象のキャラクターを墓地に移せた場合、選んだカードを兵士として表向きかつチャージ状態で場に出す。移せない場合、選んだカードを墓地に戻す。

\sphinxAtStartPar
\sphinxstylestrong{導入バージョン:}  3.0

\sphinxAtStartPar
\sphinxstylestrong{更新バージョン:}  8.0


\bigskip\hrule\bigskip



\paragraph{ハンデス (通常魔法)}
\label{\detokenize{auto/actionlist:act-handeth}}\label{\detokenize{auto/actionlist:id37}}
\sphinxAtStartPar
\sphinxstylestrong{フォーマット:}


\begin{savenotes}\sphinxattablestart
\sphinxthistablewithglobalstyle
\centering
\begin{tabular}[t]{|\X{1}{4}|\X{1}{4}|\X{1}{4}|\X{1}{4}|}
\sphinxtoprule
\sphinxstyletheadfamily 
\sphinxAtStartPar
ライト
&\sphinxstyletheadfamily 
\sphinxAtStartPar
スタンダード
&\sphinxstyletheadfamily 
\sphinxAtStartPar
プロ
&\sphinxstyletheadfamily 
\sphinxAtStartPar
マスター
\\
\sphinxmidrule
\sphinxtableatstartofbodyhook
\sphinxAtStartPar
x
&
\sphinxAtStartPar
◯
&
\sphinxAtStartPar
◯
&
\sphinxAtStartPar
◯
\\
\sphinxbottomrule
\end{tabular}
\sphinxtableafterendhook\par
\sphinxattableend\end{savenotes}

\sphinxAtStartPar
\sphinxstylestrong{トリガー:} 直接

\sphinxAtStartPar
\sphinxstylestrong{スピード:} 通常

\sphinxAtStartPar
\sphinxstylestrong{タイミング:} メイン

\sphinxAtStartPar
\sphinxstylestrong{キーカード:} {\normalsize $\diamondsuit$} A〜K と {\normalsize $\clubsuit$} A〜K

\sphinxAtStartPar
\sphinxstylestrong{対象:}

\sphinxAtStartPar
対戦相手1人を対象とする。

\sphinxAtStartPar
\sphinxstylestrong{効果:}

\sphinxAtStartPar
対戦相手の手札を見て1枚カードを指定する。対戦相手は指定されたカードを手札から捨てる。

\sphinxAtStartPar
\sphinxstylestrong{導入バージョン:}  3.0

\sphinxAtStartPar
\sphinxstylestrong{更新バージョン:}  8.0


\bigskip\hrule\bigskip



\paragraph{フォース (通常魔法)}
\label{\detokenize{auto/actionlist:act-force}}\label{\detokenize{auto/actionlist:id38}}
\sphinxAtStartPar
\sphinxstylestrong{フォーマット:}


\begin{savenotes}\sphinxattablestart
\sphinxthistablewithglobalstyle
\centering
\begin{tabular}[t]{|\X{1}{4}|\X{1}{4}|\X{1}{4}|\X{1}{4}|}
\sphinxtoprule
\sphinxstyletheadfamily 
\sphinxAtStartPar
ライト
&\sphinxstyletheadfamily 
\sphinxAtStartPar
スタンダード
&\sphinxstyletheadfamily 
\sphinxAtStartPar
プロ
&\sphinxstyletheadfamily 
\sphinxAtStartPar
マスター
\\
\sphinxmidrule
\sphinxtableatstartofbodyhook
\sphinxAtStartPar
x
&
\sphinxAtStartPar
x
&
\sphinxAtStartPar
x
&
\sphinxAtStartPar
◯
\\
\sphinxbottomrule
\end{tabular}
\sphinxtableafterendhook\par
\sphinxattableend\end{savenotes}

\sphinxAtStartPar
\sphinxstylestrong{トリガー:} 直接

\sphinxAtStartPar
\sphinxstylestrong{スピード:} 通常

\sphinxAtStartPar
\sphinxstylestrong{タイミング:} メイン

\sphinxAtStartPar
\sphinxstylestrong{コスト:} BB

\sphinxAtStartPar
\sphinxstylestrong{キーカード:} {\normalsize $\heartsuit$} A〜10 を 2枚

\sphinxAtStartPar
\sphinxstylestrong{特記事項:} ※このアクションはカウンターアクションの対象にならない。

\sphinxAtStartPar
\sphinxstylestrong{効果:}

\sphinxAtStartPar
このターンが終わるまで自分の兵士全ての数字は、キーカードの合計値分加算される。

\sphinxAtStartPar
その印としてフォグにキーカードを置き、フォースとする。

\sphinxAtStartPar
\sphinxstylestrong{導入バージョン:}  3.0

\sphinxAtStartPar
\sphinxstylestrong{更新バージョン:}  8.0


\bigskip\hrule\bigskip



\paragraph{剣の雨 (通常魔法)}
\label{\detokenize{auto/actionlist:act-swordrain}}\label{\detokenize{auto/actionlist:id39}}
\sphinxAtStartPar
\sphinxstylestrong{フォーマット:}


\begin{savenotes}\sphinxattablestart
\sphinxthistablewithglobalstyle
\centering
\begin{tabular}[t]{|\X{1}{4}|\X{1}{4}|\X{1}{4}|\X{1}{4}|}
\sphinxtoprule
\sphinxstyletheadfamily 
\sphinxAtStartPar
ライト
&\sphinxstyletheadfamily 
\sphinxAtStartPar
スタンダード
&\sphinxstyletheadfamily 
\sphinxAtStartPar
プロ
&\sphinxstyletheadfamily 
\sphinxAtStartPar
マスター
\\
\sphinxmidrule
\sphinxtableatstartofbodyhook
\sphinxAtStartPar
x
&
\sphinxAtStartPar
x
&
\sphinxAtStartPar
x
&
\sphinxAtStartPar
◯
\\
\sphinxbottomrule
\end{tabular}
\sphinxtableafterendhook\par
\sphinxattableend\end{savenotes}

\sphinxAtStartPar
\sphinxstylestrong{トリガー:} 直接

\sphinxAtStartPar
\sphinxstylestrong{スピード:} 通常

\sphinxAtStartPar
\sphinxstylestrong{タイミング:} メイン

\sphinxAtStartPar
\sphinxstylestrong{コスト:} BB

\sphinxAtStartPar
\sphinxstylestrong{キーカード:} {\normalsize $\spadesuit$} A〜10 を 2枚

\sphinxAtStartPar
\sphinxstylestrong{特記事項:} ※このアクションはカウンターアクションの対象にならない。

\sphinxAtStartPar
\sphinxstylestrong{効果:}

\sphinxAtStartPar
キーカードの合計値以下の全ての兵士を墓地に移す。

\sphinxAtStartPar
\sphinxstylestrong{ストーリー:}

\sphinxAtStartPar
死{\normalsize $\spadesuit$} と剣{\normalsize $\spadesuit$} が合わさり雨のように戦場に降り注いだ。

\sphinxAtStartPar
\sphinxstylestrong{フレーバー:}

\sphinxAtStartPar
\sphinxstyleemphasis{「え!助かる訳ないじゃん!!」}

\sphinxAtStartPar
\textasciitilde{} 平和主義の{\normalsize $\clubsuit$} 3一般兵より \textasciitilde{}
\sphinxstylestrong{導入バージョン:}  3.0

\sphinxAtStartPar
\sphinxstylestrong{更新バージョン:}  8.0


\bigskip\hrule\bigskip



\paragraph{徴募 (通常魔法)}
\label{\detokenize{auto/actionlist:act-recruit}}\label{\detokenize{auto/actionlist:id40}}
\sphinxAtStartPar
\sphinxstylestrong{フォーマット:}


\begin{savenotes}\sphinxattablestart
\sphinxthistablewithglobalstyle
\centering
\begin{tabular}[t]{|\X{1}{4}|\X{1}{4}|\X{1}{4}|\X{1}{4}|}
\sphinxtoprule
\sphinxstyletheadfamily 
\sphinxAtStartPar
ライト
&\sphinxstyletheadfamily 
\sphinxAtStartPar
スタンダード
&\sphinxstyletheadfamily 
\sphinxAtStartPar
プロ
&\sphinxstyletheadfamily 
\sphinxAtStartPar
マスター
\\
\sphinxmidrule
\sphinxtableatstartofbodyhook
\sphinxAtStartPar
x
&
\sphinxAtStartPar
x
&
\sphinxAtStartPar
x
&
\sphinxAtStartPar
◯
\\
\sphinxbottomrule
\end{tabular}
\sphinxtableafterendhook\par
\sphinxattableend\end{savenotes}

\sphinxAtStartPar
\sphinxstylestrong{トリガー:} 直接

\sphinxAtStartPar
\sphinxstylestrong{スピード:} 通常

\sphinxAtStartPar
\sphinxstylestrong{タイミング:} メイン

\sphinxAtStartPar
\sphinxstylestrong{コスト:} BB

\sphinxAtStartPar
\sphinxstylestrong{キーカード:} {\normalsize $\diamondsuit$} A〜10 を 2枚

\sphinxAtStartPar
\sphinxstylestrong{特記事項:} ※このアクションはカウンターアクションの対象にならない。

\sphinxAtStartPar
\sphinxstylestrong{効果:}
\begin{enumerate}
\sphinxsetlistlabels{\arabic}{enumi}{enumii}{}{.}%
\item {} 
\sphinxAtStartPar
ライフの一番上から4枚めくり、キーカードの合計値以下のカードを兵士としてドライブ状態で場に出す。

\item {} 
\sphinxAtStartPar
残りのカードをライフの一番下に好きな順で移す。

\end{enumerate}

\sphinxAtStartPar
\sphinxstylestrong{ストーリー:}

\sphinxAtStartPar
金{\normalsize $\diamondsuit$} が手元にたくさんあれば、戦力強化が定石だろう。

\sphinxAtStartPar
\sphinxstylestrong{フレーバー:}

\sphinxAtStartPar
\sphinxstyleemphasis{「君はどう?兵士になる?...あっそう。」
「君はどう?兵士になる?...あっそう。」
「はぁ。金がなけれりゃ誰も来やしねぇ」}

\sphinxAtStartPar
\textasciitilde{} 孤独な徴募兵より \textasciitilde{}
\sphinxstylestrong{導入バージョン:}  3.0

\sphinxAtStartPar
\sphinxstylestrong{更新バージョン:}  8.0


\bigskip\hrule\bigskip



\paragraph{奇襲 (通常魔法)}
\label{\detokenize{auto/actionlist:act-surprise}}\label{\detokenize{auto/actionlist:id41}}
\sphinxAtStartPar
\sphinxstylestrong{フォーマット:}


\begin{savenotes}\sphinxattablestart
\sphinxthistablewithglobalstyle
\centering
\begin{tabular}[t]{|\X{1}{4}|\X{1}{4}|\X{1}{4}|\X{1}{4}|}
\sphinxtoprule
\sphinxstyletheadfamily 
\sphinxAtStartPar
ライト
&\sphinxstyletheadfamily 
\sphinxAtStartPar
スタンダード
&\sphinxstyletheadfamily 
\sphinxAtStartPar
プロ
&\sphinxstyletheadfamily 
\sphinxAtStartPar
マスター
\\
\sphinxmidrule
\sphinxtableatstartofbodyhook
\sphinxAtStartPar
x
&
\sphinxAtStartPar
x
&
\sphinxAtStartPar
x
&
\sphinxAtStartPar
◯
\\
\sphinxbottomrule
\end{tabular}
\sphinxtableafterendhook\par
\sphinxattableend\end{savenotes}

\sphinxAtStartPar
\sphinxstylestrong{トリガー:} 直接

\sphinxAtStartPar
\sphinxstylestrong{スピード:} 通常

\sphinxAtStartPar
\sphinxstylestrong{タイミング:} メイン

\sphinxAtStartPar
\sphinxstylestrong{コスト:} BB

\sphinxAtStartPar
\sphinxstylestrong{キーカード:} {\normalsize $\clubsuit$} A〜10 を 2枚

\sphinxAtStartPar
\sphinxstylestrong{特記事項:} ※このアクションはカウンターアクションの対象にならない。

\sphinxAtStartPar
\sphinxstylestrong{効果:}
\begin{enumerate}
\sphinxsetlistlabels{\arabic}{enumi}{enumii}{}{.}%
\item {} 
\sphinxAtStartPar
自分の場にいる防壁を全て兵士にする。このターンに場に出た防壁を兵士にする場合、その兵士はこのターンに出た兵士と同様に扱う。

\item {} 
\sphinxAtStartPar
自分の場にいる全ての兵士をチャージ状態にする。

\end{enumerate}

\sphinxAtStartPar
\sphinxstylestrong{導入バージョン:}  3.0

\sphinxAtStartPar
\sphinxstylestrong{更新バージョン:}  8.0


\subsection{キャラクターリスト}
\label{\detokenize{auto/actionlist:id42}}
\begin{sphinxShadowBox}
\sphinxstyletopictitle{目次}
\begin{itemize}
\item {} 
\sphinxAtStartPar
\phantomsection\label{\detokenize{auto/actionlist:id100}}{\hyperref[\detokenize{auto/actionlist:id44}]{\sphinxcrossref{兵士}}}
\begin{itemize}
\item {} 
\sphinxAtStartPar
\phantomsection\label{\detokenize{auto/actionlist:id101}}{\hyperref[\detokenize{auto/actionlist:char-soldier}]{\sphinxcrossref{一般兵 (兵士)}}}

\item {} 
\sphinxAtStartPar
\phantomsection\label{\detokenize{auto/actionlist:id102}}{\hyperref[\detokenize{auto/actionlist:char-hero}]{\sphinxcrossref{英雄 (兵士)}}}

\item {} 
\sphinxAtStartPar
\phantomsection\label{\detokenize{auto/actionlist:id103}}{\hyperref[\detokenize{auto/actionlist:char-ace}]{\sphinxcrossref{エース (兵士)}}}

\item {} 
\sphinxAtStartPar
\phantomsection\label{\detokenize{auto/actionlist:id104}}{\hyperref[\detokenize{auto/actionlist:char-magician}]{\sphinxcrossref{魔術士 (兵士)}}}

\item {} 
\sphinxAtStartPar
\phantomsection\label{\detokenize{auto/actionlist:id105}}{\hyperref[\detokenize{auto/actionlist:char-armedsoldier}]{\sphinxcrossref{装備兵 (兵士)}}}

\end{itemize}

\item {} 
\sphinxAtStartPar
\phantomsection\label{\detokenize{auto/actionlist:id106}}{\hyperref[\detokenize{auto/actionlist:id50}]{\sphinxcrossref{防壁}}}
\begin{itemize}
\item {} 
\sphinxAtStartPar
\phantomsection\label{\detokenize{auto/actionlist:id107}}{\hyperref[\detokenize{auto/actionlist:char-bulwark}]{\sphinxcrossref{防壁 (防壁)}}}

\end{itemize}

\end{itemize}
\end{sphinxShadowBox}


\subsubsection{兵士}
\label{\detokenize{auto/actionlist:id44}}

\paragraph{一般兵 (兵士)}
\label{\detokenize{auto/actionlist:char-soldier}}\label{\detokenize{auto/actionlist:id45}}
\sphinxAtStartPar
\sphinxstylestrong{フォーマット:}


\begin{savenotes}\sphinxattablestart
\sphinxthistablewithglobalstyle
\centering
\begin{tabular}[t]{|\X{1}{4}|\X{1}{4}|\X{1}{4}|\X{1}{4}|}
\sphinxtoprule
\sphinxstyletheadfamily 
\sphinxAtStartPar
ライト
&\sphinxstyletheadfamily 
\sphinxAtStartPar
スタンダード
&\sphinxstyletheadfamily 
\sphinxAtStartPar
プロ
&\sphinxstyletheadfamily 
\sphinxAtStartPar
マスター
\\
\sphinxmidrule
\sphinxtableatstartofbodyhook
\sphinxAtStartPar
◯
&
\sphinxAtStartPar
◯
&
\sphinxAtStartPar
◯
&
\sphinxAtStartPar
◯
\\
\sphinxbottomrule
\end{tabular}
\sphinxtableafterendhook\par
\sphinxattableend\end{savenotes}

\sphinxAtStartPar
\sphinxstylestrong{キーカード:} 2〜10

\sphinxAtStartPar
\sphinxstylestrong{サイズ:} キーカードの数字

\sphinxAtStartPar
\sphinxstylestrong{ラベル:} アタッカー, ブロッカー

\sphinxAtStartPar
\sphinxstylestrong{導入バージョン:}  1.0

\sphinxAtStartPar
\sphinxstylestrong{更新バージョン:}  8.0


\bigskip\hrule\bigskip



\paragraph{英雄 (兵士)}
\label{\detokenize{auto/actionlist:char-hero}}\label{\detokenize{auto/actionlist:id46}}
\sphinxAtStartPar
\sphinxstylestrong{フォーマット:}


\begin{savenotes}\sphinxattablestart
\sphinxthistablewithglobalstyle
\centering
\begin{tabular}[t]{|\X{1}{4}|\X{1}{4}|\X{1}{4}|\X{1}{4}|}
\sphinxtoprule
\sphinxstyletheadfamily 
\sphinxAtStartPar
ライト
&\sphinxstyletheadfamily 
\sphinxAtStartPar
スタンダード
&\sphinxstyletheadfamily 
\sphinxAtStartPar
プロ
&\sphinxstyletheadfamily 
\sphinxAtStartPar
マスター
\\
\sphinxmidrule
\sphinxtableatstartofbodyhook
\sphinxAtStartPar
◯
&
\sphinxAtStartPar
◯
&
\sphinxAtStartPar
◯
&
\sphinxAtStartPar
◯
\\
\sphinxbottomrule
\end{tabular}
\sphinxtableafterendhook\par
\sphinxattableend\end{savenotes}

\sphinxAtStartPar
\sphinxstylestrong{キーカード:} J〜K

\sphinxAtStartPar
\sphinxstylestrong{サイズ:} キーカードがJなら11,Qなら12,Kなら13

\sphinxAtStartPar
\sphinxstylestrong{ラベル:} アタッカー, ブロッカー

\sphinxAtStartPar
\sphinxstylestrong{導入バージョン:}  1.0

\sphinxAtStartPar
\sphinxstylestrong{更新バージョン:}  8.0


\bigskip\hrule\bigskip



\paragraph{エース (兵士)}
\label{\detokenize{auto/actionlist:char-ace}}\label{\detokenize{auto/actionlist:id47}}
\sphinxAtStartPar
\sphinxstylestrong{フォーマット:}


\begin{savenotes}\sphinxattablestart
\sphinxthistablewithglobalstyle
\centering
\begin{tabular}[t]{|\X{1}{4}|\X{1}{4}|\X{1}{4}|\X{1}{4}|}
\sphinxtoprule
\sphinxstyletheadfamily 
\sphinxAtStartPar
ライト
&\sphinxstyletheadfamily 
\sphinxAtStartPar
スタンダード
&\sphinxstyletheadfamily 
\sphinxAtStartPar
プロ
&\sphinxstyletheadfamily 
\sphinxAtStartPar
マスター
\\
\sphinxmidrule
\sphinxtableatstartofbodyhook
\sphinxAtStartPar
◯
&
\sphinxAtStartPar
◯
&
\sphinxAtStartPar
◯
&
\sphinxAtStartPar
◯
\\
\sphinxbottomrule
\end{tabular}
\sphinxtableafterendhook\par
\sphinxattableend\end{savenotes}

\sphinxAtStartPar
\sphinxstylestrong{キーカード:} A

\sphinxAtStartPar
\sphinxstylestrong{サイズ:} 1

\sphinxAtStartPar
\sphinxstylestrong{ラベル:} アタッカー, ブロッカー, 速攻

\sphinxAtStartPar
\sphinxstylestrong{ストーリー:}

\sphinxAtStartPar
エースは先駆者であり皆の先頭に立ち切り込んで行く力があります。一人では大きなダメージにはなりませんが、大きな力を持っています。あなたもエースがまさにエースであることに気づくでしょう。

\sphinxAtStartPar
\sphinxstylestrong{フレーバー:}

\sphinxAtStartPar
\sphinxstyleemphasis{「俺には、ちびですばしっこいだけにしか見えませんが?」}

\sphinxAtStartPar
\textasciitilde{} 再現者見習いより \textasciitilde{}
\sphinxstylestrong{導入バージョン:}  1.0

\sphinxAtStartPar
\sphinxstylestrong{更新バージョン:}  8.0


\bigskip\hrule\bigskip



\paragraph{魔術士 (兵士)}
\label{\detokenize{auto/actionlist:char-magician}}\label{\detokenize{auto/actionlist:id48}}
\sphinxAtStartPar
\sphinxstylestrong{フォーマット:}


\begin{savenotes}\sphinxattablestart
\sphinxthistablewithglobalstyle
\centering
\begin{tabular}[t]{|\X{1}{4}|\X{1}{4}|\X{1}{4}|\X{1}{4}|}
\sphinxtoprule
\sphinxstyletheadfamily 
\sphinxAtStartPar
ライト
&\sphinxstyletheadfamily 
\sphinxAtStartPar
スタンダード
&\sphinxstyletheadfamily 
\sphinxAtStartPar
プロ
&\sphinxstyletheadfamily 
\sphinxAtStartPar
マスター
\\
\sphinxmidrule
\sphinxtableatstartofbodyhook
\sphinxAtStartPar
x
&
\sphinxAtStartPar
◯
&
\sphinxAtStartPar
◯
&
\sphinxAtStartPar
◯
\\
\sphinxbottomrule
\end{tabular}
\sphinxtableafterendhook\par
\sphinxattableend\end{savenotes}

\sphinxAtStartPar
\sphinxstylestrong{キーカード:} Joker

\sphinxAtStartPar
\sphinxstylestrong{サイズ:} 0

\sphinxAtStartPar
\sphinxstylestrong{ラベル:} アタッカー, ブロッカー, 速攻

\sphinxAtStartPar
\sphinxstylestrong{能力:}

\sphinxAtStartPar
・魔力増加(このキャラクターが場にいる間、タイプ:速攻魔法のコストDが無しになる。)

\sphinxAtStartPar
\sphinxstylestrong{導入バージョン:}  3.0

\sphinxAtStartPar
\sphinxstylestrong{更新バージョン:}  8.0


\bigskip\hrule\bigskip



\paragraph{装備兵 (兵士)}
\label{\detokenize{auto/actionlist:char-armedsoldier}}\label{\detokenize{auto/actionlist:id49}}
\sphinxAtStartPar
\sphinxstylestrong{フォーマット:}


\begin{savenotes}\sphinxattablestart
\sphinxthistablewithglobalstyle
\centering
\begin{tabular}[t]{|\X{1}{4}|\X{1}{4}|\X{1}{4}|\X{1}{4}|}
\sphinxtoprule
\sphinxstyletheadfamily 
\sphinxAtStartPar
ライト
&\sphinxstyletheadfamily 
\sphinxAtStartPar
スタンダード
&\sphinxstyletheadfamily 
\sphinxAtStartPar
プロ
&\sphinxstyletheadfamily 
\sphinxAtStartPar
マスター
\\
\sphinxmidrule
\sphinxtableatstartofbodyhook
\sphinxAtStartPar
◯
&
\sphinxAtStartPar
◯
&
\sphinxAtStartPar
◯
&
\sphinxAtStartPar
◯
\\
\sphinxbottomrule
\end{tabular}
\sphinxtableafterendhook\par
\sphinxattableend\end{savenotes}

\sphinxAtStartPar
\sphinxstylestrong{キーカード:} 同じスートを2枚以上

\sphinxAtStartPar
\sphinxstylestrong{サイズ:} キーカードの数字の合計

\sphinxAtStartPar
\sphinxstylestrong{ラベル:} アタッカー, ブロッカー, 速攻(キーカードにAが含まれる場合のみ)

\sphinxAtStartPar
\sphinxstylestrong{導入バージョン:}  5.0

\sphinxAtStartPar
\sphinxstylestrong{更新バージョン:}  8.0


\subsubsection{防壁}
\label{\detokenize{auto/actionlist:id50}}

\paragraph{防壁 (防壁)}
\label{\detokenize{auto/actionlist:char-bulwark}}\label{\detokenize{auto/actionlist:id51}}
\sphinxAtStartPar
\sphinxstylestrong{フォーマット:}


\begin{savenotes}\sphinxattablestart
\sphinxthistablewithglobalstyle
\centering
\begin{tabular}[t]{|\X{1}{4}|\X{1}{4}|\X{1}{4}|\X{1}{4}|}
\sphinxtoprule
\sphinxstyletheadfamily 
\sphinxAtStartPar
ライト
&\sphinxstyletheadfamily 
\sphinxAtStartPar
スタンダード
&\sphinxstyletheadfamily 
\sphinxAtStartPar
プロ
&\sphinxstyletheadfamily 
\sphinxAtStartPar
マスター
\\
\sphinxmidrule
\sphinxtableatstartofbodyhook
\sphinxAtStartPar
◯
&
\sphinxAtStartPar
◯
&
\sphinxAtStartPar
◯
&
\sphinxAtStartPar
◯
\\
\sphinxbottomrule
\end{tabular}
\sphinxtableafterendhook\par
\sphinxattableend\end{savenotes}

\sphinxAtStartPar
\sphinxstylestrong{キーカード:} 全て(裏向き)

\sphinxAtStartPar
\sphinxstylestrong{ラベル:} ブロッカー

\sphinxAtStartPar
\sphinxstylestrong{導入バージョン:}  1.0

\sphinxAtStartPar
\sphinxstylestrong{更新バージョン:}  8.0


\subsection{フォグリスト}
\label{\detokenize{auto/actionlist:id52}}
\begin{sphinxShadowBox}
\sphinxstyletopictitle{目次}
\begin{itemize}
\item {} 
\sphinxAtStartPar
\phantomsection\label{\detokenize{auto/actionlist:id108}}{\hyperref[\detokenize{auto/actionlist:id54}]{\sphinxcrossref{速攻魔法}}}
\begin{itemize}
\item {} 
\sphinxAtStartPar
\phantomsection\label{\detokenize{auto/actionlist:id109}}{\hyperref[\detokenize{auto/actionlist:fog-upfog}]{\sphinxcrossref{アップ (速攻魔法)}}}

\item {} 
\sphinxAtStartPar
\phantomsection\label{\detokenize{auto/actionlist:id110}}{\hyperref[\detokenize{auto/actionlist:fog-downfog}]{\sphinxcrossref{ダウン (速攻魔法)}}}

\end{itemize}

\item {} 
\sphinxAtStartPar
\phantomsection\label{\detokenize{auto/actionlist:id111}}{\hyperref[\detokenize{auto/actionlist:id57}]{\sphinxcrossref{通常魔法}}}
\begin{itemize}
\item {} 
\sphinxAtStartPar
\phantomsection\label{\detokenize{auto/actionlist:id112}}{\hyperref[\detokenize{auto/actionlist:fog-forcefog}]{\sphinxcrossref{フォース (通常魔法)}}}

\end{itemize}

\end{itemize}
\end{sphinxShadowBox}


\subsubsection{速攻魔法}
\label{\detokenize{auto/actionlist:id54}}

\paragraph{アップ (速攻魔法)}
\label{\detokenize{auto/actionlist:fog-upfog}}\label{\detokenize{auto/actionlist:id55}}
\sphinxAtStartPar
\sphinxstylestrong{フォーマット:}


\begin{savenotes}\sphinxattablestart
\sphinxthistablewithglobalstyle
\centering
\begin{tabular}[t]{|\X{1}{4}|\X{1}{4}|\X{1}{4}|\X{1}{4}|}
\sphinxtoprule
\sphinxstyletheadfamily 
\sphinxAtStartPar
ライト
&\sphinxstyletheadfamily 
\sphinxAtStartPar
スタンダード
&\sphinxstyletheadfamily 
\sphinxAtStartPar
プロ
&\sphinxstyletheadfamily 
\sphinxAtStartPar
マスター
\\
\sphinxmidrule
\sphinxtableatstartofbodyhook
\sphinxAtStartPar
◯
&
\sphinxAtStartPar
◯
&
\sphinxAtStartPar
◯
&
\sphinxAtStartPar
◯
\\
\sphinxbottomrule
\end{tabular}
\sphinxtableafterendhook\par
\sphinxattableend\end{savenotes}

\sphinxAtStartPar
\sphinxstylestrong{キーカード:} {\normalsize $\heartsuit$} A〜10

\sphinxAtStartPar
\sphinxstylestrong{能力:}

\sphinxAtStartPar
対象とした兵士のサイズは、このターンが終わるまでキーカードの数字分加算される。

\sphinxAtStartPar
\sphinxstylestrong{導入バージョン:}  8.0


\bigskip\hrule\bigskip



\paragraph{ダウン (速攻魔法)}
\label{\detokenize{auto/actionlist:fog-downfog}}\label{\detokenize{auto/actionlist:id56}}
\sphinxAtStartPar
\sphinxstylestrong{フォーマット:}


\begin{savenotes}\sphinxattablestart
\sphinxthistablewithglobalstyle
\centering
\begin{tabular}[t]{|\X{1}{4}|\X{1}{4}|\X{1}{4}|\X{1}{4}|}
\sphinxtoprule
\sphinxstyletheadfamily 
\sphinxAtStartPar
ライト
&\sphinxstyletheadfamily 
\sphinxAtStartPar
スタンダード
&\sphinxstyletheadfamily 
\sphinxAtStartPar
プロ
&\sphinxstyletheadfamily 
\sphinxAtStartPar
マスター
\\
\sphinxmidrule
\sphinxtableatstartofbodyhook
\sphinxAtStartPar
◯
&
\sphinxAtStartPar
◯
&
\sphinxAtStartPar
◯
&
\sphinxAtStartPar
◯
\\
\sphinxbottomrule
\end{tabular}
\sphinxtableafterendhook\par
\sphinxattableend\end{savenotes}

\sphinxAtStartPar
\sphinxstylestrong{キーカード:} {\normalsize $\spadesuit$} A〜10

\sphinxAtStartPar
\sphinxstylestrong{能力:}

\sphinxAtStartPar
対象とした兵士のサイズは、このターンが終わるまでキーカードの数字分減算される。

\sphinxAtStartPar
\sphinxstylestrong{導入バージョン:}  8.0


\subsubsection{通常魔法}
\label{\detokenize{auto/actionlist:id57}}

\paragraph{フォース (通常魔法)}
\label{\detokenize{auto/actionlist:fog-forcefog}}\label{\detokenize{auto/actionlist:id58}}
\sphinxAtStartPar
\sphinxstylestrong{フォーマット:}


\begin{savenotes}\sphinxattablestart
\sphinxthistablewithglobalstyle
\centering
\begin{tabular}[t]{|\X{1}{4}|\X{1}{4}|\X{1}{4}|\X{1}{4}|}
\sphinxtoprule
\sphinxstyletheadfamily 
\sphinxAtStartPar
ライト
&\sphinxstyletheadfamily 
\sphinxAtStartPar
スタンダード
&\sphinxstyletheadfamily 
\sphinxAtStartPar
プロ
&\sphinxstyletheadfamily 
\sphinxAtStartPar
マスター
\\
\sphinxmidrule
\sphinxtableatstartofbodyhook
\sphinxAtStartPar
x
&
\sphinxAtStartPar
x
&
\sphinxAtStartPar
x
&
\sphinxAtStartPar
◯
\\
\sphinxbottomrule
\end{tabular}
\sphinxtableafterendhook\par
\sphinxattableend\end{savenotes}

\sphinxAtStartPar
\sphinxstylestrong{キーカード:} {\normalsize $\heartsuit$} A〜10 を 2枚

\sphinxAtStartPar
\sphinxstylestrong{能力:}

\sphinxAtStartPar
このターンが終わるまで自分の兵士全ての数字は、キーカードの合計値分加算される。

\sphinxAtStartPar
\sphinxstylestrong{導入バージョン:}  8.0

\sphinxstepscope


\chapter{フレーム}
\label{\detokenize{frame/frame:frame-rst}}\label{\detokenize{frame/frame:id1}}\label{\detokenize{frame/frame::doc}}
\index{フレーム@\spxentry{フレーム}}\ignorespaces 

\section{フレームとは}
\label{\detokenize{frame/frame:index-0}}\label{\detokenize{frame/frame:id2}}
\sphinxAtStartPar
フレームとはフォーマットとは違った角度で対戦方法を決める枠組みです。

\sphinxAtStartPar
フレームはフォーマットと異なり、アクション以外にも領域(ゾーン)、ゲームの始め方などについても定義しています。


\section{定義項目}
\label{\detokenize{frame/frame:id3}}
\sphinxAtStartPar
フレームには次の項目が定義されています。
\begin{description}
\sphinxlineitem{配置図}
\sphinxAtStartPar
共通ルールにて定義されていない領域を使う場合やゲームの始め方の補足説明で必要な場合、定義されます。
共通ルールで定義された領域:ライフ、墓地、場、手札、フォグ

\sphinxlineitem{フォーマット}
\sphinxAtStartPar
対応しているフォーマット

\sphinxlineitem{デッキ条件}
\sphinxAtStartPar
デッキのカード構成が54枚(2枚のJokerを含む)でない場合、定義されます。

\sphinxlineitem{ゲームの始め方}
\sphinxAtStartPar
共通ルールに記載されているゲームの始め方(カードの初期配置)を変更する場合、定義されます。

\sphinxlineitem{アクションリスト}
\sphinxAtStartPar
追加で起こせるアクション(フレーム依存アクション)がある場合、定義されます。

\end{description}


\section{フレーム定義}
\label{\detokenize{frame/frame:id4}}
\sphinxAtStartPar
各フレームの詳細は次の公式フレームを参照してください。

\sphinxstepscope


\subsection{フレームリスト}
\label{\detokenize{auto/framelist:id1}}\label{\detokenize{auto/framelist::doc}}
\sphinxAtStartPar
本ドキュメントでは、BlackPokerのすべてのフレームについて記載します。

\begin{sphinxShadowBox}
\sphinxstyletopictitle{目次}
\begin{itemize}
\item {} 
\sphinxAtStartPar
\phantomsection\label{\detokenize{auto/framelist:id11}}{\hyperref[\detokenize{auto/framelist:frame-entry20}]{\sphinxcrossref{エントリー20}}}

\item {} 
\sphinxAtStartPar
\phantomsection\label{\detokenize{auto/framelist:id12}}{\hyperref[\detokenize{auto/framelist:frame-pack}]{\sphinxcrossref{パック}}}

\item {} 
\sphinxAtStartPar
\phantomsection\label{\detokenize{auto/framelist:id13}}{\hyperref[\detokenize{auto/framelist:frame-rarepack}]{\sphinxcrossref{レアパック}}}

\item {} 
\sphinxAtStartPar
\phantomsection\label{\detokenize{auto/framelist:id14}}{\hyperref[\detokenize{auto/framelist:frame-strategy}]{\sphinxcrossref{ストラテジー}}}

\end{itemize}
\end{sphinxShadowBox}


\subsubsection{エントリー20}
\label{\detokenize{auto/framelist:frame-entry20}}\label{\detokenize{auto/framelist:id3}}
\sphinxAtStartPar
\sphinxstylestrong{配置図:}

\begin{figure}[htbp]
\centering
\capstart

\noindent\sphinxincludegraphics{{entry20}.pdf}
\caption{配置図\sphinxhyphen{}エントリー20}\label{\detokenize{auto/framelist:id7}}\label{\detokenize{auto/framelist:frame-entry20-image}}\end{figure}

\sphinxAtStartPar
\sphinxstylestrong{フォーマット:}


\begin{savenotes}\sphinxattablestart
\sphinxthistablewithglobalstyle
\centering
\begin{tabular}[t]{|\X{1}{4}|\X{1}{4}|\X{1}{4}|\X{1}{4}|}
\sphinxtoprule
\sphinxstyletheadfamily 
\sphinxAtStartPar
ライト
&\sphinxstyletheadfamily 
\sphinxAtStartPar
スタンダード
&\sphinxstyletheadfamily 
\sphinxAtStartPar
プロ
&\sphinxstyletheadfamily 
\sphinxAtStartPar
マスター
\\
\sphinxmidrule
\sphinxtableatstartofbodyhook
\sphinxAtStartPar
◯
&
\sphinxAtStartPar
◯
&
\sphinxAtStartPar
x
&
\sphinxAtStartPar
x
\\
\sphinxbottomrule
\end{tabular}
\sphinxtableafterendhook\par
\sphinxattableend\end{savenotes}

\sphinxAtStartPar
\sphinxstylestrong{デッキ条件:}

\sphinxAtStartPar
次のカードのみで構成された20枚のデッキを使う
\begin{itemize}
\item {} 
\sphinxAtStartPar
{\normalsize $\spadesuit$} A, 2, 3, 4, 5

\item {} 
\sphinxAtStartPar
{\normalsize $\heartsuit$} A, 8, 9, 10, J

\item {} 
\sphinxAtStartPar
{\normalsize $\diamondsuit$} A, 3, 7, 10, Q

\item {} 
\sphinxAtStartPar
{\normalsize $\clubsuit$} A, 5, 6, 10, K

\end{itemize}

\sphinxAtStartPar
\sphinxstylestrong{ゲームの始め方:}
\begin{itemize}
\item {} 
\sphinxAtStartPar
デッキをライフとし配置図の場所に伏せておく。

\item {} 
\sphinxAtStartPar
ライフの一番上のカードを防壁として場に出し、次のカードを兵士として場に出す。(プリセット)

\end{itemize}

\sphinxAtStartPar
以降は、 \hyperref[\detokenize{common/common:common-gamestart-field}]{\ref{\detokenize{common/common:common-gamestart-field}} \nameref{\detokenize{common/common:common-gamestart-field}}} 以降に沿ってゲームを開始します。


\bigskip\hrule\bigskip



\subsubsection{パック}
\label{\detokenize{auto/framelist:frame-pack}}\label{\detokenize{auto/framelist:id4}}
\sphinxAtStartPar
\sphinxstylestrong{配置図:}

\begin{figure}[htbp]
\centering
\capstart

\noindent\sphinxincludegraphics{{pack}.pdf}
\caption{配置図\sphinxhyphen{}パック}\label{\detokenize{auto/framelist:id8}}\label{\detokenize{auto/framelist:frame-pack-image}}\end{figure}

\sphinxAtStartPar
\sphinxstylestrong{フォーマット:}


\begin{savenotes}\sphinxattablestart
\sphinxthistablewithglobalstyle
\centering
\begin{tabular}[t]{|\X{1}{4}|\X{1}{4}|\X{1}{4}|\X{1}{4}|}
\sphinxtoprule
\sphinxstyletheadfamily 
\sphinxAtStartPar
ライト
&\sphinxstyletheadfamily 
\sphinxAtStartPar
スタンダード
&\sphinxstyletheadfamily 
\sphinxAtStartPar
プロ
&\sphinxstyletheadfamily 
\sphinxAtStartPar
マスター
\\
\sphinxmidrule
\sphinxtableatstartofbodyhook
\sphinxAtStartPar
◯
&
\sphinxAtStartPar
◯
&
\sphinxAtStartPar
◯
&
\sphinxAtStartPar
◯
\\
\sphinxbottomrule
\end{tabular}
\sphinxtableafterendhook\par
\sphinxattableend\end{savenotes}

\sphinxAtStartPar
\sphinxstylestrong{デッキ条件:}

\sphinxAtStartPar
なし

\sphinxAtStartPar
\sphinxstylestrong{ゲームの始め方:}
\begin{itemize}
\item {} 
\sphinxAtStartPar
デッキの上から14枚を取り除いてパックとし、配置図の場所に伏せておく。

\item {} 
\sphinxAtStartPar
残りのデッキをライフとし配置図の場所に伏せておく。

\item {} 
\sphinxAtStartPar
ライフの一番上のカードを防壁として場に出し、次のカードを兵士として場に出す。(プリセット)

\item {} 
\sphinxAtStartPar
フォーマットがライトの場合、プリセットされた兵士カードがJokerならばそれを墓地に移し、次のカードを兵士として場に出す。(ライトに兵士「魔術士」がいないため。)

\end{itemize}

\sphinxAtStartPar
以降は、 \hyperref[\detokenize{common/common:common-gamestart-first}]{\ref{\detokenize{common/common:common-gamestart-first}} \nameref{\detokenize{common/common:common-gamestart-first}}} 以降に沿ってゲームを開始します。

\sphinxAtStartPar
\sphinxstylestrong{アクションリスト:}

\sphinxAtStartPar
次のアクションが追加で利用可能です。
\begin{itemize}
\item {} 
\sphinxAtStartPar
パック開封

\end{itemize}

\sphinxAtStartPar
詳細は次を参照してください。

\sphinxAtStartPar
\hyperref[\detokenize{auto/frameActionlist:act-act-frame}]{\ref{\detokenize{auto/frameActionlist:act-act-frame}} \nameref{\detokenize{auto/frameActionlist:act-act-frame}}}

\sphinxAtStartPar
\sphinxstylestrong{公開レベル:}

\sphinxAtStartPar
このフレームで定義されている領域の公開レベルは次の通りです。
\begin{description}
\sphinxlineitem{パック}
\begin{DUlineblock}{0em}
\item[] 完全公開:開封済み・未開封
\item[] 個人公開:開封後のパックの中身
\item[] 非公開:未開封のパックの中身
\end{DUlineblock}

\end{description}


\bigskip\hrule\bigskip



\subsubsection{レアパック}
\label{\detokenize{auto/framelist:frame-rarepack}}\label{\detokenize{auto/framelist:id5}}
\sphinxAtStartPar
\sphinxstylestrong{配置図:}

\begin{figure}[htbp]
\centering
\capstart

\noindent\sphinxincludegraphics{{rarePack}.pdf}
\caption{配置図\sphinxhyphen{}レアパック}\label{\detokenize{auto/framelist:id9}}\label{\detokenize{auto/framelist:frame-rarepack-image}}\end{figure}

\sphinxAtStartPar
\sphinxstylestrong{フォーマット:}


\begin{savenotes}\sphinxattablestart
\sphinxthistablewithglobalstyle
\centering
\begin{tabular}[t]{|\X{1}{4}|\X{1}{4}|\X{1}{4}|\X{1}{4}|}
\sphinxtoprule
\sphinxstyletheadfamily 
\sphinxAtStartPar
ライト
&\sphinxstyletheadfamily 
\sphinxAtStartPar
スタンダード
&\sphinxstyletheadfamily 
\sphinxAtStartPar
プロ
&\sphinxstyletheadfamily 
\sphinxAtStartPar
マスター
\\
\sphinxmidrule
\sphinxtableatstartofbodyhook
\sphinxAtStartPar
x
&
\sphinxAtStartPar
◯
&
\sphinxAtStartPar
◯
&
\sphinxAtStartPar
◯
\\
\sphinxbottomrule
\end{tabular}
\sphinxtableafterendhook\par
\sphinxattableend\end{savenotes}

\sphinxAtStartPar
\sphinxstylestrong{デッキ条件:}

\sphinxAtStartPar
なし

\sphinxAtStartPar
\sphinxstylestrong{ゲームの始め方:}
\begin{itemize}
\item {} 
\sphinxAtStartPar
デッキからレアカード1枚を選び配置図の場所に伏せておく。

\item {} 
\sphinxAtStartPar
デッキをシャッフルする。

\item {} 
\sphinxAtStartPar
デッキの上から14枚を取り除いてパックとし、配置図の場所に伏せておく。

\item {} 
\sphinxAtStartPar
残りのデッキをライフとし配置図の場所に伏せておく。

\item {} 
\sphinxAtStartPar
ライフの一番上のカードを防壁として場に出し、次のカードを兵士として場に出す。(プリセット)

\end{itemize}

\sphinxAtStartPar
以降は、 \hyperref[\detokenize{common/common:common-gamestart-first}]{\ref{\detokenize{common/common:common-gamestart-first}} \nameref{\detokenize{common/common:common-gamestart-first}}} 以降に沿ってゲームを開始します。

\sphinxAtStartPar
\sphinxstylestrong{アクションリスト:}

\sphinxAtStartPar
次のアクションが追加で利用可能です。
\begin{itemize}
\item {} 
\sphinxAtStartPar
パック開封

\item {} 
\sphinxAtStartPar
レアドロー

\item {} 
\sphinxAtStartPar
レア召喚

\item {} 
\sphinxAtStartPar
罠カウンター

\end{itemize}

\sphinxAtStartPar
詳細は次を参照してください。

\sphinxAtStartPar
\hyperref[\detokenize{auto/frameActionlist:act-act-frame}]{\ref{\detokenize{auto/frameActionlist:act-act-frame}} \nameref{\detokenize{auto/frameActionlist:act-act-frame}}}

\sphinxAtStartPar
\sphinxstylestrong{公開レベル:}

\sphinxAtStartPar
このフレームで定義されている領域の公開レベルは次の通りです。
\begin{description}
\sphinxlineitem{パック}
\begin{DUlineblock}{0em}
\item[] 完全公開:開封済み・未開封
\item[] 個人公開:開封後のパックの中身
\item[] 非公開:未開封のパックの中身
\end{DUlineblock}

\sphinxlineitem{レアカード}
\begin{DUlineblock}{0em}
\item[] 完全公開:レアカードの枚数
\item[] 個人公開:レアカードの中身
\item[] 非公開:なし
\end{DUlineblock}

\end{description}


\bigskip\hrule\bigskip



\subsubsection{ストラテジー}
\label{\detokenize{auto/framelist:frame-strategy}}\label{\detokenize{auto/framelist:id6}}
\sphinxAtStartPar
\sphinxstylestrong{配置図:}

\begin{figure}[htbp]
\centering
\capstart

\noindent\sphinxincludegraphics{{rarePack}.pdf}
\caption{配置図\sphinxhyphen{}ストラテジー}\label{\detokenize{auto/framelist:id10}}\label{\detokenize{auto/framelist:frame-strategy-image}}\end{figure}

\sphinxAtStartPar
\sphinxstylestrong{フォーマット:}


\begin{savenotes}\sphinxattablestart
\sphinxthistablewithglobalstyle
\centering
\begin{tabular}[t]{|\X{1}{4}|\X{1}{4}|\X{1}{4}|\X{1}{4}|}
\sphinxtoprule
\sphinxstyletheadfamily 
\sphinxAtStartPar
ライト
&\sphinxstyletheadfamily 
\sphinxAtStartPar
スタンダード
&\sphinxstyletheadfamily 
\sphinxAtStartPar
プロ
&\sphinxstyletheadfamily 
\sphinxAtStartPar
マスター
\\
\sphinxmidrule
\sphinxtableatstartofbodyhook
\sphinxAtStartPar
x
&
\sphinxAtStartPar
x
&
\sphinxAtStartPar
◯
&
\sphinxAtStartPar
◯
\\
\sphinxbottomrule
\end{tabular}
\sphinxtableafterendhook\par
\sphinxattableend\end{savenotes}

\sphinxAtStartPar
\sphinxstylestrong{デッキ条件:}

\sphinxAtStartPar
なし

\sphinxAtStartPar
\sphinxstylestrong{ゲームの始め方:}
\begin{itemize}
\item {} 
\sphinxAtStartPar
デッキから最初の手札に入れる3枚を選ぶ。(ストラテジー)

\item {} 
\sphinxAtStartPar
デッキからレアカード1枚を選び配置図の場所に伏せておく。

\item {} 
\sphinxAtStartPar
残りのデッキをシャッフルし、上から14枚を取り除いてパックとし、配置図の場所に伏せておく。

\item {} 
\sphinxAtStartPar
残りのデッキをライフとし配置図の場所に伏せておく。

\item {} 
\sphinxAtStartPar
ライフの一番上のカードを防壁として場に出し、次のカードを兵士として場に出す。(プリセット)

\item {} 
\sphinxAtStartPar
4枚引いて手札を7枚とする。

\end{itemize}

\sphinxAtStartPar
以降は、 \hyperref[\detokenize{common/common:common-gamestart-first}]{\ref{\detokenize{common/common:common-gamestart-first}} \nameref{\detokenize{common/common:common-gamestart-first}}} 以降に沿ってゲームを開始します。

\sphinxAtStartPar
\sphinxstylestrong{アクションリスト:}

\sphinxAtStartPar
次のアクションが追加で利用可能です。
\begin{itemize}
\item {} 
\sphinxAtStartPar
パック開封

\item {} 
\sphinxAtStartPar
レアドロー

\item {} 
\sphinxAtStartPar
レア召喚

\item {} 
\sphinxAtStartPar
罠カウンター

\end{itemize}

\sphinxAtStartPar
詳細は次を参照してください。

\sphinxAtStartPar
\hyperref[\detokenize{auto/frameActionlist:act-act-frame}]{\ref{\detokenize{auto/frameActionlist:act-act-frame}} \nameref{\detokenize{auto/frameActionlist:act-act-frame}}}

\sphinxAtStartPar
\sphinxstylestrong{公開レベル:}

\sphinxAtStartPar
このフレームで定義されている領域の公開レベルは次の通りです。
\begin{description}
\sphinxlineitem{パック}
\begin{DUlineblock}{0em}
\item[] 完全公開:開封済み・未開封
\item[] 個人公開:開封後のパックの中身
\item[] 非公開:未開封のパックの中身
\end{DUlineblock}

\sphinxlineitem{レアカード}
\begin{DUlineblock}{0em}
\item[] 完全公開:レアカードの枚数
\item[] 個人公開:レアカードの中身
\item[] 非公開:なし
\end{DUlineblock}

\end{description}

\sphinxAtStartPar
各フレームで追加で使えるアクションについては次を参照してください。

\sphinxstepscope


\subsection{フレームアクションリスト}
\label{\detokenize{auto/frameActionlist:act-act-frame}}\label{\detokenize{auto/frameActionlist:id1}}\label{\detokenize{auto/frameActionlist::doc}}
\begin{sphinxShadowBox}
\sphinxstyletopictitle{目次}
\begin{itemize}
\item {} 
\sphinxAtStartPar
\phantomsection\label{\detokenize{auto/frameActionlist:id8}}{\hyperref[\detokenize{auto/frameActionlist:id3}]{\sphinxcrossref{パック}}}
\begin{itemize}
\item {} 
\sphinxAtStartPar
\phantomsection\label{\detokenize{auto/frameActionlist:id9}}{\hyperref[\detokenize{auto/frameActionlist:act-packopen}]{\sphinxcrossref{パック開封 (パック)}}}

\item {} 
\sphinxAtStartPar
\phantomsection\label{\detokenize{auto/frameActionlist:id10}}{\hyperref[\detokenize{auto/frameActionlist:act-raredraw}]{\sphinxcrossref{レアドロー (パック)}}}

\item {} 
\sphinxAtStartPar
\phantomsection\label{\detokenize{auto/frameActionlist:id11}}{\hyperref[\detokenize{auto/frameActionlist:act-summonsrare}]{\sphinxcrossref{レア召喚 (パック)}}}

\item {} 
\sphinxAtStartPar
\phantomsection\label{\detokenize{auto/frameActionlist:id12}}{\hyperref[\detokenize{auto/frameActionlist:act-trapcounter}]{\sphinxcrossref{罠カウンター (パック)}}}

\end{itemize}

\end{itemize}
\end{sphinxShadowBox}


\subsubsection{パック}
\label{\detokenize{auto/frameActionlist:id3}}

\paragraph{パック開封 (パック)}
\label{\detokenize{auto/frameActionlist:act-packopen}}\label{\detokenize{auto/frameActionlist:id4}}
\sphinxAtStartPar
\sphinxstylestrong{フレーム:}

\sphinxAtStartPar
・パック

\sphinxAtStartPar
・レアパック

\sphinxAtStartPar
・ストラテジー

\sphinxAtStartPar
\sphinxstylestrong{トリガー:} 直接

\sphinxAtStartPar
\sphinxstylestrong{スピード:} 即時

\sphinxAtStartPar
\sphinxstylestrong{タイミング:} クイック

\sphinxAtStartPar
\sphinxstylestrong{起動条件:}

\sphinxAtStartPar
パックが開封済みでない場合のみこのアクションを起こすことができる。

\sphinxAtStartPar
\sphinxstylestrong{効果:}
\begin{enumerate}
\sphinxsetlistlabels{\arabic}{enumi}{enumii}{}{.}%
\item {} 
\sphinxAtStartPar
パックの中から好きなカードを1枚選び対戦相手に見せ手札に加える。レアカードがある場合、代わりにそれを対戦相手に見せ手札に加えてもよい。

\item {} 
\sphinxAtStartPar
パックを表向きにし、開封済みとする。

\end{enumerate}


\bigskip\hrule\bigskip



\paragraph{レアドロー (パック)}
\label{\detokenize{auto/frameActionlist:act-raredraw}}\label{\detokenize{auto/frameActionlist:id5}}
\sphinxAtStartPar
\sphinxstylestrong{フレーム:}

\sphinxAtStartPar
・パック

\sphinxAtStartPar
・レアパック

\sphinxAtStartPar
・ストラテジー

\sphinxAtStartPar
\sphinxstylestrong{トリガー:} 直接

\sphinxAtStartPar
\sphinxstylestrong{スピード:} 即時

\sphinxAtStartPar
\sphinxstylestrong{タイミング:} クイック

\sphinxAtStartPar
\sphinxstylestrong{起動条件:}

\sphinxAtStartPar
プレイヤーのライフが9以下の場合にしかこのアクションを起こすことができない。

\sphinxAtStartPar
\sphinxstylestrong{効果:}

\sphinxAtStartPar
レアカードを1枚手札に加える。


\bigskip\hrule\bigskip



\paragraph{レア召喚 (パック)}
\label{\detokenize{auto/frameActionlist:act-summonsrare}}\label{\detokenize{auto/frameActionlist:id6}}
\sphinxAtStartPar
\sphinxstylestrong{フレーム:}

\sphinxAtStartPar
・パック

\sphinxAtStartPar
・レアパック

\sphinxAtStartPar
・ストラテジー

\sphinxAtStartPar
\sphinxstylestrong{トリガー:} 直接

\sphinxAtStartPar
\sphinxstylestrong{スピード:} 通常

\sphinxAtStartPar
\sphinxstylestrong{タイミング:} クイック

\sphinxAtStartPar
\sphinxstylestrong{コスト:} S

\sphinxAtStartPar
\sphinxstylestrong{特記事項:} ※このアクションが起こされた時、キーカード(レアカード)はステージ上に置かれ完全公開となる。

\sphinxAtStartPar
\sphinxstylestrong{効果:}

\sphinxAtStartPar
キーカードを兵士として表向きかつチャージ状態で場に出す。


\bigskip\hrule\bigskip



\paragraph{罠カウンター (パック)}
\label{\detokenize{auto/frameActionlist:act-trapcounter}}\label{\detokenize{auto/frameActionlist:id7}}
\sphinxAtStartPar
\sphinxstylestrong{フレーム:}

\sphinxAtStartPar
・パック

\sphinxAtStartPar
・レアパック

\sphinxAtStartPar
・ストラテジー

\sphinxAtStartPar
\sphinxstylestrong{トリガー:} 誘発

\sphinxAtStartPar
\sphinxstylestrong{スピード:} 通常

\sphinxAtStartPar
\sphinxstylestrong{タイミング:} クイック

\sphinxAtStartPar
\sphinxstylestrong{特記事項:} ※このアクションが起こされた時、キーカード(レアカード)はステージ上に置かれ完全公開となる。

\sphinxAtStartPar
\sphinxstylestrong{対象:}

\sphinxAtStartPar
キーカードが1枚または2枚のアクションを対象とする。

\sphinxAtStartPar
\sphinxstylestrong{効果:}

\sphinxAtStartPar
対象アクションのキーカードのいずれかがこのアクションのキーカードと同じ場合、対象アクションをステージから取り除き、対象アクションのキーカードを墓地に移す。

\sphinxstepscope


\chapter{コアルール}
\label{\detokenize{core/core:core-rst}}\label{\detokenize{core/core:id1}}\label{\detokenize{core/core::doc}}
\sphinxAtStartPar
トランプのみでトレーディングカードゲームのように遊ぶために、
一般的なトレーディングカードゲームを抽象化し、トランプのみで遊べるように再構築しました。

\sphinxAtStartPar
コアルールは、割込みが可能なターン制ゲームの開始から勝敗が決まるまでを定義します。(\hyperref[\detokenize{core/core:abstract-core-rule}]{Fig.\@ \ref{\detokenize{core/core:abstract-core-rule}}})

\sphinxAtStartPar
BlackPokerはコアルールを実装しているため、ルールを変更する場合は、コアルールに従う必要があります。

\sphinxAtStartPar
このルールは、割込みが可能なターン制ゲームの標準化したモデルとして構築されており、他のゲームへの応用も可能です。

\begin{figure}[htbp]
\centering
\capstart

\noindent\sphinxincludegraphics[scale=0.5]{{plantuml-5869adc776d20b7021b88175e8de80f0f436330c}.pdf}
\caption{コアルールのイメージ}\label{\detokenize{core/core:id34}}\label{\detokenize{core/core:abstract-core-rule}}\end{figure}


\section{基礎概念}
\label{\detokenize{core/core:id2}}
\sphinxAtStartPar
割込み可能なターン制ゲームでは、現実のゲーム盤面に影響を与える前に、
どの行動を先に行うかを決める仮想的な場所があります。

\sphinxAtStartPar
割込み可能なターン制ゲームは、見方を変えると“許可制のゲーム”とも表現できます。

\sphinxAtStartPar
つまり、プレイヤーは相手に許可を得て、自分の行動を実行できるかどうか確認しながら進める形式のゲームです。

\sphinxAtStartPar
いくつかの基礎概念を紹介します。
その後、それらを組み合わせたイメージを説明します。

\index{ターン@\spxentry{ターン}}\ignorespaces 

\subsection{ターン}
\label{\detokenize{core/core:index-0}}\label{\detokenize{core/core:id3}}
\sphinxAtStartPar
プレイヤーはターンを“持つ”ことができます。
ターンを持っているプレイヤーは先に行動できます。
ターンを持っているプレイヤーをターンプレイヤーといいます。

\index{アクション(コア)@\spxentry{アクション(コア)}}\ignorespaces 

\subsection{アクション}
\label{\detokenize{core/core:index-1}}\label{\detokenize{core/core:id4}}
\sphinxAtStartPar
ターン制のゲームでは、プレイヤーは様々な行動を行います。
チェスであればコマを進めたり、ババ抜きであれば隣の人からカードを引いたりします。
また、ゲームのルールシステムが行う行動もあります。それらの行動をアクションと定義します。
プレイヤーが行う行動を直接アクション、ルールシステムが行うアクションを誘発アクションといいます。

\index{リクエスト@\spxentry{リクエスト}}\ignorespaces 

\subsection{リクエスト}
\label{\detokenize{core/core:index-2}}\label{\detokenize{core/core:id5}}
\sphinxAtStartPar
アクションを実行するために、ゲームシステム(コアフロー)に対して
どのようなアクションを実行するかを要求します。
この要求をリクエストと呼びます。正式名称はアクションリクエストです。

\sphinxAtStartPar
リクエストの種類によっては、ステージに蓄積されず、すぐに実行(解決)されるものもあります。


\subsection{解決}
\label{\detokenize{core/core:id6}}
\sphinxAtStartPar
リクエストを処理すること(要求されたアクションを実行すること)を解決と言います。

\index{チャンス@\spxentry{チャンス}}\ignorespaces 

\subsection{チャンス}
\label{\detokenize{core/core:index-3}}\label{\detokenize{core/core:id7}}
\sphinxAtStartPar
プレイヤーがリクエストできる機会をチャンスといいます。
チャンスを持っている間は何度でもリクエストすることができます。
正式名称はアクションチャンスです。

\index{ステージ@\spxentry{ステージ}}\ignorespaces 

\subsection{ステージ}
\label{\detokenize{core/core:index-4}}\label{\detokenize{core/core:id8}}
\sphinxAtStartPar
ステージはリクエストの解決順を整理するために使う領域です。
後入れ先出し方式で、最後に積まれたリクエストから順に解決されます。
リクエストの種類によっては、ステージに蓄積されず、すぐに解決されるものもあります。

\index{コンポーネント@\spxentry{コンポーネント}}\ignorespaces 

\subsection{コンポーネント}
\label{\detokenize{core/core:component}}\label{\detokenize{core/core:index-5}}\label{\detokenize{core/core:id9}}
\sphinxAtStartPar
コンポーネントとは、ゲーム盤面に配置されるモンスターや駒の定義です。
例えば、将棋の「歩」は駒としては複数存在しますが、「歩」の定義は1つです。
この定義をコンポーネントといいます。

\index{コンポーネントインスタンス@\spxentry{コンポーネントインスタンス}}\ignorespaces 

\subsection{コンポーネントインスタンス}
\label{\detokenize{core/core:index-6}}\label{\detokenize{core/core:id10}}
\sphinxAtStartPar
コンポーネントとして定義されたものがゲーム盤面に配置されているとき、それをコンポーネントインスタンスといいます。
例えば、将棋の「歩」はコンポーネントとして定義されていますが、 実際の盤面に配置される「歩」は、コンポーネント定義という設計図から作られたコンポーネントインスタンスです。


\subsection{コアフロー(ルールシステム)}
\label{\detokenize{core/core:id11}}
\sphinxAtStartPar
リクエストはコアフロー(ルールシステム)によって整理され、処理(解決)されます。
リクエストは即時解決されるものと、ステージに蓄積されるものに分類され、順番に処理されます。

\sphinxAtStartPar
ゲームによって具体的に行う内容は異なりますが、処理する順番の制御はコアフローが担います。

\sphinxAtStartPar
コアフローはゲームの開始から勝敗が決まるまで動作し続けます。


\section{詳細}
\label{\detokenize{core/core:id12}}
\sphinxAtStartPar
基礎概念を図で表すと次のようになります。(\hyperref[\detokenize{core/core:abstract-core-image}]{Fig.\@ \ref{\detokenize{core/core:abstract-core-image}}})

\begin{figure}[htbp]
\centering
\capstart

\noindent\sphinxincludegraphics{{abstract}.pdf}
\caption{割込み可能なターン制ゲーム}\label{\detokenize{core/core:id35}}\label{\detokenize{core/core:abstract-core-image}}\end{figure}

\sphinxAtStartPar
仮想的な場所でリクエストを整理し、現実のゲーム盤面に承認された順で変更を反映します。

\sphinxAtStartPar
リクエストを整理することで割込みを実現しています。どのようにリクエストを処理するかは、コアフローに従います。

\sphinxAtStartPar
さらに、アクションとリクエスト、コンポーネントとコンポーネントインスタンスの関係は次のようになります。(\hyperref[\detokenize{core/core:action-request-image}]{Fig.\@ \ref{\detokenize{core/core:action-request-image}}})

\begin{figure}[htbp]
\centering
\capstart

\noindent\sphinxincludegraphics[scale=0.5]{{plantuml-cef0f4720204bfabdb7f2ab72dc494cfa1172912}.pdf}
\caption{リクエストとコンポーネントインスタンスの関係}\label{\detokenize{core/core:id36}}\label{\detokenize{core/core:action-request-image}}\end{figure}

\sphinxAtStartPar
ゲーム盤面には複数のコンポーネントインスタンスが生成されます。

\sphinxAtStartPar
リクエストが解決されるたびに、コンポーネントインスタンスが生成されたり、既存のコンポーネントインスタンスの状態が変化したりします。

\sphinxAtStartPar
ここからは、アクションとコンポーネントの各項目について説明します。


\subsection{アクションの定義項目}
\label{\detokenize{core/core:id13}}
\sphinxAtStartPar
アクションを定義する際には、次の項目を設定する必要があります。
その他の項目は、具体的なアクションに応じて追加してください。
\begin{itemize}
\item {} 
\sphinxAtStartPar
オーナー

\item {} 
\sphinxAtStartPar
トリガー

\item {} 
\sphinxAtStartPar
スピード

\item {} 
\sphinxAtStartPar
タイミング

\item {} 
\sphinxAtStartPar
起動条件

\item {} 
\sphinxAtStartPar
誘発条件

\item {} 
\sphinxAtStartPar
効果

\end{itemize}

\index{トリガー(アクション)@\spxentry{トリガー(アクション)}}\ignorespaces 

\subsubsection{オーナー}
\label{\detokenize{core/core:index-7}}\label{\detokenize{core/core:id14}}
\sphinxAtStartPar
アクションの所有者。
アクションの定義はプレイヤー事に保持するため、同じ内容のアクションでもオーナーが異なります。

\index{トリガー(コア)@\spxentry{トリガー(コア)}}\ignorespaces 

\subsubsection{トリガー}
\label{\detokenize{core/core:trigger-core}}\label{\detokenize{core/core:index-8}}\label{\detokenize{core/core:id15}}
\sphinxAtStartPar
アクションが要求される方法は、大きく分けて次の2種類に分類されます。
\begin{description}
\sphinxlineitem{\sphinxstylestrong{直接}}
\sphinxAtStartPar
プレイヤーがコストを支払うなどの手続きを経てリクエストされるアクション。

\sphinxlineitem{\sphinxstylestrong{誘発}}
\sphinxAtStartPar
条件が満たされた場合、自動でリクエストされるアクション。

\end{description}

\sphinxAtStartPar
トリガー項目には「直接」または「誘発」のいずれかが設定されます。

\index{スピード(コア)@\spxentry{スピード(コア)}}\ignorespaces 

\subsubsection{スピード}
\label{\detokenize{core/core:speed-core}}\label{\detokenize{core/core:index-9}}\label{\detokenize{core/core:id16}}
\sphinxAtStartPar
リクエストが処理される速度は次の2種類に分類されます。
\begin{description}
\sphinxlineitem{\sphinxstylestrong{即時}}
\sphinxAtStartPar
リクエストはステージを用いず解決されます。

\sphinxlineitem{\sphinxstylestrong{通常}}
\sphinxAtStartPar
リクエストはステージを経由して解決されます。

\end{description}

\sphinxAtStartPar
スポード項目には「即時」または「通常」のいずれかが設定されます。

\index{タイミング(コア)@\spxentry{タイミング(コア)}}\ignorespaces 

\subsubsection{タイミング}
\label{\detokenize{core/core:timing}}\label{\detokenize{core/core:index-10}}\label{\detokenize{core/core:id17}}
\sphinxAtStartPar
タイミングとは、アクションがリクエストされるタイミングを示します。
タイミングには次の2種類があります。

\index{メイン@\spxentry{メイン}}\ignorespaces \begin{description}
\sphinxlineitem{\sphinxstylestrong{メイン}}
\sphinxAtStartPar
ターンプレイヤーかつステージが空のときに起こせるアクション。
実行条件:
\sphinxhyphen{} チャンスを持っている
\sphinxhyphen{} 自分のターンである
\sphinxhyphen{} ステージが空である

\end{description}

\index{クイック@\spxentry{クイック}}\ignorespaces \begin{description}
\sphinxlineitem{\sphinxstylestrong{クイック}}
\sphinxAtStartPar
いつでも起こすことができ、アクションをステージに積み重ねることが可能。
実行条件:
\sphinxhyphen{} チャンスを持っている

\end{description}

\begin{sphinxadmonition}{note}{注釈:}
\sphinxAtStartPar
エンドアクションの定義

\sphinxAtStartPar
最低1つはターンを別のプレイヤーに渡すアクションを定義してください。
これがないと、ターンが進行せずゲームが停止する可能性があります。
\end{sphinxadmonition}

\begin{sphinxadmonition}{note}{注釈:}
\sphinxAtStartPar
アクションのコントローラー

\sphinxAtStartPar
アクションを実行したプレイヤーを \sphinxstylestrong{アクションのコントローラー} と呼びます。
効果の解釈は、このコントローラーの視点で行われます。
\end{sphinxadmonition}

\index{き\textbar{}起動条件(コア)@\spxentry{き\textbar{}起動条件(コア)}}\ignorespaces 

\subsubsection{起動条件}
\label{\detokenize{core/core:index-13}}\label{\detokenize{core/core:id18}}
\sphinxAtStartPar
アクションを起こす(リクエストする)ための条件を示します。

\sphinxAtStartPar
トリガーが「直接」の場合、起動条件が定義されます。
コストの支払いや対象の指定など、様々な条件がアクションごとに設定されます。

\sphinxAtStartPar
BlackPokerでは、コストの支払いや対象の指定の記述が冗長にならないよう、省略されることが多いです。

\index{ゆ\textbar{}誘発条件(コア)@\spxentry{ゆ\textbar{}誘発条件(コア)}}\ignorespaces 

\subsubsection{誘発条件}
\label{\detokenize{core/core:index-14}}\label{\detokenize{core/core:id19}}
\sphinxAtStartPar
アクションが誘発される条件を示します。
条件が満たされると自動的にリクエストされます。

\sphinxAtStartPar
トリガーが「誘発」の場合、この項目が定義されます。

\sphinxAtStartPar
例:
\sphinxhyphen{} ダメージを受けたとき
\sphinxhyphen{} カードが墓地に移動したとき

\sphinxAtStartPar
これらの状況で誘発するアクションが設定されることがあります。

\index{こ\textbar{}効果(コア)@\spxentry{こ\textbar{}効果(コア)}}\ignorespaces 

\subsubsection{効果}
\label{\detokenize{core/core:index-15}}\label{\detokenize{core/core:id20}}
\sphinxAtStartPar
効果とは、アクションが解決された際に実行される処理を指します。


\subsection{リクエストの定義項目}
\label{\detokenize{core/core:id21}}\begin{description}
\sphinxlineitem{\sphinxstylestrong{アクション定義}}
\sphinxAtStartPar
リクエストが解決された際に実行されるアクションの内容。

\sphinxlineitem{\sphinxstylestrong{スピード}}
\sphinxAtStartPar
アクション定義のスピードとなります。

\sphinxlineitem{\sphinxstylestrong{タイミング}}
\sphinxAtStartPar
アクション定義のタイミングとなります。

\sphinxlineitem{\sphinxstylestrong{コントローラー}}
\sphinxAtStartPar
アクションを実行したプレイヤーがリクエストのコントローラーとなります。

\end{description}


\subsection{コンポーネントの定義項目}
\label{\detokenize{core/core:id22}}
\sphinxAtStartPar
コンポーネントには、次の項目が定義されます。
必要に応じて、ゲームに合わせた追加設定をしてください。
\begin{description}
\sphinxlineitem{\sphinxstylestrong{オーナー}}
\sphinxAtStartPar
コンポーネンの所有者。
コンポーネントの定義はプレイヤー事に保持するため、同じ内容のコンポーネントでもオーナーが異なります。

\sphinxlineitem{\sphinxstylestrong{能力}}
\sphinxAtStartPar
能力の詳細については、後述のセクションを参照してください。(\hyperref[\detokenize{core/core:ability}]{\ref{\detokenize{core/core:ability}} \nameref{\detokenize{core/core:ability}}})

\end{description}


\subsection{コンポーネントインスタンスの定義項目}
\label{\detokenize{core/core:id23}}
\sphinxAtStartPar
コンポーネントインスタンスには、次の項目が設定されます。
必要に応じて、ゲームに合わせた追加設定をしてください。
\begin{description}
\sphinxlineitem{\sphinxstylestrong{コンポーネント定義}}
\sphinxAtStartPar
どのコンポーネント定義から生成されたのかを保持します。

\end{description}

\index{オーナー@\spxentry{オーナー}}\ignorespaces \begin{description}
\sphinxlineitem{\sphinxstylestrong{オーナー}}
\sphinxAtStartPar
コンポーネントインスタンスの所有者。
一般的なトランプゲームでは無視されることが多いですが、TCGのようにデッキを個人所有するゲームでは必要な情報です。

\end{description}

\index{コントローラー@\spxentry{コントローラー}}\ignorespaces \begin{description}
\sphinxlineitem{\sphinxstylestrong{コントローラー}}
\sphinxAtStartPar
現在、そのコンポーネントインスタンスを操作しているプレイヤー。
通常はオーナーとコントローラーは同じですが、コントロールを奪うアクションがある場合、異なることがあります。

\end{description}

\begin{sphinxadmonition}{note}{注釈:}
\sphinxAtStartPar
コンポーネントインスタンスとリクエストのコントローラー

\sphinxAtStartPar
コントローラーは制御している人という意味になるため、コンポーネントインスタンスとリクエストのコントローラーは制御する対象が異なります。
コンポーネントインスタンスとリクエストの属性を次の図に示します。(\hyperref[\detokenize{core/core:controller-attr}]{Fig.\@ \ref{\detokenize{core/core:controller-attr}}})
\end{sphinxadmonition}

\begin{figure}[htbp]
\centering
\capstart

\noindent\sphinxincludegraphics[scale=0.5]{{plantuml-601946f3f8eb470fbf97e87d640335e4c349780a}.pdf}
\caption{コントローラー属性}\label{\detokenize{core/core:id37}}\label{\detokenize{core/core:controller-attr}}\end{figure}

\index{の\textbar{}能力(コア)@\spxentry{の\textbar{}能力(コア)}}\ignorespaces 

\section{能力}
\label{\detokenize{core/core:ability}}\label{\detokenize{core/core:index-18}}\label{\detokenize{core/core:id24}}
\sphinxAtStartPar
アクション、コンポーネントの定義項目を見てきました。
これらとは別の概念である \sphinxstylestrong{能力} について説明します。

\sphinxAtStartPar
能力とはアクションの効果とは異なる概念で、アクションを起こす際や効果を解釈する際に参照されます。

\sphinxAtStartPar
能力は解釈される際にコストは支払われず、ステージに置かれません。

\sphinxAtStartPar
能力を持つことができるのは、プレイヤーの他に駒やカードなどのゲームに登場するコンポーネントも含まれます。
(\hyperref[\detokenize{core/core:ability-image}]{Fig.\@ \ref{\detokenize{core/core:ability-image}}})

\begin{figure}[htbp]
\centering
\capstart

\noindent\sphinxincludegraphics[scale=0.5]{{plantuml-5c3b9d2c43d8740600145b89955601a615170288}.pdf}
\caption{能力のイメージ}\label{\detokenize{core/core:id38}}\label{\detokenize{core/core:ability-image}}\end{figure}

\begin{sphinxadmonition}{note}{注釈:}
\sphinxAtStartPar
7版までは、能力に誘発能力と常在型能力がありました。
8版からは、誘発型能力とアクションを起こせる能力をアクションの定義側に移動しました。
能力はそれ以外の常在型能力を示すものになりました。
\end{sphinxadmonition}

\index{コアフロー@\spxentry{コアフロー}}\ignorespaces 

\section{コアフロー}
\label{\detokenize{core/core:coreflowsec}}\label{\detokenize{core/core:index-19}}\label{\detokenize{core/core:id25}}
\sphinxAtStartPar
今まで説明してきた概念を用いて \sphinxstylestrong{コアフローの具体的な処理} を説明します。
この図は \sphinxstylestrong{ゲームの開始から勝敗が決まるまでの流れ(コアフロー)} を示しています。(\hyperref[\detokenize{core/core:coreflow-2}]{Fig.\@ \ref{\detokenize{core/core:coreflow-2}}})

\sphinxAtStartPar
BlackPokerはこのコアフローに則りリクエストが処理されます。

\sphinxAtStartPar
アナログゲーム用に作成したコアフローであるため、なるべく記憶する容量を減らすように設計しています。
デジタルゲームに応用する場合は、細部をゲームに合わせて変更してください。

\begin{figure}[htbp]
\centering
\capstart

\noindent\sphinxincludegraphics[scale=0.5]{{plantuml-56e0b8d8835bda254d3bb4ae5813ef3482975b0f}.pdf}
\caption{コアフロー}\label{\detokenize{core/core:id39}}\label{\detokenize{core/core:coreflow-2}}\end{figure}
\phantomsection\label{\detokenize{core/core:core-gamestart}}\begin{description}
\sphinxlineitem{\sphinxstylestrong{{[}1{]} ゲーム開始}}
\sphinxAtStartPar
先攻を決め、ゲームを始める準備を行います。

\sphinxlineitem{\sphinxstylestrong{{[}2{]} ターンプレイヤーにチャンスを移動}}
\sphinxAtStartPar
ターンを持っているプレイヤーにチャンスを移動します。

\sphinxlineitem{\sphinxstylestrong{{[}3{]} アクションをリクエストするか?}}
\sphinxAtStartPar
チャンスを持っているプレイヤーはアクションを起こすかを判断します。

\sphinxlineitem{\sphinxstylestrong{{[}4{]} パス記録のリセット}}
\sphinxAtStartPar
パスしたプレイヤーの記録をリセットします。

\sphinxlineitem{\sphinxstylestrong{{[}5{]} アクションをリクエストする}}
\sphinxAtStartPar
アクションをリクエストし、これからプレイヤーが行うことを宣言します。
ゲームによってアクションをリクエストする方法は異なります。
BlackPokerではアクション名を言い、コストの支払や対象を指定しアクションをリクエストします。
一方ババ抜きでは、隣のプレイヤーからカードを引く際に宣言せず暗黙にアクションがリクエストされている場合もあります。

\sphinxAtStartPar
このときに有効になっている能力を考慮します。
能力の適用順については \hyperref[\detokenize{core/core:ability-order}]{\ref{\detokenize{core/core:ability-order}} \nameref{\detokenize{core/core:ability-order}}} 参照してください。

\sphinxlineitem{\sphinxstylestrong{{[}6{]} 誘発チェック}}
\sphinxAtStartPar
ここに至るまでに誘発したアクションがないかチェックします。
誘発した場合、対象のアクションをリクエストします。
詳しいフローは \hyperref[\detokenize{core/core:trigger-check}]{\ref{\detokenize{core/core:trigger-check}} \nameref{\detokenize{core/core:trigger-check}}} を参照してください。

\sphinxlineitem{\sphinxstylestrong{{[}7{]} 即時か?}}
\sphinxAtStartPar
リクエストのスピードが即時か判定します。

\end{description}
\phantomsection\label{\detokenize{core/core:actresolve}}\begin{description}
\sphinxlineitem{\sphinxstylestrong{{[}8{]} リクエストの解決}}
\sphinxAtStartPar
アクションの効果に定義されている内容を実行します。
その他にコンポーネントを捨て山に移動するなどゲームによって決まった処理があれば行います。
アクションの解決の中でも効果に定義されている内容を実行することのみを指す場合「効果を発揮する」と言います。

\sphinxAtStartPar
このときに有効になっている能力を考慮します。
能力の適用順については \hyperref[\detokenize{core/core:ability-order}]{\ref{\detokenize{core/core:ability-order}} \nameref{\detokenize{core/core:ability-order}}} 参照してください。

\end{description}
\phantomsection\label{\detokenize{core/core:winlose}}\begin{description}
\sphinxlineitem{\sphinxstylestrong{{[}9{]} 勝敗判定}}
\sphinxAtStartPar
ゲームの勝敗を判定します。決着した場合ゲームが終了します。判定の方法はゲームにより異なります。

\sphinxlineitem{\sphinxstylestrong{{[}10{]} ステージに追加}}
\sphinxAtStartPar
リクエストをステージに追加します。

\sphinxlineitem{\sphinxstylestrong{{[}11{]} パス記録に登録}}
\sphinxAtStartPar
パスしたプレイヤーを記録します。パス記録がリセットされるため、同じプレイヤー名は2回登録されません。

\sphinxlineitem{\sphinxstylestrong{{[}12{]} 全員がパスしたか?}}
\sphinxAtStartPar
パス記録に全てのプレイヤー名が記録されているか判定します。

\sphinxlineitem{\sphinxstylestrong{{[}13{]} ルールシステムにチャンスを移動}}
\sphinxAtStartPar
ルールシステムにチャンスを移動します。

\sphinxlineitem{\sphinxstylestrong{{[}14{]} ステージにリクエストが存在するか?}}
\sphinxAtStartPar
ステージにリクエストが存在するか判定します。

\sphinxlineitem{\sphinxstylestrong{{[}15{]} 最後のリクエストを解決}}
\sphinxAtStartPar
最後にステージに追加されたリクエストを解決します。
解決方法は {\hyperref[\detokenize{core/core:actresolve}]{\sphinxcrossref{\DUrole{std,std-ref}{{[}8{]} リクエストの解決}}}} 参照してください。
ステージ上のリクエストを解決する場合、 {\hyperref[\detokenize{core/core:actresolve}]{\sphinxcrossref{\DUrole{std,std-ref}{{[}8{]} リクエストの解決}}}} を行った後、次の内容も合わせて行います。
\begin{itemize}
\item {} 
\sphinxAtStartPar
リクエストをステージから取り除く

\end{itemize}

\sphinxlineitem{\sphinxstylestrong{{[}16{]} チャンス移動}}
\sphinxAtStartPar
チャンスを持っているプレイヤーからチャンスを持っていないプレイヤーにチャンスを移動します。
チャンスを移動するルールはゲームによって異なります。

\end{description}


\subsection{誘発チェック}
\label{\detokenize{core/core:trigger-check}}\label{\detokenize{core/core:id26}}
\sphinxAtStartPar
アクションの中には誘発条件を持っているアクションがあります。
誘発条件に該当した場合、アクションからリクエストが誘発されます。

\sphinxAtStartPar
誘発チェックでは、誘発したリクエストを解決またはステージに追加します。
誘発したリクエストのコントローラーは起因となったアクションのオーナーがコントローラーとなります。
誘発チェックは次の図のように行います。(\hyperref[\detokenize{core/core:trigger-flow}]{Fig.\@ \ref{\detokenize{core/core:trigger-flow}}})

\begin{sphinxadmonition}{note}{注釈:}
\sphinxAtStartPar
バッファ

\sphinxAtStartPar
誘発したリクエストを一時的に溜めておくバッファという領域があります。正式名称はアクションバッファです。
\end{sphinxadmonition}

\begin{figure}[htbp]
\centering
\capstart

\noindent\sphinxincludegraphics[scale=0.5]{{plantuml-6f625fd3091913318d2a09c4f5dda21eb1e08d44}.pdf}
\caption{誘発チェック}\label{\detokenize{core/core:id40}}\label{\detokenize{core/core:trigger-flow}}\end{figure}
\phantomsection\label{\detokenize{core/core:trigger-act-gather}}\begin{description}
\sphinxlineitem{\sphinxstylestrong{{[}6\sphinxhyphen{}1{]} 誘発したリクエストを分類しバッファに追加}}
\sphinxAtStartPar
各プレイヤーが誘発させたリクエストを、スピードおよび
タイミングに基づいて分類し、一旦バッファに追加します。

\sphinxlineitem{\sphinxstylestrong{{[}6\sphinxhyphen{}2{]} バッファは空か?}}
\sphinxAtStartPar
バッファが空であるかどうかを判定します。
未処理のリクエストが残っている場合は、以降の処理ループを継続します。

\sphinxlineitem{\sphinxstylestrong{{[}6\sphinxhyphen{}3{]} バッファに即時はあるか?}}
\sphinxAtStartPar
バッファ内にスピードが即時のリクエストが存在するかを判定します。
存在する場合、
ターンプレイヤーから順に即時の処理グループへ進みます。

\sphinxlineitem{\sphinxstylestrong{{[}6\sphinxhyphen{}4{]} 該当プレイヤーに即時があるか?}}
\sphinxAtStartPar
現在処理対象となっているプレイヤーがコントローラーとなっているリクエストがバッファにあるかを確認します。
スピードが即時のリクエストが存在するかを判定し、
存在しない場合は、そのプレイヤーでの処理を終了し、次のプレイヤーへ移行します。

\sphinxlineitem{\sphinxstylestrong{{[}6\sphinxhyphen{}5{]} タイミング=メインの即時アクションを処理}}
\sphinxAtStartPar
該当プレイヤーについて、タイミングが「メイン」、スピードが「即時」のアクションを実行します。
詳細は \hyperref[\detokenize{core/core:trigger-act-s}]{\ref{\detokenize{core/core:trigger-act-s}} \nameref{\detokenize{core/core:trigger-act-s}}} をタイミング=メインとして参照してください。

\sphinxlineitem{\sphinxstylestrong{{[}6\sphinxhyphen{}6{]} タイミング=クイックの即時アクションを処理}}
\sphinxAtStartPar
同じプレイヤーについて、タイミングが「クイック」、スピードが「即時」のアクションを実行します。
詳細は \hyperref[\detokenize{core/core:trigger-act-s}]{\ref{\detokenize{core/core:trigger-act-s}} \nameref{\detokenize{core/core:trigger-act-s}}} をタイミング=クイックとして参照してください。

\sphinxlineitem{\sphinxstylestrong{{[}6\sphinxhyphen{}7{]} 次のプレイヤーへ}}
\sphinxAtStartPar
現在のプレイヤーでの処理が完了した後、
ターン順に次のプレイヤーへ処理を移行します。

\sphinxlineitem{\sphinxstylestrong{{[}6\sphinxhyphen{}8{]} バッファに通常はあるか?}}
\sphinxAtStartPar
バッファ内にスピードが通常のリクエストが存在するかを判定します。
存在する場合、
ターンプレイヤーから順に即時の処理グループへ進みます。

\sphinxlineitem{\sphinxstylestrong{{[}6\sphinxhyphen{}9{]} 該当プレイヤーに通常があるか?}}
\sphinxAtStartPar
現在処理対象となっているプレイヤーがコントローラーとなっているリクエストがバッファにあるかを確認します。
スピードが即時のリクエストが存在するかを判定し、
存在しない場合は、そのプレイヤーでの処理を終了し、次のプレイヤーへ移行します。

\sphinxlineitem{\sphinxstylestrong{{[}6\sphinxhyphen{}10{]} タイミング=メインの通常アクションを処理}}
\sphinxAtStartPar
該当プレイヤーについて、タイミングが「メイン」、スピードが「通常」のアクションを実行します。
詳細は \hyperref[\detokenize{core/core:trigger-act-n}]{\ref{\detokenize{core/core:trigger-act-n}} \nameref{\detokenize{core/core:trigger-act-n}}} をタイミング=メインとして参照してください。

\sphinxlineitem{\sphinxstylestrong{{[}6\sphinxhyphen{}11{]} タイミング=クイックの通常アクションを処理}}
\sphinxAtStartPar
同じプレイヤーについて、タイミングが「クイック」、スピードが「通常」のアクションを実行します。
詳細は \hyperref[\detokenize{core/core:trigger-act-n}]{\ref{\detokenize{core/core:trigger-act-n}} \nameref{\detokenize{core/core:trigger-act-n}}} をタイミング=クイックとして参照してください。

\sphinxlineitem{\sphinxstylestrong{{[}6\sphinxhyphen{}12{]} 次のプレイヤーへ}}
\sphinxAtStartPar
現在のプレイヤーでの処理が完了した後、
ターン順に次のプレイヤーへ処理を移行します。

\end{description}

\begin{sphinxadmonition}{note}{注釈:}
\sphinxAtStartPar
各処理グループ内では、必ずターンプレイヤーから始まり、
ターンが回る順に全プレイヤーに対して確認およびアクションの処理を実施します。
また、ループはバッファに未処理のアクションが存在する限り繰り返されます。
\end{sphinxadmonition}


\subsubsection{誘発即時解決}
\label{\detokenize{core/core:trigger-act-s}}\label{\detokenize{core/core:id27}}
\sphinxAtStartPar
誘発チェックで誘発したスピードが即時のリクエストを処理します。
呼び出し元で指定されたプレイヤーおよびタイミングに基づいて処理します。
誘発チェックは次の図のように行います。(\hyperref[\detokenize{core/core:trigger-flow-s}]{Fig.\@ \ref{\detokenize{core/core:trigger-flow-s}}})

\begin{figure}[htbp]
\centering
\capstart

\noindent\sphinxincludegraphics[scale=0.5]{{plantuml-caaca34586e95540adb102e9bc8309243f59b393}.pdf}
\caption{誘発チェック\sphinxhyphen{}即時処理}\label{\detokenize{core/core:id41}}\label{\detokenize{core/core:trigger-flow-s}}\end{figure}
\begin{description}
\sphinxlineitem{\sphinxstylestrong{{[}6\sphinxhyphen{}5\sphinxhyphen{}1{]} バッファから該当の即時はあるか?}}
\sphinxAtStartPar
バッファから対象プレイヤーかつ、該当するタイミングかつ、スピードが即時のリクエストが存在するかを判定します。
存在する場合、以降の処理へ進み、存在しなくなるまでこのループを継続します。

\sphinxlineitem{\sphinxstylestrong{{[}6\sphinxhyphen{}5\sphinxhyphen{}2{]} バッファからリクエストを取り出す}}
\sphinxAtStartPar
条件を満たしたリクエストを、バッファから1つ取り出します。
どのリクエストを取り出すかは対象プレイヤーが決定します。
取り出されたリクエストは、解決処理の対象となります。

\sphinxlineitem{\sphinxstylestrong{{[}6\sphinxhyphen{}5\sphinxhyphen{}3{]} 取り出したリクエストを解決する}}
\sphinxAtStartPar
取り出されたリクエストの効果を実行し、解決します。
詳しくは {\hyperref[\detokenize{core/core:actresolve}]{\sphinxcrossref{\DUrole{std,std-ref}{{[}8{]} リクエストの解決}}}} 参照してください。

\sphinxlineitem{\sphinxstylestrong{{[}6\sphinxhyphen{}5\sphinxhyphen{}4{]} 勝敗判定}}
\sphinxAtStartPar
勝敗を判定します。
詳しくは {\hyperref[\detokenize{core/core:winlose}]{\sphinxcrossref{\DUrole{std,std-ref}{{[}9{]} 勝敗判定}}}} 参照。

\sphinxlineitem{\sphinxstylestrong{{[}6\sphinxhyphen{}5\sphinxhyphen{}5{]} リクエストが新たな誘発を発生させたか?}}
\sphinxAtStartPar
効果の解決後、それが起因となり新たな誘発を発生させたかどうかを確認します。
発生している場合は、その誘発アクションを再度バッファに追加する必要があります。

\sphinxlineitem{\sphinxstylestrong{{[}6\sphinxhyphen{}5\sphinxhyphen{}6{]} 誘発したリクエストをバッファに追加}}
\sphinxAtStartPar
新たに誘発されたリクエストが存在する場合、該当アクションをバッファに追加します。
これにより、再帰的なアクション処理が可能となり、次のループで該当するリクエストの取り出し処理が実行されます。

\end{description}


\subsubsection{通常:アクション毎に処理}
\label{\detokenize{core/core:trigger-act-n}}\label{\detokenize{core/core:id28}}
\sphinxAtStartPar
誘発チェックで誘発したスピードが通常のリクエストを処理します。
呼び出し元で指定されたプレイヤーおよびタイミングに基づいて処理します。
誘発チェックは次の図のように行います。(\hyperref[\detokenize{core/core:trigger-flow-n}]{Fig.\@ \ref{\detokenize{core/core:trigger-flow-n}}})

\begin{figure}[htbp]
\centering
\capstart

\noindent\sphinxincludegraphics[scale=0.5]{{plantuml-5d2051ec7bb14a770950bcb23c25d203c53ec5f1}.pdf}
\caption{誘発チェック\sphinxhyphen{}通常処理}\label{\detokenize{core/core:id42}}\label{\detokenize{core/core:trigger-flow-n}}\end{figure}
\begin{description}
\sphinxlineitem{\sphinxstylestrong{{[}6\sphinxhyphen{}10\sphinxhyphen{}1{]} バッファから該当の通常はあるか?}}
\sphinxAtStartPar
バッファから対象プレイヤーかつ、該当するタイミングかつ、スピードが通常のリクエストが存在するかを判定します。
存在する場合、以降の処理へ進み、存在しなくなるまでこのループを継続します。

\sphinxlineitem{\sphinxstylestrong{{[}6\sphinxhyphen{}10\sphinxhyphen{}2{]} バッファからリクエストを取り出す}}
\sphinxAtStartPar
条件を満たしたリクエストを、バッファから1つ取り出します。
どのリクエストを取り出すかは対象プレイヤーが決定します。
取り出されたリクエストは、解決処理の対象となります。

\sphinxlineitem{\sphinxstylestrong{{[}6\sphinxhyphen{}10\sphinxhyphen{}3{]} 取り出したリクエストのタイミングがメインか判定}}
\sphinxAtStartPar
取り出したリクエストのタイミングが「メイン」であるかどうかを判定します。
「Yes」と判定された場合は、ステージへの追加前に空き状況の確認へ進みます。
「No」の場合は、クイックタイミングとして処理を行います。

\sphinxlineitem{\sphinxstylestrong{{[}6\sphinxhyphen{}10\sphinxhyphen{}4{]} ステージが空か判定}}
\sphinxAtStartPar
タイミングがメインの場合、そのリクエストをステージに追加できるかどうか、
すなわちステージに空きがあるかを判定します。

\sphinxlineitem{\sphinxstylestrong{{[}6\sphinxhyphen{}10\sphinxhyphen{}5{]} リクエストをステージに追加}}
\sphinxAtStartPar
ステージが空いている場合、取り出したリクエストをステージに追加します。

\sphinxlineitem{\sphinxstylestrong{{[}6\sphinxhyphen{}10\sphinxhyphen{}6{]} リクエストを破棄}}
\sphinxAtStartPar
ステージが埋まっている場合、取り出したタイミングがメインのリクエストを破棄します。

\sphinxlineitem{\sphinxstylestrong{{[}6\sphinxhyphen{}10\sphinxhyphen{}7{]} リクエストをステージに追加}}
\sphinxAtStartPar
取り出したリクエストのタイミングが「クイック」である場合、
ステージの空き状況にかかわらず無条件でリクエストをステージに追加します。

\sphinxlineitem{\sphinxstylestrong{{[}6\sphinxhyphen{}10\sphinxhyphen{}8{]} リクエストが新たな誘発を発生させたか?}}
\sphinxAtStartPar
リクエストをステージに追加した後、それが起因となり新たな誘発を発生させたかどうかを確認します。
誘発が発生している場合は、後続の処理でアクションバッファへの追加が行われます。

\sphinxlineitem{\sphinxstylestrong{{[}6\sphinxhyphen{}10\sphinxhyphen{}9{]} 誘発したリクエストをバッファに追加}}
\sphinxAtStartPar
新たに誘発されたリクエストが存在する場合、該当アクションをバッファに追加します。
これにより、誘発処理の再実行が可能となります。

\end{description}


\subsection{能力の適応順}
\label{\detokenize{core/core:ability-order}}\label{\detokenize{core/core:id29}}
\sphinxAtStartPar
複数の能力が同時に有効になった場合、効果の内容によっては矛盾を引き起こす可能性があります。
その場合は、次の \hyperref[\detokenize{core/core:ability-order-rule}]{\ref{\detokenize{core/core:ability-order-rule}} \nameref{\detokenize{core/core:ability-order-rule}}} に沿って能力を解釈してください。

\sphinxAtStartPar
なお、ゲームによって細部の裁定が異なる場合があります。ここでは BlackPoker を例に説明します。


\subsubsection{考え方の原則}
\label{\detokenize{core/core:id30}}
\sphinxAtStartPar
影響範囲が広い能力ほど優先順位を低く扱い、影響範囲が狭い(個別の対象を限定する)能力ほど優先順位を高く扱います。
つまり、「広く浅く影響する能力」は先に適用し、「狭く深く影響する能力」は後から適用します。


\subsubsection{能力の適用ルール}
\label{\detokenize{core/core:ability-order-rule}}\label{\detokenize{core/core:id31}}\begin{enumerate}
\sphinxsetlistlabels{\arabic}{enumi}{enumii}{}{.}%
\item {} 
\sphinxAtStartPar
能力を分類する
各能力を「影響範囲」に注目して、次の 3 つに分類します。
\begin{description}
\sphinxlineitem{\sphinxstylestrong{{[}1{]} 全ての (all)}}\begin{itemize}
\item {} 
\sphinxAtStartPar
ゲームの場全体、または全プレイヤーなど「すべて」を有効範囲に含む能力です。

\item {} 
\sphinxAtStartPar
例: 「全ての{\normalsize $\spadesuit$} の兵士はサイズを1加算する」など。

\end{itemize}

\sphinxlineitem{\sphinxstylestrong{{[}2{]} あなたの (your)}}\begin{itemize}
\item {} 
\sphinxAtStartPar
あなたの場や、あなたがコントロールしているコンポーネントだけを有効範囲にする能力です。

\item {} 
\sphinxAtStartPar
例: 「あなたの兵士はドライブ状態でも攻撃できる」など。

\end{itemize}

\sphinxlineitem{\sphinxstylestrong{{[}3{]} この (this)}}\begin{itemize}
\item {} 
\sphinxAtStartPar
対象のコンポーネントインスタンス 1 つを有効範囲にします。

\item {} 
\sphinxAtStartPar
例: 「この兵士はサイズを3加算する」「対象の兵士のサイズを2減算する」など。

\end{itemize}

\end{description}

\item {} 
\sphinxAtStartPar
能力を適用する
上記で分類した番号の小さい順、すなわち \sphinxstylestrong{{[}1{]} \(\rightarrow\) {[}2{]} \(\rightarrow\) {[}3{]}} の順で能力を適用します。
\begin{enumerate}
\sphinxsetlistlabels{\arabic}{enumii}{enumiii}{}{.}%
\item {} \begin{description}
\sphinxlineitem{\sphinxstylestrong{{[}1{]} 全ての (all)}}
\sphinxAtStartPar
まず「全ての ~」といった広範囲を対象とする能力を先に処理します。

\end{description}

\item {} \begin{description}
\sphinxlineitem{\sphinxstylestrong{{[}2{]} あなたの (your)}}
\sphinxAtStartPar
次に「あなたの ~」等、特定プレイヤーやその場を限定する能力を処理します。

\end{description}

\item {} \begin{description}
\sphinxlineitem{\sphinxstylestrong{{[}3{]} この (this)}}
\sphinxAtStartPar
最後に、個別のコンポーネント単位でより限定的な能力を処理し、必要に応じて上書きします。

\end{description}

\end{enumerate}

\sphinxAtStartPar
\sphinxstylestrong{同じ範囲で矛盾した能力がある場合}
それらの能力が発動(または有効化)された順番に応じて “タイムスタンプ” が古いものから適用し、
新しいものが後から上書きする形で処理します。

\sphinxAtStartPar
このように、同時に有効な効果が互いに矛盾する場合でも、最終的により限定的な効果が優先されます。

\end{enumerate}


\subsubsection{運用上の注意}
\label{\detokenize{core/core:id32}}\begin{itemize}
\item {} \begin{description}
\sphinxlineitem{\sphinxstylestrong{複数の「同じ範囲」の能力同士の扱い}}
\sphinxAtStartPar
たとえば「全ての兵士は攻撃できない」という能力と、「全ての兵士のサイズはこのターン中1加算される」という能力が同じ “{[}1{]} 全ての (all)” 範囲に当たる場合、どちらも基本的には同時に成立し共存します。ただし、まったく相反する内容で整合が取れない場合は、対戦ルールやジャッジの裁定に従ってください。

\end{description}

\item {} \begin{description}
\sphinxlineitem{\sphinxstylestrong{適用タイミング}}
\sphinxAtStartPar
能力の適用は、コアフロー内の特定のステップでまとめて行います。何度か再適用が行われる場面があっても、適用のたびに「{[}1{]} \(\rightarrow\) {[}2{]} \(\rightarrow\) {[}3{]}」の順でまっさらな状態から再計算します。そのため、兵士のサイズが重複して加算され続けるようなことはありません。

\end{description}

\item {} \begin{description}
\sphinxlineitem{\sphinxstylestrong{互いに矛盾しない場合}}
\sphinxAtStartPar
たとえば「全ての兵士のサイズを \sphinxhyphen{}1」「あなたの兵士はさらに +2」「この兵士はさらに +3」という 3 つの効果が同時に有効なとき、BlackPoker ではサイズ 0 以下でも兵士が場に存在し続けられるため、最終的にその兵士には合計 +5 の修正が加わります。特に抵触しないかぎりすべて重複適用されます。

\end{description}

\item {} \begin{description}
\sphinxlineitem{\sphinxstylestrong{互いに矛盾する場合}}
\sphinxAtStartPar
「全ての兵士は攻撃できない」と「この兵士は攻撃できる」というように真っ向から対立する効果が同時に発生した場合、最終的には範囲の狭い「この兵士は攻撃できる」の方が優先されます。

\end{description}

\end{itemize}

\sphinxAtStartPar
以上のルールが、複数の能力が同時に有効になった場合の処理基準となります。


\section{まとめ}
\label{\detokenize{core/core:id33}}
\sphinxAtStartPar
コアルールについて説明しました。
すでにあるターン制のゲームからアクションを洗い出し、能力を整理することで割込処理を可能としゲームの新しい遊び方が見つけられます。
また、新しく作成するゲームに関してもコアルールを意識して作成することで、ルール追加がしやすいゲームが考えやすいと思います。

\sphinxstepscope


\chapter{付録}
\label{\detokenize{appendix/appendix:appendix-rst}}\label{\detokenize{appendix/appendix:id1}}\label{\detokenize{appendix/appendix::doc}}

\section{PDF版ルール}
\label{\detokenize{appendix/appendix:pdf}}
\sphinxAtStartPar
\sphinxurl{https://blackpoker.github.io/BlackPoker/master/blackpoker.pdf}


\section{アクションリスト}
\label{\detokenize{appendix/appendix:id2}}

\subsection{ライト}
\label{\detokenize{appendix/appendix:actionlist-lite}}\label{\detokenize{appendix/appendix:id3}}\begin{description}
\sphinxlineitem{URL}
\sphinxAtStartPar
\sphinxurl{https://blackpoker.github.io/BlackPoker/master/actionlist/html/lite.html}

\sphinxlineitem{PDF}
\sphinxAtStartPar
\sphinxurl{https://blackpoker.github.io/BlackPoker/master/actionlist/pdf/blackpoker-lite.pdf}

\sphinxAtStartPar
\sphinxurl{https://blackpoker.github.io/BlackPoker/master/actionlist/pdf/blackpoker-lite-2up.pdf}

\end{description}


\subsection{スタンダード}
\label{\detokenize{appendix/appendix:actionlist-std}}\label{\detokenize{appendix/appendix:id4}}\begin{description}
\sphinxlineitem{URL}
\sphinxAtStartPar
\sphinxurl{https://blackpoker.github.io/BlackPoker/master/actionlist/html/std.html}

\sphinxlineitem{PDF}
\sphinxAtStartPar
\sphinxurl{https://blackpoker.github.io/BlackPoker/master/actionlist/pdf/blackpoker-std.pdf}

\sphinxAtStartPar
\sphinxurl{https://blackpoker.github.io/BlackPoker/master/actionlist/pdf/blackpoker-std-2up.pdf}

\end{description}


\subsection{プロ}
\label{\detokenize{appendix/appendix:actionlist-pro}}\label{\detokenize{appendix/appendix:id5}}\begin{description}
\sphinxlineitem{URL}
\sphinxAtStartPar
\sphinxurl{https://blackpoker.github.io/BlackPoker/master/actionlist/html/pro.html}

\sphinxlineitem{PDF}
\sphinxAtStartPar
\sphinxurl{https://blackpoker.github.io/BlackPoker/master/actionlist/pdf/blackpoker-pro.pdf}

\sphinxAtStartPar
\sphinxurl{https://blackpoker.github.io/BlackPoker/master/actionlist/pdf/blackpoker-pro-2up.pdf}

\end{description}


\subsection{マスター}
\label{\detokenize{appendix/appendix:actionlist-master}}\label{\detokenize{appendix/appendix:id6}}\begin{description}
\sphinxlineitem{URL}
\sphinxAtStartPar
\sphinxurl{https://blackpoker.github.io/BlackPoker/master/actionlist/html/mast.html}

\sphinxlineitem{PDF}
\sphinxAtStartPar
\sphinxurl{https://blackpoker.github.io/BlackPoker/master/actionlist/pdf/blackpoker-mast.pdf}

\sphinxAtStartPar
\sphinxurl{https://blackpoker.github.io/BlackPoker/master/actionlist/pdf/blackpoker-mast-2up.pdf}

\end{description}
\phantomsection\label{\detokenize{appendix/appendix:extralist}}


\renewcommand{\indexname}{索引}
\printindex
\end{document}