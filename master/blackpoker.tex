%% Generated by Sphinx.
\def\sphinxdocclass{jsbook}
\documentclass[letterpaper,10pt,dvipdfmx]{sphinxmanual}
\ifdefined\pdfpxdimen
   \let\sphinxpxdimen\pdfpxdimen\else\newdimen\sphinxpxdimen
\fi \sphinxpxdimen=.75bp\relax
\ifdefined\pdfimageresolution
    \pdfimageresolution= \numexpr \dimexpr1in\relax/\sphinxpxdimen\relax
\fi
%% let collapsible pdf bookmarks panel have high depth per default
\PassOptionsToPackage{bookmarksdepth=5}{hyperref}


\PassOptionsToPackage{warn}{textcomp}


\usepackage{cmap}
\usepackage[T1]{fontenc}
\usepackage{amsmath,amssymb,amstext}




\usepackage{tgtermes}
\usepackage{tgheros}
\renewcommand{\ttdefault}{txtt}




\usepackage[,numfigreset=1,mathnumfig]{sphinx}

\fvset{fontsize=auto}
\usepackage[dvipdfm]{geometry}


% Include hyperref last.
\usepackage{hyperref}
% Fix anchor placement for figures with captions.
\usepackage{hypcap}% it must be loaded after hyperref.
% Set up styles of URL: it should be placed after hyperref.
\urlstyle{same}
\usepackage{pxjahyper}


\usepackage{sphinxmessages}
\setcounter{tocdepth}{1}


\makeatletter
\renewcommand\sphinxlineitem[2]{%
  % safe test of whether #2 is \sphinxlineitem
  \sphinx@gobto@sphinxlineitem#2\@gobbletwo\sphinxlineitem\unless
  \iftrue
    % Accumulate successive terms until actual definition or sub-list is found
    \spx@lineitemlabel\expandafter{\the\spx@lineitemlabel\strut#1\\}%
  \else
    % Issue the \item command with possibly multi-line contents
    \item[\kern\labelwidth\kern-\itemindent\kern-\leftmargin
          {\parbox[t]{1.4\linewidth}{% <- ここで幅を調整
          \raggedright
          \the\spx@lineitemlabel% Accumulated terms before this one, CR separated
          \strut#1}}% No \par token allowed here, but the \parbox will insert one tacitly at end
          \kern-\labelsep]%
    \spx@lineitemlabel{}%
    % This causes the label to be typeset (filling up the line), clearing up
    % things in case a nested list follows.
    \leavevmode
  \fi #2%
}
\makeatother


\title{BlackPoker}
\date{2024年11月25日}
\release{2024/4/1}
\author{BlackPoker}
\newcommand{\sphinxlogo}{\vbox{}}
\renewcommand{\releasename}{リリース}
\makeindex
\begin{document}

\pagestyle{empty}
\sphinxmaketitle
\pagestyle{plain}
\sphinxtableofcontents
\pagestyle{normal}
\phantomsection\label{\detokenize{index::doc}}


\sphinxAtStartPar
release: 2024/4/1

\sphinxstepscope


\chapter{はじめに}
\label{\detokenize{init/init:init-rst}}\label{\detokenize{init/init:id1}}\label{\detokenize{init/init::doc}}
\sphinxAtStartPar
この文章はトランプゲーム「BlackPoker」の全てのルールをまとめた文章です。

\sphinxAtStartPar
詳細なルールが記載されており、初心者の方は文章の量に圧倒されます。
ゲームをプレイする際に全てを熟読する必要はありませんが、
ルールについて深く知りたい、または新しいルールに触れたい方はぜひ熟読してください。


\section{BlackPokerとは}
\label{\detokenize{init/init:blackpoker}}
\sphinxAtStartPar
当サークルが考案した1人1セットのトランプを使ったターン制の
トレーディングカードゲームのようなトランプゲームです。
自分だけのオリジナルトランプを使って友達と遊べます。
プレイしている時のイメージは図のような感じです。(\hyperref[\detokenize{init/init:play-image}]{Fig.\@ \ref{\detokenize{init/init:play-image}}})

\begin{figure}[htbp]
\centering
\capstart

\noindent\sphinxincludegraphics{{play-image}.pdf}
\caption{プレイ風景}\label{\detokenize{init/init:id8}}\label{\detokenize{init/init:play-image}}\end{figure}


\section{きっかけ}
\label{\detokenize{init/init:id2}}
\begin{sphinxVerbatim}[commandchars=\\\{\}]
「また昔みたいに友達とカードゲームがしたい」
\end{sphinxVerbatim}

\sphinxAtStartPar
けど、仕事に追われ時間もなく、お金もかけられないのでカード資産も抱えられない。
昔のデッキを引っ張りだしてもカードパワーが違って平等に遊べないし、コミニティが狭い。
そんな悩みを解消しようと誰でも持っているトランプでカードゲーム風のルールを作ろうと思いました。


\section{ゲームのストーリー}
\label{\detokenize{init/init:id3}}
\sphinxAtStartPar
あなたのトランプはただのトランプではありません。
かつて大地を支配していた偉大な国々や強固な絆で結ばれた組織の魂が宿る遺産です。

\sphinxAtStartPar
これらのカードには、世界の歴史を彩る多彩な物語が織り込まれており、その物語たちはまるで時間を超えたタペストリーのように絡み合っています。

\sphinxAtStartPar
プレイヤーは歴史の再現者、リアニメーターとしてデッキに宿る過去の栄光を再現し、その国や組織がいかに素晴らしかったかを
戦いを通して対戦相手に示してください。

\sphinxAtStartPar
え?そんな国知らない?でしたらあなたのカードから創造してみてはいかがでしょうか。


\section{読み方}
\label{\detokenize{init/init:id4}}
\sphinxAtStartPar
各章は次のことを説明しています。
\begin{description}
\sphinxlineitem{\hyperref[\detokenize{init/init:init-rst}]{\ref{\detokenize{init/init:init-rst}} \nameref{\detokenize{init/init:init-rst}}}}
\sphinxAtStartPar
ルールの指針や全体像を説明

\sphinxlineitem{\hyperref[\detokenize{core/core:core-rst}]{\ref{\detokenize{core/core:core-rst}} \nameref{\detokenize{core/core:core-rst}}}}
\sphinxAtStartPar
BlackPokerのコアであるターン制ゲームのルールを説明

\sphinxlineitem{\hyperref[\detokenize{common/common:common-rst}]{\ref{\detokenize{common/common:common-rst}} \nameref{\detokenize{common/common:common-rst}}}}
\sphinxAtStartPar
ゲームの開始方法など全体的なゲームの流れを説明

\sphinxlineitem{\hyperref[\detokenize{format/format:format-rst}]{\ref{\detokenize{format/format:format-rst}} \nameref{\detokenize{format/format:format-rst}}}}
\sphinxAtStartPar
ゲーム内で使えるアクションの定義方法を説明

\sphinxlineitem{\hyperref[\detokenize{match-regulations/match-regulations:match-regulations-rst}]{\ref{\detokenize{match-regulations/match-regulations:match-regulations-rst}} \nameref{\detokenize{match-regulations/match-regulations:match-regulations-rst}}}}
\sphinxAtStartPar
ゲーム開始時に決定する規則について説明

\end{description}


\subsection{用途別読み方}
\label{\detokenize{init/init:id5}}
\sphinxAtStartPar
文章が複雑であるため、用途に合わせて読むことをおすすめします。
\begin{description}
\sphinxlineitem{ざっくりゲームの流れを理解したい方}
\sphinxAtStartPar
\hyperref[\detokenize{common/common:common-rst}]{\ref{\detokenize{common/common:common-rst}} \nameref{\detokenize{common/common:common-rst}}} から読み、部分的に参照されている \hyperref[\detokenize{core/core:core-rst}]{\ref{\detokenize{core/core:core-rst}} \nameref{\detokenize{core/core:core-rst}}} を読む

\sphinxlineitem{アクションの解決順で悩んでいる方}
\sphinxAtStartPar
\hyperref[\detokenize{core/core:core-rst}]{\ref{\detokenize{core/core:core-rst}} \nameref{\detokenize{core/core:core-rst}}} を熟読する

\sphinxlineitem{BlackPokerとは何か理解した方}
\sphinxAtStartPar
\hyperref[\detokenize{core/core:core-rst}]{\ref{\detokenize{core/core:core-rst}} \nameref{\detokenize{core/core:core-rst}}} を軽く目を通し、\hyperref[\detokenize{common/common:common-rst}]{\ref{\detokenize{common/common:common-rst}} \nameref{\detokenize{common/common:common-rst}}} 、\hyperref[\detokenize{format/format:format-rst}]{\ref{\detokenize{format/format:format-rst}} \nameref{\detokenize{format/format:format-rst}}} と読み進める

\end{description}


\section{ルール指針}
\label{\detokenize{init/init:id6}}
\sphinxAtStartPar
ルールを作成・修正するための指針を示します。
\begin{description}
\sphinxlineitem{\sphinxstylestrong{誰とでも戦える \textasciitilde{}目指すは老若男女\textasciitilde{}}}
\sphinxAtStartPar
ルールを知りトランプを持っていれば誰とでも遊べるゲームを目指します。

\sphinxlineitem{\sphinxstylestrong{個性が出せる \textasciitilde{}オリジナルトランプ・デッキ構築\textasciitilde{}}}
\sphinxAtStartPar
さまざまなトランプが使え見た目で個性を出せるのはもちろんのこと、
デッキ構築の面でも自分のしたい戦い方が表現できることを目指します。

\sphinxlineitem{\sphinxstylestrong{短く終わる \textasciitilde{}1戦15分\textasciitilde{}}}
\sphinxAtStartPar
時間をかけずさっと遊べることを目指します。

\sphinxlineitem{\sphinxstylestrong{ずっと使えるデッキ}}
\sphinxAtStartPar
愛着のあるカードがずっと使えるようなルールとします。

\sphinxlineitem{\sphinxstylestrong{必要な物は最小限 \textasciitilde{}トランプのみ\textasciitilde{}}}
\sphinxAtStartPar
用意するものはトランプのみ。それ以外の道具は必要ないルールとします。

\sphinxlineitem{\sphinxstylestrong{プレイング重視 \textasciitilde{}5:3:2=技:運:構築\textasciitilde{}}}
\sphinxAtStartPar
運やデッキ構築より技量を重視したルールを目指します。

\sphinxlineitem{\sphinxstylestrong{ベースルールはトレーディングカードゲーム}}
\sphinxAtStartPar
カードゲームプレイヤーが覚えやすいルールを目指します。

\sphinxlineitem{\sphinxstylestrong{カスタマイズ可能 \textasciitilde{}基本と拡張の分離\textasciitilde{}}}
\sphinxAtStartPar
基本ルールと拡張ルールを分離し、大富豪のようにローカルルールが作成できることを目指します。

\sphinxlineitem{\sphinxstylestrong{ルールの更新 \textasciitilde{}飽き防止&不備改善\textasciitilde{}}}
\sphinxAtStartPar
新たなルールを度々公開し、飽きを防止します。またルールに不備がある場合、随時改善します。

\sphinxlineitem{\sphinxstylestrong{相手のカードに触らない}}
\sphinxAtStartPar
盗難防止とネット対戦対応に努めます。

\end{description}


\section{ルールの構成}
\label{\detokenize{init/init:rule-constract}}\label{\detokenize{init/init:id7}}
\sphinxAtStartPar
ルールの構成は次のようになっています。
ルールを階層化し、ルール指針を具体化しています。(\hyperref[\detokenize{init/init:rule-puml}]{Fig.\@ \ref{\detokenize{init/init:rule-puml}}})

\begin{figure}[htbp]
\centering
\capstart

\noindent\sphinxincludegraphics[scale=0.5]{{plantuml-0b4f05a00976af29a919ccd57da91a1d511d7894}.pdf}
\caption{ルール構成}\label{\detokenize{init/init:id9}}\label{\detokenize{init/init:rule-puml}}\end{figure}

\sphinxAtStartPar
更にルールを詳しく記載すると次のようになります。
専門的な表現になるので、理解出来なくても構いません。(\hyperref[\detokenize{init/init:rule-class-puml}]{Fig.\@ \ref{\detokenize{init/init:rule-class-puml}}})

\begin{figure}[htbp]
\centering
\capstart

\noindent\sphinxincludegraphics[scale=0.5]{{plantuml-25d895c0cb10acacb2584c83ee11d0fe0b970036}.pdf}
\caption{ルール構成(詳細)}\label{\detokenize{init/init:id10}}\label{\detokenize{init/init:rule-class-puml}}\end{figure}

\sphinxstepscope


\chapter{コアルール}
\label{\detokenize{core/core:core-rst}}\label{\detokenize{core/core:id1}}\label{\detokenize{core/core::doc}}
\sphinxAtStartPar
コアルールは割込み処理が可能なターン制ゲームの開始から勝敗が決まるまでを定義します。

\index{ターン@\spxentry{ターン}}\ignorespaces 

\section{ターン}
\label{\detokenize{core/core:index-0}}\label{\detokenize{core/core:id2}}
\sphinxAtStartPar
このルールを説明する上でターンとは持つことができるものとします。
ターンを持っているプレイヤーは先に行動できます。
ターンを持っているプレイヤーをターンプレイヤーといいます。

\index{アクション(コア)@\spxentry{アクション(コア)}}\ignorespaces 

\section{アクション}
\label{\detokenize{core/core:index-1}}\label{\detokenize{core/core:id3}}
\sphinxAtStartPar
アクションとは、プレイヤーの行動を示します。
ターン制のゲームでは、プレイヤーは様々な行動を行います。
チェスであればコマを進めたり、ババ抜きであれば隣の人からカードを引くなどがあります。
それらをアクションと定義します。

\index{チャンス@\spxentry{チャンス}}\ignorespaces 

\section{チャンス}
\label{\detokenize{core/core:index-2}}\label{\detokenize{core/core:id4}}
\sphinxAtStartPar
アクションを起こすことができる機会をチャンスといいます。
チャンスを持っている間は何度でもアクションを起こすことができます。

\index{ステージ@\spxentry{ステージ}}\ignorespaces 

\section{ステージ}
\label{\detokenize{core/core:index-3}}\label{\detokenize{core/core:id5}}
\sphinxAtStartPar
アクションの解決順を整理するために使う領域です。
後入れ先出し方式で最後に積まれたアクションから順に解決されていきます。


\section{アクションの定義項目}
\label{\detokenize{core/core:id6}}
\sphinxAtStartPar
アクション、チャンス、ステージについて簡単に説明しました。
これらの概念を用いて、アクションに定義する項目を説明します。

\sphinxAtStartPar
アクションは次の項目を定義する必要があります。
他の項目は具体的にアクションを定義する際に、ゲームに合わせて追加して下さい。
\begin{itemize}
\item {} 
\sphinxAtStartPar
効果(通常効果・即時効果)

\item {} 
\sphinxAtStartPar
タイミング

\end{itemize}

\index{つ\textbar{}通常効果(コア)@\spxentry{つ\textbar{}通常効果(コア)}!そ\textbar{}即時効果(コア)@\spxentry{そ\textbar{}即時効果(コア)}}\index{そ\textbar{}即時効果(コア)@\spxentry{そ\textbar{}即時効果(コア)}!つ\textbar{}通常効果(コア)@\spxentry{つ\textbar{}通常効果(コア)}}\ignorespaces 

\subsection{効果}
\label{\detokenize{core/core:index-4}}\label{\detokenize{core/core:id7}}
\sphinxAtStartPar
効果とはアクションの解決時にプレイヤーが行う行動です。
効果の中には、通常効果と即時効果があります。
違いについては、図(\hyperref[\detokenize{core/core:coreflow-2}]{Fig.\@ \ref{\detokenize{core/core:coreflow-2}}})を説明する際に分岐条件として登場します。

\index{タイミング(コア)@\spxentry{タイミング(コア)}}\ignorespaces 

\subsection{タイミング}
\label{\detokenize{core/core:timing}}\label{\detokenize{core/core:index-5}}\label{\detokenize{core/core:id8}}
\sphinxAtStartPar
タイミングとは、アクションを起こすことができる時を示します。
タイミングには「メイン」と「クイック」の2種類あります。

\index{メイン@\spxentry{メイン}}\ignorespaces \begin{description}
\sphinxlineitem{メイン}
\sphinxAtStartPar
ターンプレイヤーかつステージが空の時に起こせるアクションです。

\sphinxAtStartPar
条件をまとめると次のようになります。
\begin{itemize}
\item {} 
\sphinxAtStartPar
チャンスを持っている

\item {} 
\sphinxAtStartPar
自分のターン

\item {} 
\sphinxAtStartPar
ステージが空

\end{itemize}

\end{description}

\index{クイック@\spxentry{クイック}}\ignorespaces \begin{description}
\sphinxlineitem{クイック}
\sphinxAtStartPar
いつでも起こせるため、アクションをステージに積み重ねることができます。

\sphinxAtStartPar
条件をまとめると次のようになります。
\begin{itemize}
\item {} 
\sphinxAtStartPar
チャンスを持っている

\end{itemize}

\end{description}


\subsubsection{エンドアクションの定義}
\label{\detokenize{core/core:id9}}
\sphinxAtStartPar
定義するアクションの中で最低1つは
ターンを別のプレイヤーにわたす効果を定義してください。
そうしないと、ターンが別のプレイヤーに渡らす、ゲームが進行しなくなります。


\subsubsection{アクションのコントローラー}
\label{\detokenize{core/core:id10}}
\sphinxAtStartPar
アクションを起こしたプレイヤーをそのアクションのコントローラーと呼びます。
効果はこのコントローラー視点で解釈されることになります。

\index{コンポーネント@\spxentry{コンポーネント}}\ignorespaces 

\section{コンポーネント}
\label{\detokenize{core/core:component}}\label{\detokenize{core/core:index-8}}\label{\detokenize{core/core:id11}}
\sphinxAtStartPar
ゲームにてプレイヤーが保有する駒やカードのことをコンポーネントと定義します。
コンポーネントは次の項目を持っています。

\index{オーナー@\spxentry{オーナー}}\ignorespaces \begin{description}
\sphinxlineitem{オーナー}
\sphinxAtStartPar
コンポーネントの所有者を示します。大体のトランプゲームではトランプを1セットしか用いないため無視されますが、TCGのデッキなど個人所有のものを用いるゲームでは必要な項目となります。

\end{description}

\index{コントローラー@\spxentry{コントローラー}}\ignorespaces \begin{description}
\sphinxlineitem{コントローラー}
\sphinxAtStartPar
現在そのコンポーネントを操作しているプレイヤーを示します。オーナーとコントローラーは基本同じプレイヤーが設定されますが、コントロールを奪うアクションがある場合、オーナーとコントローラーは異なります。

\end{description}

\begin{sphinxadmonition}{note}{注釈:}
\sphinxAtStartPar
コンポーネントとアクションのコントローラー

\sphinxAtStartPar
コントローラーは制御している人という意味になるため、コンポーネントとアクションのコントローラー制御する対象が異なることになります。
コンポーネントとアクションの属性を次の図に示します。アクションにはオーナーがいない点が異なります。
\end{sphinxadmonition}

\begin{figure}[htbp]
\centering
\capstart

\noindent\sphinxincludegraphics[scale=0.5]{{plantuml-5973875432da5c6c9c30841afa726d28ed2ec522}.pdf}
\caption{コンポーネントとアクションの属性}\label{\detokenize{core/core:id16}}\end{figure}

\index{の\textbar{}能力(コア)@\spxentry{の\textbar{}能力(コア)}}\ignorespaces 

\section{能力}
\label{\detokenize{core/core:index-11}}\label{\detokenize{core/core:id12}}
\sphinxAtStartPar
能力とはアクションの効果とは異なる概念で、アクションを起こすことができたり、 アクションを誘発したりすることがでる力です。

\sphinxAtStartPar
能力を持つことができるのは、プレイヤーの他に駒やカードなどのゲームに登場するコンポーネントも持つことができます。
(\hyperref[\detokenize{core/core:ability-image}]{Fig.\@ \ref{\detokenize{core/core:ability-image}}})

\begin{figure}[htbp]
\centering
\capstart

\noindent\sphinxincludegraphics[scale=0.5]{{plantuml-976f45327f77d4f4ee1da59bff1cd262ba402df4}.pdf}
\caption{能力のイメージ}\label{\detokenize{core/core:id17}}\label{\detokenize{core/core:ability-image}}\end{figure}

\sphinxAtStartPar
能力には、次の種類があります。

\index{じ\textbar{}常在型能力@\spxentry{じ\textbar{}常在型能力}}\ignorespaces \begin{description}
\sphinxlineitem{常在型能力}
\sphinxAtStartPar
能力が有効である場合、継続的に発揮される能力

\end{description}

\index{ゆ\textbar{}誘発型能力@\spxentry{ゆ\textbar{}誘発型能力}}\ignorespaces \begin{description}
\sphinxlineitem{誘発型能力}
\sphinxAtStartPar
能力が有効である間に何かの契機でアクションを起こす能力

\end{description}

\sphinxAtStartPar
概ねのゲームでは、
ターン終了や駒をすすめるなどのアクションが定義されています。
そして、そのアクションを起こせる能力(常在型能力)を
プレイヤーは保持しています。

\index{コアフロー@\spxentry{コアフロー}}\ignorespaces 

\section{コアフロー}
\label{\detokenize{core/core:coreflowsec}}\label{\detokenize{core/core:index-14}}\label{\detokenize{core/core:id13}}
\sphinxAtStartPar
この図にゲームの開始から勝敗が決まるまでの流れが集約されいます。(\hyperref[\detokenize{core/core:coreflow-2}]{Fig.\@ \ref{\detokenize{core/core:coreflow-2}}})

\begin{figure}[htbp]
\centering
\capstart

\noindent\sphinxincludegraphics[scale=0.5]{{plantuml-06d74c8ca532cbff699e6d17eac527fc25dc81ee}.pdf}
\caption{コアフロー}\label{\detokenize{core/core:id18}}\label{\detokenize{core/core:coreflow-2}}\end{figure}
\phantomsection\label{\detokenize{core/core:core-gamestart}}\begin{description}
\sphinxlineitem{{[}1{]}ゲーム開始}
\sphinxAtStartPar
先攻を決め、ゲームを始める準備を行います。

\sphinxlineitem{{[}2{]}ターンプレイヤーにチャンスを移動}
\sphinxAtStartPar
ターンを持っているプレイヤーにチャンスを移動します。

\sphinxlineitem{{[}3{]}アクションを起こすか?}
\sphinxAtStartPar
チャンスを持っているプレイヤーはアクションを起こすかを判断します。

\sphinxlineitem{{[}4{]}パス記録のリセット}
\sphinxAtStartPar
パスしたプレイヤーの記録をリセットします。

\sphinxlineitem{{[}5{]}アクションを起こす}
\sphinxAtStartPar
アクションを起こしこれからプレイヤーが行うことを宣言します。
ゲームによってアクションの起こし方は異なります。BlackPokerではアクション名を言い、コストの支払や対象を指定しアクションを起こします。
一方ババ抜きでは、隣のプレイヤーからカードを引く際に宣言せず暗黙にアクションが起きている場合もあります。

\sphinxlineitem{{[}6{]}誘発チェック}
\sphinxAtStartPar
ここに至るまでに誘発したアクションがないかチェックします。誘発した場合、効果を解決するかスタックに追加します。詳しいフローは \hyperref[\detokenize{core/core:trigger-check}]{\ref{\detokenize{core/core:trigger-check}} \nameref{\detokenize{core/core:trigger-check}}} を参照してください。

\sphinxlineitem{{[}7{]}即時効果か?}
\sphinxAtStartPar
起こしたアクションが即時効果か通常効果か判定します。

\end{description}
\phantomsection\label{\detokenize{core/core:actresolve}}\begin{description}
\sphinxlineitem{{[}8{]}アクションの解決}
\sphinxAtStartPar
アクションの効果に定義されている内容を実行します。
その他にコンポーネントを捨て山に移動するなどゲームによって決まった処理があれば行います。
アクションの解決の中でも効果に定義されている内容を実行することのみを指す場合「効果を発揮する」と言います。

\end{description}
\phantomsection\label{\detokenize{core/core:winlose}}\begin{description}
\sphinxlineitem{{[}9{]}勝敗判定}
\sphinxAtStartPar
ゲームの勝敗を判定します。決着した場合ゲームが終了します。判定の方法はゲームにより異なります。

\sphinxlineitem{{[}10{]}ステージに追加}
\sphinxAtStartPar
ステージというアクションを貯めておける領域に追加します。

\sphinxlineitem{{[}11{]}パス記録に登録}
\sphinxAtStartPar
パスしたプレイヤーを記録します。パス記録がリセットされるため、同じプレイヤー名は2回登録されません。

\sphinxlineitem{{[}12{]}全員がパスしたか?}
\sphinxAtStartPar
パス記録に全てのプレイヤー名が記録されているか判定します。

\sphinxlineitem{{[}13{]}ステージにアクションが存在するか?}
\sphinxAtStartPar
ステージにアクションが存在するか判定します。

\sphinxlineitem{{[}14{]}ステージから取出し}
\sphinxAtStartPar
最後にステージに追加されたアクションをステージから取出します。

\sphinxlineitem{{[}15{]}チャンス移動}
\sphinxAtStartPar
チャンスを持っているプレイヤーからチャンスを持っていないプレイヤーにチャンスを移動します。
チャンスを移動するルールはゲームによって異なります。

\end{description}


\subsection{誘発チェック}
\label{\detokenize{core/core:trigger-check}}\label{\detokenize{core/core:id14}}
\sphinxAtStartPar
能力の中でも誘発型能力は、なにかをきっかけにしてアクションが起きる条件が定義されています。
誘発する条件は「〜の場合」、「〜時」などで記載されており、誘発するアクションは「〜を誘発する」と記載されています。

\sphinxAtStartPar
誘発チェックでは、誘発したアクションの効果を解決もしくは、ステージに追加します。
誘発したアクションのコントローラーは起因となった誘発型能力を持ったコンポーネントのコントローラーになります。
誘発チェックは次の図のように行います。(\hyperref[\detokenize{core/core:trigger-flow}]{Fig.\@ \ref{\detokenize{core/core:trigger-flow}}})

\begin{figure}[htbp]
\centering
\capstart

\noindent\sphinxincludegraphics[scale=0.5]{{plantuml-3706c977d1e3edc6c020c1b97b3098f0c6d32a19}.pdf}
\caption{誘発チェック}\label{\detokenize{core/core:id19}}\label{\detokenize{core/core:trigger-flow}}\end{figure}
\phantomsection\label{\detokenize{core/core:trigger-act-gather}}\begin{description}
\sphinxlineitem{{[}6\sphinxhyphen{}1{]}誘発したアクションをプレイヤー毎の誘発即時リストと誘発通常リストに追加}
\sphinxAtStartPar
全てのプレイヤー、コンポーネントが持っている誘発型能力を確認します。
誘発したアクションをコントローラーのプレイヤー毎に即時効果と通常効果に分け、
プレイヤー毎の誘発即時リスト、誘発通常リストに追加します。

\sphinxlineitem{{[}6\sphinxhyphen{}2{]}誘発即時リスト、誘発通常リスト全体の件数判定}
\sphinxAtStartPar
プレイヤー毎の誘発即時リスト、誘発通常リストの合計件数を判定します。

\sphinxlineitem{{[}6\sphinxhyphen{}3{]}プレイヤー毎に誘発即時リストの即時効果のアクションを解決}
\sphinxAtStartPar
プレイヤー毎に誘発即時リストの即時効果のアクションを解決を行います。
順番はターンプレイヤーからターンが回る順にプレイヤー毎に行います。

\sphinxlineitem{{[}6\sphinxhyphen{}4{]}誘発即時リストから即時効果のアクションを1つ取り出す}
\sphinxAtStartPar
順番のプレイヤーは、 プレイヤー毎の誘発即時リストから1つ即時効果のアクションを取り出します。
取り出すアクションは任意に選択できます。

\sphinxlineitem{{[}6\sphinxhyphen{}5{]}即時効果のアクションを解決}
\sphinxAtStartPar
アクションの効果を解決します。
詳しくは {\hyperref[\detokenize{core/core:actresolve}]{\sphinxcrossref{\DUrole{std,std-ref}{{[}8{]}アクションの解決}}}} 参照。

\sphinxlineitem{{[}6\sphinxhyphen{}6{]}勝敗判定}
\sphinxAtStartPar
勝敗を判定します。
詳しくは {\hyperref[\detokenize{core/core:winlose}]{\sphinxcrossref{\DUrole{std,std-ref}{{[}9{]}勝敗判定}}}} 参照。

\sphinxlineitem{{[}6\sphinxhyphen{}7{]}誘発したアクションをプレイヤー毎の誘発即時リスト、誘発通常リストに追加}
\sphinxAtStartPar
詳しくは {\hyperref[\detokenize{core/core:trigger-act-gather}]{\sphinxcrossref{\DUrole{std,std-ref}{{[}6\sphinxhyphen{}1{]}誘発したアクションをプレイヤー毎の誘発即時リストと誘発通常リストに追加}}}} 参照。

\sphinxlineitem{{[}6\sphinxhyphen{}8{]}誘発即時リストの件数が0件でなけば繰り返す}
\sphinxAtStartPar
順番のプレイヤーの誘発即時リストに未解決の即時効果がある場合、
即時効果の解決を繰返します。

\sphinxlineitem{{[}6\sphinxhyphen{}9{]}全ての誘発即時リストの件数が0件でなければ繰り返す}
\sphinxAtStartPar
プレイヤー毎の誘発即時リストに未解決のアクションがある場合、
再びプレイヤー毎に誘発即時リストの即時効果の解決を繰返します。

\sphinxlineitem{{[}6\sphinxhyphen{}10{]}プレイヤー毎に誘発通常リストのアクションをステージに追加}
\sphinxAtStartPar
プレイヤー毎に誘発通常リストのアクションをステージに追加します。
順番はターンプレイヤーからターンが回る順にプレイヤー毎に行います。

\sphinxlineitem{{[}6\sphinxhyphen{}11{]}通常効果のアクションを任意の順でステージに追加}
\sphinxAtStartPar
順番のプレイヤーは、 プレイヤー毎の誘発通常リストからアクションを任意の順でステージに追加します。

\sphinxlineitem{{[}6\sphinxhyphen{}12{]}誘発したアクションをプレイヤー毎に誘発即時リストと誘発通常リストにまとめる}
\sphinxAtStartPar
詳しくは {\hyperref[\detokenize{core/core:trigger-act-gather}]{\sphinxcrossref{\DUrole{std,std-ref}{{[}6\sphinxhyphen{}1{]}誘発したアクションをプレイヤー毎の誘発即時リストと誘発通常リストに追加}}}} 参照。

\sphinxlineitem{{[}6\sphinxhyphen{}13{]}誘発通常リストにアクションがあれば繰り返す}
\sphinxAtStartPar
プレイヤー毎の誘発通常リストにアクションがある場合、
順番を次のプレイヤーに渡し、プレイヤー毎に誘発通常リストのアクションをステージに追加します。

\end{description}


\section{まとめ}
\label{\detokenize{core/core:id15}}
\sphinxAtStartPar
コアルールについて説明しました。
すでにあるターン制のゲームからアクションを洗い出し、能力を整理することで割込処理を可能としゲームの新しい遊び方が見つけられます。
また、新しく作成するゲームに関してもコアルールを意識して作成することで、ルール追加がしやすいゲームが考えやすいと思います。

\sphinxstepscope


\chapter{共通ルール}
\label{\detokenize{common/common:common-rst}}\label{\detokenize{common/common:id1}}\label{\detokenize{common/common::doc}}
\sphinxAtStartPar
この章では、カードの配置などコアルールで定義されていない内容を定義します。


\section{プレイ人数}
\label{\detokenize{common/common:id2}}
\sphinxAtStartPar
フォーマット、対戦レギュレーションに定義されていない場合、2人です。
プレイする際に確認してください。


\section{用意するもの}
\label{\detokenize{common/common:id3}}\begin{itemize}
\item {} 
\sphinxAtStartPar
1人1セットのトランプが必要です。

\item {} 
\sphinxAtStartPar
覚えていない場合、フォーマットに応じてアクションリスト、エクストラリストがあると便利です。

\end{itemize}


\section{使用できるトランプ}
\label{\detokenize{common/common:id4}}
\sphinxAtStartPar
BlackPokerでは次の条件を満たしたトランプを使うことができます。
一般的なトランプなら満たす条件となっています。
\begin{quote}
\begin{itemize}
\item {} 
\sphinxAtStartPar
スートと数字が分かる

\item {} 
\sphinxAtStartPar
スートの{\normalsize $\spadesuit$} {\normalsize $\heartsuit$} {\normalsize $\diamondsuit$} {\normalsize $\clubsuit$} が判断できる

\item {} 
\sphinxAtStartPar
数字のA\sphinxhyphen{}K(1\sphinxhyphen{}13)が判断できる

\item {} 
\sphinxAtStartPar
スートと数字の組合せが重複していない

\item {} 
\sphinxAtStartPar
裏から表がわからない

\item {} 
\sphinxAtStartPar
縦向き横向きが判断できる

\item {} 
\sphinxAtStartPar
Jokerは2枚まで入れられる

\item {} 
\sphinxAtStartPar
54枚無くてもよい

\end{itemize}

\sphinxAtStartPar
対戦レギュレーションにより使用できるトランプの枚数など異なる場合があるため、
対戦する際に、対戦レギュレーション、フォーマットを確認してください。
\end{quote}


\section{トランプの数字}
\label{\detokenize{common/common:id5}}
\sphinxAtStartPar
ゲーム全体を通してトランプの数字は次のような数値として扱います。(\hyperref[\detokenize{common/common:cardrank}]{Table \ref{\detokenize{common/common:cardrank}}})


\begin{savenotes}\sphinxattablestart
\sphinxthistablewithglobalstyle
\centering
\sphinxcapstartof{table}
\sphinxthecaptionisattop
\sphinxcaption{トランプの数字}\label{\detokenize{common/common:id49}}\label{\detokenize{common/common:cardrank}}
\sphinxaftertopcaption
\begin{tabulary}{\linewidth}[t]{|T|T|}
\sphinxtoprule
\sphinxstyletheadfamily 
\sphinxAtStartPar
カード
&\sphinxstyletheadfamily 
\sphinxAtStartPar
数字
\\
\sphinxmidrule
\sphinxtableatstartofbodyhook
\sphinxAtStartPar
A
&
\sphinxAtStartPar
1
\\
\sphinxhline
\sphinxAtStartPar
2〜10
&
\sphinxAtStartPar
表記どおり
\\
\sphinxhline
\sphinxAtStartPar
J
&
\sphinxAtStartPar
11
\\
\sphinxhline
\sphinxAtStartPar
Q
&
\sphinxAtStartPar
12
\\
\sphinxhline
\sphinxAtStartPar
K
&
\sphinxAtStartPar
13
\\
\sphinxhline
\sphinxAtStartPar
Joker
&
\sphinxAtStartPar
0
\\
\sphinxbottomrule
\end{tabulary}
\sphinxtableafterendhook\par
\sphinxattableend\end{savenotes}


\section{カードの配置}
\label{\detokenize{common/common:id6}}
\sphinxAtStartPar
カードの配置には次のような場所があります。(\hyperref[\detokenize{common/common:field-ex}]{Fig.\@ \ref{\detokenize{common/common:field-ex}}})

\begin{figure}[htbp]
\centering
\capstart

\noindent\sphinxincludegraphics{{field-ex}.pdf}
\caption{プレイ中のカードの配置}\label{\detokenize{common/common:id50}}\label{\detokenize{common/common:field-ex}}\end{figure}

\index{ライフ@\spxentry{ライフ}}\ignorespaces \begin{description}
\sphinxlineitem{ライフ}
\sphinxAtStartPar
山札。ゲームを始める時に自分のトランプを裏向きに置く場所です。
ダメージを受けるとライフの一番上から墓地にカードを移します。

\end{description}

\index{ぼ\textbar{}墓地@\spxentry{ぼ\textbar{}墓地}}\ignorespaces \begin{description}
\sphinxlineitem{墓地}
\sphinxAtStartPar
捨て札置き場。ダメージを受けた時などに表向きでカードを重ねて置きます。

\end{description}

\index{ば\textbar{}場@\spxentry{ば\textbar{}場}}\ignorespaces \begin{description}
\sphinxlineitem{場}
\sphinxAtStartPar
兵士や防壁などのキャラクターを置きます。

\end{description}

\index{て\textbar{}手札@\spxentry{て\textbar{}手札}}\ignorespaces \begin{description}
\sphinxlineitem{手札}
\sphinxAtStartPar
ライフから引いたカードを持っておく場所です。相手から見えないようにしましょう。

\end{description}

\index{き\textbar{}切札(場所)@\spxentry{き\textbar{}切札(場所)}}\ignorespaces \begin{description}
\sphinxlineitem{切札}
\sphinxAtStartPar
能力が割り当てられたカードを置きます。エクストラフォーマットのみで使用します。
エクストラのルールについては、 \hyperref[\detokenize{common/common:extra}]{\ref{\detokenize{common/common:extra}} \nameref{\detokenize{common/common:extra}}} で説明します。

\end{description}


\subsection{デッキとライフ}
\label{\detokenize{common/common:id7}}
\sphinxAtStartPar
対戦レギュレーションなどでデッキという表現が出てきます。

\index{デッキ@\spxentry{デッキ}}\ignorespaces \begin{description}
\sphinxlineitem{デッキ}
\sphinxAtStartPar
ゲーム開始前にゲームで使用するカードの束(カード構成)

\end{description}

\sphinxAtStartPar
ゲームの始め方を経てデッキはライフとなります。詳細は \hyperref[\detokenize{common/common:common-gamestart}]{\ref{\detokenize{common/common:common-gamestart}} \nameref{\detokenize{common/common:common-gamestart}}} で説明します。


\section{勝利条件}
\label{\detokenize{common/common:id8}}
\sphinxAtStartPar
プレイヤーは順に対戦相手に対し攻撃を行い、ダメージを与え先に相手のライフを0枚にした方が勝ちです。ダメージは1点につき1枚ライフが減ります。

\index{ダメージ@\spxentry{ダメージ}}\ignorespaces 

\section{ダメージ}
\label{\detokenize{common/common:index-6}}\label{\detokenize{common/common:id9}}
\sphinxAtStartPar
プレイヤーがダメージを受けた場合、ライフの一番上から受けた点数分墓地にカードを表向きで移動します。移動する際は、カードの表を対戦相手に見せる必要はありません。

\index{キャラクター@\spxentry{キャラクター}}\ignorespaces 

\section{キャラクター}
\label{\detokenize{common/common:index-7}}\label{\detokenize{common/common:id10}}
\sphinxAtStartPar
キャラクターとは、場に存在する兵士や防壁のことを指します。
コアルールのコンポーネントにあたります。

\sphinxAtStartPar
キャラクターは1枚のカードで1体を表すこともあれば、
複数枚で1体を表すこともあります。(\hyperref[\detokenize{common/common:character}]{Fig.\@ \ref{\detokenize{common/common:character}}})

\begin{figure}[htbp]
\centering
\capstart

\noindent\sphinxincludegraphics{{character}.pdf}
\caption{キャラクターの例}\label{\detokenize{common/common:id51}}\label{\detokenize{common/common:character}}\end{figure}


\subsection{キャラクターのもつ項目}
\label{\detokenize{common/common:id11}}
\sphinxAtStartPar
キャラクターのもつ項目について説明します。
凡例のキャラクター「一般兵」を見てみましょう。(\hyperref[\detokenize{common/common:character-sample}]{Fig.\@ \ref{\detokenize{common/common:character-sample}}})

\begin{figure}[htbp]
\centering
\capstart

\noindent\sphinxincludegraphics{{character-sample}.pdf}
\caption{一般兵}\label{\detokenize{common/common:id52}}\label{\detokenize{common/common:character-sample}}\end{figure}

\index{キャラクター名@\spxentry{キャラクター名}}\ignorespaces \begin{description}
\sphinxlineitem{キャラクター名}
\sphinxAtStartPar
キャラクターの名称を示します。

\end{description}

\index{タイプ(キャラクター)@\spxentry{タイプ(キャラクター)}}\ignorespaces \begin{description}
\sphinxlineitem{タイプ}
\sphinxAtStartPar
キャラクターのタイプを示します。タイプは兵士と防壁の2種類が存在します。

\end{description}

\index{キーカード@\spxentry{キーカード}}\ignorespaces \begin{description}
\sphinxlineitem{キーカード}
\sphinxAtStartPar
キャラクターを示すカードが記載されています。複数のカードで1体のキャラクターを示す場合もあります。

\end{description}

\index{の\textbar{}能力(キャラクター)@\spxentry{の\textbar{}能力(キャラクター)}}\ignorespaces \begin{description}
\sphinxlineitem{能力}
\sphinxAtStartPar
キャラクターが持っている能力を記載しています。

\end{description}


\subsection{キャラクターの数字}
\label{\detokenize{common/common:id12}}
\sphinxAtStartPar
トランプの数字は、キャラクターの強さを示します。
基本はカードに記載された数字を示しますが、魔法などのアクションを使うことで
加算したり減算されたりします。


\subsection{キャラクターの注意点}
\label{\detokenize{common/common:id13}}

\subsubsection{複数枚で1体となるキャラクターが防壁になったら?}
\label{\detokenize{common/common:id14}}
\sphinxAtStartPar
アクションの効果で兵士を防壁にすることがあります。
防壁は1枚で1体のキャラクターであるため、
複数枚からなるキャラクターが防壁となった場合、
複数体の防壁となります。

\sphinxAtStartPar
なお、複数枚からなるキャラクターが
墓地や手札に移った場合、
1体のキャラクターとして
扱うため複数枚合わせて移します。
チャージ状態、ドライブ状態となった場合も同様に1体のキャラクター
として扱います。

\index{チャージ@\spxentry{チャージ}}\index{ドライブ@\spxentry{ドライブ}}\ignorespaces 

\subsection{チャージとドライブ}
\label{\detokenize{common/common:index-12}}\label{\detokenize{common/common:id15}}
\sphinxAtStartPar
キャラクターには、チャージ状態とドライブ状態が存在します。
チャージ状態は未使用状態を示し、ドライブ状態は使用済み状態を示しています。
また、キャラクターを横向きにすることを「ドライブ」、縦向きにすることを「チャージ」と言います。(\hyperref[\detokenize{common/common:chargedrive}]{Fig.\@ \ref{\detokenize{common/common:chargedrive}}})

\begin{figure}[htbp]
\centering
\capstart

\noindent\sphinxincludegraphics{{charge&drive}.pdf}
\caption{チャージとドライブ}\label{\detokenize{common/common:id53}}\label{\detokenize{common/common:chargedrive}}\end{figure}


\section{ゲームの始め方}
\label{\detokenize{common/common:common-gamestart}}\label{\detokenize{common/common:id16}}
\sphinxAtStartPar
次の手順でゲームを始めます。
\begin{enumerate}
\sphinxsetlistlabels{\arabic}{enumi}{enumii}{}{.}%
\item {} 
\sphinxAtStartPar
デッキをよく切る。

\item {} 
\sphinxAtStartPar
デッキより7枚引き手札にする。

\item {} 
\sphinxAtStartPar
デッキをライフの場所に置き、ライフとする。

\item {} 
\sphinxAtStartPar
両者ライフの一番上を表にする。

\item {} 
\sphinxAtStartPar
大きい数字のプレイヤーが先攻。数字については、 \hyperref[\detokenize{common/common:cardrank}]{Table \ref{\detokenize{common/common:cardrank}}} 参照。

\item {} 
\sphinxAtStartPar
数字が同じ場合、さらにライフの一番上を表にし同様のルールで比べる。

\item {} 
\sphinxAtStartPar
表にしたカードを墓地へ移す。

\item {} 
\sphinxAtStartPar
先攻プレイヤーはライフより1枚引き手札に加える。

\item {} 
\sphinxAtStartPar
先攻プレイヤーがターンとチャンスをもちゲームを開始する。

\end{enumerate}

\sphinxAtStartPar
この行動が
\hyperref[\detokenize{core/core:coreflowsec}]{\ref{\detokenize{core/core:coreflowsec}} \nameref{\detokenize{core/core:coreflowsec}}} の
{\hyperref[\detokenize{core/core:core-gamestart}]{\sphinxcrossref{\DUrole{std,std-ref}{{[}1{]}ゲーム開始}}}} に該当します。
この後は
\hyperref[\detokenize{core/core:coreflowsec}]{\ref{\detokenize{core/core:coreflowsec}} \nameref{\detokenize{core/core:coreflowsec}}}
に準じアクションを起こしてゲームを進行します。

\sphinxAtStartPar
ゲーム内で起こせるアクションは対戦レギュレーション、フォーマットより異なります。
対戦前に確認してください。

\index{アクション@\spxentry{アクション}}\ignorespaces 

\section{アクション}
\label{\detokenize{common/common:index-13}}\label{\detokenize{common/common:id17}}

\subsection{アクションが持つ項目}
\label{\detokenize{common/common:id18}}
\sphinxAtStartPar
アクションが持つ項目について説明します。
凡例の「サンプル」アクションを見てみましょう。(\hyperref[\detokenize{common/common:action-sample}]{Fig.\@ \ref{\detokenize{common/common:action-sample}}})

\begin{figure}[htbp]
\centering
\capstart

\noindent\sphinxincludegraphics{{action-sample}.pdf}
\caption{サンプルアクション}\label{\detokenize{common/common:id54}}\label{\detokenize{common/common:action-sample}}\end{figure}

\index{アクション名@\spxentry{アクション名}}\ignorespaces \begin{description}
\sphinxlineitem{アクション名}
\sphinxAtStartPar
アクションの名称を示します。

\end{description}

\index{キーカード(アクション)@\spxentry{キーカード(アクション)}}\ignorespaces \begin{description}
\sphinxlineitem{キーカード}
\sphinxAtStartPar
アクションの核となるカードを示します。
キーカードは★を使って表記します。
凡例の場合、手札からコストとは別に{\normalsize $\heartsuit$} A〜10に該当するカードを1枚
キーカードとして使用します。

\end{description}

\index{と\textbar{}特記事項@\spxentry{と\textbar{}特記事項}}\ignorespaces \begin{description}
\sphinxlineitem{特記事項}
\sphinxAtStartPar
特記事項は※を使って表記し、その他の項目では書き表せない条件を示します。

\end{description}

\index{た\textbar{}対象@\spxentry{た\textbar{}対象}}\ignorespaces \begin{description}
\sphinxlineitem{対象}
\sphinxAtStartPar
効果の対象を示します。

\end{description}

\index{つ\textbar{}通常効果@\spxentry{つ\textbar{}通常効果}!そ\textbar{}即時効果@\spxentry{そ\textbar{}即時効果}}\index{そ\textbar{}即時効果@\spxentry{そ\textbar{}即時効果}!つ\textbar{}通常効果@\spxentry{つ\textbar{}通常効果}}\ignorespaces \begin{description}
\sphinxlineitem{即時効果/通常効果}
\sphinxAtStartPar
効果の内容を示します。

\end{description}

\index{コスト@\spxentry{コスト}}\ignorespaces \begin{description}
\sphinxlineitem{コスト}
\sphinxAtStartPar
アクションを起こすのに必要な対価です。
コストは$を使って表記し、コストの支払いはアクションを起こすプレイヤーが行います。コストの種類は \hyperref[\detokenize{common/common:cost}]{\ref{\detokenize{common/common:cost}} \nameref{\detokenize{common/common:cost}}} で説明します。

\end{description}

\index{タイミング@\spxentry{タイミング}}\ignorespaces \begin{description}
\sphinxlineitem{タイミング}
\sphinxAtStartPar
アクションを起こせる時を示します。
タイミングはコアルール \hyperref[\detokenize{core/core:timing}]{\ref{\detokenize{core/core:timing}} \nameref{\detokenize{core/core:timing}}} を参照してください。

\end{description}

\index{タイプ(アクション)@\spxentry{タイプ(アクション)}}\ignorespaces \begin{description}
\sphinxlineitem{タイプ}
\sphinxAtStartPar
アクションの種類を表します。アクション名の後に括弧書きで記載します。

\end{description}


\subsubsection{記載されていないアクションの項目}
\label{\detokenize{common/common:id19}}
\sphinxAtStartPar
アクションによっては記載されていない項目もあります。
記載されていない項目は無視して構いません。
たとえばコスト項目がなければコストを支払う必要はありません。


\subsection{コストの種類}
\label{\detokenize{common/common:cost}}\label{\detokenize{common/common:id20}}
\sphinxAtStartPar
アクションによって支払うコストが異なります。
コストには次の種類があり、それぞれ支払い方が異なります。(\hyperref[\detokenize{common/common:table-cost}]{Table \ref{\detokenize{common/common:table-cost}}})


\begin{savenotes}\sphinxattablestart
\sphinxthistablewithglobalstyle
\centering
\sphinxcapstartof{table}
\sphinxthecaptionisattop
\sphinxcaption{コストの種類}\label{\detokenize{common/common:id55}}\label{\detokenize{common/common:table-cost}}
\sphinxaftertopcaption
\begin{tabulary}{\linewidth}[t]{|T|T|}
\sphinxtoprule
\sphinxstyletheadfamily 
\sphinxAtStartPar
表記(名称)
&\sphinxstyletheadfamily 
\sphinxAtStartPar
対価
\\
\sphinxmidrule
\sphinxtableatstartofbodyhook
\sphinxAtStartPar
B (Bulwark)
&
\sphinxAtStartPar
防壁をドライブする
\\
\sphinxhline
\sphinxAtStartPar
L (Life)
&
\sphinxAtStartPar
1点ダメージを受ける
\\
\sphinxhline
\sphinxAtStartPar
D (Discard)
&
\sphinxAtStartPar
手札を1枚捨てる
\\
\sphinxhline
\sphinxAtStartPar
S (Sacrifice)
&
\sphinxAtStartPar
キャラクター1体を墓地に移す
\\
\sphinxbottomrule
\end{tabulary}
\sphinxtableafterendhook\par
\sphinxattableend\end{savenotes}

\sphinxAtStartPar
たとえばコストが \sphinxstylestrong{「\$BL」} の場合、自分の場にいるチャージ状態の防壁を1体ドライブし、1点ダメージを受けることでコストが支払われたことになります。


\subsection{アクションの起こし方}
\label{\detokenize{common/common:id21}}
\sphinxAtStartPar
次の手順でアクションを起こします。
\begin{enumerate}
\sphinxsetlistlabels{\arabic}{enumi}{enumii}{}{.}%
\item {} 
\sphinxAtStartPar
起こすアクションを対戦相手に伝える。

\item {} 
\sphinxAtStartPar
アクションに応じたコストを支払う。

\item {} 
\sphinxAtStartPar
必要なら手札からキーカードを出す。

\item {} 
\sphinxAtStartPar
対象の指定が必要な場合、対象を指定する。

\end{enumerate}

\sphinxAtStartPar
「サンプル」アクションを起こす例を見てみましょう。(\hyperref[\detokenize{common/common:action-sample2}]{Fig.\@ \ref{\detokenize{common/common:action-sample2}}})

\begin{figure}[htbp]
\centering
\capstart

\noindent\sphinxincludegraphics{{action-sample2}.pdf}
\caption{アクションを起こす例}\label{\detokenize{common/common:id56}}\label{\detokenize{common/common:action-sample2}}\end{figure}


\subsubsection{アクションを起こすときの注意点}
\label{\detokenize{common/common:id22}}

\paragraph{対象を指定しないでアクションを起こせるか?}
\label{\detokenize{common/common:id23}}
\sphinxAtStartPar
「サンプル」アクションのように対象を指定するアクションがあります。
「対象」項目がある場合、記載された条件を満たした対象を指定できなければ、
そのアクションを起こすことはできません。


\paragraph{アクションを対象とするアクションは自身を対象にできるか?}
\label{\detokenize{common/common:id24}}
\sphinxAtStartPar
アクションは、自分自身を対象とすることはできません。
そのため、「カウンター」アクションのようにアクションを対象とするアクションは
自身を対象とすることはできません。


\subsection{アクションの解決}
\label{\detokenize{common/common:id25}}
\sphinxAtStartPar
\hyperref[\detokenize{core/core:coreflowsec}]{\ref{\detokenize{core/core:coreflowsec}} \nameref{\detokenize{core/core:coreflowsec}}} の
{\hyperref[\detokenize{core/core:actresolve}]{\sphinxcrossref{\DUrole{std,std-ref}{{[}8{]}アクションの解決}}}} に行うことを順に示します。


\subsubsection{対象条件を確認}
\label{\detokenize{common/common:id26}}
\sphinxAtStartPar
対象を指定するアクションが効果を発揮しようとした時に次の条件に該当する場合、効果を発揮する対象を失うため効果が発揮されず
アクションが解決されます。
\begin{itemize}
\item {} 
\sphinxAtStartPar
対象が存在していない場合

\item {} 
\sphinxAtStartPar
対象が分裂した場合

\end{itemize}

\sphinxAtStartPar
たとえば兵士に対して「アップ」アクションを起こし、対応して「ダウン」
アクションを起こされました。
「ダウン」の方が先に解決されるため、「アップ」を解決する時には
兵士が墓地に移っていたとします。その場合、「アップ」アクションは効果を発揮せず解決されます。

\sphinxAtStartPar
「リバース」による対象分裂も同様です。
たとえば装備兵に対して「ツイスト」アクションを起こし、対応して「リバース」アクションを起こしたとします。
この場合、「リバース」が先に解決され、装備兵が分裂します。
その場合、「ツイスト」は対象を失いアクションの効果を発揮せず解決されます。


\subsubsection{効果を発揮}
\label{\detokenize{common/common:id27}}
\sphinxAtStartPar
アクションの効果に定義されている内容を実行します。
効果の中に実行不可能な部分がある場合、可能な部分のみ実行します。

\sphinxAtStartPar
たとえば、ライフの枚数が残1枚の時に5点のダメージを受けたとします。
ライフは1枚しかないので5点ダメージを受けることはできませんが、
1点までなら受けることが可能なため、
この場合1点のダメージを受けることになります。


\subsubsection{キーカードを墓地に移す}
\label{\detokenize{common/common:keycard-gy}}\label{\detokenize{common/common:id28}}
\sphinxAtStartPar
効果を発揮した後、そのアクションをステージから取り除き、キーカードを墓地に移します。
ただし効果によってキーカードを場に出した場合や手札に戻した場合、
そのカードを移す先が明確になっているため、墓地には移しません。


\subsection{勝敗判定}
\label{\detokenize{common/common:id29}}
\sphinxAtStartPar
{\hyperref[\detokenize{core/core:winlose}]{\sphinxcrossref{\DUrole{std,std-ref}{{[}9{]}勝敗判定}}}} で確認する内容は次になります。

\sphinxAtStartPar
ライフを確認し0枚の場合そのプレイヤーは敗北となります。両プレイヤーのライフが0枚の場合、引き分けとなります。


\subsection{その他補足事項}
\label{\detokenize{common/common:id30}}

\subsubsection{防壁の置き方}
\label{\detokenize{common/common:id31}}
\sphinxAtStartPar
防壁を場に出すときは次のルールにしたがって場に出して下さい。(\hyperref[\detokenize{common/common:set-bulwork}]{Fig.\@ \ref{\detokenize{common/common:set-bulwork}}})
\begin{itemize}
\item {} 
\sphinxAtStartPar
防壁を置く時はライフ側に詰めて置いて下さい。

\item {} 
\sphinxAtStartPar
防壁の左右の入れ替えは行わないでください。

\end{itemize}

\begin{figure}[htbp]
\centering
\capstart

\noindent\sphinxincludegraphics{{set-bulwork}.pdf}
\caption{防壁の置き方}\label{\detokenize{common/common:id57}}\label{\detokenize{common/common:set-bulwork}}\end{figure}


\subsubsection{1ターンに1回制限}
\label{\detokenize{common/common:id32}}
\sphinxAtStartPar
特記事項に「プレイヤーは1ターンに1回しかこのアクションを起こすことができない。」と記載されているアクションは、
ターンを持っているプレイヤーが変わるまでの間に1回しか起こす
ことができません。

\sphinxAtStartPar
ターンを持っているプレイヤーが変わればまた起こすことができます。


\subsubsection{直接起こせないアクション}
\label{\detokenize{common/common:id33}}
\sphinxAtStartPar
特記事項に「プレイヤーはこのアクションを直接起こすことが出来ない。」
と記載されているアクションは、
プレイヤーがチャンスを持っていても
アクションを起こすことができません。
また、この特記事項が記載されたアクションが何らかの起因で起きても、プレイヤーが起こした訳ではないためパスは自動的に発生せず、チャンスは移りません。

\index{エクストラ@\spxentry{エクストラ}}\ignorespaces 

\section{エクストラ}
\label{\detokenize{common/common:extra}}\label{\detokenize{common/common:index-22}}\label{\detokenize{common/common:id34}}
\sphinxAtStartPar
エクストラではアクションに加え切札の能力を使うことができます。
使用できるアクション、切札は対戦レギュレーションを確認してください。

\index{き\textbar{}切札@\spxentry{き\textbar{}切札}}\ignorespaces 

\subsection{切札}
\label{\detokenize{common/common:index-23}}\label{\detokenize{common/common:id35}}
\sphinxAtStartPar
切札とは、切札領域に置かれたカードを示します。
具体的な切札の置き場所については、 \hyperref[\detokenize{common/common:field-ex}]{Fig.\@ \ref{\detokenize{common/common:field-ex}}} を参照して下さい。
切札には各々能力が割り当てられており、表にするとその能力が有効になります。
切札を操作するアクションは、「エクストラリスト」を参照して下さい。


\subsection{バージョン}
\label{\detokenize{common/common:id36}}
\sphinxAtStartPar
エクストラには、バージョンが存在します。
対戦を開始する前に対戦相手とバージョンの確認をしましょう。


\subsubsection{版数との関係}
\label{\detokenize{common/common:id37}}
\sphinxAtStartPar
版数毎に使える切札の種類が異なります。
たとえば、第一版、第二版ではエクストラで遊ぶことはできません。
第三版以降は、次版が出るまでの間に公開された切札であれば使用できます。

\sphinxAtStartPar
バージョンは以下のような命名規則になっています。

\begin{sphinxVerbatim}[commandchars=\\\{\}]
\PYG{n}{ex}\PYG{p}{\PYGZob{}}\PYG{n}{版数}\PYG{p}{\PYGZcb{}}\PYG{o}{.}\PYG{p}{\PYGZob{}}\PYG{n}{切札枚数}\PYG{p}{\PYGZcb{}}\PYG{o}{.}\PYG{p}{\PYGZob{}}\PYG{n}{更新回数}\PYG{p}{\PYGZcb{}}
\end{sphinxVerbatim}

\sphinxAtStartPar
各々は次の意味になります。
\begin{description}
\sphinxlineitem{版数}
\sphinxAtStartPar
対応する版数

\sphinxlineitem{切札枚数}
\sphinxAtStartPar
定義されている切札の枚数

\sphinxlineitem{更新回数}
\sphinxAtStartPar
定義されてから時点から更新された回数。0始まりで、版数が更新されるたびにリセットされます。

\end{description}

\sphinxAtStartPar
例えば、次のように表記されています。

\begin{sphinxVerbatim}[commandchars=\\\{\}]
\PYG{n}{ex5}\PYG{l+m+mf}{.30}\PYG{l+m+mf}{.2}
\end{sphinxVerbatim}


\subsection{ゲームのはじめ方}
\label{\detokenize{common/common:extra-start}}\label{\detokenize{common/common:id38}}
\sphinxAtStartPar
エクストラでは、切札を置いてからゲームを始めます。
切札を置くルールは次のようになっています。(\hyperref[\detokenize{common/common:trump}]{Fig.\@ \ref{\detokenize{common/common:trump}}})
\begin{itemize}
\item {} 
\sphinxAtStartPar
対戦前に裏向きで2枚まで切札を置くことができる。

\item {} 
\sphinxAtStartPar
切札はライフと角度を変えて交わるようにライフの下に置く。

\item {} 
\sphinxAtStartPar
切札を表にするときはスートと数字が見えるようにし、対応する能力の名称を言う。

\item {} 
\sphinxAtStartPar
ライフが0枚になった場合、切札が残っていても敗北する。

\item {} 
\sphinxAtStartPar
能力が割り当てられていないカードも切札にできるが、表になっても能力が有効にならない。

\end{itemize}

\begin{figure}[htbp]
\centering
\capstart

\noindent\sphinxincludegraphics{{trump}.pdf}
\caption{切札の置き方}\label{\detokenize{common/common:id58}}\label{\detokenize{common/common:trump}}\end{figure}

\sphinxAtStartPar
これ以降は、通常のゲームの始め方と同様です。


\subsection{切札の能力}
\label{\detokenize{common/common:id39}}
\sphinxAtStartPar
エクストラでは切札を使って能力を得ることができます。
切札1枚毎に異なった能力が割り当てられており、
表にすることで能力が有効になります。
割り当てられている能力については、「エクストラリスト」を参照して下さい。


\subsubsection{能力を有効にする}
\label{\detokenize{common/common:id40}}
\sphinxAtStartPar
切札に割り当てられた能力は
「オープン」アクションを起こし表にすることで有効になります。(\hyperref[\detokenize{common/common:trump-open}]{Fig.\@ \ref{\detokenize{common/common:trump-open}}})
「オープン」アクションの詳細は、 \hyperref[\detokenize{appendix/appendix:extralist}]{\ref{\detokenize{appendix/appendix:extralist}} \nameref{\detokenize{appendix/appendix:extralist}}} を参照して下さい。
切札が表でいる限り、
その切札の能力は持続的に有効になります。
また切札を表にする時は、
対戦相手に有効となった能力が分かるように、
能力の名称を言いスートと数字が見えるようにしましょう。

\begin{figure}[htbp]
\centering
\capstart

\noindent\sphinxincludegraphics{{trump-open}.pdf}
\caption{切札を表にする例}\label{\detokenize{common/common:id59}}\label{\detokenize{common/common:trump-open}}\end{figure}


\subsubsection{能力を無効する}
\label{\detokenize{common/common:id41}}
\sphinxAtStartPar
切札は裏向きもしくは、
墓地に移されると能力が無効になります。
切札を無効化するためには、「クローズ」アクションを用い
切札を裏向きにするか、
「切札破壊」アクションを用いて切札を破壊しましょう。
「クローズ」アクション、
「切札破壊」アクションの詳細は、 \hyperref[\detokenize{appendix/appendix:extralist}]{\ref{\detokenize{appendix/appendix:extralist}} \nameref{\detokenize{appendix/appendix:extralist}}} を参照して下さい。


\subsection{エクストラ注意事項}
\label{\detokenize{common/common:id42}}

\subsubsection{1ターンに1回制限のアクションについて}
\label{\detokenize{common/common:id43}}
\sphinxAtStartPar
切札がもたらすアクションの中には「プレイヤーは1ターンに1回しかこのアクションを起こすことができない。」
と特記事項に記載されているものがあります。
このアクションは1ターンに1回しか起こすことができないため、
切札が無効化され再度オープンし有効となっても、そのターンを通して1回しか起こすことができません。


\section{その他のルール}
\label{\detokenize{common/common:id44}}
\sphinxAtStartPar
この章では、
公開・非公開情報やシャッフルの仕方といった
細かな決まりごとを説明します。


\subsection{公開・非公開情報}
\label{\detokenize{common/common:id45}}
\sphinxAtStartPar
配置されているカードには、アクションの効果
を使わなくても中身や枚数を知れるものがあります。
知れる度合いには次の種類があります。
\begin{description}
\sphinxlineitem{完全公開}
\sphinxAtStartPar
全てのプレイヤーが知ることができ、
聞かれたプレイヤーは正しく答える必要がある

\sphinxlineitem{個人公開}
\sphinxAtStartPar
ライフの持ち主のみ知ることができる

\sphinxlineitem{非公開}
\sphinxAtStartPar
全てのプレイヤーは知ることができない

\end{description}

\sphinxAtStartPar
完全公開の情報であれば、ゲーム中いつでも対戦相手に聞くことができます。
各カードの配置と公開・非公開の度合いは次のとおりです。
\begin{description}
\sphinxlineitem{ライフ}
\begin{DUlineblock}{0em}
\item[] 完全公開:10枚未満のライフ枚数
\item[] 個人公開:ライフの枚数
\item[] 非公開:ライフの中身
\end{DUlineblock}

\sphinxlineitem{墓地}
\begin{DUlineblock}{0em}
\item[] 完全公開:墓地の一番上のカード
\item[] 個人公開:墓地の中身
\item[] 非公開:なし
\end{DUlineblock}

\sphinxlineitem{場}
\begin{DUlineblock}{0em}
\item[] 完全公開:表裏を変えずに見えるカード
\item[] 個人公開:伏せてあるカード
\item[] 非公開:なし
\end{DUlineblock}

\sphinxlineitem{手札}
\begin{DUlineblock}{0em}
\item[] 完全公開:手札の枚数
\item[] 個人公開:手札の中身
\item[] 非公開:なし
\end{DUlineblock}

\sphinxlineitem{切札}
\begin{DUlineblock}{0em}
\item[] 完全公開:表裏を変えずに見えるカード
\item[] 個人公開:伏せてあるカード
\item[] 非公開:なし
\end{DUlineblock}

\end{description}


\subsubsection{残りライフを聞かれたらどうしたらいいの?}
\label{\detokenize{common/common:id46}}
\sphinxAtStartPar
対戦相手から残りのライフを聞かれた場合、自分のライフの枚数を上から10枚まで数え、相手に数えたカードの枚数が分かるように裏向きで見せます。
10枚未満であれば枚数を答え、10枚以上の場合「10枚以上です」と答えて下さい。
10枚以上の場合、正確な枚数を答える必要はありません。


\subsubsection{墓地の一番上のカードはいつ決まるのか?}
\label{\detokenize{common/common:id47}}
\sphinxAtStartPar
カードを墓地に移す際に移すカードの中から1枚を公開してください。
すでに墓地にあるカードを改めて公開しないでください。


\subsection{デッキのシャッフルについて}
\label{\detokenize{common/common:id48}}
\sphinxAtStartPar
BlackPokerでは
コンセプトの1つに”相手のカードに触らない”があるため、
対戦相手にデッキのシャッフルをお願いする必要はありません。

\sphinxAtStartPar
ただシャッフルしてほしいのであれば、お願いしても構いません。
逆に、対戦相手があまりシャッフルしていない場合は、
さらにシャッフルをお願いできます。

\sphinxstepscope


\chapter{フォーマット}
\label{\detokenize{format/format:format-rst}}\label{\detokenize{format/format:id1}}\label{\detokenize{format/format::doc}}
\index{フォーマット@\spxentry{フォーマット}}\ignorespaces 

\section{フォーマットとは}
\label{\detokenize{format/format:index-0}}\label{\detokenize{format/format:id2}}
\sphinxAtStartPar
BlackPokerにはいくつかのフォーマットがあり、
フォーマットによりゲーム内でできる行動が異なります。
同じトランプでもフォーマットを変えることで様々な遊び方ができます。

\sphinxAtStartPar
BlackPokerはアクションという行動を起こし、兵士などのキャラクターを出してターンを進めていくゲームです。
アクション、キャラクターの種類は、ライト < スタンダード < プロ < マスター < エクストラの順に増えていきます。
カードゲームでいうところのカードの種類が増えていくイメージです。
覚える量が多いほど難易度が高いため、初心者はライトから始めることをお勧めします。

\sphinxAtStartPar
アクションはアクションリストに記載されており、
フォーマットによって参照するアクションリストが異なります。


\subsection{対戦レギュレーションとの違い}
\label{\detokenize{format/format:id3}}
\sphinxAtStartPar
フォーマットと対戦レギュレーションの違いは、
フォーマットはゲーム内でできるアクション等のできる行動を定義している
のに対して、対戦レギュレーションはフォーマットを前提としてそれを加工している位置づけになります。

\sphinxAtStartPar
各々の関係性は \hyperref[\detokenize{init/init:rule-constract}]{\ref{\detokenize{init/init:rule-constract}} \nameref{\detokenize{init/init:rule-constract}}} を参照してください。


\section{定義項目}
\label{\detokenize{format/format:id4}}
\sphinxAtStartPar
フォーマットには次の項目が定義されています。
\begin{description}
\sphinxlineitem{アクションリスト}
\sphinxAtStartPar
起こせるアクション、キャラクターのリスト

\sphinxlineitem{エクストラリスト}
\sphinxAtStartPar
切札のリスト

\end{description}


\section{フォーマット定義}
\label{\detokenize{format/format:id5}}
\sphinxAtStartPar
公式として次のフォーマットを定義しています。(\hyperref[\detokenize{format/format:format-list}]{Table \ref{\detokenize{format/format:format-list}}})


\begin{savenotes}\sphinxattablestart
\sphinxthistablewithglobalstyle
\centering
\sphinxcapstartof{table}
\sphinxthecaptionisattop
\sphinxcaption{フォーマット一覧}\label{\detokenize{format/format:id6}}\label{\detokenize{format/format:format-list}}
\sphinxaftertopcaption
\begin{tabulary}{\linewidth}[t]{|T|T|T|}
\sphinxtoprule
\sphinxstyletheadfamily 
\sphinxAtStartPar
フォーマット
&\sphinxstyletheadfamily 
\sphinxAtStartPar
アクションリスト
&\sphinxstyletheadfamily 
\sphinxAtStartPar
エクストラリスト
\\
\sphinxmidrule
\sphinxtableatstartofbodyhook
\sphinxAtStartPar
ライト
&
\sphinxAtStartPar
\hyperref[\detokenize{appendix/appendix:actionlist-lite}]{\ref{\detokenize{appendix/appendix:actionlist-lite}} \nameref{\detokenize{appendix/appendix:actionlist-lite}}}
&
\sphinxAtStartPar
x
\\
\sphinxhline
\sphinxAtStartPar
スタンダード
&
\sphinxAtStartPar
\hyperref[\detokenize{appendix/appendix:actionlist-std}]{\ref{\detokenize{appendix/appendix:actionlist-std}} \nameref{\detokenize{appendix/appendix:actionlist-std}}}
&
\sphinxAtStartPar
x
\\
\sphinxhline
\sphinxAtStartPar
プロ
&
\sphinxAtStartPar
\hyperref[\detokenize{appendix/appendix:actionlist-pro}]{\ref{\detokenize{appendix/appendix:actionlist-pro}} \nameref{\detokenize{appendix/appendix:actionlist-pro}}}
&
\sphinxAtStartPar
x
\\
\sphinxhline
\sphinxAtStartPar
マスター
&
\sphinxAtStartPar
\hyperref[\detokenize{appendix/appendix:actionlist-master}]{\ref{\detokenize{appendix/appendix:actionlist-master}} \nameref{\detokenize{appendix/appendix:actionlist-master}}}
&
\sphinxAtStartPar
x
\\
\sphinxhline
\sphinxAtStartPar
エクストラ
&
\sphinxAtStartPar
\hyperref[\detokenize{appendix/appendix:actionlist-master}]{\ref{\detokenize{appendix/appendix:actionlist-master}} \nameref{\detokenize{appendix/appendix:actionlist-master}}}
&
\sphinxAtStartPar
\hyperref[\detokenize{appendix/appendix:extralist}]{\ref{\detokenize{appendix/appendix:extralist}} \nameref{\detokenize{appendix/appendix:extralist}}}
\\
\sphinxbottomrule
\end{tabulary}
\sphinxtableafterendhook\par
\sphinxattableend\end{savenotes}

\sphinxAtStartPar
エクストラの始め方は \hyperref[\detokenize{common/common:extra-start}]{\ref{\detokenize{common/common:extra-start}} \nameref{\detokenize{common/common:extra-start}}} 参照。

\sphinxstepscope


\chapter{対戦レギュレーション}
\label{\detokenize{match-regulations/match-regulations:match-regulations-rst}}\label{\detokenize{match-regulations/match-regulations:id1}}\label{\detokenize{match-regulations/match-regulations::doc}}
\index{た\textbar{}対戦レギュレーション@\spxentry{た\textbar{}対戦レギュレーション}}\ignorespaces 

\section{対戦レギュレーションとは}
\label{\detokenize{match-regulations/match-regulations:index-0}}\label{\detokenize{match-regulations/match-regulations:id2}}
\sphinxAtStartPar
対戦レギュレーションとは、
BlackPokerで対戦する前にプレイヤー間で決定する
規則のことです。

\sphinxAtStartPar
BlackPokerはトランプだけで遊べるため、
対戦する前にプレイヤー間でルールのすり合わせをする必要があります。


\section{定義項目}
\label{\detokenize{match-regulations/match-regulations:id3}}
\sphinxAtStartPar
対戦レギュレーションは次の各項目を決めることで決定します。

\index{フォーマット(対戦レギュレーション)@\spxentry{フォーマット(対戦レギュレーション)}}\ignorespaces \begin{description}
\sphinxlineitem{フォーマット}
\sphinxAtStartPar
使用するフォーマット。詳しくは \hyperref[\detokenize{format/format:format-rst}]{\ref{\detokenize{format/format:format-rst}} \nameref{\detokenize{format/format:format-rst}}} 参照

\end{description}

\index{フレーム@\spxentry{フレーム}}\ignorespaces \begin{description}
\sphinxlineitem{フレーム}
\sphinxAtStartPar
デッキの構成方法、マッチ形式。

\end{description}

\index{オプション@\spxentry{オプション}}\ignorespaces \begin{description}
\sphinxlineitem{オプション}
\sphinxAtStartPar
フレームとフォーマットで規定されていないルールを追加。複数指定することが出来ます。

\end{description}


\subsection{対戦レギュレーションの表記}
\label{\detokenize{match-regulations/match-regulations:id4}}
\sphinxAtStartPar
対戦レギュレーションは次のように表記します。

\begin{sphinxVerbatim}[commandchars=\\\{\}]
\PYG{n}{フォーマット}\PYG{o}{+}\PYG{n}{フレーム}\PYG{o}{+}\PYG{n}{オプション}
\end{sphinxVerbatim}

\sphinxAtStartPar
例えば、次のように表記します。

\begin{sphinxVerbatim}[commandchars=\\\{\}]
\PYG{n}{マスター}\PYG{o}{+}\PYG{n}{ランダムハーフ}\PYG{o}{+}\PYG{n}{プリセット2}
\end{sphinxVerbatim}

\sphinxAtStartPar
オプションは複数指定することが出来ます。
指定した例は次のようになります。

\begin{sphinxVerbatim}[commandchars=\\\{\}]
\PYG{n}{マスター}\PYG{o}{+}\PYG{n}{ランダム40}\PYG{o}{+}\PYG{n}{プリセット2}\PYG{o}{+}\PYG{n}{パック}
\end{sphinxVerbatim}

\index{た\textbar{}対戦@\spxentry{た\textbar{}対戦}!マッチ@\spxentry{マッチ}}\index{マッチ@\spxentry{マッチ}!た\textbar{}対戦@\spxentry{た\textbar{}対戦}}\index{ゲーム@\spxentry{ゲーム}!し\textbar{}試合@\spxentry{し\textbar{}試合}}\index{し\textbar{}試合@\spxentry{し\textbar{}試合}!ゲーム@\spxentry{ゲーム}}\ignorespaces 

\subsection{対戦(マッチ)とゲーム(試合)}
\label{\detokenize{match-regulations/match-regulations:index-4}}\label{\detokenize{match-regulations/match-regulations:id5}}
\sphinxAtStartPar
対戦(マッチ)とゲーム(試合)について説明します。
\begin{description}
\sphinxlineitem{対戦(マッチ)}
\sphinxAtStartPar
プレイヤー間の一連のゲーム(試合)で構成される対戦全体を指します。

\sphinxlineitem{ゲーム(試合)}
\sphinxAtStartPar
対戦の中で行われる個々の勝負のことを示します。各ゲームは対戦の結果に貢献し、ゲームの集合体が最終的に対戦の勝者を決定します。

\end{description}

\sphinxAtStartPar
例えば、3回中2回先に勝った方を勝者とする場合、これ全体を対戦(マッチ)と呼び、個々の回をゲームと呼びます。


\section{フレーム定義}
\label{\detokenize{match-regulations/match-regulations:id6}}
\sphinxAtStartPar
公式として次のフレームを定義しています。(\hyperref[\detokenize{match-regulations/match-regulations:frame-table}]{Table \ref{\detokenize{match-regulations/match-regulations:frame-table}}})


\begin{savenotes}\sphinxattablestart
\sphinxthistablewithglobalstyle
\centering
\sphinxcapstartof{table}
\sphinxthecaptionisattop
\sphinxcaption{フレーム一覧}\label{\detokenize{match-regulations/match-regulations:id13}}\label{\detokenize{match-regulations/match-regulations:frame-table}}
\sphinxaftertopcaption
\begin{tabulary}{\linewidth}[t]{|T|T|T|T|T|T|}
\sphinxtoprule
\sphinxstyletheadfamily 
\sphinxAtStartPar
フレーム
&\sphinxstyletheadfamily 
\sphinxAtStartPar
構築タイプ
&\sphinxstyletheadfamily 
\sphinxAtStartPar
マッチ形式
&\sphinxstyletheadfamily 
\sphinxAtStartPar
デッキ構成方法
&\sphinxstyletheadfamily 
\sphinxAtStartPar
デッキ1
&\sphinxstyletheadfamily 
\sphinxAtStartPar
デッキ2
\\
\sphinxmidrule
\sphinxtableatstartofbodyhook
\sphinxAtStartPar
エントリー
&
\sphinxAtStartPar
固定
&
\sphinxAtStartPar
1デッキマッチ
&
\sphinxAtStartPar
{\normalsize $\spadesuit$} A2345 {\normalsize $\heartsuit$} A8910J {\normalsize $\diamondsuit$} A3710Q {\normalsize $\clubsuit$} A5610K Jokerの21枚
&
\sphinxAtStartPar
21枚
&
\sphinxAtStartPar
x
\\
\sphinxhline
\sphinxAtStartPar
ランダム40
&
\sphinxAtStartPar
ランダム
&
\sphinxAtStartPar
1デッキマッチ
&
\sphinxAtStartPar
54枚からランダムに14枚のカードを抜く
&
\sphinxAtStartPar
40枚
&
\sphinxAtStartPar
x
\\
\sphinxhline
\sphinxAtStartPar
ランダムハーフ
&
\sphinxAtStartPar
ランダム
&
\sphinxAtStartPar
3デッキマッチ
&
\sphinxAtStartPar
54枚をランダムに2つに分ける(27枚ずつでなくてもよい)
&
\sphinxAtStartPar
n枚
&
\sphinxAtStartPar
(54\sphinxhyphen{}n)枚
\\
\sphinxhline
\sphinxAtStartPar
構築ハーフ
&
\sphinxAtStartPar
構築
&
\sphinxAtStartPar
3デッキマッチ
&
\sphinxAtStartPar
54枚で2つのデッキを作る(カードを抜いてもよい)
&
\sphinxAtStartPar
n枚
&
\sphinxAtStartPar
(54\sphinxhyphen{}n)枚以下
\\
\sphinxhline
\sphinxAtStartPar
構築40
&
\sphinxAtStartPar
構築
&
\sphinxAtStartPar
1デッキマッチ
&
\sphinxAtStartPar
54枚から14枚以上のカードを抜く
&
\sphinxAtStartPar
40枚以下
&
\sphinxAtStartPar
x
\\
\sphinxhline
\sphinxAtStartPar
レギュラー
&
\sphinxAtStartPar
構築
&
\sphinxAtStartPar
1デッキマッチ
&
\sphinxAtStartPar
54枚からカードを抜いてもよい
&
\sphinxAtStartPar
54枚以下
&
\sphinxAtStartPar
x
\\
\sphinxhline
\sphinxAtStartPar
フル
&
\sphinxAtStartPar
固定
&
\sphinxAtStartPar
1デッキマッチ
&
\sphinxAtStartPar
54枚すべて
&
\sphinxAtStartPar
54枚
&
\sphinxAtStartPar
x
\\
\sphinxbottomrule
\end{tabulary}
\sphinxtableafterendhook\par
\sphinxattableend\end{savenotes}


\subsection{マッチ形式}
\label{\detokenize{match-regulations/match-regulations:id7}}\begin{description}
\sphinxlineitem{1デッキマッチ}\begin{itemize}
\item {} 
\sphinxAtStartPar
各プレイヤーが1つのデッキを使って1ゲームを行い勝ったプレイヤーのマッチ勝利とする対戦形式

\end{itemize}

\sphinxlineitem{2デッキマッチ}\begin{itemize}
\item {} 
\sphinxAtStartPar
各プレイヤーが2つのデッキを使い、最大3ゲームを行って2ゲームに勝ったプレイヤーの対戦勝利とする対戦形式

\item {} 
\sphinxAtStartPar
各プレイヤーはゲームで使用するデッキを選択してよい

\item {} 
\sphinxAtStartPar
ただし、一度ゲームに勝ったデッキはそれ以降使用できない

\item {} 
\sphinxAtStartPar
デッキは非公開(自分のデッキも見ることができない)とする

\item {} 
\sphinxAtStartPar
ただし、ゲーム終了直後のみ、そのゲームで使用したデッキは個人公開(自分だけ見てよい)となる

\end{itemize}

\sphinxlineitem{2デッキマッチの手順}\begin{itemize}
\item {} 
\sphinxAtStartPar
2デッキマッチのゲームの進め方と対戦勝敗を決める手順

\item {} 
\sphinxAtStartPar
各プレイヤーは使用するデッキを選び、使用しないデッキを左上に裏向き横向きに置く

\item {} 
\sphinxAtStartPar
1ゲーム目を行う

\item {} 
\sphinxAtStartPar
1ゲーム目が終わったら一時的に1ゲーム目で使用した自分のデッキを見てよい(個人公開)

\item {} 
\sphinxAtStartPar
1ゲーム目で勝ったプレイヤーは1ゲーム目で使用したデッキを左上に表向き横向きに置く

\item {} 
\sphinxAtStartPar
1ゲーム目で勝ったプレイヤーは2ゲーム目以降はもう1つのデッキを使用する

\item {} 
\sphinxAtStartPar
1ゲーム目で負けたプレイヤーは2ゲーム目で使用するデッキを選び、使用しないデッキを左上に裏向き横向きに置く

\item {} 
\sphinxAtStartPar
2ゲーム目を行う

\item {} 
\sphinxAtStartPar
1ゲーム目に勝ったプレイヤーが勝った場合は、そのプレイヤーの対戦勝利とする

\item {} 
\sphinxAtStartPar
1ゲーム目に勝ったプレイヤーが負けた場合は、3ゲーム目を行う

\item {} 
\sphinxAtStartPar
3ゲーム目を行う前に一時的に2ゲーム目で使用した自分のデッキを見てよい(個人公開)

\item {} 
\sphinxAtStartPar
両プレイヤーはゲームに勝ったことのないデッキを使用して3ゲーム目を行う

\item {} 
\sphinxAtStartPar
使用しないデッキを左上に表向き横向きに置く

\item {} 
\sphinxAtStartPar
2ゲーム勝ったプレイヤーの対戦勝利とする

\end{itemize}

\end{description}


\section{オプション定義}
\label{\detokenize{match-regulations/match-regulations:id8}}
\sphinxAtStartPar
公式として次のオプションを定義しています。


\subsection{プリセット2}
\label{\detokenize{match-regulations/match-regulations:id9}}
\sphinxAtStartPar
各プレイヤーがゲーム開始時(切札配置後)にデッキからランダムにキャラクター2体を場に出すオプションルール

\sphinxAtStartPar
切札配置後のデッキをシャッフルし上から1枚を防壁、次の1枚を兵士として場に出す


\subsection{プリセット4}
\label{\detokenize{match-regulations/match-regulations:id10}}
\sphinxAtStartPar
各プレイヤーがゲーム開始時(切札配置後)にデッキからランダムにキャラクター4体を場に出すオプションルール

\sphinxAtStartPar
切札配置後のデッキをシャッフルし上から2枚を防壁、次の2枚を兵士として場に出す


\subsection{パック}
\label{\detokenize{match-regulations/match-regulations:id11}}
\sphinxAtStartPar
ゲーム開始前にデッキに入らなかったカードをパックとして扱い、利用するオプションルール

\sphinxAtStartPar
パックはゲーム開始前に裏向きにして場の外に置く。

\sphinxAtStartPar
「パック開封」アクションの追加
\begin{quote}

\sphinxAtStartPar
【パック開封】\textless{}速攻魔法\textgreater{} @クイック

\sphinxAtStartPar
※プレイヤーは1ゲームに1回しかこのアクションを起こすことができない。

\sphinxAtStartPar
(即時効果)
\begin{enumerate}
\sphinxsetlistlabels{\arabic}{enumi}{enumii}{}{.}%
\item {} 
\sphinxAtStartPar
パックの中から好きなカードを1枚選び対戦相手に見せ手札に加える。

\item {} 
\sphinxAtStartPar
パックを表向きにする。

\end{enumerate}

\sphinxAtStartPar
注)キーカードなし、コストなしのアクション
\end{quote}


\section{対戦レギュレーションの決め方}
\label{\detokenize{match-regulations/match-regulations:id12}}
\sphinxAtStartPar
対戦レギュレーションを決定する手順を記載します。

\sphinxAtStartPar
公式では対応していない組み合わせが存在するため、手順に従って対戦レギュレーションを決定してください。
\begin{enumerate}
\sphinxsetlistlabels{\arabic}{enumi}{enumii}{}{.}%
\item {} 
\sphinxAtStartPar
\sphinxstylestrong{フォーマットの決定}
\begin{quote}

\sphinxAtStartPar
「ライト」「スタンダード」などフォーマットを決めます。
\end{quote}

\item {} 
\sphinxAtStartPar
\sphinxstylestrong{フレームの決定}
\begin{quote}

\sphinxAtStartPar
手順1で選択したフォーマットをもとに次のフレーム対応一覧より、フレームを決めます。

\sphinxAtStartPar
「◯」と表記されている組み合わせが選択出来ます。(\hyperref[\detokenize{match-regulations/match-regulations:frame-format}]{Table \ref{\detokenize{match-regulations/match-regulations:frame-format}}})


\begin{savenotes}\sphinxattablestart
\sphinxthistablewithglobalstyle
\centering
\sphinxcapstartof{table}
\sphinxthecaptionisattop
\sphinxcaption{フレーム対応一覧}\label{\detokenize{match-regulations/match-regulations:id14}}\label{\detokenize{match-regulations/match-regulations:frame-format}}
\sphinxaftertopcaption
\begin{tabulary}{\linewidth}[t]{|T|T|T|T|T|T|}
\sphinxtoprule
\sphinxtableatstartofbodyhook
\sphinxAtStartPar
【フレーム】
&
\sphinxAtStartPar
ライト
&
\sphinxAtStartPar
スタンダード
&
\sphinxAtStartPar
プロ
&
\sphinxAtStartPar
マスター
&
\sphinxAtStartPar
エクストラ
\\
\sphinxhline
\sphinxAtStartPar
エントリー
&
\sphinxAtStartPar
◯
&
\sphinxAtStartPar
◯
&
\sphinxAtStartPar
◯
&
\sphinxAtStartPar
◯
&
\sphinxAtStartPar
x
\\
\sphinxhline
\sphinxAtStartPar
ランダム40
&
\sphinxAtStartPar
◯
&
\sphinxAtStartPar
◯
&
\sphinxAtStartPar
◯
&
\sphinxAtStartPar
◯
&
\sphinxAtStartPar
x
\\
\sphinxhline
\sphinxAtStartPar
ランダムハーフ
&
\sphinxAtStartPar
◯
&
\sphinxAtStartPar
◯
&
\sphinxAtStartPar
◯
&
\sphinxAtStartPar
◯
&
\sphinxAtStartPar
x
\\
\sphinxhline
\sphinxAtStartPar
構築ハーフ
&
\sphinxAtStartPar
◯
&
\sphinxAtStartPar
◯
&
\sphinxAtStartPar
◯
&
\sphinxAtStartPar
◯
&
\sphinxAtStartPar
x
\\
\sphinxhline
\sphinxAtStartPar
構築40
&
\sphinxAtStartPar
◯
&
\sphinxAtStartPar
◯
&
\sphinxAtStartPar
◯
&
\sphinxAtStartPar
◯
&
\sphinxAtStartPar
x
\\
\sphinxhline
\sphinxAtStartPar
レギュラー
&
\sphinxAtStartPar
◯
&
\sphinxAtStartPar
◯
&
\sphinxAtStartPar
◯
&
\sphinxAtStartPar
◯
&
\sphinxAtStartPar
◯
\\
\sphinxhline
\sphinxAtStartPar
フル
&
\sphinxAtStartPar
◯
&
\sphinxAtStartPar
◯
&
\sphinxAtStartPar
◯
&
\sphinxAtStartPar
◯
&
\sphinxAtStartPar
◯
\\
\sphinxbottomrule
\end{tabulary}
\sphinxtableafterendhook\par
\sphinxattableend\end{savenotes}
\end{quote}

\item {} 
\sphinxAtStartPar
\sphinxstylestrong{オプションの選択}
\begin{quote}

\sphinxAtStartPar
手順1,2で決定したフォーマット、フレームをもとに次のオプション対応一覧より、オプションを決めます。

\sphinxAtStartPar
フォーマット、フレームの両方で「◯」と表記されている組み合わせが選択出来ます。(\hyperref[\detokenize{match-regulations/match-regulations:options-depenson}]{Table \ref{\detokenize{match-regulations/match-regulations:options-depenson}}})

\sphinxAtStartPar
条件を満たせば複数のオプションを選択することも可能です。また、オプションは選択しなくても構いません。


\begin{savenotes}\sphinxattablestart
\sphinxthistablewithglobalstyle
\centering
\sphinxcapstartof{table}
\sphinxthecaptionisattop
\sphinxcaption{オプション対応一覧}\label{\detokenize{match-regulations/match-regulations:id15}}\label{\detokenize{match-regulations/match-regulations:options-depenson}}
\sphinxaftertopcaption
\begin{tabulary}{\linewidth}[t]{|T|T|T|T|}
\sphinxtoprule
\sphinxtableatstartofbodyhook&
\sphinxAtStartPar
プリセット2
&
\sphinxAtStartPar
プリセット4
&
\sphinxAtStartPar
パック
\\
\sphinxhline
\sphinxAtStartPar
【フォーマット】
&&&\\
\sphinxhline
\sphinxAtStartPar
ライト
&
\sphinxAtStartPar
x
&
\sphinxAtStartPar
x
&
\sphinxAtStartPar
◯
\\
\sphinxhline
\sphinxAtStartPar
スタンダード
&
\sphinxAtStartPar
◯
&
\sphinxAtStartPar
◯
&
\sphinxAtStartPar
◯
\\
\sphinxhline
\sphinxAtStartPar
プロ
&
\sphinxAtStartPar
◯
&
\sphinxAtStartPar
◯
&
\sphinxAtStartPar
◯
\\
\sphinxhline
\sphinxAtStartPar
マスター
&
\sphinxAtStartPar
◯
&
\sphinxAtStartPar
◯
&
\sphinxAtStartPar
◯
\\
\sphinxhline
\sphinxAtStartPar
エクストラ
&
\sphinxAtStartPar
◯
&
\sphinxAtStartPar
◯
&
\sphinxAtStartPar
x
\\
\sphinxhline
\sphinxAtStartPar
【フレーム】
&&&\\
\sphinxhline
\sphinxAtStartPar
エントリー
&
\sphinxAtStartPar
◯
&
\sphinxAtStartPar
x
&
\sphinxAtStartPar
x
\\
\sphinxhline
\sphinxAtStartPar
ランダム40
&
\sphinxAtStartPar
◯
&
\sphinxAtStartPar
◯
&
\sphinxAtStartPar
◯
\\
\sphinxhline
\sphinxAtStartPar
ランダムハーフ
&
\sphinxAtStartPar
◯
&
\sphinxAtStartPar
x
&
\sphinxAtStartPar
x
\\
\sphinxhline
\sphinxAtStartPar
構築ハーフ
&
\sphinxAtStartPar
◯
&
\sphinxAtStartPar
x
&
\sphinxAtStartPar
x
\\
\sphinxhline
\sphinxAtStartPar
構築40
&
\sphinxAtStartPar
◯
&
\sphinxAtStartPar
◯
&
\sphinxAtStartPar
◯
\\
\sphinxhline
\sphinxAtStartPar
レギュラー
&
\sphinxAtStartPar
◯
&
\sphinxAtStartPar
◯
&
\sphinxAtStartPar
x
\\
\sphinxhline
\sphinxAtStartPar
フル
&
\sphinxAtStartPar
◯
&
\sphinxAtStartPar
◯
&
\sphinxAtStartPar
x
\\
\sphinxbottomrule
\end{tabulary}
\sphinxtableafterendhook\par
\sphinxattableend\end{savenotes}
\end{quote}

\end{enumerate}

\begin{sphinxadmonition}{note}{注釈:}
\sphinxAtStartPar
対戦レギュレーション決定例
\begin{enumerate}
\sphinxsetlistlabels{\arabic}{enumi}{enumii}{}{.}%
\item {} 
\sphinxAtStartPar
フォーマットを「スタンダード」に決めたとします。

\item {} 
\sphinxAtStartPar
\hyperref[\detokenize{match-regulations/match-regulations:frame-format}]{Table \ref{\detokenize{match-regulations/match-regulations:frame-format}}} より「スタンダード」で選択できるフォーマットを選びます。今回は「ランダムハーフ」を選択します。

\item {} 
\sphinxAtStartPar
\hyperref[\detokenize{match-regulations/match-regulations:options-depenson}]{Table \ref{\detokenize{match-regulations/match-regulations:options-depenson}}} よりフォーマット「スタンダード」、フレーム「ランダムハーフ」として両方が◯となる列を確認します。すると「プリセット2」の列が該当しました。今回は「プリセット2」を選択します。

\item {} 
\sphinxAtStartPar
これまでの選択より対戦レギュレーションは「スタンダード+ランダムハーフ+プリセット2」となります。

\end{enumerate}
\end{sphinxadmonition}

\sphinxstepscope


\chapter{付録}
\label{\detokenize{appendix/appendix:appendix-rst}}\label{\detokenize{appendix/appendix:id1}}\label{\detokenize{appendix/appendix::doc}}

\section{PDF版ルール}
\label{\detokenize{appendix/appendix:pdf}}
\sphinxAtStartPar
\sphinxurl{https://blackpoker.github.io/BlackPoker/master/blackpoker.pdf}


\section{アクションリスト}
\label{\detokenize{appendix/appendix:id2}}

\subsection{ライト}
\label{\detokenize{appendix/appendix:actionlist-lite}}\label{\detokenize{appendix/appendix:id3}}\begin{description}
\sphinxlineitem{URL}
\sphinxAtStartPar
\sphinxurl{https://blackpoker.github.io/BlackPoker/master/actionlist/html/lite.html}

\sphinxlineitem{PDF}
\sphinxAtStartPar
\sphinxurl{https://blackpoker.github.io/BlackPoker/master/actionlist/pdf/blackpoker-lite.pdf}

\sphinxAtStartPar
\sphinxurl{https://blackpoker.github.io/BlackPoker/master/actionlist/pdf/blackpoker-lite-2up.pdf}

\end{description}


\subsection{スタンダード}
\label{\detokenize{appendix/appendix:actionlist-std}}\label{\detokenize{appendix/appendix:id4}}\begin{description}
\sphinxlineitem{URL}
\sphinxAtStartPar
\sphinxurl{https://blackpoker.github.io/BlackPoker/master/actionlist/html/std.html}

\sphinxlineitem{PDF}
\sphinxAtStartPar
\sphinxurl{https://blackpoker.github.io/BlackPoker/master/actionlist/pdf/blackpoker-std.pdf}

\sphinxAtStartPar
\sphinxurl{https://blackpoker.github.io/BlackPoker/master/actionlist/pdf/blackpoker-std-2up.pdf}

\end{description}


\subsection{プロ}
\label{\detokenize{appendix/appendix:actionlist-pro}}\label{\detokenize{appendix/appendix:id5}}\begin{description}
\sphinxlineitem{URL}
\sphinxAtStartPar
\sphinxurl{https://blackpoker.github.io/BlackPoker/master/actionlist/html/pro.html}

\sphinxlineitem{PDF}
\sphinxAtStartPar
\sphinxurl{https://blackpoker.github.io/BlackPoker/master/actionlist/pdf/blackpoker-pro.pdf}

\sphinxAtStartPar
\sphinxurl{https://blackpoker.github.io/BlackPoker/master/actionlist/pdf/blackpoker-pro-2up.pdf}

\end{description}


\subsection{マスター}
\label{\detokenize{appendix/appendix:actionlist-master}}\label{\detokenize{appendix/appendix:id6}}\begin{description}
\sphinxlineitem{URL}
\sphinxAtStartPar
\sphinxurl{https://blackpoker.github.io/BlackPoker/master/actionlist/html/mast.html}

\sphinxlineitem{PDF}
\sphinxAtStartPar
\sphinxurl{https://blackpoker.github.io/BlackPoker/master/actionlist/pdf/blackpoker-mast.pdf}

\sphinxAtStartPar
\sphinxurl{https://blackpoker.github.io/BlackPoker/master/actionlist/pdf/blackpoker-mast-2up.pdf}

\end{description}


\section{エクストラリスト}
\label{\detokenize{appendix/appendix:extralist}}\label{\detokenize{appendix/appendix:id7}}\begin{description}
\sphinxlineitem{URL}
\sphinxAtStartPar
\sphinxurl{https://blackpoker.github.io/BlackPoker/master/actionlist/html/ex.html}

\sphinxlineitem{PDF}
\sphinxAtStartPar
\sphinxurl{https://blackpoker.github.io/BlackPoker/master/actionlist/pdf/blackpoker-extra.pdf}

\sphinxAtStartPar
\sphinxurl{https://blackpoker.github.io/BlackPoker/master/actionlist/pdf/blackpoker-extra-2up.pdf}

\end{description}



\renewcommand{\indexname}{索引}
\printindex
\end{document}