%% Generated by Sphinx.
\def\sphinxdocclass{jsbook}
\documentclass[letterpaper,10pt,dvipdfmx]{sphinxmanual}
\ifdefined\pdfpxdimen
   \let\sphinxpxdimen\pdfpxdimen\else\newdimen\sphinxpxdimen
\fi \sphinxpxdimen=.75bp\relax

\PassOptionsToPackage{warn}{textcomp}


\usepackage{cmap}
\usepackage[T1]{fontenc}
\usepackage{amsmath,amssymb,amstext}



\usepackage{times}


\usepackage[,numfigreset=1,mathnumfig]{sphinx}

\fvset{fontsize=\small}
\usepackage[dvipdfm]{geometry}


% Include hyperref last.
\usepackage{hyperref}
% Fix anchor placement for figures with captions.
\usepackage{hypcap}% it must be loaded after hyperref.
% Set up styles of URL: it should be placed after hyperref.
\urlstyle{same}
\renewcommand{\contentsname}{Contents:}

\usepackage{sphinxmessages}
\setcounter{tocdepth}{2}



\title{BlackPoker}
\date{2021年06月27日}
\release{}
\author{BlackPoker}
\newcommand{\sphinxlogo}{\vbox{}}
\renewcommand{\releasename}{}
\makeindex
\begin{document}

\pagestyle{empty}
\sphinxmaketitle
\pagestyle{plain}
\sphinxtableofcontents
\pagestyle{normal}
\phantomsection\label{\detokenize{index::doc}}



\chapter{はじめに}
\label{\detokenize{init/init:id1}}\label{\detokenize{init/init::doc}}
この文章はトランプゲーム「BlackPoker」の全てのルールをまとめた文章です。

詳細なルールが記載されており、初心者の方は文章の量に圧倒されます。
ゲームをプレイする際に全てを熟読する必要はありませんが、
ルールについて深く知りたい、または新しいルールに触れたい方はぜひ熟読してください。


\section{ルールの構成}
\label{\detokenize{init/init:id2}}
ルールの構成は次のようになっています。

\noindent\sphinxincludegraphics{{plantuml-d144433912d7de61d8f7772389281396a917cd98}.pdf}


\chapter{コアルール}
\label{\detokenize{core/core:id1}}\label{\detokenize{core/core::doc}}
コアルールは割込み処理が可能なターン制ゲームの開始から勝敗が決まるまでを定義します。


\section{ターン}
\label{\detokenize{core/core:id2}}
このルールを説明する上でターンとは持つことができるものとします。
ターンを持っているプレイヤーは先に行動することができます。
ターンを持っているプレイヤーをターンプレイヤーといいます。


\section{アクション}
\label{\detokenize{core/core:id3}}
アクションとは、プレイヤーの行動を示します。
ターン制のゲームでは、プレイヤーは様々な行動を行います。
チェスであればコマを進めたり、ババ抜きであれば隣の人からカードを引くなどがあります。
それらをアクションと定義します。


\section{チャンス}
\label{\detokenize{core/core:id4}}
アクションを起こすことができる機会をチャンスといいます。
チャンスを持っている間は何度でもアクションを起こすことができます。


\section{ステージ}
\label{\detokenize{core/core:id5}}
アクションの解決順を整理するために使う領域です。
後入れ先出し方式で最後に積まれたアクションから順に解決されていきます。


\section{アクションの定義項目}
\label{\detokenize{core/core:id6}}
アクション、チャンス、ステージについて簡単に説明しました。
これらの概念を用いて、アクションに定義する項目を説明します。

アクションは次の項目を定義する必要があります。
他の項目は具体的にアクションを定義する際に、ゲームに合わせて追加して下さい。
\begin{itemize}
\item {} 
効果(通常効果・即時効果)

\item {} 
タイミング

\end{itemize}


\subsection{効果}
\label{\detokenize{core/core:id7}}
効果とはその効果を発揮した際に、プレイヤーが行う行動です。
効果の中には、通常効果と即時効果があります。
違いについては、図(\hyperref[\detokenize{core/core:coreflow-2}]{Fig.\@ \ref{\detokenize{core/core:coreflow-2}}})を説明する際に分岐条件として登場します。


\subsection{タイミング}
\label{\detokenize{core/core:timing}}\label{\detokenize{core/core:id8}}
タイミングとは、アクションを起こすことができる時を示します。
タイミングには「メイン」と「クイック」の2種類あります。
\begin{description}
\item[{メイン}] \leavevmode
ターンプレイヤーかつステージが空の時に起こせるアクションです。

条件をまとめると次のようになります。
\begin{itemize}
\item {} 
チャンスを持っている

\item {} 
自分のターン

\item {} 
ステージが空

\end{itemize}

\item[{クイック}] \leavevmode
いつでも起こせるため、アクションをステージに積み重ねることができます。

条件をまとめると次のようになります。
\begin{itemize}
\item {} 
チャンスを持っている

\end{itemize}

\end{description}


\subsubsection{エンドアクションの定義}
\label{\detokenize{core/core:id9}}
定義するアクションの中で最低1つは
ターンを別のプレイヤーにわたす効果を定義してください。
そうしないと、ターンが別のプレイヤーに渡らす、ゲームが進行しなくなります。


\subsubsection{アクションのコントローラー}
\label{\detokenize{core/core:id10}}
アクションを起こしたプレイヤーをそのアクションのコントローラーと呼びます。
効果はこのコントローラー視点で解釈されることになります。


\section{コンポーネント}
\label{\detokenize{core/core:component}}\label{\detokenize{core/core:id11}}
ゲームにてプレイヤーが保有する駒やカードのことをコンポーネントと定義します。
コンポーネントは次の項目を持っています。
\begin{description}
\item[{オーナー}] \leavevmode
コンポーネントの所有者を示します。大体のトランプゲームではトランプを1セットしか用いないため無視されますが、TCGのデッキなど個人所有のものを用いるゲームでは必要な項目となります。

\item[{コントローラー}] \leavevmode
現在そのコンポーネントを操作しているプレイヤーを示します。オーナーとコントローラーは基本同じプレイヤーが設定されますが、コントロールを奪うアクションがある場合、オーナーとコントローラーは異なります。

\end{description}

\begin{sphinxadmonition}{note}{注釈:}
コンポーネントとアクションのコントローラー

コントローラーは制御している人という意味になるため、コンポーネントとアクションのコントローラー制御する対象が異なることになります。
コンポーネントとアクションの属性を次の図に示します。アクションにはオーナーがいない点が異なります。

\begin{figure}[H]
\centering
\capstart

\noindent\sphinxincludegraphics{{plantuml-97eb2fb6ceaeae89223169fc6f937aee84be4fe1}.pdf}
\caption{コンポーネントとアクションの属性}\label{\detokenize{core/core:id36}}\end{figure}
\end{sphinxadmonition}


\section{能力}
\label{\detokenize{core/core:id12}}
能力とはアクションの効果とは異なる概念で、アクションを起こすことができたり、
アクションを誘発したりすることができます。

能力を持つことができるのは、プレイヤーの他に駒やカードなどのゲームに登場するコンポーネントも持つことができます。
(\hyperref[\detokenize{core/core:ability-image}]{Fig.\@ \ref{\detokenize{core/core:ability-image}}})

\begin{figure}[htbp]
\centering
\capstart

\noindent\sphinxincludegraphics{{plantuml-976f45327f77d4f4ee1da59bff1cd262ba402df4}.pdf}
\caption{能力のイメージ}\label{\detokenize{core/core:id37}}\label{\detokenize{core/core:ability-image}}\end{figure}

能力には、次の種類があります。
\begin{description}
\item[{常在型能力}] \leavevmode
能力が有効である場合、継続的に発揮される能力

\item[{誘発型能力}] \leavevmode
能力が有効である間に何かの契機でアクションを起こす能力

\end{description}

概ねのゲームでは、
ターン終了や駒をすすめるなどのアクションが定義されています。
そして、そのアクションを起こせる能力(常在型能力)を
プレイヤーは保持しています。


\section{コアフロー}
\label{\detokenize{core/core:coreflowsec}}\label{\detokenize{core/core:id13}}
この図にゲームの開始から勝敗が決まるまでの流れが集約されいます。(\hyperref[\detokenize{core/core:coreflow-2}]{Fig.\@ \ref{\detokenize{core/core:coreflow-2}}})

\begin{figure}[htbp]
\centering
\capstart

\noindent\sphinxincludegraphics{{plantuml-1f7308387c9dc0dd58a0199510bada736a26d4e5}.pdf}
\caption{コアフロー}\label{\detokenize{core/core:id38}}\label{\detokenize{core/core:coreflow-2}}\end{figure}


\subsection{{[}1{]}ゲーム開始}
\label{\detokenize{core/core:id14}}
先攻を決め、ゲームを始める準備を行います。


\subsection{{[}2{]}ターンプレイヤーにチャンスを移動}
\label{\detokenize{core/core:id15}}
ターンを持っているプレイヤーにチャンスを移動します。


\subsection{{[}3{]}ステージが空か?}
\label{\detokenize{core/core:id16}}
ステージにアクションが存在していないか判定します。


\subsection{{[}4{]}パス名簿リセット}
\label{\detokenize{core/core:id17}}
パスしたプレイヤーを記録するパス名簿をリセットします。


\subsection{{[}5{]}アクション起こす}
\label{\detokenize{core/core:id18}}
アクションを起こしこれからプレイヤーが行うことを宣言します。
ゲームによってアクションの起こし方は異なります。BlackPokerではアクション名を言い、コストの支払や対象を指定しアクションを起こします。
一方ババ抜きでは、隣のプレイヤーからカードを引く際に宣言せず暗黙にアクションが起きている場合もあります。


\subsection{{[}6{]}即時効果か?}
\label{\detokenize{core/core:id19}}
起こしたアクションが即時効果か通常効果か判定します。


\subsection{{[}7{]}効果解決}
\label{\detokenize{core/core:id20}}
アクションの効果に定義されている内容を実行します。


\subsection{{[}8{]}勝敗判定}
\label{\detokenize{core/core:winlose}}\label{\detokenize{core/core:id21}}
ゲームの勝敗を判定します。決着した場合ゲームが終了します。判定の方法はゲームにより異なります。


\subsection{{[}9{]}ステージに追加}
\label{\detokenize{core/core:id22}}
ステージというアクションを貯めておける領域に追加します。


\subsection{{[}10{]}誘発チェック}
\label{\detokenize{core/core:id23}}
ここに至るまでに誘発したアクションがないかチェックします。誘発した場合、効果を解決するかスタックに追加します。詳しいフローは \hyperref[\detokenize{core/core:trigger-check}]{\ref{\detokenize{core/core:trigger-check}} \nameref{\detokenize{core/core:trigger-check}}} を参照してください。


\subsection{{[}11{]}アクションを起こすか?}
\label{\detokenize{core/core:id24}}
チャンスを持っているプレイヤーはアクションを起こすかを判断します。


\subsection{{[}12{]}パス名簿に登録}
\label{\detokenize{core/core:id25}}
パスしたプレイヤーを記録するパス名簿に登録します。同じプレイヤー名は2回登録されません。


\subsection{{[}13{]}パス名簿の件数=プレイヤー数か?}
\label{\detokenize{core/core:id26}}
パス名簿の件数がゲームに参加しているプレイヤーの数と一致しているか判定します。


\subsection{{[}14{]}ステージから取出し}
\label{\detokenize{core/core:id27}}
最後にステージに追加されたアクションをステージから取出します。


\subsection{{[}15{]}チャンス移動}
\label{\detokenize{core/core:id28}}
チャンスを持っているプレイヤーからチャンスを持っていないプレイヤーにチャンスを移動します。
チャンスを移動するルールはゲームによって異なります。


\subsection{誘発チェック}
\label{\detokenize{core/core:trigger-check}}\label{\detokenize{core/core:id29}}
能力の中でも誘発型能力は、なにかをきっかけにしてアクションが起きる条件が定義されています。
誘発する条件は「〜の場合」、「〜時」などで記載されており、誘発するアクションは「〜を誘発する」と記載されています。

誘発チェックでは、誘発したアクションの効果を解決もしくは、ステージに追加します。
誘発したアクションのコントローラーは起因となった誘発型能力を持ったコンポーネントのコントローラーになります。
誘発チェックは次の図のように行います。(\hyperref[\detokenize{core/core:trigger-flow}]{Fig.\@ \ref{\detokenize{core/core:trigger-flow}}})

\begin{figure}[htbp]
\centering
\capstart

\noindent\sphinxincludegraphics{{plantuml-a91828c766d562c760f38e1517ad1bbf09672f77}.pdf}
\caption{誘発チェック}\label{\detokenize{core/core:id39}}\label{\detokenize{core/core:trigger-flow}}\end{figure}


\subsubsection{{[}10\sphinxhyphen{}1{]}誘発チェック}
\label{\detokenize{core/core:id30}}
全てのプレイヤー、コンポーネントが持っている誘発型能力を確認し、
アクションが誘発していないか判定します。


\subsubsection{{[}10\sphinxhyphen{}2{]}即時誘発有無判定}
\label{\detokenize{core/core:id31}}
即時効果を持つアクション誘発していないか判定します。


\subsubsection{{[}10\sphinxhyphen{}3{]}効果解決\&勝敗判定}
\label{\detokenize{core/core:id32}}
誘発した即時効果をプレイヤー毎に任意の順番で解決します。
解決するプレイヤーの順序は、
ターンプレイヤーがコントローラーとなっているアクションを全て解決してから、
ターンプレイヤー以外がコントローラーとなっているアクションを解決します。
この解決順序は、ゲームによって変更できます。

効果を解決する毎に勝敗判定を行ってください。


\subsubsection{{[}10\sphinxhyphen{}4{]}誘発有無判定}
\label{\detokenize{core/core:id33}}
通常効果を持つアクションが誘発していないか判定します。


\subsubsection{{[}10\sphinxhyphen{}5{]}ステージに追加}
\label{\detokenize{core/core:id34}}
誘発したアクションをプレイヤー毎に任意の順番でステージに追加します。
ステージに追加するプレイヤーの順序は、
ターンプレイヤーがコントローラーとなっているアクションを全てステージに追加してから、
ターンプレイヤー以外がコントローラーとなっているアクションをステージに追加します。
この解決順序は、ゲームによって変更できます。


\section{まとめ}
\label{\detokenize{core/core:id35}}
コアルールについて説明しました。
すでにあるターン制のゲームからアクションを洗い出し、能力を整理することで割込処理を可能としゲームの新しい遊び方が見つけられます。
また、新しく作成するゲームに関してもコアルールを意識して作成することで、ルール追加がしやすいゲームが考えやすいと思います。


\chapter{共通ルール}
\label{\detokenize{common/common:id1}}\label{\detokenize{common/common::doc}}
この章では、カードの配置などコアルールで定義されていない内容を定義します。


\section{プレイ人数}
\label{\detokenize{common/common:id2}}
フォーマット、対戦レギュレーションに定義されていない場合、2人です。
プレイする際に確認してください。


\section{用意するもの}
\label{\detokenize{common/common:id3}}\begin{itemize}
\item {} 
1人1セットのトランプが必要です。

\item {} 
覚えていない場合、フォーマットに応じてアクションリスト、エクストラリストがあると便利です。

\end{itemize}


\section{使用できるトランプ}
\label{\detokenize{common/common:id4}}
BlackPokerでは次の条件を満たしたトランプを使うことができます。
一般的なトランプなら満たす条件となっています。
\begin{quote}
\begin{itemize}
\item {} 
スートと数字が分かる

\item {} 
スートの{\normalsize $\spadesuit$} {\normalsize $\heartsuit$} {\normalsize $\diamondsuit$} {\normalsize $\clubsuit$} が判断できる

\item {} 
数字のA\sphinxhyphen{}K(1\sphinxhyphen{}13)が判断できる

\item {} 
スートと数字の組合せが重複していない

\item {} 
裏から表がわからない

\item {} 
縦向き横向きが判断できる

\item {} 
Jokerは2枚まで入れられる

\item {} 
54枚無くてもよい

\end{itemize}

対戦レギュレーションにより使用できるトランプの枚数など異なる場合があるため、
対戦する際に、対戦レギュレーション、フォーマットを確認してください。
\end{quote}


\section{トランプの数字}
\label{\detokenize{common/common:id5}}
ゲーム全体を通してトランプの数字は次のような数値として扱います。(\hyperref[\detokenize{common/common:cardrank}]{Table \ref{\detokenize{common/common:cardrank}}})


\begin{savenotes}\sphinxattablestart
\centering
\sphinxcapstartof{table}
\sphinxthecaptionisattop
\sphinxcaption{トランプの数字}\label{\detokenize{common/common:id49}}\label{\detokenize{common/common:cardrank}}
\sphinxaftertopcaption
\begin{tabulary}{\linewidth}[t]{|T|T|}
\hline
\sphinxstyletheadfamily 
カード
&\sphinxstyletheadfamily 
数字
\\
\hline
A
&
1
\\
\hline
2〜10
&
表記どおり
\\
\hline
J
&
11
\\
\hline
Q
&
12
\\
\hline
K
&
13
\\
\hline
Joker
&
0
\\
\hline
\end{tabulary}
\par
\sphinxattableend\end{savenotes}


\section{カードの配置}
\label{\detokenize{common/common:id6}}
カードの配置には次のような場所があります。(\hyperref[\detokenize{common/common:field-ex}]{Fig.\@ \ref{\detokenize{common/common:field-ex}}})

\begin{figure}[htbp]
\centering
\capstart

\noindent\sphinxincludegraphics{{field-ex}.pdf}
\caption{プレイ中のカードの配置}\label{\detokenize{common/common:id50}}\label{\detokenize{common/common:field-ex}}\end{figure}
\begin{description}
\item[{デッキ}] \leavevmode
山札。ゲームを始める時に自分のトランプを裏向きに置く場所です。
ダメージを受けるとデッキの一番上から墓地にカードを移します。

\item[{墓地}] \leavevmode
捨て札置き場。ダメージを受けた時などに表向きでカードを重ねて置きます。

\item[{場}] \leavevmode
兵士や防壁などのキャラクターを置きます。

\item[{手札}] \leavevmode
デッキから引いたカードを持っておく場所です。相手から見えないようにしましょう。

\item[{切札}] \leavevmode
能力が割り当てられたカードを置きます。エクストラフォーマットのみで使用します。
エクストラのルールについては、 \hyperref[\detokenize{common/common:extra}]{\ref{\detokenize{common/common:extra}} \nameref{\detokenize{common/common:extra}}} で説明します。

\end{description}


\section{勝利条件}
\label{\detokenize{common/common:id7}}
プレイヤーは順に対戦相手に対し攻撃を行い、ダメージを与え先に相手のデッキを0枚にした方が勝ちです。ダメージは1点につき1枚デッキが減ります。


\section{ダメージ}
\label{\detokenize{common/common:id8}}
プレイヤーがダメージを受けた場合、デッキの一番上から受けた点数分墓地にカードを表向きで移動します。移動する際は、カードの表を対戦相手に見せる必要はありません。


\section{キャラクター}
\label{\detokenize{common/common:id9}}
キャラクターとは、場に存在する兵士や防壁のことを指します。
コアルールのコンポーネントにあたります。

キャラクターは1枚のカードで1体を表すこともあれば、
複数枚で1体を表すこともあります。(\hyperref[\detokenize{common/common:character}]{Fig.\@ \ref{\detokenize{common/common:character}}})

\begin{figure}[htbp]
\centering
\capstart

\noindent\sphinxincludegraphics{{character}.pdf}
\caption{キャラクターの例}\label{\detokenize{common/common:id51}}\label{\detokenize{common/common:character}}\end{figure}


\subsection{キャラクターのもつ項目}
\label{\detokenize{common/common:id10}}
キャラクターのもつ項目について説明します。
凡例のキャラクター「一般兵」を見てみましょう。(\hyperref[\detokenize{common/common:character-sample}]{Fig.\@ \ref{\detokenize{common/common:character-sample}}})

\begin{figure}[htbp]
\centering
\capstart

\noindent\sphinxincludegraphics{{character-sample}.pdf}
\caption{一般兵}\label{\detokenize{common/common:id52}}\label{\detokenize{common/common:character-sample}}\end{figure}
\begin{description}
\item[{キャラクター名}] \leavevmode
キャラクターの名称を示します。

\item[{タイプ}] \leavevmode
キャラクターのタイプを示します。タイプは兵士と防壁の2種類が存在します。

\item[{キーカード}] \leavevmode
キャラクターを示すカードが記載されています。複数のカードで1体のキャラクターを示す場合もあります。

\item[{能力}] \leavevmode
キャラクターが持っている能力を記載しています。

\end{description}


\subsection{キャラクターの数字}
\label{\detokenize{common/common:id11}}
トランプの数字は、キャラクターの強さを示します。
基本はカードに記載された数字を示しますが、魔法などのアクションを使うことで
加算したり減算されたりします。


\subsection{キャラクターの注意点}
\label{\detokenize{common/common:id12}}

\subsubsection{複数枚で1体となるキャラクターが防壁になったら?}
\label{\detokenize{common/common:id13}}
アクションの効果で兵士を防壁にすることがあります。
防壁は1枚で1体のキャラクターであるため、
複数枚からなるキャラクターが防壁となった場合、
複数体の防壁となります。

なお、複数枚からなるキャラクターが
墓地や手札に移った場合、
1体のキャラクターとして
扱うため複数枚合わせて移します。
チャージ状態、ドライブ状態となった場合も同様に1体のキャラクター
として扱います。


\subsection{チャージとドライブ}
\label{\detokenize{common/common:id14}}
キャラクターには、チャージ状態とドライブ状態が存在します。
チャージ状態は未使用状態を示し、ドライブ状態は使用済み状態を示しています。
また、キャラクターを横向きにすることを「ドライブ」、縦向きにすることを「チャージ」と言います。(\hyperref[\detokenize{common/common:chargedrive}]{Fig.\@ \ref{\detokenize{common/common:chargedrive}}})

\begin{figure}[htbp]
\centering
\capstart

\noindent\sphinxincludegraphics{{charge&drive}.pdf}
\caption{チャージとドライブ}\label{\detokenize{common/common:id53}}\label{\detokenize{common/common:chargedrive}}\end{figure}


\section{ゲームの始め方}
\label{\detokenize{common/common:id15}}\begin{quote}

次の手順でゲームを始めます。
\begin{enumerate}
\sphinxsetlistlabels{\arabic}{enumi}{enumii}{}{.}%
\item {} 
デッキをよく切る。

\item {} 
デッキより7枚引き手札にする。

\item {} 
両者デッキの一番上を表にする。

\item {} 
大きい数字のプレイヤーが先攻。数字については、 \hyperref[\detokenize{common/common:cardrank}]{Table \ref{\detokenize{common/common:cardrank}}} 参照。

\item {} 
数字が同じ場合、さらにデッキの一番上を表にし同様のルールで比べる。

\item {} 
表にしたカードを墓地へ移す。

\item {} 
先攻プレイヤーはデッキより1枚引き手札に加える。

\item {} 
先攻プレイヤーがターンとチャンスをもちゲームを開始する。

\end{enumerate}
\end{quote}

ゲーム開始後はコアフローに準じアクションを起こしてゲームを進行します。

ゲーム内で起こせるアクションは対戦レギュレーション、フォーマットより異なります。
対戦前に確認してください。


\section{アクション}
\label{\detokenize{common/common:id16}}

\subsection{アクションが持つ項目}
\label{\detokenize{common/common:id17}}
アクションが持つ項目について説明します。
凡例の「サンプル」アクションを見てみましょう。(\hyperref[\detokenize{common/common:action-sample}]{Fig.\@ \ref{\detokenize{common/common:action-sample}}})

\begin{figure}[htbp]
\centering
\capstart

\noindent\sphinxincludegraphics{{action-sample}.pdf}
\caption{サンプルアクション}\label{\detokenize{common/common:id54}}\label{\detokenize{common/common:action-sample}}\end{figure}
\begin{description}
\item[{アクション名}] \leavevmode
アクションの名称を示します。

\item[{キーカード}] \leavevmode
アクションの核となるカードを示します。
キーカードは★を使って表記します。
凡例の場合、手札からコストとは別に{\normalsize $\heartsuit$} A〜10に該当するカードを1枚
キーカードとして使用します。

\item[{特記事項}] \leavevmode
特記事項は※を使って表記し、その他の項目では書き表せない条件を示します。

\item[{対象}] \leavevmode
効果を発揮する対象を示します。

\item[{即時効果/通常効果}] \leavevmode
発揮する効果の内容を示します。

\item[{コスト}] \leavevmode
アクションを起こすのに必要な対価です。
コストは$を使って表記し、コストの支払いはアクションを起こすプレイヤーが行います。コストの種類は \hyperref[\detokenize{common/common:cost}]{\ref{\detokenize{common/common:cost}} \nameref{\detokenize{common/common:cost}}} で説明します。

\item[{タイミング}] \leavevmode
アクションを起こせる時を示します。
タイミングはコアルール \hyperref[\detokenize{core/core:timing}]{\ref{\detokenize{core/core:timing}} \nameref{\detokenize{core/core:timing}}} を参照してください。

\item[{タイプ}] \leavevmode
アクションの種類を表します。アクション名の後に括弧書きで記載します。

\end{description}


\subsubsection{記載されていないアクションの項目}
\label{\detokenize{common/common:id18}}
アクションによっては記載されていない項目もあります。
記載されていない項目は無視して構いません。
たとえばコスト項目がなければコストを支払う必要はありません。


\subsection{コストの種類}
\label{\detokenize{common/common:cost}}\label{\detokenize{common/common:id19}}
アクションによって支払うコストが異なります。
コストには次の種類があり、それぞれ支払い方が異なります。(\hyperref[\detokenize{common/common:table-cost}]{Table \ref{\detokenize{common/common:table-cost}}})


\begin{savenotes}\sphinxattablestart
\centering
\sphinxcapstartof{table}
\sphinxthecaptionisattop
\sphinxcaption{コストの種類}\label{\detokenize{common/common:id55}}\label{\detokenize{common/common:table-cost}}
\sphinxaftertopcaption
\begin{tabulary}{\linewidth}[t]{|T|T|}
\hline
\sphinxstyletheadfamily 
表記(名称)
&\sphinxstyletheadfamily 
対価
\\
\hline
B (Bulwark)
&
防壁をドライブする
\\
\hline
L (Life)
&
1点ダメージを受ける
\\
\hline
D (Discard)
&
手札を1枚捨てる
\\
\hline
S (Sacrifice)
&
キャラクター1体を墓地に移す
\\
\hline
\end{tabulary}
\par
\sphinxattableend\end{savenotes}

たとえばコストが \sphinxstylestrong{「\$BL」} の場合、自分の場にいるチャージ状態の防壁を1体ドライブし、1点ダメージを受けることでコストが支払われたことになります。


\subsection{アクションの起こし方}
\label{\detokenize{common/common:id20}}
次の手順でアクションを起こします。
\begin{enumerate}
\sphinxsetlistlabels{\arabic}{enumi}{enumii}{}{.}%
\item {} 
起こすアクションを対戦相手に伝える。

\item {} 
アクションに応じたコストを支払う。

\item {} 
必要なら手札からキーカードを出す。

\item {} 
対象の指定が必要な場合、対象を指定する。

\end{enumerate}

「サンプル」アクションを起こす例を見てみましょう。(\hyperref[\detokenize{common/common:action-sample2}]{Fig.\@ \ref{\detokenize{common/common:action-sample2}}})

\begin{figure}[htbp]
\centering
\capstart

\noindent\sphinxincludegraphics{{action-sample2}.pdf}
\caption{アクションを起こす例}\label{\detokenize{common/common:id56}}\label{\detokenize{common/common:action-sample2}}\end{figure}


\subsubsection{アクションを起こすときの注意点}
\label{\detokenize{common/common:id21}}

\paragraph{対象を指定しないでアクションを起こせるか?}
\label{\detokenize{common/common:id22}}
「サンプル」アクションのように対象を指定するアクションがあります。
「対象」項目がある場合、記載された条件を満たした対象を指定できなければ、
そのアクションを起こすことはできません。


\paragraph{アクションを対象とするアクションは自身を対象にできるか?}
\label{\detokenize{common/common:id23}}
アクションは、自分自身を対象とすることはできません。
そのため、「カウンター」アクションのようにアクションを対象とするアクションは
自身を対象とすることはできません。


\subsection{アクションの効果解決}
\label{\detokenize{common/common:id24}}
\hyperref[\detokenize{core/core:coreflowsec}]{\ref{\detokenize{core/core:coreflowsec}} \nameref{\detokenize{core/core:coreflowsec}}} に準じ起こしたアクションの効果を解決します。
解決する際の次のことを確認します。


\subsubsection{対象条件を満たしているか}
\label{\detokenize{common/common:id25}}
対象を指定するアクションが効果を発揮しようとした時に
対象が存在していない場合、効果を発揮する対象を失うため効果が発揮されず
アクションが解決されます。

たとえば兵士に対して「アップ」アクションを起こし、対応して「ダウン」
アクションを起こされました。
「ダウン」の方が先に解決されるため、「アップ」を解決する時には
兵士が墓地に移っていたとします。その場合、「アップ」アクションは効果を発揮せず解決されます。


\subsubsection{効果の中に実行不可能な部分があるか}
\label{\detokenize{common/common:id26}}
効果の中に実行不可能な部分がある場合、可能な部分のみ実行します。

たとえば、デッキの枚数が残1枚の時に5点のダメージを受けたとします。
デッキは1枚しかないので5点ダメージを受けることはできませんが、
1点までなら受けることが可能なため、
この場合1点のダメージを受けることになります。


\subsection{アクションの効果解決後}
\label{\detokenize{common/common:id27}}
アクションの効果を解決した後、次のことを行います。


\subsubsection{キーカードを墓地に移す}
\label{\detokenize{common/common:keycard-gy}}\label{\detokenize{common/common:id28}}
1つのアクションが解決された後そのアクションをステージから取り除き、キーカードを墓地に移します。
ただし効果によってキーカードを場に出した場合や手札に戻した場合、
そのカードを移す先が明確になっているため、墓地には移しません。


\subsection{勝敗判定}
\label{\detokenize{common/common:id29}}
\hyperref[\detokenize{core/core:winlose}]{\ref{\detokenize{core/core:winlose}} \nameref{\detokenize{core/core:winlose}}} で確認する内容は次になります。

デッキを確認し0枚の場合そのプレイヤーは敗北となります。両プレイヤーのデッキが0枚の場合、引き分けとなります。


\subsection{その他補足事項}
\label{\detokenize{common/common:id30}}

\subsubsection{防壁の置き方}
\label{\detokenize{common/common:id31}}
防壁を場に出すときは次のルールにしたがって場に出して下さい。(\hyperref[\detokenize{common/common:set-bulwork}]{Fig.\@ \ref{\detokenize{common/common:set-bulwork}}})
\begin{itemize}
\item {} 
防壁を置く時はデッキ側に詰めて置いて下さい。

\item {} 
防壁の左右の入れ替えは行わないでください。

\end{itemize}

\begin{figure}[htbp]
\centering
\capstart

\noindent\sphinxincludegraphics{{set-bulwork}.pdf}
\caption{防壁の置き方}\label{\detokenize{common/common:id57}}\label{\detokenize{common/common:set-bulwork}}\end{figure}


\subsubsection{1ターンに1回制限}
\label{\detokenize{common/common:id32}}
特記事項に「プレイヤーは1ターンに1回しかこのアクションを起こすことができない。」と記載されているアクションは、
ターンを持っているプレイヤーが変わるまでの間に1回しか起こす
ことができません。

ターンを持っているプレイヤーが変わればまた起こすことができます。


\subsubsection{直接起こせないアクション}
\label{\detokenize{common/common:id33}}
特記事項に「プレイヤーはこのアクションを直接起こすことが出来ない。」
と記載されているアクションは、
プレイヤーがチャンスを持っていても
アクションを起こすことができません。
また、この特記事項が記載されたアクションが何らかの起因で起きても、プレイヤーが起こした訳ではないためパスは自動的に発生せず、チャンスは移りません。


\section{エクストラ}
\label{\detokenize{common/common:extra}}\label{\detokenize{common/common:id34}}
エクストラではアクションに加え切札の能力を使うことができます。
使用できるアクション、切札は対戦レギュレーションを確認してください。


\subsection{切札}
\label{\detokenize{common/common:id35}}
切札とは、切札領域に置かれたカードを示します。
具体的な切札の置き場所については、 \hyperref[\detokenize{common/common:field-ex}]{Fig.\@ \ref{\detokenize{common/common:field-ex}}} を参照して下さい。
切札には各々能力が割り当てられており、表にするとその能力を発揮します。
切札を操作するアクションは、「エクストラリスト」を参照して下さい。


\subsection{バージョン}
\label{\detokenize{common/common:id36}}
エクストラには、バージョンが存在します。
対戦を開始する前に対戦相手とバージョンの確認をしましょう。


\subsection{版数との関係}
\label{\detokenize{common/common:id37}}
版数毎に使える切札の種類が異なります。
たとえば、第一版、第二版ではエクストラで遊ぶことはできません。
第三版以降は、次版が出るまでの間に公開された切札であれば
使用することができます。(\hyperref[\detokenize{common/common:ver-ex}]{Table \ref{\detokenize{common/common:ver-ex}}})


\begin{savenotes}\sphinxattablestart
\centering
\sphinxcapstartof{table}
\sphinxthecaptionisattop
\sphinxcaption{版数とエクストラのバージョン}\label{\detokenize{common/common:id58}}\label{\detokenize{common/common:ver-ex}}
\sphinxaftertopcaption
\begin{tabulary}{\linewidth}[t]{|T|T|}
\hline
\sphinxstyletheadfamily 
版数
&\sphinxstyletheadfamily 
エクストラのバージョン
\\
\hline
第一版
&
−
\\
\hline
第二版
&
−
\\
\hline
第三版
&
ex3.4.0 〜 ex3.10.0
\\
\hline
第四版
&
ex4.14.0 〜 ex4.22.0
\\
\hline
第五版
&
ex5.22.0 〜
\\
\hline
\end{tabulary}
\par
\sphinxattableend\end{savenotes}


\subsection{ゲームのはじめ方}
\label{\detokenize{common/common:extra-start}}\label{\detokenize{common/common:id38}}
エクストラでは、切札を置いてからゲームを始めます。
切札を置くルールは次のようになっています。(\hyperref[\detokenize{common/common:trump}]{Fig.\@ \ref{\detokenize{common/common:trump}}})
\begin{itemize}
\item {} 
対戦前に裏向きで2枚まで切札を置くことができる。

\item {} 
切札はデッキと角度を変えて交わるようにデッキの下に置く。

\item {} 
切札を表にするときはスートと数字が見えるようにし、対応する能力の名称を言う。

\item {} 
デッキが0枚になった場合、切札が残っていても敗北する。

\item {} 
能力が割り当てられていないカードも切札とすることができるが、表になっても能力は発揮されない。

\end{itemize}

\begin{figure}[htbp]
\centering
\capstart

\noindent\sphinxincludegraphics{{trump}.pdf}
\caption{切札の置き方}\label{\detokenize{common/common:id59}}\label{\detokenize{common/common:trump}}\end{figure}

これ以降は、通常のゲームの始め方と同様です。


\subsection{切札の能力}
\label{\detokenize{common/common:id39}}
エクストラでは切札を使って能力を得ることができます。
切札1枚1枚に異なった能力が割り当てられており、
表にすることで能力を発揮します。
割り当てられている能力については、「エクストラリスト」を参照して下さい。


\subsubsection{能力を発揮する}
\label{\detokenize{common/common:id40}}
切札に割り当てられた能力は
「オープン」アクションを起こし表にすることで発揮します。(\hyperref[\detokenize{common/common:trump-open}]{Fig.\@ \ref{\detokenize{common/common:trump-open}}})
「オープン」アクションの詳細は、
「エクストラリスト」を参照して下さい。
切札が表でいる限り、
その切札の能力は持続的に発揮されます。
また切札を表にする時は、
対戦相手に有効となった能力が分かるように、
能力の名称を言いスートと数字が見えるようにしましょう。

\begin{figure}[htbp]
\centering
\capstart

\noindent\sphinxincludegraphics{{trump-open}.pdf}
\caption{切札を表にする例}\label{\detokenize{common/common:id60}}\label{\detokenize{common/common:trump-open}}\end{figure}


\subsubsection{能力を無効化する}
\label{\detokenize{common/common:id41}}
切札は裏向きもしくは、
墓地に移されると能力を発揮しなくなります。
切札を無効化するためには、「クローズ」アクションを用い
切札を裏向きにするか、
「切札破壊」アクションを用いて切札を破壊しましょう。
「クローズ」アクション、
「切札破壊」アクションの詳細は、
「エクストラリスト」を参照して下さい。


\subsection{エクストラ注意事項}
\label{\detokenize{common/common:id42}}

\subsubsection{1ターンに1回制限のアクションについて}
\label{\detokenize{common/common:id43}}
切札がもたらすアクションの中には「プレイヤーは1ターンに1回しかこのアクションを起こすことができない。」
と特記事項に記載されているものがあります。
このアクションは1ターンに1回しか起こすことができないため、
切札が無効化され再度オープンし有効となっても、そのターンを通して1回しか起こすことができません。


\section{その他のルール}
\label{\detokenize{common/common:id44}}
この章では、
公開・非公開情報やシャッフルの仕方といった
細かな決まりごとを説明します。


\subsection{公開・非公開情報}
\label{\detokenize{common/common:id45}}
配置されているカードには、アクションの効果
を使わなくても中身や枚数を知れるものがあります。
知れる度合いには次の種類があります。
\begin{description}
\item[{完全公開}] \leavevmode
全てのプレイヤーが知ることができ、
聞かれたプレイヤーは正しく答える必要がある

\item[{個人公開}] \leavevmode
デッキの持ち主のみ知ることができる

\item[{非公開}] \leavevmode
全てのプレイヤーは知ることができない

\end{description}

完全公開の情報であれば、ゲーム中いつでも対戦相手に聞くことができます。
各カードの配置と公開・非公開の度合いは次のとおりです。
\begin{description}
\item[{デッキ}] \leavevmode
\begin{DUlineblock}{0em}
\item[] 完全公開:10枚未満のデッキ枚数
\item[] 個人公開:デッキの枚数
\item[] 非公開:デッキの中身
\end{DUlineblock}

\item[{墓地}] \leavevmode
\begin{DUlineblock}{0em}
\item[] 完全公開:墓地の一番上のカード
\item[] 個人公開:墓地の中身
\item[] 非公開:なし
\end{DUlineblock}

\item[{場}] \leavevmode
\begin{DUlineblock}{0em}
\item[] 完全公開:表裏を変えずに見えるカード
\item[] 個人公開:伏せてあるカード
\item[] 非公開:なし
\end{DUlineblock}

\item[{手札}] \leavevmode
\begin{DUlineblock}{0em}
\item[] 完全公開:手札の枚数
\item[] 個人公開:手札の中身
\item[] 非公開:なし
\end{DUlineblock}

\item[{切札}] \leavevmode
\begin{DUlineblock}{0em}
\item[] 完全公開:表裏を変えずに見えるカード
\item[] 個人公開:伏せてあるカード
\item[] 非公開:なし
\end{DUlineblock}

\end{description}


\subsubsection{残りのデッキ枚数を聞かれたらどうしたらいいの?}
\label{\detokenize{common/common:id46}}
対戦相手から残りのデッキ枚数を聞かれた場合、自分のデッキの枚数を上から10枚まで数え、相手に数えたカードの枚数が分かるように裏向きで見せます。
10枚未満であれば枚数を答え、10枚以上の場合「10枚以上です」と答えて下さい。
10枚以上の場合、正確な枚数を答える必要はありません。


\subsubsection{墓地の一番上のカードはいつ決まるのか?}
\label{\detokenize{common/common:id47}}
カードを墓地に移す際に移すカードの中から1枚を公開してください。
すでに墓地にあるカードを改めて公開しないでください。


\subsection{デッキのシャッフルについて}
\label{\detokenize{common/common:id48}}
BlackPokerでは
コンセプトの1つに”相手のカードに触らない”があるため、
対戦相手にデッキのシャッフルをお願いする必要はありません。

ただシャッフルしてほしいのであれば、お願いしても構いません。
逆に、対戦相手があまりシャッフルしていない場合は、
さらにシャッフルをお願いすることができます。


\chapter{フォーマット}
\label{\detokenize{format/format:id1}}\label{\detokenize{format/format::doc}}

\section{フォーマットとは}
\label{\detokenize{format/format:id2}}
BlackPokerにはいくつかのフォーマットがあり、
フォーマットによりゲーム内でできる行動が異なります。
同じトランプでもフォーマットを変えることで様々な遊び方をすることができます。

BlackPokerはアクションという行動を起こし、兵士などのキャラクターを出してターンを進めていくゲームです。
アクション、キャラクターの種類は、ライト < スタンダード < プロ < マスター < エクストラの順に増えていきます。
カードゲームでいうところのカードの種類が増えていくイメージです。
覚える量が多いほど難易度が高いため、初心者はライトから始めることをお勧めします。

アクションはアクションリストに記載されており、
フォーマットによって参照するアクションリストが異なります。


\subsection{対戦レギュレーションとの違い}
\label{\detokenize{format/format:id3}}
フォーマットと対戦レギュレーションの違いは、
フォーマットはゲーム内でできるアクション等のできる行動を定義している
のに対して、対戦レギュレーションはフォーマットを前提としてそれを加工している位置づけになります。

フォーマット名と対戦レギュレーション名が等しい対戦レギュレーションは、
フォーマットのルールが無加工で遊べます。


\subsection{共通ルールとの関係}
\label{\detokenize{format/format:id4}}
フォーマットは共通ルールを参照しており、
共通ルールにBlackPokerの主なルールが定義されています。

フォーマット、対戦レギュレーション、共通ルールの関係は次の図のようになります。

\noindent\sphinxincludegraphics{{plantuml-c4b165770c9149e49e8be64e90ba8bbbafdb51b5}.pdf}


\section{種類}
\label{\detokenize{format/format:id5}}
フォーマットは次の種類があります。


\begin{savenotes}\sphinxattablestart
\centering
\begin{tabulary}{\linewidth}[t]{|T|T|T|T|T|}
\hline
\sphinxstyletheadfamily 
フォーマット名
&\sphinxstyletheadfamily 
難易度
&\sphinxstyletheadfamily 
アクション数
&\sphinxstyletheadfamily 
キャラクター数
&\sphinxstyletheadfamily 
切札有無
\\
\hline
ライト
&
★
&
17
&
5
&
無
\\
\hline
スタンダード
&
★★
&
24
&
6
&
無
\\
\hline
プロ
&
★★★
&
29
&
6
&
無
\\
\hline
マスター
&
★★★★
&
35
&
6
&
無
\\
\hline
エクストラ
&
★★★★★
&
39〜
&
6〜
&
有
\\
\hline
\end{tabulary}
\par
\sphinxattableend\end{savenotes}

アクション数、キャラクター数の増加にともない覚える数が増えるため、難易度が上がります。


\section{定義項目}
\label{\detokenize{format/format:id6}}
フォーマットには次の項目が定義されています。
\begin{description}
\item[{アクションリスト}] \leavevmode
起こせるアクション、キャラクターのリスト

\item[{エクストラリスト}] \leavevmode
切札のリスト

\item[{事前準備}] \leavevmode
対戦前に行う事項

\item[{その他事項}] \leavevmode
上記項目で説明できないルール

\end{description}


\section{フォーマット定義}
\label{\detokenize{format/format:id7}}
公式として次のフォーマットを定義しています。


\subsection{ライト}
\label{\detokenize{format/light:id1}}\label{\detokenize{format/light::doc}}

\subsubsection{アクションリスト}
\label{\detokenize{format/light:id2}}\begin{itemize}
\item {} 
\hyperref[\detokenize{appendix/appendix:actionlist-lite}]{\ref{\detokenize{appendix/appendix:actionlist-lite}} \nameref{\detokenize{appendix/appendix:actionlist-lite}}}

\end{itemize}


\subsubsection{エクストラリスト}
\label{\detokenize{format/light:id3}}\begin{itemize}
\item {} 
なし

\end{itemize}


\subsubsection{事前準備}
\label{\detokenize{format/light:id4}}\begin{itemize}
\item {} 
なし

\end{itemize}


\subsubsection{その他事項}
\label{\detokenize{format/light:id5}}\begin{itemize}
\item {} 
なし

\end{itemize}


\subsection{スタンダード}
\label{\detokenize{format/standard:id1}}\label{\detokenize{format/standard::doc}}

\subsubsection{アクションリスト}
\label{\detokenize{format/standard:id2}}\begin{itemize}
\item {} 
\hyperref[\detokenize{appendix/appendix:actionlist-std}]{\ref{\detokenize{appendix/appendix:actionlist-std}} \nameref{\detokenize{appendix/appendix:actionlist-std}}}

\end{itemize}


\subsubsection{エクストラリスト}
\label{\detokenize{format/standard:id3}}\begin{itemize}
\item {} 
なし

\end{itemize}


\subsubsection{事前準備}
\label{\detokenize{format/standard:id4}}\begin{itemize}
\item {} 
なし

\end{itemize}


\subsubsection{その他事項}
\label{\detokenize{format/standard:id5}}\begin{itemize}
\item {} 
なし

\end{itemize}


\subsection{プロ}
\label{\detokenize{format/pro:id1}}\label{\detokenize{format/pro::doc}}

\subsubsection{アクションリスト}
\label{\detokenize{format/pro:id2}}\begin{itemize}
\item {} 
\sphinxcode{\sphinxupquote{actionlist\sphinxhyphen{}pro}}

\end{itemize}


\subsubsection{エクストラリスト}
\label{\detokenize{format/pro:id3}}\begin{itemize}
\item {} 
なし

\end{itemize}


\subsubsection{事前準備}
\label{\detokenize{format/pro:id4}}\begin{itemize}
\item {} 
なし

\end{itemize}


\subsubsection{その他事項}
\label{\detokenize{format/pro:id5}}\begin{itemize}
\item {} 
なし

\end{itemize}


\subsection{マスター}
\label{\detokenize{format/master:id1}}\label{\detokenize{format/master::doc}}

\subsubsection{アクションリスト}
\label{\detokenize{format/master:id2}}\begin{itemize}
\item {} 
\hyperref[\detokenize{appendix/appendix:actionlist-master}]{\ref{\detokenize{appendix/appendix:actionlist-master}} \nameref{\detokenize{appendix/appendix:actionlist-master}}}

\end{itemize}


\subsubsection{エクストラリスト}
\label{\detokenize{format/master:id3}}\begin{itemize}
\item {} 
なし

\end{itemize}


\subsubsection{事前準備}
\label{\detokenize{format/master:id4}}\begin{itemize}
\item {} 
なし

\end{itemize}


\subsubsection{その他事項}
\label{\detokenize{format/master:id5}}\begin{itemize}
\item {} 
なし

\end{itemize}


\subsection{エクストラ}
\label{\detokenize{format/extra:id1}}\label{\detokenize{format/extra::doc}}

\subsubsection{アクションリスト}
\label{\detokenize{format/extra:id2}}\begin{itemize}
\item {} 
\hyperref[\detokenize{appendix/appendix:actionlist-master}]{\ref{\detokenize{appendix/appendix:actionlist-master}} \nameref{\detokenize{appendix/appendix:actionlist-master}}}

\end{itemize}


\subsubsection{エクストラリスト}
\label{\detokenize{format/extra:id3}}\begin{itemize}
\item {} 
\hyperref[\detokenize{appendix/appendix:extralist}]{\ref{\detokenize{appendix/appendix:extralist}} \nameref{\detokenize{appendix/appendix:extralist}}}

\end{itemize}


\subsubsection{事前準備}
\label{\detokenize{format/extra:id4}}\begin{itemize}
\item {} 
切札を置く。詳細は、 \hyperref[\detokenize{common/common:extra-start}]{\ref{\detokenize{common/common:extra-start}} \nameref{\detokenize{common/common:extra-start}}} 参照。

\end{itemize}


\subsubsection{その他事項}
\label{\detokenize{format/extra:id5}}\begin{itemize}
\item {} 
なし

\end{itemize}


\chapter{対戦レギュレーション}
\label{\detokenize{match-regulations/match-regulations:id1}}\label{\detokenize{match-regulations/match-regulations::doc}}
対戦レギュレーションとは、
BlackPokerで対戦する前にプレイヤー間で決定する
規則のことです。

BlackPokerはトランプだけで遊べるため、
対戦する前にプレイヤー間でルールのすり合わせをする必要があります。


\section{定義項目}
\label{\detokenize{match-regulations/match-regulations:id2}}
各対戦レギュレーションには次の項目が定義されています。
\begin{description}
\item[{フォーマット}] \leavevmode
使用するフォーマット。詳しくは {\hyperref[\detokenize{format/format::doc}]{\sphinxcrossref{\DUrole{doc}{フォーマット}}}} 参照

\item[{デッキ条件}] \leavevmode
対戦に使用するデッキの条件

\item[{対戦前準備事項}] \leavevmode
切札の選定など対戦前に行う事項

\item[{その他制約事項}] \leavevmode
上記項目で説明できない制約事項

\end{description}


\section{レギュレーション定義}
\label{\detokenize{match-regulations/match-regulations:id3}}
公式として次の対戦レギュレーションを定義しています。


\subsection{ライト}
\label{\detokenize{match-regulations/light:id1}}\label{\detokenize{match-regulations/light::doc}}

\subsubsection{フォーマット}
\label{\detokenize{match-regulations/light:id2}}
ライト


\subsubsection{デッキ条件}
\label{\detokenize{match-regulations/light:id3}}
次のカードの中から54までカードを選びデッキとする


\begin{savenotes}\sphinxattablestart
\centering
\begin{tabulary}{\linewidth}[t]{|T|T|}
\hline

{\normalsize $\spadesuit$} 
&
A〜K
\\
\hline
{\normalsize $\heartsuit$} 
&
A〜K
\\
\hline
{\normalsize $\diamondsuit$} 
&
A〜K
\\
\hline
{\normalsize $\clubsuit$} 
&
A〜K
\\
\hline\sphinxstartmulticolumn{2}%
\begin{varwidth}[t]{\sphinxcolwidth{2}{2}}
Joker x2
\par
\vskip-\baselineskip\vbox{\hbox{\strut}}\end{varwidth}%
\sphinxstopmulticolumn
\\
\hline
\end{tabulary}
\par
\sphinxattableend\end{savenotes}


\subsubsection{対戦前準備事項}
\label{\detokenize{match-regulations/light:id4}}\begin{itemize}
\item {} 
なし

\end{itemize}


\subsubsection{その他制限事項}
\label{\detokenize{match-regulations/light:id5}}\begin{itemize}
\item {} 
なし

\end{itemize}


\subsection{ライト40}
\label{\detokenize{match-regulations/light40:id1}}\label{\detokenize{match-regulations/light40::doc}}

\subsubsection{フォーマット}
\label{\detokenize{match-regulations/light40:id2}}
ライト


\subsubsection{デッキ条件}
\label{\detokenize{match-regulations/light40:id3}}
次のカードの中から40までカードを選びデッキとする


\begin{savenotes}\sphinxattablestart
\centering
\begin{tabulary}{\linewidth}[t]{|T|T|}
\hline

{\normalsize $\spadesuit$} 
&
A〜K
\\
\hline
{\normalsize $\heartsuit$} 
&
A〜K
\\
\hline
{\normalsize $\diamondsuit$} 
&
A〜K
\\
\hline
{\normalsize $\clubsuit$} 
&
A〜K
\\
\hline\sphinxstartmulticolumn{2}%
\begin{varwidth}[t]{\sphinxcolwidth{2}{2}}
Joker x2
\par
\vskip-\baselineskip\vbox{\hbox{\strut}}\end{varwidth}%
\sphinxstopmulticolumn
\\
\hline
\end{tabulary}
\par
\sphinxattableend\end{savenotes}


\subsubsection{対戦前準備事項}
\label{\detokenize{match-regulations/light40:id4}}\begin{itemize}
\item {} 
なし

\end{itemize}


\subsubsection{その他制限事項}
\label{\detokenize{match-regulations/light40:id5}}\begin{itemize}
\item {} 
なし

\end{itemize}


\subsection{スタンダード}
\label{\detokenize{match-regulations/standard:id1}}\label{\detokenize{match-regulations/standard::doc}}

\subsubsection{フォーマット}
\label{\detokenize{match-regulations/standard:id2}}
スタンダード


\subsubsection{デッキ条件}
\label{\detokenize{match-regulations/standard:id3}}
次のカードの中から54までカードを選びデッキとする


\begin{savenotes}\sphinxattablestart
\centering
\begin{tabulary}{\linewidth}[t]{|T|T|}
\hline

{\normalsize $\spadesuit$} 
&
A〜K
\\
\hline
{\normalsize $\heartsuit$} 
&
A〜K
\\
\hline
{\normalsize $\diamondsuit$} 
&
A〜K
\\
\hline
{\normalsize $\clubsuit$} 
&
A〜K
\\
\hline\sphinxstartmulticolumn{2}%
\begin{varwidth}[t]{\sphinxcolwidth{2}{2}}
Joker x2
\par
\vskip-\baselineskip\vbox{\hbox{\strut}}\end{varwidth}%
\sphinxstopmulticolumn
\\
\hline
\end{tabulary}
\par
\sphinxattableend\end{savenotes}


\subsubsection{対戦前準備事項}
\label{\detokenize{match-regulations/standard:id4}}\begin{itemize}
\item {} 
なし

\end{itemize}


\subsubsection{その他制限事項}
\label{\detokenize{match-regulations/standard:id5}}\begin{itemize}
\item {} 
なし

\end{itemize}


\subsection{スタンダード40}
\label{\detokenize{match-regulations/standard40:id1}}\label{\detokenize{match-regulations/standard40::doc}}

\subsubsection{フォーマット}
\label{\detokenize{match-regulations/standard40:id2}}
スタンダード


\subsubsection{デッキ条件}
\label{\detokenize{match-regulations/standard40:id3}}
次のカードの中から40までカードを選びデッキとする


\begin{savenotes}\sphinxattablestart
\centering
\begin{tabulary}{\linewidth}[t]{|T|T|}
\hline

{\normalsize $\spadesuit$} 
&
A〜K
\\
\hline
{\normalsize $\heartsuit$} 
&
A〜K
\\
\hline
{\normalsize $\diamondsuit$} 
&
A〜K
\\
\hline
{\normalsize $\clubsuit$} 
&
A〜K
\\
\hline\sphinxstartmulticolumn{2}%
\begin{varwidth}[t]{\sphinxcolwidth{2}{2}}
Joker x2
\par
\vskip-\baselineskip\vbox{\hbox{\strut}}\end{varwidth}%
\sphinxstopmulticolumn
\\
\hline
\end{tabulary}
\par
\sphinxattableend\end{savenotes}


\subsubsection{対戦前準備事項}
\label{\detokenize{match-regulations/standard40:id4}}\begin{itemize}
\item {} 
なし

\end{itemize}


\subsubsection{その他制限事項}
\label{\detokenize{match-regulations/standard40:id5}}\begin{itemize}
\item {} 
なし

\end{itemize}


\subsection{プロ}
\label{\detokenize{match-regulations/pro:id1}}\label{\detokenize{match-regulations/pro::doc}}

\subsubsection{フォーマット}
\label{\detokenize{match-regulations/pro:id2}}
プロ


\subsubsection{デッキ条件}
\label{\detokenize{match-regulations/pro:id3}}
次のカードの中から54までカードを選びデッキとする


\begin{savenotes}\sphinxattablestart
\centering
\begin{tabulary}{\linewidth}[t]{|T|T|}
\hline

{\normalsize $\spadesuit$} 
&
A〜K
\\
\hline
{\normalsize $\heartsuit$} 
&
A〜K
\\
\hline
{\normalsize $\diamondsuit$} 
&
A〜K
\\
\hline
{\normalsize $\clubsuit$} 
&
A〜K
\\
\hline\sphinxstartmulticolumn{2}%
\begin{varwidth}[t]{\sphinxcolwidth{2}{2}}
Joker x2
\par
\vskip-\baselineskip\vbox{\hbox{\strut}}\end{varwidth}%
\sphinxstopmulticolumn
\\
\hline
\end{tabulary}
\par
\sphinxattableend\end{savenotes}


\subsubsection{対戦前準備事項}
\label{\detokenize{match-regulations/pro:id4}}\begin{itemize}
\item {} 
なし

\end{itemize}


\subsubsection{その他制限事項}
\label{\detokenize{match-regulations/pro:id5}}\begin{itemize}
\item {} 
なし

\end{itemize}


\subsection{プロ40}
\label{\detokenize{match-regulations/pro40:id1}}\label{\detokenize{match-regulations/pro40::doc}}

\subsubsection{フォーマット}
\label{\detokenize{match-regulations/pro40:id2}}
プロ


\subsubsection{デッキ条件}
\label{\detokenize{match-regulations/pro40:id3}}
次のカードの中から40までカードを選びデッキとする


\begin{savenotes}\sphinxattablestart
\centering
\begin{tabulary}{\linewidth}[t]{|T|T|}
\hline

{\normalsize $\spadesuit$} 
&
A〜K
\\
\hline
{\normalsize $\heartsuit$} 
&
A〜K
\\
\hline
{\normalsize $\diamondsuit$} 
&
A〜K
\\
\hline
{\normalsize $\clubsuit$} 
&
A〜K
\\
\hline\sphinxstartmulticolumn{2}%
\begin{varwidth}[t]{\sphinxcolwidth{2}{2}}
Joker x2
\par
\vskip-\baselineskip\vbox{\hbox{\strut}}\end{varwidth}%
\sphinxstopmulticolumn
\\
\hline
\end{tabulary}
\par
\sphinxattableend\end{savenotes}


\subsubsection{対戦前準備事項}
\label{\detokenize{match-regulations/pro40:id4}}\begin{itemize}
\item {} 
なし

\end{itemize}


\subsubsection{その他制限事項}
\label{\detokenize{match-regulations/pro40:id5}}\begin{itemize}
\item {} 
なし

\end{itemize}


\subsection{マスター}
\label{\detokenize{match-regulations/master:id1}}\label{\detokenize{match-regulations/master::doc}}

\subsubsection{フォーマット}
\label{\detokenize{match-regulations/master:id2}}
マスター


\subsubsection{デッキ条件}
\label{\detokenize{match-regulations/master:id3}}
次のカードの中から54までカードを選びデッキとする


\begin{savenotes}\sphinxattablestart
\centering
\begin{tabulary}{\linewidth}[t]{|T|T|}
\hline

{\normalsize $\spadesuit$} 
&
A〜K
\\
\hline
{\normalsize $\heartsuit$} 
&
A〜K
\\
\hline
{\normalsize $\diamondsuit$} 
&
A〜K
\\
\hline
{\normalsize $\clubsuit$} 
&
A〜K
\\
\hline\sphinxstartmulticolumn{2}%
\begin{varwidth}[t]{\sphinxcolwidth{2}{2}}
Joker x2
\par
\vskip-\baselineskip\vbox{\hbox{\strut}}\end{varwidth}%
\sphinxstopmulticolumn
\\
\hline
\end{tabulary}
\par
\sphinxattableend\end{savenotes}


\subsubsection{対戦前準備事項}
\label{\detokenize{match-regulations/master:id4}}\begin{itemize}
\item {} 
なし

\end{itemize}


\subsubsection{その他制限事項}
\label{\detokenize{match-regulations/master:id5}}\begin{itemize}
\item {} 
なし

\end{itemize}


\subsection{マスター40}
\label{\detokenize{match-regulations/master40:id1}}\label{\detokenize{match-regulations/master40::doc}}

\subsubsection{フォーマット}
\label{\detokenize{match-regulations/master40:id2}}
マスター


\subsubsection{デッキ条件}
\label{\detokenize{match-regulations/master40:id3}}
次のカードの中から40までカードを選びデッキとする


\begin{savenotes}\sphinxattablestart
\centering
\begin{tabulary}{\linewidth}[t]{|T|T|}
\hline

{\normalsize $\spadesuit$} 
&
A〜K
\\
\hline
{\normalsize $\heartsuit$} 
&
A〜K
\\
\hline
{\normalsize $\diamondsuit$} 
&
A〜K
\\
\hline
{\normalsize $\clubsuit$} 
&
A〜K
\\
\hline\sphinxstartmulticolumn{2}%
\begin{varwidth}[t]{\sphinxcolwidth{2}{2}}
Joker x2
\par
\vskip-\baselineskip\vbox{\hbox{\strut}}\end{varwidth}%
\sphinxstopmulticolumn
\\
\hline
\end{tabulary}
\par
\sphinxattableend\end{savenotes}


\subsubsection{対戦前準備事項}
\label{\detokenize{match-regulations/master40:id4}}\begin{itemize}
\item {} 
なし

\end{itemize}


\subsubsection{その他制限事項}
\label{\detokenize{match-regulations/master40:id5}}\begin{itemize}
\item {} 
なし

\end{itemize}


\chapter{付録}
\label{\detokenize{appendix/appendix:id1}}\label{\detokenize{appendix/appendix::doc}}

\section{PDF版ルール}
\label{\detokenize{appendix/appendix:pdf}}
\sphinxurl{https://blackpoker.github.io/BlackPoker/issue\_36/blackpoker.pdf}


\section{アクションリスト}
\label{\detokenize{appendix/appendix:id2}}

\subsection{ライト}
\label{\detokenize{appendix/appendix:actionlist-lite}}\label{\detokenize{appendix/appendix:id3}}\begin{quote}
\begin{description}
\item[{URL}] \leavevmode
\sphinxurl{https://blackpoker.github.io/BlackPoker/issue\_36/actionlist/html/v6-lite.html}

\item[{PDF}] \leavevmode
\sphinxurl{https://blackpoker.github.io/BlackPoker/issue\_36/actionlist/pdf/blackpoker-v6-lite.pdf}
\sphinxurl{https://blackpoker.github.io/BlackPoker/issue\_36/actionlist/pdf/blackpoker-v6-lite-2up.pdf}

\end{description}
\end{quote}


\subsection{スタンダード}
\label{\detokenize{appendix/appendix:actionlist-std}}\label{\detokenize{appendix/appendix:id4}}\begin{quote}
\begin{description}
\item[{URL}] \leavevmode
\sphinxurl{https://blackpoker.github.io/BlackPoker/issue\_36/actionlist/html/v6-std.html}

\item[{PDF}] \leavevmode
\sphinxurl{https://blackpoker.github.io/BlackPoker/issue\_36/actionlist/pdf/blackpoker-v6-std.pdf}
\sphinxurl{https://blackpoker.github.io/BlackPoker/issue\_36/actionlist/pdf/blackpoker-v6-std-2up.pdf}

\end{description}
\end{quote}


\subsection{プロ}
\label{\detokenize{appendix/appendix:id5}}\begin{quote}
\begin{description}
\item[{URL}] \leavevmode
\sphinxurl{https://blackpoker.github.io/BlackPoker/issue\_36/actionlist/html/v6-pro.html}

\item[{PDF}] \leavevmode
\sphinxurl{https://blackpoker.github.io/BlackPoker/issue\_36/actionlist/pdf/blackpoker-v6-pro.pdf}
\sphinxurl{https://blackpoker.github.io/BlackPoker/issue\_36/actionlist/pdf/blackpoker-v6-pro-2up.pdf}

\end{description}
\end{quote}


\subsection{マスター}
\label{\detokenize{appendix/appendix:actionlist-master}}\label{\detokenize{appendix/appendix:id6}}\begin{quote}
\begin{description}
\item[{URL}] \leavevmode
\sphinxurl{https://blackpoker.github.io/BlackPoker/issue\_36/actionlist/html/v6-mast.html}

\item[{PDF}] \leavevmode
\sphinxurl{https://blackpoker.github.io/BlackPoker/issue\_36/actionlist/pdf/blackpoker-v6-mast.pdf}
\sphinxurl{https://blackpoker.github.io/BlackPoker/issue\_36/actionlist/pdf/blackpoker-v6-mast-2up.pdf}

\end{description}
\end{quote}


\section{エクストラリスト}
\label{\detokenize{appendix/appendix:extralist}}\label{\detokenize{appendix/appendix:id7}}\begin{quote}
\begin{description}
\item[{URL}] \leavevmode
\sphinxurl{https://blackpoker.github.io/BlackPoker/issue\_36/actionlist/html/v6-ex.html}

\item[{PDF}] \leavevmode
\sphinxurl{https://blackpoker.github.io/BlackPoker/issue\_36/actionlist/pdf/blackpoker-v6-extra.pdf}
\sphinxurl{https://blackpoker.github.io/BlackPoker/issue\_36/actionlist/pdf/blackpoker-v6-extra-2up.pdf}

\end{description}
\end{quote}



\renewcommand{\indexname}{索引}
\printindex
\end{document}