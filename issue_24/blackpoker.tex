%% Generated by Sphinx.
\def\sphinxdocclass{jsbook}
\documentclass[letterpaper,10pt,dvipdfmx]{sphinxmanual}
\ifdefined\pdfpxdimen
   \let\sphinxpxdimen\pdfpxdimen\else\newdimen\sphinxpxdimen
\fi \sphinxpxdimen=.75bp\relax

\PassOptionsToPackage{warn}{textcomp}


\usepackage{cmap}
\usepackage[T1]{fontenc}
\usepackage{amsmath,amssymb,amstext}



\usepackage{times}


\usepackage[,numfigreset=1,mathnumfig]{sphinx}

\fvset{fontsize=\small}
\usepackage[dvipdfm]{geometry}


% Include hyperref last.
\usepackage{hyperref}
% Fix anchor placement for figures with captions.
\usepackage{hypcap}% it must be loaded after hyperref.
% Set up styles of URL: it should be placed after hyperref.
\urlstyle{same}
\renewcommand{\contentsname}{Contents:}

\usepackage{sphinxmessages}
\setcounter{tocdepth}{1}



\title{BlackPoker}
\date{2021年06月05日}
\release{}
\author{BlackPoker}
\newcommand{\sphinxlogo}{\vbox{}}
\renewcommand{\releasename}{}
\makeindex
\begin{document}

\pagestyle{empty}
\sphinxmaketitle
\pagestyle{plain}
\sphinxtableofcontents
\pagestyle{normal}
\phantomsection\label{\detokenize{index::doc}}



\chapter{はじめに}
\label{\detokenize{init/init:id1}}\label{\detokenize{init/init::doc}}
この文章はトランプゲーム「BlackPoker」の全てのルールをまとめた文章です。

詳細なルールが記載されており、初心者の方は文章の量に圧倒されます。
ゲームをプレイする際に全てを熟読する必要はありませんが、
ルールについて深く知りたい、または新しいルールに触れたい方はぜひ熟読してください。


\section{ルールの構成}
\label{\detokenize{init/init:id2}}
ルールの構成は次のようになっています。

\noindent\sphinxincludegraphics{{plantuml-539e0c031fa17f075d01c25c766b6741f05fab73}.pdf}


\chapter{対戦レギュレーション}
\label{\detokenize{match-regulations/match-regulations:id1}}\label{\detokenize{match-regulations/match-regulations::doc}}
対戦レギュレーションとは、
BlackPokerで対戦する前にプレイヤー間で決定する
規則のことです。

BlackPokerはトランプだけで遊べるため、
対戦する前にプレイヤー間でルールのすり合わせをする必要があります。


\section{定義項目}
\label{\detokenize{match-regulations/match-regulations:id2}}
各対戦レギュレーションには次の項目が定義されています。
\begin{description}
\item[{フォーマット}] \leavevmode
使用するフォーマット。詳しくは {\hyperref[\detokenize{format/format::doc}]{\sphinxcrossref{\DUrole{doc}{フォーマット}}}} 参照

\item[{デッキ条件}] \leavevmode
対戦に使用するデッキの条件

\item[{対戦前準備事項}] \leavevmode
切札の選定など対戦前に行う事項

\item[{その他制約事項}] \leavevmode
上記項目で説明できない制約事項

\end{description}


\section{レギュレーション定義}
\label{\detokenize{match-regulations/match-regulations:id3}}
公式として次の対戦レギュレーションを定義しています。


\subsection{ライト}
\label{\detokenize{match-regulations/light:id1}}\label{\detokenize{match-regulations/light::doc}}

\subsubsection{フォーマット}
\label{\detokenize{match-regulations/light:id2}}
ライト


\subsubsection{デッキ条件}
\label{\detokenize{match-regulations/light:id3}}
次のカードの中から54までカードを選びデッキとする


\begin{savenotes}\sphinxattablestart
\centering
\begin{tabulary}{\linewidth}[t]{|T|T|}
\hline

{\normalsize $\spadesuit$} 
&
A〜K
\\
\hline
{\normalsize $\heartsuit$} 
&
A〜K
\\
\hline
{\normalsize $\diamondsuit$} 
&
A〜K
\\
\hline
{\normalsize $\clubsuit$} 
&
A〜K
\\
\hline\sphinxstartmulticolumn{2}%
\begin{varwidth}[t]{\sphinxcolwidth{2}{2}}
Joker x2
\par
\vskip-\baselineskip\vbox{\hbox{\strut}}\end{varwidth}%
\sphinxstopmulticolumn
\\
\hline
\end{tabulary}
\par
\sphinxattableend\end{savenotes}


\subsubsection{対戦前準備事項}
\label{\detokenize{match-regulations/light:id4}}\begin{itemize}
\item {} 
なし

\end{itemize}


\subsubsection{その他制限事項}
\label{\detokenize{match-regulations/light:id5}}\begin{itemize}
\item {} 
なし

\end{itemize}


\subsection{ライト40}
\label{\detokenize{match-regulations/light40:id1}}\label{\detokenize{match-regulations/light40::doc}}

\subsubsection{フォーマット}
\label{\detokenize{match-regulations/light40:id2}}
ライト


\subsubsection{デッキ条件}
\label{\detokenize{match-regulations/light40:id3}}
次のカードの中から40までカードを選びデッキとする


\begin{savenotes}\sphinxattablestart
\centering
\begin{tabulary}{\linewidth}[t]{|T|T|}
\hline

{\normalsize $\spadesuit$} 
&
A〜K
\\
\hline
{\normalsize $\heartsuit$} 
&
A〜K
\\
\hline
{\normalsize $\diamondsuit$} 
&
A〜K
\\
\hline
{\normalsize $\clubsuit$} 
&
A〜K
\\
\hline\sphinxstartmulticolumn{2}%
\begin{varwidth}[t]{\sphinxcolwidth{2}{2}}
Joker x2
\par
\vskip-\baselineskip\vbox{\hbox{\strut}}\end{varwidth}%
\sphinxstopmulticolumn
\\
\hline
\end{tabulary}
\par
\sphinxattableend\end{savenotes}


\subsubsection{対戦前準備事項}
\label{\detokenize{match-regulations/light40:id4}}\begin{itemize}
\item {} 
なし

\end{itemize}


\subsubsection{その他制限事項}
\label{\detokenize{match-regulations/light40:id5}}\begin{itemize}
\item {} 
なし

\end{itemize}


\subsection{スタンダード}
\label{\detokenize{match-regulations/standard:id1}}\label{\detokenize{match-regulations/standard::doc}}

\subsubsection{フォーマット}
\label{\detokenize{match-regulations/standard:id2}}
スタンダード


\subsubsection{デッキ条件}
\label{\detokenize{match-regulations/standard:id3}}
次のカードの中から54までカードを選びデッキとする


\begin{savenotes}\sphinxattablestart
\centering
\begin{tabulary}{\linewidth}[t]{|T|T|}
\hline

{\normalsize $\spadesuit$} 
&
A〜K
\\
\hline
{\normalsize $\heartsuit$} 
&
A〜K
\\
\hline
{\normalsize $\diamondsuit$} 
&
A〜K
\\
\hline
{\normalsize $\clubsuit$} 
&
A〜K
\\
\hline\sphinxstartmulticolumn{2}%
\begin{varwidth}[t]{\sphinxcolwidth{2}{2}}
Joker x2
\par
\vskip-\baselineskip\vbox{\hbox{\strut}}\end{varwidth}%
\sphinxstopmulticolumn
\\
\hline
\end{tabulary}
\par
\sphinxattableend\end{savenotes}


\subsubsection{対戦前準備事項}
\label{\detokenize{match-regulations/standard:id4}}\begin{itemize}
\item {} 
なし

\end{itemize}


\subsubsection{その他制限事項}
\label{\detokenize{match-regulations/standard:id5}}\begin{itemize}
\item {} 
なし

\end{itemize}


\subsection{スタンダード40}
\label{\detokenize{match-regulations/standard40:id1}}\label{\detokenize{match-regulations/standard40::doc}}

\subsubsection{フォーマット}
\label{\detokenize{match-regulations/standard40:id2}}
スタンダード


\subsubsection{デッキ条件}
\label{\detokenize{match-regulations/standard40:id3}}
次のカードの中から40までカードを選びデッキとする


\begin{savenotes}\sphinxattablestart
\centering
\begin{tabulary}{\linewidth}[t]{|T|T|}
\hline

{\normalsize $\spadesuit$} 
&
A〜K
\\
\hline
{\normalsize $\heartsuit$} 
&
A〜K
\\
\hline
{\normalsize $\diamondsuit$} 
&
A〜K
\\
\hline
{\normalsize $\clubsuit$} 
&
A〜K
\\
\hline\sphinxstartmulticolumn{2}%
\begin{varwidth}[t]{\sphinxcolwidth{2}{2}}
Joker x2
\par
\vskip-\baselineskip\vbox{\hbox{\strut}}\end{varwidth}%
\sphinxstopmulticolumn
\\
\hline
\end{tabulary}
\par
\sphinxattableend\end{savenotes}


\subsubsection{対戦前準備事項}
\label{\detokenize{match-regulations/standard40:id4}}\begin{itemize}
\item {} 
なし

\end{itemize}


\subsubsection{その他制限事項}
\label{\detokenize{match-regulations/standard40:id5}}\begin{itemize}
\item {} 
なし

\end{itemize}


\subsection{プロ}
\label{\detokenize{match-regulations/pro:id1}}\label{\detokenize{match-regulations/pro::doc}}

\subsubsection{フォーマット}
\label{\detokenize{match-regulations/pro:id2}}
プロ


\subsubsection{デッキ条件}
\label{\detokenize{match-regulations/pro:id3}}
次のカードの中から54までカードを選びデッキとする


\begin{savenotes}\sphinxattablestart
\centering
\begin{tabulary}{\linewidth}[t]{|T|T|}
\hline

{\normalsize $\spadesuit$} 
&
A〜K
\\
\hline
{\normalsize $\heartsuit$} 
&
A〜K
\\
\hline
{\normalsize $\diamondsuit$} 
&
A〜K
\\
\hline
{\normalsize $\clubsuit$} 
&
A〜K
\\
\hline\sphinxstartmulticolumn{2}%
\begin{varwidth}[t]{\sphinxcolwidth{2}{2}}
Joker x2
\par
\vskip-\baselineskip\vbox{\hbox{\strut}}\end{varwidth}%
\sphinxstopmulticolumn
\\
\hline
\end{tabulary}
\par
\sphinxattableend\end{savenotes}


\subsubsection{対戦前準備事項}
\label{\detokenize{match-regulations/pro:id4}}\begin{itemize}
\item {} 
なし

\end{itemize}


\subsubsection{その他制限事項}
\label{\detokenize{match-regulations/pro:id5}}\begin{itemize}
\item {} 
なし

\end{itemize}


\subsection{プロ40}
\label{\detokenize{match-regulations/pro40:id1}}\label{\detokenize{match-regulations/pro40::doc}}

\subsubsection{フォーマット}
\label{\detokenize{match-regulations/pro40:id2}}
プロ


\subsubsection{デッキ条件}
\label{\detokenize{match-regulations/pro40:id3}}
次のカードの中から40までカードを選びデッキとする


\begin{savenotes}\sphinxattablestart
\centering
\begin{tabulary}{\linewidth}[t]{|T|T|}
\hline

{\normalsize $\spadesuit$} 
&
A〜K
\\
\hline
{\normalsize $\heartsuit$} 
&
A〜K
\\
\hline
{\normalsize $\diamondsuit$} 
&
A〜K
\\
\hline
{\normalsize $\clubsuit$} 
&
A〜K
\\
\hline\sphinxstartmulticolumn{2}%
\begin{varwidth}[t]{\sphinxcolwidth{2}{2}}
Joker x2
\par
\vskip-\baselineskip\vbox{\hbox{\strut}}\end{varwidth}%
\sphinxstopmulticolumn
\\
\hline
\end{tabulary}
\par
\sphinxattableend\end{savenotes}


\subsubsection{対戦前準備事項}
\label{\detokenize{match-regulations/pro40:id4}}\begin{itemize}
\item {} 
なし

\end{itemize}


\subsubsection{その他制限事項}
\label{\detokenize{match-regulations/pro40:id5}}\begin{itemize}
\item {} 
なし

\end{itemize}


\subsection{マスター}
\label{\detokenize{match-regulations/master:id1}}\label{\detokenize{match-regulations/master::doc}}

\subsubsection{フォーマット}
\label{\detokenize{match-regulations/master:id2}}
マスター


\subsubsection{デッキ条件}
\label{\detokenize{match-regulations/master:id3}}
次のカードの中から54までカードを選びデッキとする


\begin{savenotes}\sphinxattablestart
\centering
\begin{tabulary}{\linewidth}[t]{|T|T|}
\hline

{\normalsize $\spadesuit$} 
&
A〜K
\\
\hline
{\normalsize $\heartsuit$} 
&
A〜K
\\
\hline
{\normalsize $\diamondsuit$} 
&
A〜K
\\
\hline
{\normalsize $\clubsuit$} 
&
A〜K
\\
\hline\sphinxstartmulticolumn{2}%
\begin{varwidth}[t]{\sphinxcolwidth{2}{2}}
Joker x2
\par
\vskip-\baselineskip\vbox{\hbox{\strut}}\end{varwidth}%
\sphinxstopmulticolumn
\\
\hline
\end{tabulary}
\par
\sphinxattableend\end{savenotes}


\subsubsection{対戦前準備事項}
\label{\detokenize{match-regulations/master:id4}}\begin{itemize}
\item {} 
なし

\end{itemize}


\subsubsection{その他制限事項}
\label{\detokenize{match-regulations/master:id5}}\begin{itemize}
\item {} 
なし

\end{itemize}


\subsection{マスター40}
\label{\detokenize{match-regulations/master40:id1}}\label{\detokenize{match-regulations/master40::doc}}

\subsubsection{フォーマット}
\label{\detokenize{match-regulations/master40:id2}}
マスター


\subsubsection{デッキ条件}
\label{\detokenize{match-regulations/master40:id3}}
次のカードの中から40までカードを選びデッキとする


\begin{savenotes}\sphinxattablestart
\centering
\begin{tabulary}{\linewidth}[t]{|T|T|}
\hline

{\normalsize $\spadesuit$} 
&
A〜K
\\
\hline
{\normalsize $\heartsuit$} 
&
A〜K
\\
\hline
{\normalsize $\diamondsuit$} 
&
A〜K
\\
\hline
{\normalsize $\clubsuit$} 
&
A〜K
\\
\hline\sphinxstartmulticolumn{2}%
\begin{varwidth}[t]{\sphinxcolwidth{2}{2}}
Joker x2
\par
\vskip-\baselineskip\vbox{\hbox{\strut}}\end{varwidth}%
\sphinxstopmulticolumn
\\
\hline
\end{tabulary}
\par
\sphinxattableend\end{savenotes}


\subsubsection{対戦前準備事項}
\label{\detokenize{match-regulations/master40:id4}}\begin{itemize}
\item {} 
なし

\end{itemize}


\subsubsection{その他制限事項}
\label{\detokenize{match-regulations/master40:id5}}\begin{itemize}
\item {} 
なし

\end{itemize}


\chapter{フォーマット}
\label{\detokenize{format/format:id1}}\label{\detokenize{format/format::doc}}

\section{フォーマットとは}
\label{\detokenize{format/format:id2}}
BlackPokerにはいくつかのフォーマットがあり、
フォーマットによりゲーム内でできる行動が異なります。
同じトランプでもフォーマットを変えることで様々な遊び方をすることができます。

BlackPokerはアクションという行動を起こし、兵士などのキャラクターを出してターンを進めていくゲームです。
アクション、キャラクターの種類は、ライト < スタンダード < プロ < マスター < エクストラの順に増えていきます。
カードゲームでいうところのカードの種類が増えていくイメージです。
覚える量が多いほど難易度が高いため、初心者はライトから始めることをお勧めします。

アクションはアクションリストに記載されており、
フォーマットによって参照するアクションリストが異なります。


\subsection{対戦レギュレーションとの違い}
\label{\detokenize{format/format:id3}}
フォーマットと対戦レギュレーションの違いは、
フォーマットはゲーム内でできるアクション等のできる行動を定義している
のに対して、対戦レギュレーションはフォーマットを前提としてそれを加工している位置づけになります。

フォーマット名と対戦レギュレーション名が等しい対戦レギュレーションは、
フォーマットのルールが無加工で遊べます。


\subsection{共通ルールとの関係}
\label{\detokenize{format/format:id4}}
フォーマットは共通ルールを参照しており、
共通ルールにBlackPokerの主なルールが定義されています。

フォーマット、対戦レギュレーション、共通ルールの関係は次の図のようになります。

\noindent\sphinxincludegraphics{{plantuml-c4b165770c9149e49e8be64e90ba8bbbafdb51b5}.pdf}


\section{種類}
\label{\detokenize{format/format:id5}}
フォーマットは次の種類があります。


\begin{savenotes}\sphinxattablestart
\centering
\begin{tabulary}{\linewidth}[t]{|T|T|T|T|T|}
\hline
\sphinxstyletheadfamily 
フォーマット名
&\sphinxstyletheadfamily 
難易度
&\sphinxstyletheadfamily 
アクション数
&\sphinxstyletheadfamily 
キャラクター数
&\sphinxstyletheadfamily 
切札有無
\\
\hline
ライト
&
★
&
17
&
5
&
無
\\
\hline
スタンダード
&
★★
&
24
&
6
&
無
\\
\hline
プロ
&
★★★
&
29
&
6
&
無
\\
\hline
マスター
&
★★★★
&
35
&
6
&
無
\\
\hline
エクストラ
&
★★★★★
&
39〜
&
6〜
&
有
\\
\hline
\end{tabulary}
\par
\sphinxattableend\end{savenotes}

アクション数、キャラクター数の増加にともない覚える数が増えるため、難易度が上がります。


\section{定義項目}
\label{\detokenize{format/format:id6}}
フォーマットには次の項目が定義されています。
\begin{description}
\item[{アクションリスト}] \leavevmode
起こせるアクション、キャラクターのリスト

\item[{エクストラリスト}] \leavevmode
切札のリスト

\item[{事前準備}] \leavevmode
対戦前に行う事項

\item[{その他事項}] \leavevmode
上記項目で説明できないルール

\end{description}


\section{フォーマット定義}
\label{\detokenize{format/format:id7}}
公式として次のフォーマットを定義しています。


\subsection{ライト}
\label{\detokenize{format/light:id1}}\label{\detokenize{format/light::doc}}

\subsubsection{アクションリスト}
\label{\detokenize{format/light:id2}}\begin{itemize}
\item {} 
ライトのアクションリストパスを記載する

\end{itemize}


\subsubsection{エクストラリスト}
\label{\detokenize{format/light:id3}}\begin{itemize}
\item {} 
なし

\end{itemize}


\subsubsection{事前準備}
\label{\detokenize{format/light:id4}}\begin{itemize}
\item {} 
なし

\end{itemize}


\subsubsection{その他事項}
\label{\detokenize{format/light:id5}}\begin{itemize}
\item {} 
なし

\end{itemize}


\chapter{共通ルール}
\label{\detokenize{common/common:id1}}\label{\detokenize{common/common::doc}}

\section{基本ルール}
\label{\detokenize{common/01-base:id1}}\label{\detokenize{common/01-base::doc}}
この章では、カードの配置などゲームを始める前
の基本的なルールを説明します。


\subsection{プレイ人数}
\label{\detokenize{common/01-base:id2}}
BlackPokerは基本2人で遊ぶゲームです。

フォーマット、対戦レギュレーションによりプレイ人数が変わることがあるため、
プレイする際に確認してください。


\subsection{用意するもの}
\label{\detokenize{common/01-base:id3}}\begin{itemize}
\item {} 
1人1セットのトランプが必要です。

\item {} 
覚えていない場合、フォーマットに応じてアクションリスト、エクストラリストがあると便利です。

\end{itemize}


\subsection{使用できるトランプ}
\label{\detokenize{common/01-base:id4}}
BlackPokerでは次の条件を満たしたトランプを使うことができます。
一般的なトランプなら満たす条件となっています。
\begin{quote}
\begin{itemize}
\item {} 
スートと数字が分かる

\item {} 
スートの{\normalsize $\spadesuit$} {\normalsize $\heartsuit$} {\normalsize $\diamondsuit$} {\normalsize $\clubsuit$} が判断できる

\item {} 
数字のA\sphinxhyphen{}K(1\sphinxhyphen{}13)が判断できる

\item {} 
スートと数字の組合せが重複していない

\item {} 
裏から表がわからない

\item {} 
縦向き横向きが判断できる

\item {} 
Jokerは2枚まで入れられる

\item {} 
54枚無くてもよい

\end{itemize}

対戦レギュレーションにより使用できるトランプが枚数など異なる場合があるため、
対戦する際に、対戦レギュレーション、フォーマットを確認してください。
\end{quote}


\subsection{トランプの数字}
\label{\detokenize{common/01-base:id5}}
ゲーム全体を通してトランプの数字は次のような数値として扱います。(\hyperref[\detokenize{common/01-base:cardrank}]{表 \ref{\detokenize{common/01-base:cardrank}}})


\begin{savenotes}\sphinxattablestart
\centering
\sphinxcapstartof{table}
\sphinxthecaptionisattop
\sphinxcaption{トランプの数字}\label{\detokenize{common/01-base:id11}}\label{\detokenize{common/01-base:cardrank}}
\sphinxaftertopcaption
\begin{tabulary}{\linewidth}[t]{|T|T|}
\hline
\sphinxstyletheadfamily 
カード
&\sphinxstyletheadfamily 
数字
\\
\hline
A
&
1
\\
\hline
2〜10
&
表記どおり
\\
\hline
J
&
11
\\
\hline
Q
&
12
\\
\hline
K
&
13
\\
\hline
Joker
&
0
\\
\hline
\end{tabulary}
\par
\sphinxattableend\end{savenotes}


\subsection{カードの配置}
\label{\detokenize{common/01-base:id6}}
カードの配置には次のような場所があります。(\hyperref[\detokenize{common/01-base:field-ex}]{図 \ref{\detokenize{common/01-base:field-ex}}})

\begin{figure}[htbp]
\centering
\capstart

\noindent\sphinxincludegraphics{{field-ex}.pdf}
\caption{プレイ中のカードの配置}\label{\detokenize{common/01-base:id12}}\label{\detokenize{common/01-base:field-ex}}\end{figure}
\begin{description}
\item[{デッキ}] \leavevmode
山札。ゲームを始める時に自分のトランプを裏向きに置く場所です。
ダメージを受けるとデッキの一番上から墓地にカードを移します。

\item[{墓地}] \leavevmode
捨て札置き場。ダメージを受けた時などに表向きでカードを重ねて置きます。

\item[{場}] \leavevmode
兵士や防壁などのキャラクターを置きます。

\item[{手札}] \leavevmode
デッキから引いたカードを持っておく場所です。相手から見えないようにしましょう。

\item[{切札}] \leavevmode
能力が割り当てられたカードを置きます。エクストラフォーマットのみで使用します。
エクストラのルールについては、 \hyperref[\detokenize{common/06-extra:extra}]{\ref{\detokenize{common/06-extra:extra}} 章} で説明します。

\end{description}


\subsection{勝利条件}
\label{\detokenize{common/01-base:id7}}
プレイヤーは順に対戦相手に対し攻撃を行い、ダメージを与え先に相手のデッキを0枚にした方が勝ちです。ダメージは1点につき1枚デッキが減ります。


\subsection{ダメージ}
\label{\detokenize{common/01-base:id8}}
プレイヤーがダメージを受けた場合、デッキの一番上から受けた点数分墓地にカードを表向きで移動します。移動する際は、カードの表を対戦相手に見せる必要はありません。


\subsection{キャラクター}
\label{\detokenize{common/01-base:id9}}
BlackPokerは主に兵士と防壁の攻防で対戦相手にダメージを与えます。
兵士や防壁など、破壊されるまで場に留まるカードをキャラクターと呼びます。
防壁は守備、兵士は主に攻撃を行います。
詳しくは {\hyperref[\detokenize{common/04-character::doc}]{\sphinxcrossref{\DUrole{doc}{キャラクターと能力}}}} にて説明します。


\subsection{チャージとドライブ}
\label{\detokenize{common/01-base:id10}}
キャラクターには、チャージ状態とドライブ状態が存在します。
チャージ状態は未使用状態を示し、ドライブ状態は使用済み状態を示しています。
また、キャラクターを横向きにすることを「ドライブ」、縦向きにすることを「チャージ」と言います。(\hyperref[\detokenize{common/01-base:chargedrive}]{図 \ref{\detokenize{common/01-base:chargedrive}}})

\begin{figure}[htbp]
\centering
\capstart

\noindent\sphinxincludegraphics{{charge&drive}.pdf}
\caption{チャージとドライブ}\label{\detokenize{common/01-base:id13}}\label{\detokenize{common/01-base:chargedrive}}\end{figure}

\begin{sphinxadmonition}{note}{注釈:}
【補足】ドライブ状態のキャラクターをドライブしたらどうなるの?

ドライブ状態のキャラクターをドライブした場合、
そのキャラクターはドライブ状態のままとなります。
チャージも同様に、チャージ状態のキャラクターをチャージしてもチャージ状態のままとなります。

 チャージ、ドライブという行為はすでにその状態となっている場合でもチャージ、ドライブという行為を行ったことになることに注意が必要です。
たとえばあるキャラクターをドライブするという効果があり、
そのキャラクターがすでにドライブ状態の場合、効果を発揮してドライブという行為を行った上でキャラクターはドライブ状態のままということになります。
\end{sphinxadmonition}


\section{ゲームの流れ}
\label{\detokenize{common/02-turn:id1}}\label{\detokenize{common/02-turn::doc}}
この章では、ゲームの始め方やターン進行について説明します。


\subsection{ゲームの始め方}
\label{\detokenize{common/02-turn:id2}}\begin{quote}

次の手順でゲームを始めましょう。
\begin{enumerate}
\sphinxsetlistlabels{\arabic}{enumi}{enumii}{}{.}%
\item {} 
デッキをよく切る。

\item {} 
デッキより7枚引き手札にする。

\item {} 
両者デッキの一番上を表にする。

\item {} 
大きい数字のプレイヤーが先攻。数字については、 {\hyperref[\detokenize{common/01-base:cardrank}]{\sphinxcrossref{\DUrole{std,std-ref}{トランプの数字}}}} 参照。

\item {} 
数字が同じ場合、さらにデッキの一番上を表にし同様のルールで比べる。

\item {} 
表にしたカードを墓地へ移す。

\item {} 
先攻プレイヤーはデッキより1枚引き手札に加えターンを開始する。

\end{enumerate}
\end{quote}


\subsection{自分のターンの始め方}
\label{\detokenize{common/02-turn:id3}}
先攻が決まったらいよいよターン開始です。
自分のターンが来たら毎回次の手順を行いターンを開始して下さい。
\begin{quote}
\begin{enumerate}
\sphinxsetlistlabels{\arabic}{enumi}{enumii}{}{.}%
\item {} 
自分の場にあるドライブ状態のキャラクターをチャージします。

\item {} 
デッキからカードを1〜2枚引き手札に加えます。

\end{enumerate}
\begin{itemize}
\item {} 
先攻初回ターンはゲームの始め方で1枚引いているため、引くことができません。

\item {} 
1枚目を見てから2枚目を引いても構いません。

\end{itemize}
\end{quote}


\subsection{自分のターンにできること}
\label{\detokenize{common/02-turn:id4}}
自分のターンではさまざまな「アクション」とよばれる行動を起こすことができます。
アクションは自分のターンに何度でも起こすことができ、
アクションを起こすことで対戦相手にダメージを与えるなどしてゲームに勝つことができます。
アクションには主に次の種類があります。
\begin{quote}

展開系:キャラクターを場に出したり戻したりする。

攻撃系:対戦相手に攻撃を行う。

魔法系:兵士の数字を上げるなどの魔法。

ターン制御系:ターン終了の「エンド」アクション等のターン制御。
\end{quote}


\subsection{ターンの流れ}
\label{\detokenize{common/02-turn:id5}}
「自分のターンの始め方」、
「自分のターンにできること」をまとめると次のような流れになります。(\hyperref[\detokenize{common/02-turn:turn}]{図 \ref{\detokenize{common/02-turn:turn}}})
「自分のターンの始め方」で行っているキャラクターをチャージしたり、
デッキからカードを引いたりする行動もアクションとして定義されています。

\begin{figure}[htbp]
\centering

\noindent\sphinxincludegraphics{{turn}.pdf}
\end{figure}


\subsection{まとめ}
\label{\detokenize{common/02-turn:id6}}
ゲームの始め方、ターンの流れを説明しました。ポイントとなる部分をまとめます。
\begin{itemize}
\item {} 
ゲームを始めるときは7枚引いて手札にし、デッキの一番上をめくって大きい方が先攻

\item {} 
先攻は1ドローしてターンを開始、後攻からは2ドローしてターンを開始

\item {} 
ターンを終えたいときは、「エンド」アクションを起こす

\end{itemize}


\section{アクションの基本}
\label{\detokenize{common/03-action:id1}}\label{\detokenize{common/03-action::doc}}
この章では、アクションの基本について説明します。
アクションとはプレイヤーの行動を示しており、
相手を攻撃することやターンを終了することもアクションとして定義されています。

BlackPokerではアクションを起こした後に、
対戦相手が割込んでアクションを起こすことができるため、複雑なルールとなっています。
ルールは、 {\hyperref[\detokenize{common/03-action::doc}]{\sphinxcrossref{\DUrole{doc}{アクションの基本}}}} と {\hyperref[\detokenize{common/05-action_detail::doc}]{\sphinxcrossref{\DUrole{doc}{アクションの詳細}}}} に分けて説明ます。


\subsection{アクションが持つ項目}
\label{\detokenize{common/03-action:id2}}
アクションが持つ項目について説明します。
凡例の「サンプル」アクションを見てみましょう。(\hyperref[\detokenize{common/03-action:action-sample}]{図 \ref{\detokenize{common/03-action:action-sample}}})

\begin{figure}[htbp]
\centering
\capstart

\noindent\sphinxincludegraphics{{action-sample}.pdf}
\caption{サンプルアクション}\label{\detokenize{common/03-action:id32}}\label{\detokenize{common/03-action:action-sample}}\end{figure}
\begin{description}
\item[{アクション名}] \leavevmode
アクションの名称を示します。

\item[{キーカード}] \leavevmode
アクションの核となるカードを示します。
キーカードは★を使って表記します。
凡例の場合、手札からコストとは別に{\normalsize $\heartsuit$} A〜10に該当するカードを1枚
キーカードとして使用します。

\item[{特記事項}] \leavevmode
特記事項は※を使って表記し、その他の項目では書き表せない条件を示します。

\item[{対象}] \leavevmode
効果を発揮する対象を示します。

\item[{即時効果/通常効果}] \leavevmode
発揮する効果の内容を示します。

\item[{コスト}] \leavevmode
アクションを起こすのに必要な対価です。
コストは$を使って表記し、コストの支払いはアクションを起こすプレイヤーが行います。コストの種類は {\hyperref[\detokenize{common/03-action:cost}]{\sphinxcrossref{\DUrole{std,std-ref}{コストの種類}}}} で説明します。

\item[{タイミング}] \leavevmode
アクションを起こせる時を示します。
タイミングについては、 {\hyperref[\detokenize{common/03-action:action-chance}]{\sphinxcrossref{\DUrole{std,std-ref}{アクションが起こせる時と効果を発揮する時}}}} で説明します。

\item[{タイプ}] \leavevmode
アクションの種類を表します。アクション名の後に括弧書きで記載します。

\end{description}


\subsubsection{コストの種類}
\label{\detokenize{common/03-action:cost}}\label{\detokenize{common/03-action:id3}}
アクションによって支払うコストが異なります。
コストには次の種類があり、それぞれ支払い方が異なります。(\hyperref[\detokenize{common/03-action:table-cost}]{表 \ref{\detokenize{common/03-action:table-cost}}})


\begin{savenotes}\sphinxattablestart
\centering
\sphinxcapstartof{table}
\sphinxthecaptionisattop
\sphinxcaption{コストの種類}\label{\detokenize{common/03-action:id33}}\label{\detokenize{common/03-action:table-cost}}
\sphinxaftertopcaption
\begin{tabulary}{\linewidth}[t]{|T|T|}
\hline
\sphinxstyletheadfamily 
表記(名称)
&\sphinxstyletheadfamily 
対価
\\
\hline
B (Bulwark)
&
防壁をドライブする
\\
\hline
L (Life)
&
1点ダメージを受ける
\\
\hline
D (Discard)
&
手札を1枚捨てる
\\
\hline
S (Sacrifice)
&
キャラクター1体を墓地に移す
\\
\hline
\end{tabulary}
\par
\sphinxattableend\end{savenotes}

たとえばコストが \sphinxstylestrong{「\$BL」} の場合、自分の場にいるチャージ状態の防壁を1体ドライブし、1点ダメージを受けることでコストが支払われたことになります。


\subsubsection{即時効果と通常効果}
\label{\detokenize{common/03-action:id4}}
アクションの効果には、即時効果と通常効果の2種類があります。
即時効果と通常効果は次のような違いがあります。
\begin{description}
\item[{即時効果}] \leavevmode
アクションを起こしたらすぐに効果を発揮します。

\item[{通常効果}] \leavevmode
アクションを起こした際にステージと呼ばれる領域に置かれます。効果を発揮するタイミングについては、
{\hyperref[\detokenize{common/03-action:action-exe}]{\sphinxcrossref{\DUrole{std,std-ref}{アクションの効果を発揮}}}} 〜 {\hyperref[\detokenize{common/03-action:action-flow}]{\sphinxcrossref{\DUrole{std,std-ref}{効果を発揮するまでの流れ}}}} で説明します。

\end{description}

即時効果の場合は対戦相手が割込んでアクションを起こせないため単純ですが、
通常効果の場合は割込みを許容するため、
起こしたアクションを一旦ステージに置いた後、
効果を発揮する流れになります。
\begin{description}
\item[{ステージとは}] \leavevmode
ステージとは、アクションの解決順を整理するために使う領域です。
後入れ先出し方式で最後に積まれたアクションから順に解決されていきます。

\item[{ステージを用いた通常効果の効果を発揮するまでの流れについては、}] \leavevmode
{\hyperref[\detokenize{common/03-action:action-chance}]{\sphinxcrossref{\DUrole{std,std-ref}{アクションが起こせる時と効果を発揮する時}}}} にて仕組みを説明し、

\end{description}

{\hyperref[\detokenize{common/03-action:action-flow}]{\sphinxcrossref{\DUrole{std,std-ref}{効果を発揮するまでの流れ}}}} にて例を用いて流れを説明します。


\paragraph{記載されていないアクションの項目}
\label{\detokenize{common/03-action:id5}}
アクションによっては記載されていない項目もあります。
記載されていない項目は無視して構いません。
たとえばコスト項目がなければコストを支払う必要はありません。


\subsection{アクションの起こし方}
\label{\detokenize{common/03-action:action-howto}}\label{\detokenize{common/03-action:id6}}
次の手順でアクションを起こします。
\begin{enumerate}
\sphinxsetlistlabels{\arabic}{enumi}{enumii}{}{.}%
\item {} 
起こすアクションを対戦相手に伝える。

\item {} 
アクションに応じたコストを支払う。

\item {} 
必要なら手札からキーカードを出す。

\item {} 
対象の指定が必要な場合、対象を指定する。

\end{enumerate}

「サンプル」アクションを起こす例を見てみましょう。(\hyperref[\detokenize{common/03-action:action-sample2}]{図 \ref{\detokenize{common/03-action:action-sample2}}})

\begin{figure}[htbp]
\centering
\capstart

\noindent\sphinxincludegraphics{{action-sample2}.pdf}
\caption{アクションを起こす例}\label{\detokenize{common/03-action:id34}}\label{\detokenize{common/03-action:action-sample2}}\end{figure}


\subsubsection{アクションを起こすときの注意点}
\label{\detokenize{common/03-action:id7}}
アクションを起こす際の注意点を説明します。


\paragraph{対象を指定しないでアクションを起こせるか?}
\label{\detokenize{common/03-action:id8}}
「サンプル」アクションのように対象を指定するアクションがあります。
「対象」項目がある場合、記載された条件を満たした対象を指定できなければ、
そのアクションを起こすことはできません。


\paragraph{アクションを対象とするアクションは自身を対象にできるか?}
\label{\detokenize{common/03-action:id9}}
アクションは、自分自身を対象とすることはできません。
そのため、「カウンター」アクションのようにアクションを対象とするアクションは
自身を対象とすることはできません。


\subsubsection{アクションを起こした後}
\label{\detokenize{common/03-action:id10}}
アクションを起こした後、次のことを行います。
\begin{enumerate}
\sphinxsetlistlabels{\arabic}{enumi}{enumii}{}{.}%
\item {} 
即時効果の場合、アクションの効果を発揮します。 {\hyperref[\detokenize{common/03-action:action-chance}]{\sphinxcrossref{\DUrole{std,std-ref}{アクションが起こせる時と効果を発揮する時}}}} にて説明します。

\item {} 
誘発の有無判定を行います。 {\hyperref[\detokenize{common/05-action_detail::doc}]{\sphinxcrossref{\DUrole{doc}{アクションの詳細}}}} にて説明します。

\item {} 
通常効果のアクションを起こした場合、自動的にパスが発生します。パスについては {\hyperref[\detokenize{common/03-action:action-chance}]{\sphinxcrossref{\DUrole{std,std-ref}{アクションが起こせる時と効果を発揮する時}}}} にて説明します。

\end{enumerate}


\subsection{アクションの効果を発揮}
\label{\detokenize{common/03-action:action-exe}}\label{\detokenize{common/03-action:id11}}
アクションは起こした後、効果を発揮します。
アクションが効果を発揮するとは、アクションの項目にある
「即時効果/通常効果」に記載された内容を実行することです。
このことを「アクションを解決する」といいます。


\subsubsection{アクションが効果を発揮するときの注意点}
\label{\detokenize{common/03-action:id12}}
アクションを解決するときに、次の注意点があります。
\begin{enumerate}
\sphinxsetlistlabels{\arabic}{enumi}{enumii}{}{.}%
\item {} 
アクションを解決する際に指定した対象が「対象」項目に記載された条件を満たしていない場合、効果は発揮せず解決されます。

\item {} 
効果の中に実行不可能な部分がある場合、可能な部分のみ実行します。

\end{enumerate}

それぞれ、例を用いて説明します。


\paragraph{「アクションを解決する際に指定した対象が「対象」項目に記載された条件を満たしていない場合、効果は発揮せず解決されます」とは?}
\label{\detokenize{common/03-action:id13}}
対象を指定するアクションが効果を発揮しようとした時に
対象が存在していない場合、効果を発揮する対象を失うため効果が発揮されず
アクションが解決されます。

たとえば兵士に対して「アップ」アクションを起こし、対応して「ダウン」
アクションを起こされました。
「ダウン」の方が先に解決されるため、「アップ」を解決する時には
兵士が墓地に移っていたとします。その場合、「アップ」アクションは効果を発揮せず解決されます。


\paragraph{「効果の中に実行不可能な部分がある場合、可能な部分のみ実行します」とは?}
\label{\detokenize{common/03-action:id14}}
効果の内容をできる限り実行するということです。

たとえば、デッキの枚数が残1枚の時に5点のダメージを受けたとします。
デッキは1枚しかないので5点ダメージを受けることはできませんが、
1点までなら受けることが可能なため、
この場合1点のダメージを受けることになります。


\subsubsection{アクションが効果を発揮した後}
\label{\detokenize{common/03-action:id15}}
アクションが効果を発揮した後、次のことを行います。
\begin{quote}
\begin{enumerate}
\sphinxsetlistlabels{\arabic}{enumi}{enumii}{}{.}%
\item {} 
キーカードを墓地に移す

\item {} 
勝敗判定

\item {} 
誘発の有無判定

\end{enumerate}

誘発の有無判定については {\hyperref[\detokenize{common/05-action_detail::doc}]{\sphinxcrossref{\DUrole{doc}{アクションの詳細}}}} にて説明します。それ以外の「キーカードを墓地に移す」「勝敗判定」について説明します。
\end{quote}


\paragraph{「キーカードを墓地に移す」とは?}
\label{\detokenize{common/03-action:keycard-gy}}\label{\detokenize{common/03-action:id16}}
1つのアクションが解決された後そのアクションをステージから取り除き、キーカードを墓地に移します。
ただし効果によってキーカードを場に出した場合や手札に戻した場合、
そのカードを移す先が明確になっているため、墓地には移しません。


\paragraph{「勝敗判定」とは?}
\label{\detokenize{common/03-action:id17}}
デッキを確認し0枚の場合そのプレイヤーは敗北となります。両プレイヤーのデッキが0枚の場合、引き分けとなります。


\subsection{アクションが起こせる時と効果を発揮する時}
\label{\detokenize{common/03-action:action-chance}}\label{\detokenize{common/03-action:id18}}
アクションの起こし方、効果を発揮する方法をみてきました。
アクションはいつでも起こせるわけではなく、チャンスを持っている時だけ起こすことができます。
また、アクションの「タイミング」項目によって起こせる時が異なります。

ここでは、チャンス、タイミングの順で説明していきます。


\subsubsection{チャンスとは}
\label{\detokenize{common/03-action:id19}}
アクションチャンスの略で、プレイヤーがアクションを起こす機会のことをいいます。
チャンスを持っていないとアクションを起こすことはできません。
また、チャンスはパスすることで対戦相手にチャンスを移すことができます。


\paragraph{チャンスの遷移と通常効果の発揮}
\label{\detokenize{common/03-action:id20}}
チャンスの遷移と通常効果の発揮は次のようになります。
\begin{itemize}
\item {} 
ゲーム開始時は先攻のプレイヤーがチャンスを持っています。

\item {} 
通常効果を持つアクションを起こした場合パスが自動的に発生し、対戦相手にチャンスが移ります。

\item {} 
即時効果を持つアクションを起こした場合、チャンスは移りません。

\item {} 
パスをすると対戦相手にチャンスを移すことができます。

\item {} 
パスが2回続いた場合、最後に起こしたアクションの通常効果を発揮します。

\item {} 
通常効果を発揮する時はチャンスは誰も持っていない状態となり、通常効果を発揮した後チャンスはターンを持っているプレイヤーに移ります。

\end{itemize}

図にまとめると次のようになります。(\hyperref[\detokenize{common/03-action:chance}]{図 \ref{\detokenize{common/03-action:chance}}})

\begin{figure}[htbp]
\centering
\capstart

\noindent\sphinxincludegraphics{{chance}.pdf}
\caption{チャンスの遷移}\label{\detokenize{common/03-action:id35}}\label{\detokenize{common/03-action:chance}}\end{figure}


\subsubsection{タイミング}
\label{\detokenize{common/03-action:id21}}
アクションの持つ項目のタイミングは、「メイン」と「クイック」の2種類あります。
メインは自分のターンにしか起こせないアクションで、
クイックは相手のアクションに割込んで起こすこともできるアクションです。
もう少し詳しく説明します。


\subsubsection{メイン}
\label{\detokenize{common/03-action:id22}}
自ターンかつステージが空のときにこのアクションを起こすことができます。
つまりステージの一番下にしか置けないことになります。

条件をまとめると次のようになります。
\begin{itemize}
\item {} 
チャンスを持っている

\item {} 
自分のターン

\item {} 
ステージが空

\end{itemize}


\paragraph{クイック}
\label{\detokenize{common/03-action:id23}}
いつでもこのアクションを起こすことができます。
いつでも起こせるため、アクションをステージに積み重ねることができます。

条件をまとめると次のようになります。
\begin{itemize}
\item {} 
チャンスを持っている

\end{itemize}

クイックはメインと異なり、チャンスを持っていれば自分の
ターンでなくても起こせるのが特徴です。


\paragraph{タイミングとステージ}
\label{\detokenize{common/03-action:id24}}
ステージと合わせて説明すると次のようになります。(\hyperref[\detokenize{common/03-action:timing}]{図 \ref{\detokenize{common/03-action:timing}}})

\begin{figure}[htbp]
\centering
\capstart

\noindent\sphinxincludegraphics{{timing}.pdf}
\caption{タイミングとステージ}\label{\detokenize{common/03-action:id36}}\label{\detokenize{common/03-action:timing}}\end{figure}

タイミングがメインの場合ステージの一番下にしか置けず、
クイックの場合積み上げることができます。

アクションを起こす際は、
タイミングが合っていないと起こせないので気をつけましょう。


\subsection{効果を発揮するまでの流れ}
\label{\detokenize{common/03-action:action-flow}}\label{\detokenize{common/03-action:id25}}
アクションが起きてから、効果を発揮するまでの基本的なルールを説明してきました。
これまでの説明を踏まえて、効果が発揮するまでの流れを例を用いて説明します。


\subsubsection{通常効果が発揮されるまでの流れ}
\label{\detokenize{common/03-action:id26}}
通常効果が発揮されるまでの流れを例を用いて説明します。
通常効果は次の手順を行うことでアクションが効果を発揮します。
\begin{enumerate}
\sphinxsetlistlabels{\arabic}{enumi}{enumii}{}{.}%
\item {} 
アクションを起こす

\item {} 
対戦相手もアクションを起こすか聞く

\item {} 
効果を発揮

\item {} 
キーカードを墓地に移す

\end{enumerate}


\paragraph{手順1. アクションを起こす}
\label{\detokenize{common/03-action:id27}}
通常効果のアクションを起こすと、ステージという
アクションの効果を発揮する順番を整理する
場所に置かれます。
アクションを起こした時のステージの状態を図で見てみましょう。(\hyperref[\detokenize{common/03-action:action-begin}]{図 \ref{\detokenize{common/03-action:action-begin}}} 
aise0.2exhbox{	extcircled{scriptsize{1}}} )


\paragraph{手順2. 対戦相手もアクションを起こすか聞く}
\label{\detokenize{common/03-action:id28}}
通常効果のアクションを起こすと対戦相手にチャンスが移り、
起こしたアクションに割込んでアクションを起こせるようになります。
対戦相手がアクションを起こす場合も、 {\hyperref[\detokenize{common/03-action:action-howto}]{\sphinxcrossref{\DUrole{std,std-ref}{アクションの起こし方}}}} と同じ手順でアクションを起こします。(\hyperref[\detokenize{common/03-action:action-begin}]{図 \ref{\detokenize{common/03-action:action-begin}}} 
aise0.2exhbox{	extcircled{scriptsize{2}}} )

自分が通常効果のアクションを起こした後、対戦相手に「とおりますか?」
(自分の起こしたアクションに割込みしないで通してくれますか?)
など対戦相手がアクションを起こすか聞くと親切でしょう。

このように互いに起こすアクションがなくなるまで、
チャンスをやりとりしてアクションを起こします。
このやりとりを、アクションを起こさずにパスが2回続くまで繰り返します。


\paragraph{手順3. 効果を発揮}
\label{\detokenize{common/03-action:id29}}
両プレイヤーの起こすアクションがなくなったところで、
最後にステージに積まれたアクションの効果を発揮させます。
(\hyperref[\detokenize{common/03-action:action-begin}]{図 \ref{\detokenize{common/03-action:action-begin}}} 
aise0.2exhbox{	extcircled{scriptsize{3}}} )

\begin{figure}[htbp]
\centering
\capstart

\noindent\sphinxincludegraphics{{action-begin}.pdf}
\caption{通常効果が発揮されるまでの流れ}\label{\detokenize{common/03-action:id37}}\label{\detokenize{common/03-action:action-begin}}\end{figure}


\paragraph{手順4. キーカードを墓地に移す}
\label{\detokenize{common/03-action:id30}}
アクションで用いたキーカードを墓地に移します。
詳細については、
{\hyperref[\detokenize{common/03-action:action-exe}]{\sphinxcrossref{\DUrole{std,std-ref}{アクションの効果を発揮}}}} の
{\hyperref[\detokenize{common/03-action:keycard-gy}]{\sphinxcrossref{\DUrole{std,std-ref}{「キーカードを墓地に移す」とは?}}}}
を参照して下さい。

\begin{sphinxadmonition}{note}{【コラム】パスの類似語}

 チャンスを相手に渡す時に「パス」と言うルールとなっていますが、
通常効果のアクションを起こした場合自動的にパスが発生するため、
「パス」と言わないことがほとんどです。
「パス」と言う場面は、自分がアクションを起こさないで相手にチャンスを渡すときに主に使います。
このときに「パス」の代わりに次の言葉を使うことがあります。
\begin{itemize}
\item {} 
「スルーします」もしくは、「スルーで」

\item {} 
「通しで」

\item {} 
「OKです」

\item {} 
「対応しません」もしくは、「対応ありません」

\end{itemize}

どの言葉も起こしたアクションに対して、対応して何もしないことを示しています。
これらの言葉を聞いたときは、「パス」の意味だと捉えて下さい。
\end{sphinxadmonition}

\begin{sphinxadmonition}{note}{【コラム】なぜステージやチャンスといった複雑な手順があるか}

 理由は、「対戦相手の行動に対して割込むことを可能にするため」です。
いわばずる込みができます。たとえばこんな感じです。

\begin{DUlineblock}{0em}
\item[] Aくん「この兵士アップします。」
\item[] Bさん「その前にこの兵士ダウンします。」
\item[] Aくん「じゃあそのダウンをカウンターします。」
\item[] Bさん「それをさらにカウンターします。」
\item[] Aくん「・・・(泣)」
\item[] Bさん「(どやっ!)
\end{DUlineblock}

 カードゲームの醍醐味はいかに相手の行動に対して意表をついた行動ができるかだと
個人的には思っています。

 行動の割込みを可能にするため、カードゲームで用いられている手法を参考にし、
さらにシンプルにしてBlackPokerに取り入れました。

 カードゲームをやったことがある方は理解しやすいルールですが、
今までカードゲームをやったことがない方には少し難しいルールかもしれません。
ですが、ここで理解してしまえば他のカードゲームにも恐らく応用が利くので
頑張って理解してみて下さい。
\end{sphinxadmonition}


\subsection{まとめ}
\label{\detokenize{common/03-action:id31}}
この章では、アクションの基本について説明しました。
ポイントをまとめると次のようになります。
\begin{itemize}
\item {} 
アクションには、コスト、タイミングなどの項目がある

\item {} 
アクションは起こしてもすぐに効果を発揮しない

\item {} 
アクションを起こすにはチャンスを持っている必要がある

\item {} 
アクションはステージを経由して効果を発揮する

\end{itemize}

\begin{sphinxadmonition}{note}{【コラム】魔法系アクションの覚え方}

魔法系アクションは、スートのイメージに紐付けて作っています。
スートのイメージは大体こんな感じです。


\begin{savenotes}\sphinxattablestart
\centering
\begin{tabulary}{\linewidth}[t]{|T|T|}
\hline

{\normalsize $\spadesuit$} 
&
死、恐怖、剣、統制
\\
\hline
{\normalsize $\heartsuit$} 
&
愛、勇気、盾、信仰
\\
\hline
{\normalsize $\diamondsuit$} 
&
金、調整、槍、商売
\\
\hline
{\normalsize $\clubsuit$} 
&
知、奇策、杖、生産
\\
\hline
\end{tabulary}
\par
\sphinxattableend\end{savenotes}

{\normalsize $\diamondsuit$} の速攻魔法はお金で振り向かせるから「ツイスト」

{\normalsize $\heartsuit$} の速攻魔法は勇気が出そうだから「アップ」

みたいな感じで覚えられるかな??と思います。
\end{sphinxadmonition}


\section{キャラクターと能力}
\label{\detokenize{common/04-character:id1}}\label{\detokenize{common/04-character::doc}}
この章では、場に出て闘うキャラクターと能力について説明します。


\subsection{キャラクターとは}
\label{\detokenize{common/04-character:id2}}
キャラクターとは、場に存在する兵士や防壁のことを指します。

キャラクターは1枚のカードで1体を表すこともあれば、
複数枚で1体を表すこともあります。(\hyperref[\detokenize{common/04-character:character}]{図 \ref{\detokenize{common/04-character:character}}})

\begin{figure}[htbp]
\centering
\capstart

\noindent\sphinxincludegraphics{{character}.pdf}
\caption{キャラクターの例}\label{\detokenize{common/04-character:id10}}\label{\detokenize{common/04-character:character}}\end{figure}


\subsection{キャラクターのもつ項目}
\label{\detokenize{common/04-character:id3}}
キャラクターのもつ項目について説明します。
凡例のキャラクター「一般兵」を見てみましょう。(\hyperref[\detokenize{common/04-character:character-sample}]{図 \ref{\detokenize{common/04-character:character-sample}}})

\begin{figure}[htbp]
\centering
\capstart

\noindent\sphinxincludegraphics{{character-sample}.pdf}
\caption{一般兵}\label{\detokenize{common/04-character:id11}}\label{\detokenize{common/04-character:character-sample}}\end{figure}
\begin{description}
\item[{キャラクター名}] \leavevmode
キャラクターの名称を示します。

\item[{タイプ}] \leavevmode
キャラクターのタイプを示します。タイプは兵士と防壁の2種類が存在します。

\item[{キーカード}] \leavevmode
キャラクターを示すカードが記載されています。複数のカードで1体のキャラクターを示す場合もあります。

\item[{能力}] \leavevmode
キャラクターが持っている能力を記載しています。たとえば英雄の「世代交代」や、魔術士の「魔力増加」があげられます。能力については、 {\hyperref[\detokenize{common/04-character:abi}]{\sphinxcrossref{\DUrole{std,std-ref}{能力とは}}}} で説明します。

\end{description}


\subsection{数字について}
\label{\detokenize{common/04-character:id4}}
トランプの数字は、キャラクターの強さを示します。
基本はカードに記載された数字を示しますが、魔法などのアクションを使うことで
加算したり減算されたりします。

\begin{sphinxadmonition}{note}{【補足】複数枚で1体となるキャラクターが防壁になったら?}

 アクションの効果で兵士を防壁にすることがあります。
防壁は1枚で1体のキャラクターであるため、
複数枚からなるキャラクターが防壁となった場合、
複数体の防壁となります。

 なお、複数枚からなるキャラクターが
墓地や手札に移った場合、
1体のキャラクターとして
扱うため複数枚合わせて移します。
チャージ状態、ドライブ状態となった場合も同様に1体のキャラクター
として扱います。
\end{sphinxadmonition}


\subsection{能力とは}
\label{\detokenize{common/04-character:abi}}\label{\detokenize{common/04-character:id5}}
能力とはアクションの効果とは異なる概念で、
アクションを起こすことができたり、
アクションを誘発したりすることができます。
代表的な能力は、英雄の「世代交代」や魔術士の「魔力増加」があげられます。

能力を持つことができるのは、キャラクターやプレイヤー、エクストラフォーマットの切札
です。


\subsubsection{能力の種類}
\label{\detokenize{common/04-character:id6}}
能力には次の種類があります。


\paragraph{常在型能力}
\label{\detokenize{common/04-character:id7}}
能力が有効である場合、継続的に発揮される能力

例:魔力増加


\paragraph{誘発型能力}
\label{\detokenize{common/04-character:id8}}
能力が有効である間に何かの契機でアクションを起こす能力

例:世代交代


\subsection{まとめ}
\label{\detokenize{common/04-character:id9}}
この章では、キャラクターと能力について説明しました。
キャラクターのもつ能力という概念は {\hyperref[\detokenize{common/05-action_detail::doc}]{\sphinxcrossref{\DUrole{doc}{アクションの詳細}}}} でも登場するので、
理解しておきましょう。

ポイントをまとめると次のようになります。
\begin{itemize}
\item {} 
キャラクターの項目はキャラクター名、キーカード、能力などがある

\item {} 
キャラクターの数字(強さ)は魔法などによって一時的に増減する

\item {} 
能力はアクションの効果ではなく、アクションを誘発したりする概念

\item {} 
能力をもつことができるのは、キャラクター、プレイヤー、切札

\end{itemize}

\begin{sphinxadmonition}{note}{【補足】プレイヤーの能力とは?}

 BlackPokerでは、ゲームを始める前にフォーマット
を決めます。
実はこれはゲームを始める前にプレイヤーが使用できる能力を
選択していることになります。

 たとえばライト版をプレイする場合、ライト版のアクションリストに定義されている
アクションを起こせる能力をプレイヤーが持ってゲームを始めることになります。
\end{sphinxadmonition}


\section{アクションの詳細}
\label{\detokenize{common/05-action_detail:id1}}\label{\detokenize{common/05-action_detail::doc}}
この章では、アクションの基本を踏まえた詳細なルールを説明します。


\subsection{コントローラーとオーナー}
\label{\detokenize{common/05-action_detail:id2}}
アクションと場にいるキャラクターには持ち主がいます。
持ち主の決まり方について説明します。


\subsubsection{コントローラー}
\label{\detokenize{common/05-action_detail:id3}}
アクションを起こしたプレイヤーをそのアクションのコントローラーといいます。
アクションの効果はコントローラー目線で解決されます。

たとえば、
「兵士として場に出す。」というアクションの効果があるとします。
ここでは、どの場に出すか明言されていないのでこのアクションのコントローラーの場に
兵士を出すことになります。


\subsubsection{オーナー}
\label{\detokenize{common/05-action_detail:id4}}
オーナーとは、キャラクターや切札などの持ち主のプレイヤーのことを指します。

オーナーは対象が出ている場所により決定します。
つまり、
自分の場にいる兵士のオーナーは自分ということになります。

オーナーの概念は次のものに当てはまります。
\begin{itemize}
\item {} 
キャラクター

\item {} 
切札

\item {} 
デッキ

\item {} 
墓地

\end{itemize}

オーナーの概念が当てはまるこれらを「コンポーネント」といいます。
これは、コアルールのコンポーネントと同じ意味となります。( {\hyperref[\detokenize{core/core:component}]{\sphinxcrossref{\DUrole{std,std-ref}{コンポーネント}}}} )


\subsection{誘発}
\label{\detokenize{common/05-action_detail:id5}}
能力の中にはアクションを誘発する能力があります。
誘発とは、特定の条件を満たした時にアクションが自動的に起こることを指します。


\subsubsection{誘発の条件と誘発するアクション}
\label{\detokenize{common/05-action_detail:id6}}
英雄の世代交代を誘発する能力を見てみましょう。

\begin{sphinxVerbatim}[commandchars=\\\{\}]
場から墓地に行く場合、世代交代アクションを誘発する。
\end{sphinxVerbatim}

この能力から誘発する条件は
「場から墓地に行く場合」、
誘発するアクションは「世代交代」だということがわかります。

誘発する条件は「〜の場合」、「〜時」などで記載されており、
誘発するアクションは「〜を誘発する」と記載されています。


\paragraph{誘発したアクションのコントローラー}
\label{\detokenize{common/05-action_detail:id7}}
誘発したアクションのコントローラーは、
誘発する条件を決めているコンポーネントのオーナーになります。

たとえば英雄が墓地に移った場合、世代交代アクションを誘発します。
この世代交代アクションのコントローラーは、
英雄のオーナーということになります。


\subsubsection{誘発したアクションの解決}
\label{\detokenize{common/05-action_detail:id8}}
アクションの解決などによって複数のアクションを誘発することが
あります。
複数のアクションを誘発した場合に、
アクションを解決する順番を説明します。


\paragraph{誘発の有無判定}
\label{\detokenize{common/05-action_detail:id9}}
誘発の有無は次のときに確認します。
\begin{itemize}
\item {} 
アクションを起こした後

\item {} 
アクションを解決した後

\end{itemize}

アクションを起こした後は、
コストの支払いによってアクションを誘発するときがあります。


\paragraph{アクションの解決順}
\label{\detokenize{common/05-action_detail:action-resolves}}\label{\detokenize{common/05-action_detail:id10}}
誘発の有無判定で有りとなった場合、次のような順でアクションが解決されます。
誘発したアクションに即時効果のアクションが含まれている場合が
あるため、順番が複雑になっています。

\begin{sphinxadmonition}{note}{\label{\detokenize{common/05-action_detail:id11}}課題:}
コアルールへの参照を追加
\end{sphinxadmonition}
\begin{enumerate}
\sphinxsetlistlabels{\arabic}{enumi}{enumii}{}{.}%
\item {} 
即時効果の効果を発揮

\end{enumerate}
\begin{description}
\item[{1\sphinxhyphen{}1.}] \leavevmode
誘発したアクションのうち
ターンを持っているプレイヤーがコントローラーかつ
即時効果のアクションの効果を発揮する。
条件に該当するアクションが複数ある場合、
該当する全てのアクションをコントローラーの好きな順で効果を発揮する。

\item[{1\sphinxhyphen{}2.}] \leavevmode
誘発したアクションのうち
ターンを持っていないプレイヤーがコントローラーかつ
即時効果のアクションの効果を発揮する。
条件に該当するアクションが複数ある場合、
該当する全てのアクションをコントローラーの好きな順で効果を発揮する。

\item[{1\sphinxhyphen{}3.}] \leavevmode
1\sphinxhyphen{}1,1\sphinxhyphen{}2で発揮されたアクションの効果によって
誘発したアクションがある場合、1\sphinxhyphen{}1に戻る。

\end{description}
\begin{enumerate}
\sphinxsetlistlabels{\arabic}{enumi}{enumii}{}{.}%
\setcounter{enumi}{1}
\item {} 
誘発したアクションをステージに置く

\end{enumerate}
\begin{description}
\item[{2\sphinxhyphen{}1.}] \leavevmode
誘発したアクションのうち
ターンを持っているプレイヤーがコントローラーかつ
通常効果のアクションをステージに置く。
条件に該当するアクションが複数ある場合、
該当する全てのアクションをコントローラーの好きな順でステージに置く。

\item[{2\sphinxhyphen{}2.}] \leavevmode
誘発したアクションのうち
ターンを持っていないプレイヤーがコントローラーかつ
通常効果のアクションをステージに置く。
条件に該当するアクションが複数ある場合、
該当する全てのアクションをコントローラーの好きな順でステージに置く。

\end{description}

図にすると次のようになります。(\hyperref[\detokenize{common/05-action_detail:trigger}]{図 \ref{\detokenize{common/05-action_detail:trigger}}})

\begin{figure}[htbp]
\centering
\capstart

\noindent\sphinxincludegraphics{{trigger}.pdf}
\caption{アクションが誘発した時の流れ}\label{\detokenize{common/05-action_detail:id17}}\label{\detokenize{common/05-action_detail:trigger}}\end{figure}

同様の説明は、コアルールの誘発にも記載されています。( \DUrole{xref,std,std-ref}{trigger\sphinxhyphen{}check} )

\begin{sphinxadmonition}{note}{【補足】即時効果アクションを起こしたときの誘発}

 即時効果のアクションを起こしたときに、
コストとしてキャラクターを墓地に移すなどすると
即時効果アクションを誘発することがあります。
その場合のアクションは次の順で解決されます。
\begin{enumerate}
\sphinxsetlistlabels{\arabic}{enumi}{enumii}{}{.}%
\item {} 
起こした即時効果アクション

\item {} 
誘発した即時効果アクション

\end{enumerate}

 理由は、即時効果アクションが効果を発揮するのは、
アクションを起こす中に含まれているからです。
誘発の有無を判定するのは、アクションを起こした後
のため、起こした即時効果アクションの方が先に解決されることになります。
\end{sphinxadmonition}


\subsection{その他補足事項}
\label{\detokenize{common/05-action_detail:id12}}

\subsubsection{防壁の置き方}
\label{\detokenize{common/05-action_detail:id13}}
防壁を場に出すときは次のルールにしたがって場に出して下さい。(\hyperref[\detokenize{common/05-action_detail:set-bulwork}]{図 \ref{\detokenize{common/05-action_detail:set-bulwork}}})
\begin{itemize}
\item {} 
防壁を置く時はデッキ側に詰めて置いて下さい。

\item {} 
防壁の左右の入れ替えは行わないでください。

\end{itemize}

\begin{figure}[htbp]
\centering
\capstart

\noindent\sphinxincludegraphics{{set-bulwork}.pdf}
\caption{防壁の置き方}\label{\detokenize{common/05-action_detail:id18}}\label{\detokenize{common/05-action_detail:set-bulwork}}\end{figure}


\subsubsection{1ターンに1回制限}
\label{\detokenize{common/05-action_detail:id14}}
特記事項に「プレイヤーは1ターンに1回しかこのアクションを起こすことができない。」と記載されているアクションは、
ターンを持っているプレイヤーが変わるまでの間に1回しか起こす
ことができません。

ターンを持っているプレイヤーが変わればまた起こすことができます。


\subsubsection{直接起こせないアクション}
\label{\detokenize{common/05-action_detail:id15}}
特記事項に「プレイヤーはこのアクションを直接起こすことが出来ない。」
と記載されているアクションは、
プレイヤーがチャンスを持っていても
アクションを起こすことができません。
また、この特記事項が記載されたアクションが何らかの起因で起きても、プレイヤーが起こした訳ではないためパスは自動的に発生せず、チャンスは移りません。


\subsection{まとめ}
\label{\detokenize{common/05-action_detail:id16}}
この章では、アクションの詳細を説明しました。
ポイントとなる部分をまとめます。
\begin{itemize}
\item {} 
オーナーはキャラクターの持ち主、コントローラーはアクションを起こした人

\item {} 
即時効果のアクションはアクションを起こした時に効果を発揮

\item {} 
同時に誘発した時の解決順は {\hyperref[\detokenize{common/05-action_detail:action-resolves}]{\sphinxcrossref{\DUrole{std,std-ref}{アクションの解決順}}}} を参照

\item {} 
アクションは解決する度に勝敗判定が行われる

\end{itemize}

即時効果と誘発は混乱しやすいため、わからなくなったらその都度ルールを確認しましょう。


\section{エクストラ}
\label{\detokenize{common/06-extra:extra}}\label{\detokenize{common/06-extra:id1}}\label{\detokenize{common/06-extra::doc}}
この章では、エクストラフォーマットについて説明します。

エクストラフォーマットはプロの拡張であるため、
プロからの追加部分のみ説明しています。
プロ版で慣れてから遊ぶとルールが分かりやすいです。


\subsection{エクストラとは}
\label{\detokenize{common/06-extra:id2}}
エクストラではプロ版のアクション、
キャラクターに加え切札を操作するアクションとそれに対応する
切札の能力を使うことができます。

切札を使うことで、プロより
さらにデッキ構築要素が増えたルールとなっています。


\subsection{切札とは}
\label{\detokenize{common/06-extra:id3}}
切札とは、切札領域に置かれたカードを示します。
具体的な切札の置き場所については、 {\hyperref[\detokenize{common/01-base:field-ex}]{\sphinxcrossref{\DUrole{std,std-ref}{プレイ中のカードの配置}}}} を参照して下さい。
切札には各々能力が割り当てられており、表にするとその能力を発揮します。
切札を操作するアクションは、「エクストラリスト」を参照して下さい。


\subsection{バージョンについて}
\label{\detokenize{common/06-extra:id4}}
エクストラには、バージョンが存在します。
対戦を開始する前に対戦相手とバージョンの確認をしましょう。


\subsubsection{版数との関係について}
\label{\detokenize{common/06-extra:id5}}
版数毎に使える切札の種類が異なります。
たとえば、第一版、第二版ではエクストラで遊ぶことはできません。
第三版以降は、次版が出るまでの間に公開された切札であれば
使用することができます。(\hyperref[\detokenize{common/06-extra:ver-ex}]{表 \ref{\detokenize{common/06-extra:ver-ex}}})


\begin{savenotes}\sphinxattablestart
\centering
\sphinxcapstartof{table}
\sphinxthecaptionisattop
\sphinxcaption{版数とエクストラのバージョン}\label{\detokenize{common/06-extra:id11}}\label{\detokenize{common/06-extra:ver-ex}}
\sphinxaftertopcaption
\begin{tabulary}{\linewidth}[t]{|T|T|}
\hline
\sphinxstyletheadfamily 
版数
&\sphinxstyletheadfamily 
エクストラのバージョン
\\
\hline
第一版
&
−
\\
\hline
第二版
&
−
\\
\hline
第三版
&
ex3.4.0 〜 ex3.10.0
\\
\hline
第四版
&
ex4.14.0 〜 ex4.22.0
\\
\hline
第五版
&
ex5.22.0 〜
\\
\hline
\end{tabulary}
\par
\sphinxattableend\end{savenotes}


\subsection{ゲームのはじめ方}
\label{\detokenize{common/06-extra:id6}}
エクストラでは、切札を置いてからゲームを始めます。
切札を置くルールは次のようになっています。(\hyperref[\detokenize{common/06-extra:trump}]{図 \ref{\detokenize{common/06-extra:trump}}})
\begin{itemize}
\item {} 
対戦前に裏向きで2枚まで切札を置くことができる。

\item {} 
切札はデッキと角度を変えて交わるようにデッキの下に置く。

\item {} 
切札を表にするときはスートと数字が見えるようにし、対応する能力の名称を言う。

\item {} 
デッキが0枚になった場合、切札が残っていても敗北する。

\item {} 
能力が割り当てられていないカードも切札とすることができるが、表になっても能力は発揮されない。

\end{itemize}

\begin{figure}[htbp]
\centering
\capstart

\noindent\sphinxincludegraphics{{trump}.pdf}
\caption{切札の置き方}\label{\detokenize{common/06-extra:id12}}\label{\detokenize{common/06-extra:trump}}\end{figure}

これ以降は、通常のゲームの始め方と同様です。


\subsection{切札の能力}
\label{\detokenize{common/06-extra:id7}}
エクストラでは切札を使って能力を得ることができます。
切札1枚1枚に異なった能力が割り当てられており、
表にすることで能力を発揮します。
割り当てられている能力については、「エクストラリスト」を参照して下さい。


\subsubsection{能力を発揮する}
\label{\detokenize{common/06-extra:id8}}
切札に割り当てられた能力は
「オープン」アクションを起こし表にすることで発揮します。(\hyperref[\detokenize{common/06-extra:trump-open}]{図 \ref{\detokenize{common/06-extra:trump-open}}})
「オープン」アクションの詳細は、
「エクストラリスト」を参照して下さい。
切札が表でいる限り、
その切札の能力は持続的に発揮されます。
また切札を表にする時は、
対戦相手に有効となった能力が分かるように、
能力の名称を言いスートと数字が見えるようにしましょう。

\begin{figure}[htbp]
\centering
\capstart

\noindent\sphinxincludegraphics{{trump-open}.pdf}
\caption{切札を表にする例}\label{\detokenize{common/06-extra:id13}}\label{\detokenize{common/06-extra:trump-open}}\end{figure}


\subsubsection{能力を無効化する}
\label{\detokenize{common/06-extra:id9}}
切札は裏向きもしくは、
墓地に移されると能力を発揮しなくなります。
切札を無効化するためには、「クローズ」アクションを用い
切札を裏向きにするか、
「切札破壊」アクションを用いて切札を破壊しましょう。
「クローズ」アクション、
「切札破壊」アクションの詳細は、
「エクストラリスト」を参照して下さい。

\begin{sphinxadmonition}{note}{【補足】1ターンに1回制限のアクションについて}

 切札がもたらすアクションの中には「プレイヤーは1ターンに1回しかこのアクションを起こすことができない。」
と特記事項に記載されているものがあります。
このアクションは1ターンに1回しか起こすことができないため、
切札が無効化され再度オープンし有効となっても、そのターンを通して1回しか起こすことができません。
\end{sphinxadmonition}


\subsection{まとめ}
\label{\detokenize{common/06-extra:id10}}
エクストラについて説明しました。
ルールとしては、プロに切札が入っただけのため、
プロで慣れていれば簡単に遊べると思います。
切札が増えていくのもエクストラの特徴です。是非遊んでみて下さい。


\section{その他のルール}
\label{\detokenize{common/07-etc:id1}}\label{\detokenize{common/07-etc::doc}}
この章では、
公開・非公開情報やシャッフルの仕方といった
細かな決まりごとを説明します。


\subsection{公開・非公開情報}
\label{\detokenize{common/07-etc:id2}}
配置されているカードには、アクションの効果
を使わなくても中身や枚数を知れるものがあります。
知れる度合いには次の種類があります。
\begin{description}
\item[{完全公開}] \leavevmode
全てのプレイヤーが知ることができ、
聞かれたプレイヤーは正しく答える必要がある

\item[{個人公開}] \leavevmode
デッキの持ち主のみ知ることができる

\item[{非公開}] \leavevmode
全てのプレイヤーは知ることができない

\end{description}

完全公開の情報であれば、ゲーム中いつでも対戦相手に聞くことができます。
各カードの配置と公開・非公開の度合いは次のとおりです。
\begin{description}
\item[{デッキ}] \leavevmode
\begin{DUlineblock}{0em}
\item[] 完全公開:10枚未満のデッキ枚数
\item[] 個人公開:デッキの枚数
\item[] 非公開:デッキの中身
\end{DUlineblock}

\item[{墓地}] \leavevmode
\begin{DUlineblock}{0em}
\item[] 完全公開:墓地の一番上のカード
\item[] 個人公開:墓地の中身
\item[] 非公開:なし
\end{DUlineblock}

\item[{場}] \leavevmode
\begin{DUlineblock}{0em}
\item[] 完全公開:表裏を変えずに見えるカード
\item[] 個人公開:伏せてあるカード
\item[] 非公開:なし
\end{DUlineblock}

\item[{手札}] \leavevmode
\begin{DUlineblock}{0em}
\item[] 完全公開:手札の枚数
\item[] 個人公開:手札の中身
\item[] 非公開:なし
\end{DUlineblock}

\item[{切札}] \leavevmode
\begin{DUlineblock}{0em}
\item[] 完全公開:表裏を変えずに見えるカード
\item[] 個人公開:伏せてあるカード
\item[] 非公開:なし
\end{DUlineblock}

\end{description}

\begin{sphinxadmonition}{note}{【補足】残りのデッキ枚数を聞かれたらどうしたらいいの?}

 対戦相手から残りのデッキ枚数を聞かれた場合、自分のデッキの枚数を上から10枚まで数え、相手に数えたカードの枚数が分かるように裏向きで見せます。
10枚未満であれば枚数を答え、10枚以上の場合「10枚以上です」と答えて下さい。
10枚以上の場合、正確な枚数を答える必要はありません。
\end{sphinxadmonition}


\subsection{デッキのシャッフルについて}
\label{\detokenize{common/07-etc:id3}}
BlackPokerでは
コンセプトの1つに”相手のカードに触らない”があるため、
対戦相手にデッキのシャッフルをお願いする必要はありません。

ただシャッフルしてほしいのであれば、お願いしても構いません。
逆に、対戦相手があまりシャッフルしていない場合は、
さらにシャッフルをお願いすることができます。


\chapter{コアルール}
\label{\detokenize{core/core:id1}}\label{\detokenize{core/core::doc}}
コアルールでは、ターン制ゲームに適用できる割込み処理についてルールを定義します。

コアルールは、次の図を中心に説明します。(\hyperref[\detokenize{core/core:coreflow-1}]{図 \ref{\detokenize{core/core:coreflow-1}}})
詳細は後ほど説明しますが、ゲーム開始から勝敗が決まるまでこの図に集約して説明します。
登場する用語はゲームに合わせて変更してください。
まず、ターン等の用語について説明します。

\begin{figure}[htbp]
\centering
\capstart

\noindent\sphinxincludegraphics{{plantuml-a9a3274db53de589cea7114f13b676cbeb24fbb4}.pdf}
\caption{コアフロー}\label{\detokenize{core/core:id16}}\label{\detokenize{core/core:coreflow-1}}\end{figure}


\section{ターン}
\label{\detokenize{core/core:id2}}
このルールを説明する上でターンとは、ターンは持つことができるものとします。
ターンを持っているプレイヤーは先に行動することができます。
ターンを持っているプレイヤーをターンプレイヤーといいます。


\section{アクション}
\label{\detokenize{core/core:id3}}
アクションとは、プレイヤーの行動を示します。
ターン制のゲームでは、プレイヤーは様々な行動を行います。
チェスであればコマを進めたり、ババ抜きであれば隣の人からカードを引くなどがあります。
それらをアクションと定義します。


\section{チャンス}
\label{\detokenize{core/core:id4}}
アクションを起こすことができる機会をチャンスといいます。
チャンスを持っている間は何度でもアクションを起こすことができます。


\section{ステージ}
\label{\detokenize{core/core:id5}}
アクションの解決順を整理するために使う領域です。
後入れ先出し方式で最後に積まれたアクションから順に解決されていきます。


\section{アクションの定義項目}
\label{\detokenize{core/core:id6}}
アクション、チャンス、ステージについて簡単に説明しました。
これらの概念を用いて、アクションに定義する項目を説明します。

アクションは次の項目を定義する必要があります。
他の項目は具体的にアクションを定義する際に、ゲームに合わせて追加して下さい。
\begin{itemize}
\item {} 
効果(通常効果・即時効果)

\item {} 
タイミング

\end{itemize}


\subsection{効果}
\label{\detokenize{core/core:id7}}
効果とはその効果を発揮した際に、プレイヤーが行う行動です。
効果の中には、通常効果と即時効果があります。
違いについては、図(\hyperref[\detokenize{core/core:coreflow-2}]{図 \ref{\detokenize{core/core:coreflow-2}}})を説明する際に分岐条件として登場します。


\subsection{タイミング}
\label{\detokenize{core/core:id8}}
タイミングとは、アクションを起こすことができる時を示します。
タイミングには「メイン」と「クイック」の2種類あります。
\begin{description}
\item[{メイン}] \leavevmode
ターンプレイヤーかつステージが空の時に起こせるアクションです。

条件をまとめると次のようになります。
\begin{itemize}
\item {} 
チャンスを持っている

\item {} 
自分のターン

\item {} 
ステージが空

\end{itemize}

\item[{クイック}] \leavevmode
いつでも起こせるため、アクションをステージに積み重ねることができます。

条件をまとめると次のようになります。
\begin{itemize}
\item {} 
チャンスを持っている

\end{itemize}

\end{description}


\subsubsection{エンドアクションの定義}
\label{\detokenize{core/core:id9}}
定義するアクションの中で最低1つは
ターンを別のプレイヤーにわたす効果を定義してください。
そうしないと、ターンが別のプレイヤーに渡らす、ゲームが進行しなくなります。

\begin{sphinxadmonition}{note}{\label{\detokenize{core/core:id10}}課題:}
例を書く
\end{sphinxadmonition}


\subsubsection{アクションのコントローラー}
\label{\detokenize{core/core:id11}}
アクションを起こしたプレイヤーをそのアクションのコントローラーと呼びます。
効果はこのコントローラー視点で解釈されることになります。


\section{コンポーネント}
\label{\detokenize{core/core:component}}\label{\detokenize{core/core:id12}}
ゲームにてプレイヤーが保有する駒やカードのことをコンポーネントと定義します。
コンポーネントは次の項目を持っています。
\begin{description}
\item[{オーナー}] \leavevmode
コンポーネントの所有者を示します。大体のトランプゲームではトランプを1セットしか用いないため無視されますが、TCGのデッキなど個人所有のものを用いるゲームでは必要な項目となります。

\item[{コントローラー}] \leavevmode
現在そのコンポーネントを操作しているプレイヤーを示します。オーナーとコントローラーは基本同じプレイヤーが設定されますが、コントロールを奪うアクションがある場合、オーナーとコントローラーは異なります。

\end{description}

\begin{sphinxadmonition}{note}{注釈:}
コンポーネントとアクションのコントローラー

コントローラーは制御している人という意味になるため、コンポーネントとアクションのコントローラー制御する対象が異なることになります。
コンポーネントとアクションの属性を次の図に示します。アクションにはオーナーがいない点が異なります。

\begin{figure}[H]
\centering
\capstart

\noindent\sphinxincludegraphics{{plantuml-97eb2fb6ceaeae89223169fc6f937aee84be4fe1}.pdf}
\caption{コンポーネントとアクションの属性}\label{\detokenize{core/core:id17}}\end{figure}
\end{sphinxadmonition}


\section{能力}
\label{\detokenize{core/core:id13}}
能力とはアクションの効果とは異なる概念で、アクションを起こすことができたり、
アクションを誘発したりすることができます。

能力を持つことができるのは、プレイヤーの他に駒やカードなどのゲームに登場するコンポーネントも持つことができます。
(\hyperref[\detokenize{core/core:ability-image}]{図 \ref{\detokenize{core/core:ability-image}}})

\begin{figure}[htbp]
\centering
\capstart

\noindent\sphinxincludegraphics{{plantuml-976f45327f77d4f4ee1da59bff1cd262ba402df4}.pdf}
\caption{能力のイメージ}\label{\detokenize{core/core:id18}}\label{\detokenize{core/core:ability-image}}\end{figure}

能力には、次の種類があります。
\begin{description}
\item[{常在型能力}] \leavevmode
能力が有効である場合、継続的に発揮される能力

\item[{誘発型能力}] \leavevmode
能力が有効である間に何かの契機でアクションを起こす能力

\end{description}

概ねのゲームでは、
ターン終了や駒をすすめるなどのアクションが定義されています。
そして、そのアクションを起こせる能力(常在型能力)を
プレイヤーは保持しています。


\section{コアフロー}
\label{\detokenize{core/core:id14}}
ここまでの説明を踏まえて、冒頭に紹介した図を説明します。(\hyperref[\detokenize{core/core:coreflow-2}]{図 \ref{\detokenize{core/core:coreflow-2}}})
この図にゲームの開始から勝敗が決まるまでの流れが集約されいます。

\begin{figure}[htbp]
\centering
\capstart

\noindent\sphinxincludegraphics{{plantuml-a9a3274db53de589cea7114f13b676cbeb24fbb4}.pdf}
\caption{コアフロー}\label{\detokenize{core/core:id19}}\label{\detokenize{core/core:coreflow-2}}\end{figure}
\begin{description}
\item[{{[}1{]}ゲーム開始}] \leavevmode
先攻を決め、ゲームを始める準備を行います。

\item[{{[}2{]}ターンプレイヤーにチャンスを移動}] \leavevmode
ターンを持っているプレイヤーにチャンスを移動します。

\item[{{[}3{]}ステージが空か?}] \leavevmode
ステージにアクションが存在していないか判定します。

\item[{{[}4{]}パス名簿リセット}] \leavevmode
パスしたプレイヤーを記録するパス名簿をリセットします。

\item[{{[}5{]}アクション起こす}] \leavevmode
アクションを起こしこれからプレイヤーが行うことを宣言します。
ゲームによってアクションの起こし方は異なります。BlackPokerではアクション名を言い、コストの支払や対象を指定しアクションを起こします。
一方ババ抜きでは、隣のプレイヤーからカードを引く際に宣言せず暗黙にアクションが起きている場合もあります。

\item[{{[}6{]}即時効果か?}] \leavevmode
起こしたアクションが即時効果か通常効果か判定します。

\item[{{[}7{]}効果解決}] \leavevmode
アクションの効果に定義されている内容を実行します。

\item[{{[}8{]}勝敗判定}] \leavevmode
ゲームの勝敗を判定します。判定の方法はゲームにより異なります。

\item[{{[}9{]}ステージに追加}] \leavevmode
ステージというアクションを貯めておける領域に追加します。

\item[{{[}10{]}誘発チェック}] \leavevmode
ここに至るまでに誘発したアクションがないかチェックします。誘発した場合、効果を解決するかスタックに追加します。詳しいフローは \DUrole{xref,std,std-ref}{trigger\sphinxhyphen{}check} を参照してください。

\item[{{[}11{]}アクションを起こすか?}] \leavevmode
チャンスを持っているプレイヤーはアクションを起こすかを判断します。

\item[{{[}12{]}パス名簿に登録}] \leavevmode
パスしたプレイヤーを記録するパス名簿に登録します。同じプレイヤー名は2回登録されません。

\item[{{[}13{]}パス名簿の件数=プレイヤー数か?}] \leavevmode
パス名簿の件数がゲームに参加しているプレイヤーの数と一致しているか判定します。

\item[{{[}14{]}ステージから取出し}] \leavevmode
最後にステージに追加されたアクションをステージから取出します。

\item[{{[}15{]}チャンス移動}] \leavevmode
チャンスを持っているプレイヤーからチャンスを持っていないプレイヤーにチャンスを移動します。
チャンスを移動するルールはゲームによって異なります。

\end{description}
\phantomsection\label{\detokenize{core/core:trigger-check}}
能力の中でも誘発型能力は、なにかをきっかけにしてアクションが起きる条件が定義されています。
誘発する条件は「〜の場合」、「〜時」などで記載されており、誘発するアクションは「〜を誘発する」と記載されています。

誘発チェックでは、誘発したアクションの効果を解決もしくは、ステージに追加します。
誘発チェックは次の図のように行います。(\hyperref[\detokenize{core/core:trigger-flow}]{図 \ref{\detokenize{core/core:trigger-flow}}})

\begin{figure}[htbp]
\centering
\capstart

\noindent\sphinxincludegraphics{{plantuml-a91828c766d562c760f38e1517ad1bbf09672f77}.pdf}
\caption{誘発チェック}\label{\detokenize{core/core:id20}}\label{\detokenize{core/core:trigger-flow}}\end{figure}
\begin{description}
\item[{{[}10\sphinxhyphen{}1{]}誘発チェック}] \leavevmode
全てのプレイヤー、コンポーネントが持っている誘発型能力を確認し、
アクションが誘発していないか判定します。

\item[{{[}10\sphinxhyphen{}2{]}即時誘発有無判定}] \leavevmode
即時効果を持つアクション誘発していないか判定します。

\item[{{[}10\sphinxhyphen{}3{]}効果解決\&勝敗判定}] \leavevmode
誘発した即時効果をプレイヤー毎に任意の順番で解決します。
解決するプレイヤーの順序は、
ターンプレイヤーがコントローラーとなっているアクションを全て解決してから、
ターンプレイヤー以外がコントローラーとなっているアクションを解決します。
この解決順序は、ゲームによって変更できます。

効果を解決する毎に勝敗判定を行ってください。

\item[{{[}10\sphinxhyphen{}4{]}誘発有無判定}] \leavevmode
通常効果を持つアクション誘発していないか判定します。

\item[{{[}10\sphinxhyphen{}5{]}ステージに追加}] \leavevmode
誘発したアクションをプレイヤー毎に任意の順番でステージに追加します。
ステージに追加するプレイヤーの順序は、
ターンプレイヤーがコントローラーとなっているアクションを全てステージに追加してから、
ターンプレイヤー以外がコントローラーとなっているアクションをステージに追加します。
この解決順序は、ゲームによって変更できます。

\end{description}


\section{まとめ}
\label{\detokenize{core/core:id15}}
コアルールについて説明しました。
すでにあるターン制のゲームからアクションを洗い出し、能力を整理することで割込処理を可能としゲームの新しい遊び方が見つけられます。
また、新しく作成するゲームに関してもコアルールを意識して作成することで、ルール追加がしやすいゲームが考えやすいと思います。


\chapter{付録}
\label{\detokenize{appendix/appendix:id1}}\label{\detokenize{appendix/appendix::doc}}

\section{PDF版ルール}
\label{\detokenize{appendix/appendix:pdf}}
\sphinxurl{https://blackpoker.github.io/BlackPoker/issue\_24/blackpoker.pdf}


\section{アクションリスト}
\label{\detokenize{appendix/appendix:id2}}\begin{description}
\item[{ライト}] \leavevmode\begin{description}
\item[{URL}] \leavevmode
\sphinxurl{https://blackpoker.github.io/BlackPoker/issue\_24/actionlist/html/v6-lite.html}

\item[{PDF}] \leavevmode
\sphinxurl{https://blackpoker.github.io/BlackPoker/issue\_24/actionlist/pdf/blackpoker-v6-lite.pdf}
\sphinxurl{https://blackpoker.github.io/BlackPoker/issue\_24/actionlist/pdf/blackpoker-v6-lite-2up.pdf}

\end{description}

\item[{スタンダード}] \leavevmode\begin{description}
\item[{URL}] \leavevmode
\sphinxurl{https://blackpoker.github.io/BlackPoker/issue\_24/actionlist/html/v6-std.html}

\item[{PDF}] \leavevmode
\sphinxurl{https://blackpoker.github.io/BlackPoker/issue\_24/actionlist/pdf/blackpoker-v6-std.pdf}
\sphinxurl{https://blackpoker.github.io/BlackPoker/issue\_24/actionlist/pdf/blackpoker-v6-std-2up.pdf}

\end{description}

\item[{プロ}] \leavevmode\begin{description}
\item[{URL}] \leavevmode
\sphinxurl{https://blackpoker.github.io/BlackPoker/issue\_24/actionlist/html/v6-pro.html}

\item[{PDF}] \leavevmode
\sphinxurl{https://blackpoker.github.io/BlackPoker/issue\_24/actionlist/pdf/blackpoker-v6-pro.pdf}
\sphinxurl{https://blackpoker.github.io/BlackPoker/issue\_24/actionlist/pdf/blackpoker-v6-pro-2up.pdf}

\end{description}

\item[{マスター}] \leavevmode\begin{description}
\item[{URL}] \leavevmode
\sphinxurl{https://blackpoker.github.io/BlackPoker/issue\_24/actionlist/html/v6-mast.html}

\item[{PDF}] \leavevmode
\sphinxurl{https://blackpoker.github.io/BlackPoker/issue\_24/actionlist/pdf/blackpoker-v6-mast.pdf}
\sphinxurl{https://blackpoker.github.io/BlackPoker/issue\_24/actionlist/pdf/blackpoker-v6-mast-2up.pdf}

\end{description}

\end{description}


\section{エクストラリスト}
\label{\detokenize{appendix/appendix:id3}}\begin{quote}
\begin{description}
\item[{URL}] \leavevmode
\sphinxurl{https://blackpoker.github.io/BlackPoker/issue\_24/actionlist/html/v6-ex.html}

\item[{PDF}] \leavevmode
\sphinxurl{https://blackpoker.github.io/BlackPoker/issue\_24/actionlist/pdf/blackpoker-v6-extra.pdf}
\sphinxurl{https://blackpoker.github.io/BlackPoker/issue\_24/actionlist/pdf/blackpoker-v6-extra-2up.pdf}

\end{description}
\end{quote}

\begin{sphinxadmonition}{note}{\label{\detokenize{appendix/appendix:id4}}課題:}
URLを記載
\end{sphinxadmonition}


\chapter{Indices and tables}
\label{\detokenize{index:indices-and-tables}}\begin{itemize}
\item {} 
\DUrole{xref,std,std-ref}{genindex}

\item {} 
\DUrole{xref,std,std-ref}{modindex}

\item {} 
\DUrole{xref,std,std-ref}{search}

\end{itemize}



\renewcommand{\indexname}{索引}
\printindex
\end{document}