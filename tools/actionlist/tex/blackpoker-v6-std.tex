\documentclass[twocolumn,a5paper,papersize,10pt]{jarticle}
\usepackage{bxpapersize}
\usepackage{booktabs}
\usepackage{tabularx}
\usepackage[dvipdfmx]{graphicx} %画像読み込み設定
\usepackage{here} %画像を好きな位置に出力
\usepackage[absolute,overlay]{textpos}%座標指定

\usepackage{newtxtext,newtxmath} %timeフォント設定
\usepackage{titlesec}%タイトルの文字サイズ変更
\usepackage{setspace} % setspaceパッケージのインクルード

\usepackage{tcolorbox} % 枠囲み
\usepackage{okumacro} % ルビ

%\columnseprule=0.1pt %段組み罫線
\setlength{\columnsep}{0.5cm} %段組みの幅


\titleformat*{\section}{\small\bfseries} %sectionの文字サイズ
\titleformat*{\subsection}{\scriptsize\bfseries} %subsectionの文字サイズ
\titleformat*{\subsubsection}{\scriptsize\bfseries} %subsubsectionの文字サイズ

\setlength{\hoffset}{-2cm}
\setlength{\voffset}{-4cm}
\setlength{\marginparsep}{0pt}
\setlength{\marginparwidth}{0pt}
\setlength{\headheight}{5pt}
\setlength{\textheight}{19cm}
\setlength{\textwidth}{13.8cm}

\setlength\intextsep{0pt} %図の余白をなくす
\setlength\textfloatsep{0pt} %図の余白をなくす

\setlength\floatsep{0pt} %図と図の間の余白
\setlength\textfloatsep{0pt} %本文と図の間の余白
\setlength\intextsep{0pt} %本文中の図の余白
\setlength\abovecaptionskip{0pt} %図とキャプションの間の余白


\setstretch{1} % ページ全体の行間を設定
\parindent = 0pt %インデントを無効化

\title{\empty}
\author{\empty}
\date{\empty}

%%%%%%    TEXT START    %%%%%%
\begin{document}

% ロゴ出力
% この場合は (230pt, 100pt) の位置に 0.4\linewidth の幅のブロックができる.
\begin{textblock*}{0.4\linewidth}(55pt, 145pt)
    \centering
    \includegraphics[width=1.2cm]{blackpoker_logo.pdf}
\end{textblock*}

%%%%% QRコード第三版用 %%%%%
\begin{textblock*}{0.4\linewidth}(200pt, 135pt)
    \centering
    \includegraphics[width=1.7cm,keepaspectratio]{qr_blackpoker-support_v5-std.pdf}
\end{textblock*}
\begin{textblock*}{0.4\linewidth}(200pt, 130pt)
    \centering
    \textcolor{black}{Web版}
\end{textblock*}

%%%%% QRコード第三版用 %%%%%

\section*{\textrm{\Large BlackPoker}}
\vspace{-1zh}%余白削除
\noindent

\begin{center}
{\footnotesize 第五版 act5.230 スタンダード}

{\scriptsize 2019/06/16}
\end{center}

\scriptsize%本文の文字サイズを小さく設定
\renewcommand{\labelitemi}{・}%箇条書きのラベルを変更
\hrule height 0.5mm depth 0mm width 66.5mm %罫線
\vspace{-3zh}%余白削除
\subsection*{(補足)記号の意味}
\vspace{-1zh}%余白削除
{\small @:タイミング, 
\$:コスト, 
★:キーカード}

\vspace{1mm}%余白削除
\hrule height 0.1mm depth 0mm width 66.5mm %罫線
\vspace{-3zh}%余白削除

\subsection*{(補足)コストの意味}
\vspace{-1zh}%余白削除
\begin{small}
\begin{tabbing}
 \hspace{2mm} \= \hspace{2mm} \= \hspace{1mm} \= hspace{15mm} \kill
\> B \> : \>防壁をドライブする \\
\> L \> : \>1点ダメージを受ける \\
\> D \> : \>手札を1枚捨てる \\
\> S \> : \>キャラクター1体を墓地に移す \\
\end{tabbing}
\end{small}

\vspace{-3zh}%余白削除
\hrule height 0.1mm depth 0mm width 66.5mm %罫線
\vspace{-3zh}%余白削除

\subsection*{(補足)防壁の置き方}
\vspace{-1zh}%余白削除
\begin{itemize}
\setlength{\leftskip}{-0.3cm}%箇条書きを左詰め
%\setlength{\itemsep}{0pt}      %2. ブロック間の余白
\setlength{\parskip}{0pt}      %4. 段落間余白.
%\setlength{\itemindent}{-10pt}   %5. 最初のインデント
%\setlength{\labelsep}{15pt}     %6. item と文字の間

\item 防壁を置く時はデッキ側に詰めて置く。
\item 防壁の左右の入れ替えは行わない。
\end{itemize}

\vspace{-1zh}%余白削除
\hrule height 0.1mm depth 0mm width 66.5mm %罫線
\vspace{-3zh}%余白削除

\subsection*{(補足)誘発する場合}
\vspace{-1zh}%余白削除
アクションを誘発する場合、誘発したアクションをステージに乗せます。詳しくは、公式ルール第五章の誘発を参照して下さい。
\vspace{-1zh}%余白削除


%%%%%%%%%%%%%%%%%%%%%%%%%%%%%%
%%%%% ActionList %%%%%
\begin{center}
\begin{center}
\hrule height 1mm depth 0mm width 66.5mm %罫線
\vspace{1mm}%余白削除
{\Large\bf \ruby{Action List}{アクションリスト}}
\vspace{1mm}%余白削除
\hrule height 0.5mm depth 0mm width 66.5mm %罫線
\end{center}
\end{center}
\vspace{-1zh}%余白削除
%%%%%%%%%%%%%%%%%%%%%%%%%%%%%%


%%%%%%%%%%%%%%%%%%%%%%%%%%%%%%
%%%%% CharacterList %%%%%
\begin{center}
\begin{center}
\hrule height 1mm depth 0mm width 66.5mm %罫線
\vspace{1mm}%余白削除
{\Large\bf \ruby{CharacterList}{キャラクターリスト}}
\vspace{1mm}%余白削除
\hrule height 0.5mm depth 0mm width 66.5mm %罫線
\end{center}
\end{center}
\vspace{-1zh}%余白削除
%%%%%%%%%%%%%%%%%%%%%%%%%%%%%%


\thispagestyle{empty}
\vspace*{\stretch{1}}
\begin{flushright}
\begin{minipage}{0.8\hsize}
\hrule height 0.2mm depth 0mm width 50mm %罫線
\begin{description}
  \item{誌名:}BlackPoker ActionList 第五版 act5.230
  \item{発行:}2019/06/16
\end{description}
\end{minipage}
\end{flushright}

\begin{flushright}
\copyright 2013 BlackPoker
\end{flushright}

\end{document}
