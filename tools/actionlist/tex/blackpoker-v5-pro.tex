\documentclass[twocolumn,a5paper,papersize,10pt]{jarticle}
\usepackage{bxpapersize}
\usepackage{booktabs}
\usepackage{tabularx}
\usepackage[dvipdfmx]{graphicx} %画像読み込み設定
\usepackage{here} %画像を好きな位置に出力
\usepackage[absolute,overlay]{textpos}%座標指定

\usepackage{newtxtext,newtxmath} %timeフォント設定
\usepackage{titlesec}%タイトルの文字サイズ変更
\usepackage{setspace} % setspaceパッケージのインクルード

\usepackage{tcolorbox} % 枠囲み
\usepackage{okumacro} % ルビ

%\columnseprule=0.1pt %段組み罫線
\setlength{\columnsep}{0.5cm} %段組みの幅


\titleformat*{\section}{\small\bfseries} %sectionの文字サイズ
\titleformat*{\subsection}{\scriptsize\bfseries} %subsectionの文字サイズ
\titleformat*{\subsubsection}{\scriptsize\bfseries} %subsubsectionの文字サイズ

\setlength{\hoffset}{-2cm}
\setlength{\voffset}{-4cm}
\setlength{\marginparsep}{0pt}
\setlength{\marginparwidth}{0pt}
\setlength{\headheight}{5pt}
\setlength{\textheight}{19cm}
\setlength{\textwidth}{13.8cm}

\setlength\intextsep{0pt} %図の余白をなくす
\setlength\textfloatsep{0pt} %図の余白をなくす

\setlength\floatsep{0pt} %図と図の間の余白
\setlength\textfloatsep{0pt} %本文と図の間の余白
\setlength\intextsep{0pt} %本文中の図の余白
\setlength\abovecaptionskip{0pt} %図とキャプションの間の余白


\setstretch{1} % ページ全体の行間を設定
\parindent = 0pt %インデントを無効化

\title{\empty}
\author{\empty}
\date{\empty}

%%%%%%    TEXT START    %%%%%%
\begin{document}

% ロゴ出力
% この場合は (230pt, 100pt) の位置に 0.4\linewidth の幅のブロックができる.
\begin{textblock*}{0.4\linewidth}(55pt, 145pt)
    \centering
    \includegraphics[width=1.2cm]{blackpoker_logo.pdf}
\end{textblock*}

%%%%% QRコード第三版用 %%%%%
\begin{textblock*}{0.4\linewidth}(200pt, 135pt)
    \centering
    \includegraphics[width=1.7cm,keepaspectratio]{qr_blackpoker-support_v5-pro.pdf}
\end{textblock*}
\begin{textblock*}{0.4\linewidth}(200pt, 130pt)
    \centering
    \textcolor{black}{Web版}
\end{textblock*}

%%%%% QRコード第三版用 %%%%%

\section*{\textrm{\Large BlackPoker}}
\vspace{-1zh}%余白削除
\noindent

\begin{center}
{\footnotesize 第五版 act5.2}

{\scriptsize 2019/06/16}
\end{center}

\scriptsize%本文の文字サイズを小さく設定
\renewcommand{\labelitemi}{・}%箇条書きのラベルを変更
{\quad}本誌はライト、スタンダード、プロのActionList,CharacterListです。
各フォーマットで使用できる
アクション、キャラクターが異なるため、記載されている表記により使い分けて下さい。
\vspace{-1zh}%余白削除
\begin{tabbing}
 \hspace{2mm} \= \hspace{2mm} \= \hspace{3mm} \= \hspace{1mm} \= hspace{15mm} \kill
\> ■ \>lite \> : \>ライト、スタンダード、プロで使えます。\\
\> ○ \>std \> : \>スタンダード、プロで使えます。\\
\> ◎ \>pro \> : \>プロで使えます。\\
\end{tabbing}
\vspace{-2zh}%余白削除
\hrule height 0.5mm depth 0mm width 66.5mm %罫線
\vspace{-3zh}%余白削除
\subsection*{(補足)記号の意味}
\vspace{-1zh}%余白削除
{\small @:タイミング, 
\$:コスト, 
★:キーカード}

\vspace{1mm}%余白削除
\hrule height 0.1mm depth 0mm width 66.5mm %罫線
\vspace{-3zh}%余白削除

\subsection*{(補足)コストの意味}
\vspace{-1zh}%余白削除
\begin{small}
\begin{tabbing}
 \hspace{2mm} \= \hspace{2mm} \= \hspace{1mm} \= hspace{15mm} \kill
\> B \> : \>防壁をドライブする \\
\> L \> : \>1点ダメージを受ける \\
\> D \> : \>手札を1枚捨てる \\
\> S \> : \>キャラクター1体を墓地に移す \\
\end{tabbing}
\end{small}

\vspace{-3zh}%余白削除
\hrule height 0.1mm depth 0mm width 66.5mm %罫線
\vspace{-3zh}%余白削除

\subsection*{(補足)防壁の置き方}
\vspace{-1zh}%余白削除
\begin{itemize}
\setlength{\leftskip}{-0.3cm}%箇条書きを左詰め
%\setlength{\itemsep}{0pt}      %2. ブロック間の余白
\setlength{\parskip}{0pt}      %4. 段落間余白.
%\setlength{\itemindent}{-10pt}   %5. 最初のインデント
%\setlength{\labelsep}{15pt}     %6. item と文字の間

\item 防壁を置く時はデッキ側に詰めて置く。
\item 防壁の左右の入れ替えは行わない。
\end{itemize}

\vspace{-1zh}%余白削除
\hrule height 0.1mm depth 0mm width 66.5mm %罫線
\vspace{-3zh}%余白削除

\subsection*{(補足)誘発する場合}
\vspace{-1zh}%余白削除
アクションを誘発する場合、誘発したアクションをステージに乗せます。詳しくは、公式ルール第五章の誘発を参照して下さい。
\vspace{-1zh}%余白削除


%%%%%%%%%%%%%%%%%%%%%%%%%%%%%%
%%%%% ActionList %%%%%
\begin{center}
\begin{center}
\hrule height 1mm depth 0mm width 66.5mm %罫線
\vspace{1mm}%余白削除
{\Large\bf \ruby{Action List}{アクションリスト}}
\vspace{1mm}%余白削除
\hrule height 0.5mm depth 0mm width 66.5mm %罫線
\end{center}
\end{center}
\vspace{-1zh}%余白削除
%%%%%%%%%%%%%%%%%%%%%%%%%%%%%%


%%% 大項目 %%%
\tcbset{colframe=black,coltitle=black!0!black,coltext=white!0!white,colbacktitle=white!0!white,colback=black!0!black,sharp corners,top=0mm, left=0mm, bottom=0mm, right=0mm,boxrule=0mm,toprule=0mm,valign=center,halign=center}
\begin{tcolorbox}
{\scriptsize\bf 展開系}
\end{tcolorbox}
\vspace{-1zh}%余白削除
%%% Action %%%
%\vspace{1zh} %余白追加
\vspace{2mm} %余白追加
\hrule height 0.5mm depth 0mm width 66.5mm %罫線
\vspace{1mm} %余白追加
{\small\bf ■ 防壁設置 {\scriptsize (召喚) [lite]}} %Actionタイトル
\hfill 
{\footnotesize\bf @メイン }
  {\footnotesize\bf | } {\footnotesize\bf \$ L}


%特記事項
\vspace{1mm}%余白削除
\hrule height 0.1mm depth 0mm width 66.5mm %罫線
\vspace{1mm}%余白削除


\vspace{-1zh}%余白削除
\begin{enumerate}
\renewcommand{\labelenumi}{※}
\setlength{\leftskip}{-0.3cm}
\setlength{\itemsep}{0pt} %2. ブロック間の余白
\setlength{\parskip}{0pt} %4. 段落間余白.

\item プレイヤーは1ターンに1回しかこのアクションを起こすことができない。

\vspace{-3mm}%余白削除
\end{enumerate}
\vspace{1mm}%余白削除
\hrule height 0.1mm depth 0mm width 66.5mm %罫線
\vspace{1mm}%余白削除

{\bf(即時効果)}

手札からカード1枚を防壁として裏向きかつチャージ状態で場に出す。防壁の能力はキャラクターリスト参照。防壁の置き方は「防壁の置き方」参照。
%%% Action %%%
%\vspace{1zh} %余白追加
\vspace{2mm} %余白追加
\hrule height 0.5mm depth 0mm width 66.5mm %罫線
\vspace{1mm} %余白追加
{\small\bf ■ 兵士召喚 {\scriptsize (召喚) [lite]}} %Actionタイトル
\hfill 
{\footnotesize\bf @メイン }
  {\footnotesize\bf | } {\footnotesize\bf \$ BL}

{\footnotesize\bf ★2〜10}

\vspace{1mm}%余白削除
\hrule height 0.1mm depth 0mm width 66.5mm %罫線
\vspace{1mm}%余白削除

{\bf(通常効果)}

キーカードを一般兵として表向きかつチャージ状態で場に出す。一般兵の能力は、キャラクターリスト参照。
%%% Action %%%
%\vspace{1zh} %余白追加
\vspace{2mm} %余白追加
\hrule height 0.5mm depth 0mm width 66.5mm %罫線
\vspace{1mm} %余白追加
{\small\bf ■ 英雄召喚 {\scriptsize (召喚) [lite]}} %Actionタイトル
\hfill 
{\footnotesize\bf @メイン }
  {\footnotesize\bf | } {\footnotesize\bf \$ BBL}

{\footnotesize\bf ★J〜K}

\vspace{1mm}%余白削除
\hrule height 0.1mm depth 0mm width 66.5mm %罫線
\vspace{1mm}%余白削除

{\bf(通常効果)}

キーカードを英雄として表向きかつチャージ状態で場に出す。英雄の能力は、キャラクターリスト参照。
%%% Action %%%
%\vspace{1zh} %余白追加
\vspace{2mm} %余白追加
\hrule height 0.5mm depth 0mm width 66.5mm %罫線
\vspace{1mm} %余白追加
{\small\bf ■ エース召喚 {\scriptsize (召喚) [lite]}} %Actionタイトル
\hfill 
{\footnotesize\bf @メイン }
  {\footnotesize\bf | } {\footnotesize\bf \$ L}

{\footnotesize\bf ★A}

\vspace{1mm}%余白削除
\hrule height 0.1mm depth 0mm width 66.5mm %罫線
\vspace{1mm}%余白削除

{\bf(通常効果)}

キーカードをエースとして表向きかつチャージ状態で場に出す。エースの能力は、キャラクターリスト参照。
%%% Action %%%
%\vspace{1zh} %余白追加
\vspace{2mm} %余白追加
\hrule height 0.5mm depth 0mm width 66.5mm %罫線
\vspace{1mm} %余白追加
{\small\bf ○ 魔術士召喚 {\scriptsize (召喚) [std]}} %Actionタイトル
\hfill 
{\footnotesize\bf @メイン }
  {\footnotesize\bf | } {\footnotesize\bf \$ BD}

{\footnotesize\bf ★Joker}

\vspace{1mm}%余白削除
\hrule height 0.1mm depth 0mm width 66.5mm %罫線
\vspace{1mm}%余白削除

{\bf(通常効果)}

キーカードを魔術士として表向きかつチャージ状態で場に出す。魔術士の能力は、キャラクターリスト参照。
%%% Action %%%
%\vspace{1zh} %余白追加
\vspace{2mm} %余白追加
\hrule height 0.5mm depth 0mm width 66.5mm %罫線
\vspace{1mm} %余白追加
{\small\bf ■ 装備 {\scriptsize (召喚) [lite]}} %Actionタイトル
\hfill 
{\footnotesize\bf @メイン }
  {\footnotesize\bf | } {\footnotesize\bf \$ BL}

{\footnotesize\bf ★A〜K}

\vspace{1mm}%余白削除
\hrule height 0.1mm depth 0mm width 66.5mm %罫線
\vspace{1mm}%余白削除
{\bf(対象)}

自分の場にいるキーカードと同じスートの兵士1体を対象とする。Jokerは対象にできない。
\vspace{1mm}%余白削除
\hrule height 0.1mm depth 0mm width 66.5mm %罫線
\vspace{1mm}%余白削除

{\bf(通常効果)}

対象とした兵士の上にキーカードを置き装備兵とする。装備兵の能力は、キャラクターリスト参照。
%%% Action %%%
%\vspace{1zh} %余白追加
\vspace{2mm} %余白追加
\hrule height 0.5mm depth 0mm width 66.5mm %罫線
\vspace{1mm} %余白追加
{\small\bf ○ 帰還 {\scriptsize (召喚) [std]}} %Actionタイトル
\hfill 
{\footnotesize\bf @クイック }
  {\footnotesize\bf | } {\footnotesize\bf \$ B}

{\footnotesize\bf ★同じスートを2枚}

\vspace{1mm}%余白削除
\hrule height 0.1mm depth 0mm width 66.5mm %罫線
\vspace{1mm}%余白削除
{\bf(対象)}

自分の場のキャラクター1体を対象とする。
\vspace{1mm}%余白削除
\hrule height 0.1mm depth 0mm width 66.5mm %罫線
\vspace{1mm}%余白削除

{\bf(通常効果)}


\vspace{-1zh}%余白削除
\begin{enumerate}
\setlength{\leftskip}{-0.3cm}
\setlength{\parskip}{0pt} %4. 段落間余白.

\item 対象のキャラクターがチャージ状態の場合、対象のキャラクターを手札に戻す。

\item キーカードを手札に戻す。
\vspace{-1zh}%余白削除
\end{enumerate}


%%% 大項目 %%%
\tcbset{colframe=black,coltitle=black!0!black,coltext=white!0!white,colbacktitle=white!0!white,colback=black!0!black,sharp corners,top=0mm, left=0mm, bottom=0mm, right=0mm,boxrule=0mm,toprule=0mm,valign=center,halign=center}
\begin{tcolorbox}
{\scriptsize\bf ターン制御系}
\end{tcolorbox}
\vspace{-1zh}%余白削除
%%% Action %%%
%\vspace{1zh} %余白追加
\vspace{2mm} %余白追加
\hrule height 0.5mm depth 0mm width 66.5mm %罫線
\vspace{1mm} %余白追加
{\small\bf ■ チャージ {\scriptsize (ターン制御) [lite]}} %Actionタイトル
\hfill 
{\footnotesize\bf @メイン }


%特記事項
\vspace{1mm}%余白削除
\hrule height 0.1mm depth 0mm width 66.5mm %罫線
\vspace{1mm}%余白削除


\vspace{-1zh}%余白削除
\begin{enumerate}
\renewcommand{\labelenumi}{※}
\setlength{\leftskip}{-0.3cm}
\setlength{\itemsep}{0pt} %2. ブロック間の余白
\setlength{\parskip}{0pt} %4. 段落間余白.

\item プレイヤーはこのアクションを直接起こすことができない。

\vspace{-3mm}%余白削除
\end{enumerate}
\vspace{1mm}%余白削除
\hrule height 0.1mm depth 0mm width 66.5mm %罫線
\vspace{1mm}%余白削除

{\bf(即時効果)}


\vspace{-1zh}%余白削除
\begin{enumerate}
\setlength{\leftskip}{-0.3cm}
\setlength{\parskip}{0pt} %4. 段落間余白.

\item ターンを持っているプレイヤーの場にいるキャラクターを全てチャージ状態にする。

\item ドローアクションを起こす。
\vspace{-1zh}%余白削除
\end{enumerate}
%%% Action %%%
%\vspace{1zh} %余白追加
\vspace{2mm} %余白追加
\hrule height 0.5mm depth 0mm width 66.5mm %罫線
\vspace{1mm} %余白追加
{\small\bf ■ ドロー {\scriptsize (ターン制御) [lite]}} %Actionタイトル
\hfill 
{\footnotesize\bf @メイン }


%特記事項
\vspace{1mm}%余白削除
\hrule height 0.1mm depth 0mm width 66.5mm %罫線
\vspace{1mm}%余白削除


\vspace{-1zh}%余白削除
\begin{enumerate}
\renewcommand{\labelenumi}{※}
\setlength{\leftskip}{-0.3cm}
\setlength{\itemsep}{0pt} %2. ブロック間の余白
\setlength{\parskip}{0pt} %4. 段落間余白.

\item プレイヤーはこのアクションを直接起こすことができない。

\vspace{-3mm}%余白削除
\end{enumerate}
\vspace{1mm}%余白削除
\hrule height 0.1mm depth 0mm width 66.5mm %罫線
\vspace{1mm}%余白削除

{\bf(通常効果)}

ターンを持っているプレイヤーは次を行う。


\vspace{-1zh}%余白削除
\begin{enumerate}
\setlength{\leftskip}{-0.3cm}
\setlength{\parskip}{0pt} %4. 段落間余白.

\item デッキの一番上からカードを1枚引き、手札に加える。

\item 必要であれば、更にデッキの一番上からカードを1枚引き、手札に加える。
\vspace{-1zh}%余白削除
\end{enumerate}
%%% Action %%%
%\vspace{1zh} %余白追加
\vspace{2mm} %余白追加
\hrule height 0.5mm depth 0mm width 66.5mm %罫線
\vspace{1mm} %余白追加
{\small\bf ■ エンド {\scriptsize (ターン制御) [lite]}} %Actionタイトル
\hfill 
{\footnotesize\bf @メイン }


\vspace{1mm}%余白削除
\hrule height 0.1mm depth 0mm width 66.5mm %罫線
\vspace{1mm}%余白削除

{\bf(通常効果)}


\vspace{-1zh}%余白削除
\begin{enumerate}
\setlength{\leftskip}{-0.3cm}
\setlength{\parskip}{0pt} %4. 段落間余白.

\item 手札が7枚を越えた場合、7枚になるよう手札を捨てる。

\item 自分のターンを終了し、対戦相手にターンを渡す。

\item チャージアクションを起こす。
\vspace{-1zh}%余白削除
\end{enumerate}


%%% 大項目 %%%
\tcbset{colframe=black,coltitle=black!0!black,coltext=white!0!white,colbacktitle=white!0!white,colback=black!0!black,sharp corners,top=0mm, left=0mm, bottom=0mm, right=0mm,boxrule=0mm,toprule=0mm,valign=center,halign=center}
\begin{tcolorbox}
{\scriptsize\bf 攻撃系}
\end{tcolorbox}
\vspace{-1zh}%余白削除
%%% Action %%%
%\vspace{1zh} %余白追加
\vspace{2mm} %余白追加
\hrule height 0.5mm depth 0mm width 66.5mm %罫線
\vspace{1mm} %余白追加
{\small\bf ■ アタック {\scriptsize (攻防) [lite]}} %Actionタイトル
\hfill 
{\footnotesize\bf @メイン }


%特記事項
\vspace{1mm}%余白削除
\hrule height 0.1mm depth 0mm width 66.5mm %罫線
\vspace{1mm}%余白削除


\vspace{-1zh}%余白削除
\begin{enumerate}
\renewcommand{\labelenumi}{※}
\setlength{\leftskip}{-0.3cm}
\setlength{\itemsep}{0pt} %2. ブロック間の余白
\setlength{\parskip}{0pt} %4. 段落間余白.

\item プレイヤーは1ターンに1回しかこのアクションを起こすことができない。

\vspace{-3mm}%余白削除
\end{enumerate}
\vspace{1mm}%余白削除
\hrule height 0.1mm depth 0mm width 66.5mm %罫線
\vspace{1mm}%余白削除

{\bf(通常効果)}


\vspace{-1zh}%余白削除
\begin{enumerate}
\setlength{\leftskip}{-0.3cm}
\setlength{\parskip}{0pt} %4. 段落間余白.

\item 対戦相手を攻撃する兵士(アタッカー)をドライブして指定する。アタッカーは複数指定可能。ドライブ状態の兵士は指定できない。

\item ブロックアクションを起こす。
\vspace{-1zh}%余白削除
\end{enumerate}
%%% Action %%%
%\vspace{1zh} %余白追加
\vspace{2mm} %余白追加
\hrule height 0.5mm depth 0mm width 66.5mm %罫線
\vspace{1mm} %余白追加
{\small\bf ■ ブロック {\scriptsize (攻防) [lite]}} %Actionタイトル
\hfill 
{\footnotesize\bf @メイン }


%特記事項
\vspace{1mm}%余白削除
\hrule height 0.1mm depth 0mm width 66.5mm %罫線
\vspace{1mm}%余白削除


\vspace{-1zh}%余白削除
\begin{enumerate}
\renewcommand{\labelenumi}{※}
\setlength{\leftskip}{-0.3cm}
\setlength{\itemsep}{0pt} %2. ブロック間の余白
\setlength{\parskip}{0pt} %4. 段落間余白.

\item プレイヤーはこのアクションを直接起こすことができない。

\vspace{-3mm}%余白削除
\end{enumerate}
\vspace{1mm}%余白削除
\hrule height 0.1mm depth 0mm width 66.5mm %罫線
\vspace{1mm}%余白削除

{\bf(通常効果)}


\vspace{-1zh}%余白削除
\begin{enumerate}
\setlength{\leftskip}{-0.3cm}
\setlength{\parskip}{0pt} %4. 段落間余白.

\item 対戦相手はアタックアクションにて指定されたアタッカー毎にそれをブロックするキャラクター(ブロッカー)を指定する。 兵士でブロックする場合、1アタッカーに対して複数の兵士を指定できる。ドライブ状態のキャラクターは指定できない。

\item ダメージ判定アクションを起こす。
\vspace{-1zh}%余白削除
\end{enumerate}
%%% Action %%%
%\vspace{1zh} %余白追加
\vspace{2mm} %余白追加
\hrule height 0.5mm depth 0mm width 66.5mm %罫線
\vspace{1mm} %余白追加
{\small\bf ■ ダメージ判定 {\scriptsize (攻防) [lite]}} %Actionタイトル
\hfill 
{\footnotesize\bf @メイン }


%特記事項
\vspace{1mm}%余白削除
\hrule height 0.1mm depth 0mm width 66.5mm %罫線
\vspace{1mm}%余白削除


\vspace{-1zh}%余白削除
\begin{enumerate}
\renewcommand{\labelenumi}{※}
\setlength{\leftskip}{-0.3cm}
\setlength{\itemsep}{0pt} %2. ブロック間の余白
\setlength{\parskip}{0pt} %4. 段落間余白.

\item プレイヤーはこのアクションを直接起こすことができない。

\vspace{-3mm}%余白削除
\end{enumerate}
\vspace{1mm}%余白削除
\hrule height 0.1mm depth 0mm width 66.5mm %罫線
\vspace{1mm}%余白削除

{\bf(通常効果)}

アタッカーとブロッカーを比較する


\vspace{-1zh}%余白削除
\begin{enumerate}
\setlength{\leftskip}{-0.3cm}
\setlength{\parskip}{0pt} %4. 段落間余白.

\item 兵士(アタッカー)と兵士(ブロッカー)の場合、アタッカーとブロッカーで数字を比較し、少ない方を墓地に移す。同じ場合は両方を墓地に移動する。1アタッカーに対して複数ブロッカーがいる場合、ブロッカーの合計数字と比較する。

\item 兵士(アタッカー)と防壁(ブロッカー)の場合、キャラクターリストの防壁-(能力)【ダメージ判定】参照

\item アタッカーをブロックするブロッカーが場に存在しない場合、アタッカーの数字だけ対戦相手にダメージを与える。
\vspace{-1zh}%余白削除
\end{enumerate}


%%% 大項目 %%%
\tcbset{colframe=black,coltitle=black!0!black,coltext=white!0!white,colbacktitle=white!0!white,colback=black!0!black,sharp corners,top=0mm, left=0mm, bottom=0mm, right=0mm,boxrule=0mm,toprule=0mm,valign=center,halign=center}
\begin{tcolorbox}
{\scriptsize\bf 魔法系}
\end{tcolorbox}
\vspace{-1zh}%余白削除
%%% Action %%%
%\vspace{1zh} %余白追加
\vspace{2mm} %余白追加
\hrule height 0.5mm depth 0mm width 66.5mm %罫線
\vspace{1mm} %余白追加
{\small\bf ■ アップ {\scriptsize (速攻魔法) [lite]}} %Actionタイトル
\hfill 
{\footnotesize\bf @クイック }
  {\footnotesize\bf | } {\footnotesize\bf \$ D}

{\footnotesize\bf ★{\normalsize $\heartsuit$} A〜10}

\vspace{1mm}%余白削除
\hrule height 0.1mm depth 0mm width 66.5mm %罫線
\vspace{1mm}%余白削除
{\bf(対象)}

兵士1体を対象とする。
\vspace{1mm}%余白削除
\hrule height 0.1mm depth 0mm width 66.5mm %罫線
\vspace{1mm}%余白削除

{\bf(通常効果)}

対象とした兵士の数字は、このターンが終わるまでキーカードの数字分加算される。
%%% Action %%%
%\vspace{1zh} %余白追加
\vspace{2mm} %余白追加
\hrule height 0.5mm depth 0mm width 66.5mm %罫線
\vspace{1mm} %余白追加
{\small\bf ■ ダウン {\scriptsize (速攻魔法) [lite]}} %Actionタイトル
\hfill 
{\footnotesize\bf @クイック }
  {\footnotesize\bf | } {\footnotesize\bf \$ D}

{\footnotesize\bf ★{\normalsize $\spadesuit$} A〜10}

\vspace{1mm}%余白削除
\hrule height 0.1mm depth 0mm width 66.5mm %罫線
\vspace{1mm}%余白削除
{\bf(対象)}

兵士1体を対象とする。
\vspace{1mm}%余白削除
\hrule height 0.1mm depth 0mm width 66.5mm %罫線
\vspace{1mm}%余白削除

{\bf(通常効果)}

対象とした兵士の数字は、このターンが終わるまでキーカードの数字分減算される。もし対象の数字が0以下となった場合、対象を墓地に移す。
%%% Action %%%
%\vspace{1zh} %余白追加
\vspace{2mm} %余白追加
\hrule height 0.5mm depth 0mm width 66.5mm %罫線
\vspace{1mm} %余白追加
{\small\bf ■ ツイスト {\scriptsize (速攻魔法) [lite]}} %Actionタイトル
\hfill 
{\footnotesize\bf @クイック }
  {\footnotesize\bf | } {\footnotesize\bf \$ D}

{\footnotesize\bf ★{\normalsize $\diamondsuit$} A〜10}

\vspace{1mm}%余白削除
\hrule height 0.1mm depth 0mm width 66.5mm %罫線
\vspace{1mm}%余白削除
{\bf(対象)}

キャラクター1体を対象とする。
\vspace{1mm}%余白削除
\hrule height 0.1mm depth 0mm width 66.5mm %罫線
\vspace{1mm}%余白削除

{\bf(通常効果)}

対象のキャラクターをドライブ状態またはチャージ状態にする。
%%% Action %%%
%\vspace{1zh} %余白追加
\vspace{2mm} %余白追加
\hrule height 0.5mm depth 0mm width 66.5mm %罫線
\vspace{1mm} %余白追加
{\small\bf ■ カウンター {\scriptsize (速攻魔法) [lite]}} %Actionタイトル
\hfill 
{\footnotesize\bf @クイック }
  {\footnotesize\bf | } {\footnotesize\bf \$ D}

{\footnotesize\bf ★{\normalsize $\clubsuit$} A〜10}

\vspace{1mm}%余白削除
\hrule height 0.1mm depth 0mm width 66.5mm %罫線
\vspace{1mm}%余白削除
{\bf(対象)}

以下のいずれかの条件に該当するアクションを対象とする。Xはこのアクションのキーカードの数字とする。


\vspace{-1zh}%余白削除
\begin{itemize}
\setlength{\leftskip}{-0.3cm}
\setlength{\parskip}{0pt} %4. 段落間余白.

\item キーカードが1枚かつそのキーカードの数字がX以下

\item キーカードが2枚
\vspace{-1zh}%余白削除
\end{itemize}
\vspace{1mm}%余白削除
\hrule height 0.1mm depth 0mm width 66.5mm %罫線
\vspace{1mm}%余白削除

{\bf(通常効果)}

対象のアクションは効果を発揮しない。対象アクションをステージから取り除き、対象アクションのキーカードを墓地に移す。
%%% Action %%%
%\vspace{1zh} %余白追加
\vspace{2mm} %余白追加
\hrule height 0.5mm depth 0mm width 66.5mm %罫線
\vspace{1mm} %余白追加
{\small\bf ◎ 再会 {\scriptsize (速攻魔法) [pro]}} %Actionタイトル
\hfill 
{\footnotesize\bf @クイック }

{\footnotesize\bf ★{\normalsize $\heartsuit$} A〜10 を 2枚}

\vspace{1mm}%余白削除
\hrule height 0.1mm depth 0mm width 66.5mm %罫線
\vspace{1mm}%余白削除

{\bf(通常効果)}

自分の墓地からカードを1枚選び対戦相手に見せ手札に加える。
%%% Action %%%
%\vspace{1zh} %余白追加
\vspace{2mm} %余白追加
\hrule height 0.5mm depth 0mm width 66.5mm %罫線
\vspace{1mm} %余白追加
{\small\bf ◎ キル {\scriptsize (速攻魔法) [pro]}} %Actionタイトル
\hfill 
{\footnotesize\bf @クイック }

{\footnotesize\bf ★{\normalsize $\spadesuit$} A〜10 を 2枚}

\vspace{1mm}%余白削除
\hrule height 0.1mm depth 0mm width 66.5mm %罫線
\vspace{1mm}%余白削除
{\bf(対象)}

兵士1体を対象とする。
\vspace{1mm}%余白削除
\hrule height 0.1mm depth 0mm width 66.5mm %罫線
\vspace{1mm}%余白削除

{\bf(通常効果)}

対象とした兵士を墓地に移す。
%%% Action %%%
%\vspace{1zh} %余白追加
\vspace{2mm} %余白追加
\hrule height 0.5mm depth 0mm width 66.5mm %罫線
\vspace{1mm} %余白追加
{\small\bf ◎ 停戦 {\scriptsize (速攻魔法) [pro]}} %Actionタイトル
\hfill 
{\footnotesize\bf @クイック }

{\footnotesize\bf ★{\normalsize $\diamondsuit$} A〜10 を 2枚}

\vspace{1mm}%余白削除
\hrule height 0.1mm depth 0mm width 66.5mm %罫線
\vspace{1mm}%余白削除
{\bf(対象)}

ダメージ判定アクションを対象とする。
\vspace{1mm}%余白削除
\hrule height 0.1mm depth 0mm width 66.5mm %罫線
\vspace{1mm}%余白削除

{\bf(通常効果)}

対象のアクションは効果を発揮しない。
%%% Action %%%
%\vspace{1zh} %余白追加
\vspace{2mm} %余白追加
\hrule height 0.5mm depth 0mm width 66.5mm %罫線
\vspace{1mm} %余白追加
{\small\bf ◎ 対象変更 {\scriptsize (速攻魔法) [pro]}} %Actionタイトル
\hfill 
{\footnotesize\bf @クイック }

{\footnotesize\bf ★{\normalsize $\clubsuit$} A〜10 を 2枚}

\vspace{1mm}%余白削除
\hrule height 0.1mm depth 0mm width 66.5mm %罫線
\vspace{1mm}%余白削除
{\bf(対象)}

対象が指定されているアクションを対象とする。
\vspace{1mm}%余白削除
\hrule height 0.1mm depth 0mm width 66.5mm %罫線
\vspace{1mm}%余白削除

{\bf(通常効果)}

対象のアクションで指定されている対象をそのアクションが指定できる範囲で変更する。アクションを対象とするアクションの対象を対象変更に変更することは可能、アクションの対象にできないものへの対象の変更は不可能とする。
%%% Action %%%
%\vspace{1zh} %余白追加
\vspace{2mm} %余白追加
\hrule height 0.5mm depth 0mm width 66.5mm %罫線
\vspace{1mm} %余白追加
{\small\bf ■ 防壁破壊 {\scriptsize (通常魔法) [lite]}} %Actionタイトル
\hfill 
{\footnotesize\bf @メイン }

{\footnotesize\bf ★{\normalsize $\heartsuit$} A〜K と {\normalsize $\diamondsuit$} A〜K}

\vspace{1mm}%余白削除
\hrule height 0.1mm depth 0mm width 66.5mm %罫線
\vspace{1mm}%余白削除
{\bf(対象)}

防壁1体を対象とする。
\vspace{1mm}%余白削除
\hrule height 0.1mm depth 0mm width 66.5mm %罫線
\vspace{1mm}%余白削除

{\bf(通常効果)}

対象の防壁を場から墓地に移す。
%%% Action %%%
%\vspace{1zh} %余白追加
\vspace{2mm} %余白追加
\hrule height 0.5mm depth 0mm width 66.5mm %罫線
\vspace{1mm} %余白追加
{\small\bf ■ 投擲 {\scriptsize (通常魔法) [lite]}} %Actionタイトル
\hfill 
{\footnotesize\bf @メイン }

{\footnotesize\bf ★{\normalsize $\spadesuit$} A〜K と {\normalsize $\clubsuit$} A〜K}

\vspace{1mm}%余白削除
\hrule height 0.1mm depth 0mm width 66.5mm %罫線
\vspace{1mm}%余白削除
{\bf(対象)}

対戦相手1人を対象とする。
\vspace{1mm}%余白削除
\hrule height 0.1mm depth 0mm width 66.5mm %罫線
\vspace{1mm}%余白削除

{\bf(通常効果)}

対象の対戦相手にX点のダメージを与える。Xはキーカードの{\normalsize $\spadesuit$} カードの数字に等しい。
%%% Action %%%
%\vspace{1zh} %余白追加
\vspace{2mm} %余白追加
\hrule height 0.5mm depth 0mm width 66.5mm %罫線
\vspace{1mm} %余白追加
{\small\bf ○ 死の槍 {\scriptsize (通常魔法) [std]}} %Actionタイトル
\hfill 
{\footnotesize\bf @メイン }

{\footnotesize\bf ★{\normalsize $\spadesuit$} A〜K と {\normalsize $\diamondsuit$} A〜K}

\vspace{1mm}%余白削除
\hrule height 0.1mm depth 0mm width 66.5mm %罫線
\vspace{1mm}%余白削除
{\bf(対象)}

兵士1体を対象とする。
\vspace{1mm}%余白削除
\hrule height 0.1mm depth 0mm width 66.5mm %罫線
\vspace{1mm}%余白削除

{\bf(通常効果)}

対象の兵士の数字が0以外かつ、キーカードの{\normalsize $\diamondsuit$} カードの数字で割切れる場合、次を行う。


\vspace{-1zh}%余白削除
\begin{enumerate}
\setlength{\leftskip}{-0.3cm}
\setlength{\parskip}{0pt} %4. 段落間余白.

\item 対象の兵士をオーナーのデッキの一番上に裏向きで移す。兵士が複数のカードから成る場合、任意の順でデッキの一番上に裏向きで移す。

\item 対象の兵士のオーナーにX点のダメージを与える。Xはキーカードの{\normalsize $\spadesuit$} カードの数字に等しい。
\vspace{-1zh}%余白削除
\end{enumerate}
%%% Action %%%
%\vspace{1zh} %余白追加
\vspace{2mm} %余白追加
\hrule height 0.5mm depth 0mm width 66.5mm %罫線
\vspace{1mm} %余白追加
{\small\bf ○ 防壁補充 {\scriptsize (通常魔法) [std]}} %Actionタイトル
\hfill 
{\footnotesize\bf @メイン }

{\footnotesize\bf ★{\normalsize $\heartsuit$} A〜K と {\normalsize $\clubsuit$} A〜K}

\vspace{1mm}%余白削除
\hrule height 0.1mm depth 0mm width 66.5mm %罫線
\vspace{1mm}%余白削除

{\bf(通常効果)}

自分のデッキの一番上から1枚を防壁として裏向きかつチャージ状態で場に出す。もしくは、自分のデッキの一番上から2枚を防壁として裏向きかつドライブ状態で場に出す。防壁の能力はキャラクターリスト参照。

防壁の置き方は「防壁の置き方」参照
%%% Action %%%
%\vspace{1zh} %余白追加
\vspace{2mm} %余白追加
\hrule height 0.5mm depth 0mm width 66.5mm %罫線
\vspace{1mm} %余白追加
{\small\bf ○ リアニメイト {\scriptsize (通常魔法) [std]}} %Actionタイトル
\hfill 
{\footnotesize\bf @メイン }

{\footnotesize\bf ★{\normalsize $\spadesuit$} A〜K と {\normalsize $\heartsuit$} A〜K}

\vspace{1mm}%余白削除
\hrule height 0.1mm depth 0mm width 66.5mm %罫線
\vspace{1mm}%余白削除
{\bf(対象)}

自分の場のキャラクター1体を対象とする。
\vspace{1mm}%余白削除
\hrule height 0.1mm depth 0mm width 66.5mm %罫線
\vspace{1mm}%余白削除

{\bf(通常効果)}

自分の墓地にあるカード1枚を選ぶ。対象のキャラクターを墓地に移せた場合、選んだカードを兵士として表向きかつチャージ状態で場に出す。移せない場合、選んだカードを墓地に戻す。
%%% Action %%%
%\vspace{1zh} %余白追加
\vspace{2mm} %余白追加
\hrule height 0.5mm depth 0mm width 66.5mm %罫線
\vspace{1mm} %余白追加
{\small\bf ○ ハンデス {\scriptsize (通常魔法) [std]}} %Actionタイトル
\hfill 
{\footnotesize\bf @メイン }

{\footnotesize\bf ★{\normalsize $\diamondsuit$} A〜K と {\normalsize $\clubsuit$} A〜K}

\vspace{1mm}%余白削除
\hrule height 0.1mm depth 0mm width 66.5mm %罫線
\vspace{1mm}%余白削除
{\bf(対象)}

対戦相手1人を対象とする。
\vspace{1mm}%余白削除
\hrule height 0.1mm depth 0mm width 66.5mm %罫線
\vspace{1mm}%余白削除

{\bf(通常効果)}

対戦相手の手札を見て1枚カードを指定する。対戦相手は指定されたカードを手札から捨てる。
%%% Action %%%
%\vspace{1zh} %余白追加
\vspace{2mm} %余白追加
\hrule height 0.5mm depth 0mm width 66.5mm %罫線
\vspace{1mm} %余白追加
{\small\bf ◎ フォース {\scriptsize (通常魔法) [pro]}} %Actionタイトル
\hfill 
{\footnotesize\bf @メイン }
  {\footnotesize\bf | } {\footnotesize\bf \$ BB}

{\footnotesize\bf ★{\normalsize $\heartsuit$} A〜10 を 2枚}

%特記事項
\vspace{1mm}%余白削除
\hrule height 0.1mm depth 0mm width 66.5mm %罫線
\vspace{1mm}%余白削除


\vspace{-1zh}%余白削除
\begin{enumerate}
\renewcommand{\labelenumi}{※}
\setlength{\leftskip}{-0.3cm}
\setlength{\itemsep}{0pt} %2. ブロック間の余白
\setlength{\parskip}{0pt} %4. 段落間余白.

\item このアクションはカウンターアクションの対象にならない。

\vspace{-3mm}%余白削除
\end{enumerate}
\vspace{1mm}%余白削除
\hrule height 0.1mm depth 0mm width 66.5mm %罫線
\vspace{1mm}%余白削除

{\bf(通常効果)}

このターンが終わるまで自分の兵士全ての数字は、キーカードの合計値分加算される。
%%% Action %%%
%\vspace{1zh} %余白追加
\vspace{2mm} %余白追加
\hrule height 0.5mm depth 0mm width 66.5mm %罫線
\vspace{1mm} %余白追加
{\small\bf ◎ 剣の雨 {\scriptsize (通常魔法) [pro]}} %Actionタイトル
\hfill 
{\footnotesize\bf @メイン }
  {\footnotesize\bf | } {\footnotesize\bf \$ BB}

{\footnotesize\bf ★{\normalsize $\spadesuit$} A〜10 を 2枚}

%特記事項
\vspace{1mm}%余白削除
\hrule height 0.1mm depth 0mm width 66.5mm %罫線
\vspace{1mm}%余白削除


\vspace{-1zh}%余白削除
\begin{enumerate}
\renewcommand{\labelenumi}{※}
\setlength{\leftskip}{-0.3cm}
\setlength{\itemsep}{0pt} %2. ブロック間の余白
\setlength{\parskip}{0pt} %4. 段落間余白.

\item このアクションはカウンターアクションの対象にならない。

\vspace{-3mm}%余白削除
\end{enumerate}
\vspace{1mm}%余白削除
\hrule height 0.1mm depth 0mm width 66.5mm %罫線
\vspace{1mm}%余白削除

{\bf(通常効果)}

キーカードの合計値以下の全ての兵士を墓地に移す。
%%% Action %%%
%\vspace{1zh} %余白追加
\vspace{2mm} %余白追加
\hrule height 0.5mm depth 0mm width 66.5mm %罫線
\vspace{1mm} %余白追加
{\small\bf ◎ 徴募 {\scriptsize (通常魔法) [pro]}} %Actionタイトル
\hfill 
{\footnotesize\bf @メイン }
  {\footnotesize\bf | } {\footnotesize\bf \$ BB}

{\footnotesize\bf ★{\normalsize $\diamondsuit$} A〜10 を 2枚}

%特記事項
\vspace{1mm}%余白削除
\hrule height 0.1mm depth 0mm width 66.5mm %罫線
\vspace{1mm}%余白削除


\vspace{-1zh}%余白削除
\begin{enumerate}
\renewcommand{\labelenumi}{※}
\setlength{\leftskip}{-0.3cm}
\setlength{\itemsep}{0pt} %2. ブロック間の余白
\setlength{\parskip}{0pt} %4. 段落間余白.

\item このアクションはカウンターアクションの対象にならない。

\vspace{-3mm}%余白削除
\end{enumerate}
\vspace{1mm}%余白削除
\hrule height 0.1mm depth 0mm width 66.5mm %罫線
\vspace{1mm}%余白削除

{\bf(通常効果)}


\vspace{-1zh}%余白削除
\begin{enumerate}
\setlength{\leftskip}{-0.3cm}
\setlength{\parskip}{0pt} %4. 段落間余白.

\item デッキの一番上から4枚めくり、キーカードの合計値以下のカードを兵士としてドライブ状態で場に出す。

\item 残りのカードをデッキの一番下に好きな順で移す。
\vspace{-1zh}%余白削除
\end{enumerate}
%%% Action %%%
%\vspace{1zh} %余白追加
\vspace{2mm} %余白追加
\hrule height 0.5mm depth 0mm width 66.5mm %罫線
\vspace{1mm} %余白追加
{\small\bf ◎ 奇襲 {\scriptsize (通常魔法) [pro]}} %Actionタイトル
\hfill 
{\footnotesize\bf @メイン }
  {\footnotesize\bf | } {\footnotesize\bf \$ BB}

{\footnotesize\bf ★{\normalsize $\clubsuit$} A〜10 を 2枚}

%特記事項
\vspace{1mm}%余白削除
\hrule height 0.1mm depth 0mm width 66.5mm %罫線
\vspace{1mm}%余白削除


\vspace{-1zh}%余白削除
\begin{enumerate}
\renewcommand{\labelenumi}{※}
\setlength{\leftskip}{-0.3cm}
\setlength{\itemsep}{0pt} %2. ブロック間の余白
\setlength{\parskip}{0pt} %4. 段落間余白.

\item このアクションはカウンターアクションの対象にならない。

\vspace{-3mm}%余白削除
\end{enumerate}
\vspace{1mm}%余白削除
\hrule height 0.1mm depth 0mm width 66.5mm %罫線
\vspace{1mm}%余白削除

{\bf(通常効果)}


\vspace{-1zh}%余白削除
\begin{enumerate}
\setlength{\leftskip}{-0.3cm}
\setlength{\parskip}{0pt} %4. 段落間余白.

\item 自分の場にいる防壁を全て兵士にする。このターンに場に出た防壁を兵士にする場合、その兵士はこのターンに出た兵士と同様に扱う。

\item 自分の場にいる全ての兵士をチャージ状態にする。
\vspace{-1zh}%余白削除
\end{enumerate}
%%% Action %%%
%\vspace{1zh} %余白追加
\vspace{2mm} %余白追加
\hrule height 0.5mm depth 0mm width 66.5mm %罫線
\vspace{1mm} %余白追加
{\small\bf ■ サーチ {\scriptsize (速攻魔法) [lite]}} %Actionタイトル
\hfill 
{\footnotesize\bf @クイック }

{\footnotesize\bf ★Joker}

\vspace{1mm}%余白削除
\hrule height 0.1mm depth 0mm width 66.5mm %罫線
\vspace{1mm}%余白削除

{\bf(即時効果)}

デッキから好きなカードを1枚選び対戦相手に見せ手札に加える。その後デッキを切りなおす。
%%% Action %%%
%\vspace{1zh} %余白追加
\vspace{2mm} %余白追加
\hrule height 0.5mm depth 0mm width 66.5mm %罫線
\vspace{1mm} %余白追加
{\small\bf ◎ B・J {\scriptsize (通常魔法) [pro]}} %Actionタイトル
\hfill 
{\footnotesize\bf @メイン }
  {\footnotesize\bf | } {\footnotesize\bf \$ SS}

{\footnotesize\bf ★同じスートの A と J}

%特記事項
\vspace{1mm}%余白削除
\hrule height 0.1mm depth 0mm width 66.5mm %罫線
\vspace{1mm}%余白削除


\vspace{-1zh}%余白削除
\begin{enumerate}
\renewcommand{\labelenumi}{※}
\setlength{\leftskip}{-0.3cm}
\setlength{\itemsep}{0pt} %2. ブロック間の余白
\setlength{\parskip}{0pt} %4. 段落間余白.

\item コストの支払によって誘発したアクションより先に効果を発揮する。

\vspace{-3mm}%余白削除
\end{enumerate}
\vspace{1mm}%余白削除
\hrule height 0.1mm depth 0mm width 66.5mm %罫線
\vspace{1mm}%余白削除

{\bf(即時効果)}

デッキから好きなカードを2枚選び対戦相手に見せ手札に加える。その後デッキを切りなおす。
%%% Action %%%
%\vspace{1zh} %余白追加
\vspace{2mm} %余白追加
\hrule height 0.5mm depth 0mm width 66.5mm %罫線
\vspace{1mm} %余白追加
{\small\bf ◎ R・S・F {\scriptsize (通常魔法) [pro]}} %Actionタイトル
\hfill 
{\footnotesize\bf @メイン }
  {\footnotesize\bf | } {\footnotesize\bf \$ BB}

{\footnotesize\bf ★同じスートの A,10〜K を 5枚}

%特記事項
\vspace{1mm}%余白削除
\hrule height 0.1mm depth 0mm width 66.5mm %罫線
\vspace{1mm}%余白削除


\vspace{-1zh}%余白削除
\begin{enumerate}
\renewcommand{\labelenumi}{※}
\setlength{\leftskip}{-0.3cm}
\setlength{\itemsep}{0pt} %2. ブロック間の余白
\setlength{\parskip}{0pt} %4. 段落間余白.

\item このアクションはカウンターアクションの対象にならない。

\vspace{-3mm}%余白削除
\end{enumerate}
\vspace{1mm}%余白削除
\hrule height 0.1mm depth 0mm width 66.5mm %罫線
\vspace{1mm}%余白削除
{\bf(対象)}

プレイヤー1人を対象とする。
\vspace{1mm}%余白削除
\hrule height 0.1mm depth 0mm width 66.5mm %罫線
\vspace{1mm}%余白削除

{\bf(通常効果)}

対象のプレイヤーに40点のダメージを与える。
%%% Action %%%
%\vspace{1zh} %余白追加
\vspace{2mm} %余白追加
\hrule height 0.5mm depth 0mm width 66.5mm %罫線
\vspace{1mm} %余白追加
{\small\bf ○ リバース {\scriptsize (通常魔法) [std]}} %Actionタイトル
\hfill 
{\footnotesize\bf @メイン }

{\footnotesize\bf ★同じ数字を2枚}

\vspace{1mm}%余白削除
\hrule height 0.1mm depth 0mm width 66.5mm %罫線
\vspace{1mm}%余白削除
{\bf(対象)}

キャラクター1体を対象とする。
\vspace{1mm}%余白削除
\hrule height 0.1mm depth 0mm width 66.5mm %罫線
\vspace{1mm}%余白削除

{\bf(通常効果)}


\vspace{-1zh}%余白削除
\begin{enumerate}
\setlength{\leftskip}{-0.3cm}
\setlength{\parskip}{0pt} %4. 段落間余白.

\item 必要であれば、対象のキャラクターをチャージ状態またはドライブ状態にする。

\item 対象が兵士の場合、兵士を防壁にする。兵士の時に受けた効果、能力は無くなる。兵士が複数のカードから成る場合、1枚ずつ防壁にする。防壁の置き方は「防壁の置き方」参照

\item 対象が防壁の場合、防壁を兵士にする。防壁の時に受けた効果、能力は無くなる。このターンに場に出た防壁を兵士にする場合、その兵士はこのターンに出た兵士と同様に扱う。
\vspace{-1zh}%余白削除
\end{enumerate}


%%% 大項目 %%%
\tcbset{colframe=black,coltitle=black!0!black,coltext=white!0!white,colbacktitle=white!0!white,colback=black!0!black,sharp corners,top=0mm, left=0mm, bottom=0mm, right=0mm,boxrule=0mm,toprule=0mm,valign=center,halign=center}
\begin{tcolorbox}
{\scriptsize\bf 誘発系}
\end{tcolorbox}
\vspace{-1zh}%余白削除
%%% Action %%%
%\vspace{1zh} %余白追加
\vspace{2mm} %余白追加
\hrule height 0.5mm depth 0mm width 66.5mm %罫線
\vspace{1mm} %余白追加
{\small\bf ■ 世代交代 {\scriptsize (誘発) [lite]}} %Actionタイトル
\hfill 
{\footnotesize\bf @クイック }


%特記事項
\vspace{1mm}%余白削除
\hrule height 0.1mm depth 0mm width 66.5mm %罫線
\vspace{1mm}%余白削除


\vspace{-1zh}%余白削除
\begin{enumerate}
\renewcommand{\labelenumi}{※}
\setlength{\leftskip}{-0.3cm}
\setlength{\itemsep}{0pt} %2. ブロック間の余白
\setlength{\parskip}{0pt} %4. 段落間余白.

\item プレイヤーはこのアクションを直接起こすことができない。

\vspace{-3mm}%余白削除
\end{enumerate}
\vspace{1mm}%余白削除
\hrule height 0.1mm depth 0mm width 66.5mm %罫線
\vspace{1mm}%余白削除

{\bf(即時効果)}

デッキの一番上からJoker,A,J,Q,Kのいずれかが出るまで墓地にカードを移動し、出たら手札に加える。


%%%%%%%%%%%%%%%%%%%%%%%%%%%%%%
%%%%% CharacterList %%%%%
\begin{center}
\begin{center}
\hrule height 1mm depth 0mm width 66.5mm %罫線
\vspace{1mm}%余白削除
{\Large\bf \ruby{CharacterList}{キャラクターリスト}}
\vspace{1mm}%余白削除
\hrule height 0.5mm depth 0mm width 66.5mm %罫線
\end{center}
\end{center}
\vspace{-1zh}%余白削除
%%%%%%%%%%%%%%%%%%%%%%%%%%%%%%


%%% 大項目 %%%
\tcbset{colframe=black,coltitle=black!0!black,coltext=white!0!white,colbacktitle=white!0!white,colback=black!0!black,sharp corners,top=0mm, left=0mm, bottom=0mm, right=0mm,boxrule=0mm,toprule=0mm,valign=center,halign=center}
\begin{tcolorbox}
{\scriptsize\bf 兵士}
\end{tcolorbox}
\vspace{-1zh}%余白削除
%%% Character %%%
%\vspace{1zh} %余白追加
\vspace{2mm} %余白追加
\hrule height 0.5mm depth 0mm width 66.5mm %罫線
\vspace{1mm} %余白追加
{\small\bf ■ 一般兵 {\scriptsize (兵士) [lite]}} %Characterタイトル
\hfill 
{\footnotesize\bf ★2〜10 }

\vspace{1mm}%余白削除
\hrule height 0.1mm depth 0mm width 66.5mm %罫線
\vspace{1mm}%余白削除

{\bf(能力)}


\vspace{-1zh}%余白削除
\begin{itemize}
\setlength{\leftskip}{-0.3cm}
\setlength{\parskip}{0pt} %4. 段落間余白.

\item 準備(このキャラクターがこのターンに場に出たカードのみで構成されている場合、アタックアクションにて対戦相手を攻撃する兵士(アタッカー)に指定することができない。)
\vspace{-1zh}%余白削除
\end{itemize}
%%% Character %%%
%\vspace{1zh} %余白追加
\vspace{2mm} %余白追加
\hrule height 0.5mm depth 0mm width 66.5mm %罫線
\vspace{1mm} %余白追加
{\small\bf ■ 英雄 {\scriptsize (兵士) [lite]}} %Characterタイトル
\hfill 
{\footnotesize\bf ★J〜K }

\vspace{1mm}%余白削除
\hrule height 0.1mm depth 0mm width 66.5mm %罫線
\vspace{1mm}%余白削除

{\bf(能力)}


\vspace{-1zh}%余白削除
\begin{itemize}
\setlength{\leftskip}{-0.3cm}
\setlength{\parskip}{0pt} %4. 段落間余白.

\item 準備(このキャラクターがこのターンに場に出たカードのみで構成されている場合、アタックアクションにて対戦相手を攻撃する兵士(アタッカー)に指定することができない。)

\item 場から墓地に行く場合、世代交代アクションを誘発する。誘発については「誘発する場合」を参照。
\vspace{-1zh}%余白削除
\end{itemize}
%%% Character %%%
%\vspace{1zh} %余白追加
\vspace{2mm} %余白追加
\hrule height 0.5mm depth 0mm width 66.5mm %罫線
\vspace{1mm} %余白追加
{\small\bf ■ エース {\scriptsize (兵士) [lite]}} %Characterタイトル
\hfill 
{\footnotesize\bf ★A }

\vspace{1mm}%余白削除
\hrule height 0.1mm depth 0mm width 66.5mm %罫線
\vspace{1mm}%余白削除

{\bf(能力)}


\vspace{-1zh}%余白削除
\begin{itemize}
\setlength{\leftskip}{-0.3cm}
\setlength{\parskip}{0pt} %4. 段落間余白.

\item 速攻(場に出たターンからアタックアクションにて、対戦相手を攻撃する兵士(アタッカー)に指定できる。この能力は準備より優先される。)

\item 場から墓地に行く場合、世代交代アクションを誘発する。誘発については「誘発する場合」を参照。
\vspace{-1zh}%余白削除
\end{itemize}
%%% Character %%%
%\vspace{1zh} %余白追加
\vspace{2mm} %余白追加
\hrule height 0.5mm depth 0mm width 66.5mm %罫線
\vspace{1mm} %余白追加
{\small\bf ○ 魔術士 {\scriptsize (兵士) [std]}} %Characterタイトル
\hfill 
{\footnotesize\bf ★Joker }

\vspace{1mm}%余白削除
\hrule height 0.1mm depth 0mm width 66.5mm %罫線
\vspace{1mm}%余白削除

{\bf(能力)}


\vspace{-1zh}%余白削除
\begin{itemize}
\setlength{\leftskip}{-0.3cm}
\setlength{\parskip}{0pt} %4. 段落間余白.

\item 魔力増加(このキャラクターが場にいる間、タイプ:速攻魔法のコストが無しになる。)

\item 速攻(場に出たターンからアタックアクションにて、対戦相手を攻撃する兵士(アタッカー)に指定できる。この能力は準備より優先される。)

\item 場から墓地に行く場合、世代交代アクションを誘発する。誘発については「誘発する場合」を参照。
\vspace{-1zh}%余白削除
\end{itemize}
%%% Character %%%
%\vspace{1zh} %余白追加
\vspace{2mm} %余白追加
\hrule height 0.5mm depth 0mm width 66.5mm %罫線
\vspace{1mm} %余白追加
{\small\bf ■ 装備兵 {\scriptsize (兵士) [lite]}} %Characterタイトル
\hfill 
{\footnotesize\bf ★同じスートを2枚以上 }

\vspace{1mm}%余白削除
\hrule height 0.1mm depth 0mm width 66.5mm %罫線
\vspace{1mm}%余白削除

{\bf(能力)}


\vspace{-1zh}%余白削除
\begin{itemize}
\setlength{\leftskip}{-0.3cm}
\setlength{\parskip}{0pt} %4. 段落間余白.

\item キーカードにAが含まれる場合、速攻(場に出たターンからアタックアクションにて、対戦相手を攻撃する兵士(アタッカー)に指定できる。この能力は準備より優先される。)を得る。

\item キーカードにAが含まれない場合、準備(このキャラクターがこのターンに場に出たカードのみで構成されている場合、アタックアクションにて対戦相手を攻撃する兵士(アタッカー)に指定することができない。)を得る。

\item 数字はキーカードの合計値とする。

\item 場から墓地に行く場合、キーカードに含まれるA,J,Q,Kの枚数と同じ回数だけ世代交代アクションを誘発する。誘発については「誘発する場合」を参照。
\vspace{-1zh}%余白削除
\end{itemize}


%%% 大項目 %%%
\tcbset{colframe=black,coltitle=black!0!black,coltext=white!0!white,colbacktitle=white!0!white,colback=black!0!black,sharp corners,top=0mm, left=0mm, bottom=0mm, right=0mm,boxrule=0mm,toprule=0mm,valign=center,halign=center}
\begin{tcolorbox}
{\scriptsize\bf 防壁}
\end{tcolorbox}
\vspace{-1zh}%余白削除
%%% Character %%%
%\vspace{1zh} %余白追加
\vspace{2mm} %余白追加
\hrule height 0.5mm depth 0mm width 66.5mm %罫線
\vspace{1mm} %余白追加
{\small\bf ■ 防壁 {\scriptsize (防壁) [lite]}} %Characterタイトル
\hfill 
{\footnotesize\bf ★全て(裏向き) }

\vspace{1mm}%余白削除
\hrule height 0.1mm depth 0mm width 66.5mm %罫線
\vspace{1mm}%余白削除

{\bf(能力)}


\vspace{-1zh}%余白削除
\begin{itemize}
\setlength{\leftskip}{-0.3cm}
\setlength{\parskip}{0pt} %4. 段落間余白.

\item アタッカーに指定することができない。

\item 墓地に移る際に防壁がJoker,A,J,Q,Kの場合、世代交代アクションを誘発する。誘発については「誘発する場合」を参照。

\item 【ダメージ判定】 この能力はダメージ判定アクションにて兵士(アタッカー)をブロックした場合に発揮する。

\vspace{-1zh}%余白削除
\begin{enumerate}
\setlength{\leftskip}{-0.3cm}
\setlength{\parskip}{0pt} %4. 段落間余白.

\item 防壁を表にし、防壁が以下の条件に当てはまる場合、アタッカーを墓地に移す。

\begin{enumerate}
\renewcommand{\labelenumi}{\Alph{enumi}}
\setlength{\leftskip}{-0.3cm}
\setlength{\parskip}{0pt} %4. 段落間余白.

\item 防壁がJokerの場合

\item 防壁のカードに記載されている数字と同じ数字がアタッカーのカードに含まれている場合

\end{enumerate}
\item 防壁を墓地に移す。
\vspace{-1zh}%余白削除
\end{enumerate}\vspace{-1zh}%余白削除
\end{itemize}


\thispagestyle{empty}
\vspace*{\stretch{1}}
\begin{flushright}
\begin{minipage}{0.8\hsize}
\hrule height 0.2mm depth 0mm width 50mm %罫線
\begin{description}
  \item{誌名:}BlackPoker ActionList 第五版 act5.2
  \item{発行:}2019/06/16
\end{description}
\end{minipage}
\end{flushright}

\begin{flushright}
\copyright 2013 BlackPoker
\end{flushright}

\end{document}
