\documentclass[twocolumn,a5paper,papersize,10pt]{jarticle}
\usepackage{bxpapersize}
\usepackage{booktabs}
\usepackage{tabularx}
\usepackage[dvipdfmx]{graphicx} %画像読み込み設定
\usepackage{here} %画像を好きな位置に出力
\usepackage[absolute,overlay]{textpos}%座標指定

\usepackage{newtxtext,newtxmath} %timeフォント設定
\usepackage{titlesec}%タイトルの文字サイズ変更
\usepackage{setspace} % setspaceパッケージのインクルード

\usepackage{tcolorbox} % 枠囲み
\usepackage{okumacro} % ルビ

%\columnseprule=0.1pt %段組み罫線
\setlength{\columnsep}{0.5cm} %段組みの幅


\titleformat*{\section}{\large\bfseries} %sectionの文字サイズ
\titleformat*{\subsection}{\normalsize\bfseries} %subsectionの文字サイズ
\titleformat*{\subsubsection}{\small\bfseries} %subsubsectionの文字サイズ

\setlength{\hoffset}{-2cm}
\setlength{\voffset}{-4cm}
\setlength{\marginparsep}{0pt}
\setlength{\marginparwidth}{0pt}
\setlength{\headheight}{5pt}
\setlength{\textheight}{19cm}
\setlength{\textwidth}{13.8cm}

\setlength\intextsep{0pt} %図の余白をなくす
\setlength\textfloatsep{0pt} %図の余白をなくす

\setlength\floatsep{0pt} %図と図の間の余白
\setlength\textfloatsep{0pt} %本文と図の間の余白
\setlength\intextsep{0pt} %本文中の図の余白
\setlength\abovecaptionskip{0pt} %図とキャプションの間の余白


\setstretch{1} % ページ全体の行間を設定
\parindent = 0pt %インデントを無効化

\title{\empty}
\author{\empty}
\date{\empty}

%%%%%%    TEXT START    %%%%%%
\begin{document}


% ロゴ出力
% この場合は (230pt, 100pt) の位置に 0.4\linewidth の幅のブロックができる.
\begin{textblock*}{0.4\linewidth}(55pt, 145pt)
    \centering
    \includegraphics[width=1.2cm]{blackpoker_logo.pdf}
\end{textblock*}

%%%%% QRコード第三版用 %%%%%
\begin{textblock*}{0.4\linewidth}(200pt, 135pt)
    \centering
    \includegraphics[width=1.7cm,keepaspectratio]{qr_blackpoker-support_v5-ex.pdf}
\end{textblock*}
\begin{textblock*}{0.4\linewidth}(200pt, 130pt)
    \centering
    \textcolor{black}{Web版}
\end{textblock*}
%%%%% QRコード第三版用 %%%%%

\section*{\textrm{\Large BlackPoker}}
\vspace{-1zh}%余白削除
\noindent

\begin{center}
第五版 extra
ex5.30.0

{\scriptsize 2019/06/16}

\end{center}

\scriptsize%本文の文字サイズを小さく設定
\renewcommand{\labelitemi}{・}%箇条書きのラベルを変更

{\quad}extraとは、BlackPokerの遊び方の一つです。プレイヤーは切札を設置してからゲームを開始します。
切札については 切札についてを参照して下さい。extraではプロ版のアクション、キャラクターに加え切札を操作するアクションとそれに対応する切札の能力を使うことが出来ます。

{\quad}本誌はextraで使用できるアクションと切札を記載しています。

\vspace{2mm}%余白削除
\hrule height 0.5mm depth 0mm width 66.5mm %罫線
\vspace{-3zh}%余白削除
\subsection*{(補足)切札について}
%\vspace{-2mm}%余白削除
%\subsubsection*{切札について}
\vspace{-1zh}%余白削除
\begin{itemize}
\setlength{\leftskip}{-0.3cm}%箇条書きを左詰め
%\setlength{\itemsep}{0pt}      %2. ブロック間の余白
\setlength{\parskip}{0pt}      %4. 段落間余白.
%\setlength{\itemindent}{-10pt}   %5. 最初のインデント
%\setlength{\labelsep}{15pt}     %6. item と文字の間

\item 対戦前に裏向きで2枚まで切札を置くことができる。
\item 切札はデッキと直角に交わるようにデッキの下に置く。
\item 切札を表にするときはスートと数字が見えるようにし、対応する能力の名称を言う。
\item デッキが0枚になった場合、切札が残っていても敗北する。
\end{itemize}
※詳しくは、公式ルール参照。
\vspace{-1zh}%余白削除

%%%%%%%%%%%%%%%%%%%%%%%%%%%%%%
%%%%% ActionList %%%%%
\begin{center}
\begin{center}
\hrule height 1mm depth 0mm width 66.5mm %罫線
\vspace{1mm}%余白削除
{\Large\bf \ruby{Action List}{アクションリスト}}
\vspace{1mm}%余白削除
\hrule height 0.5mm depth 0mm width 66.5mm %罫線
\end{center}
\end{center}
\vspace{-1zh}%余白削除
%%%%%%%%%%%%%%%%%%%%%%%%%%%%%%

%%% Action %%%
%\vspace{1zh} %余白追加
\vspace{2mm} %余白追加
\hrule height 0.5mm depth 0mm width 66.5mm %罫線
\vspace{1mm} %余白追加
{\normalsize\bf ■ オープン {\scriptsize (通常魔法) [ex]}} %Actionタイトル
\hfill 
{\small\bf @メイン }
  {\small\bf | } {\small\bf \$ BBD}


%特記事項
\vspace{1mm}%余白削除
\hrule height 0.1mm depth 0mm width 66.5mm %罫線
\vspace{1mm}%余白削除


\vspace{-1zh}%余白削除
\begin{enumerate}
\renewcommand{\labelenumi}{※}
\setlength{\leftskip}{-0.3cm}
\setlength{\itemsep}{0pt} %2. ブロック間の余白
\setlength{\parskip}{0pt} %4. 段落間余白.

\item コストを\$DDとすればタイミングをクイックとして起こすことができる。

\vspace{-3mm}%余白削除
\end{enumerate}
\vspace{1mm}%余白削除
\hrule height 0.1mm depth 0mm width 66.5mm %罫線
\vspace{1mm}%余白削除

{\bf(即時効果)}

自分の切札を1枚表にし、切札に対応した能力を発揮する。 

切札については「切札について」参照

切札に対応する能力については「アビリティリスト」参照
%%% Action %%%
%\vspace{1zh} %余白追加
\vspace{2mm} %余白追加
\hrule height 0.5mm depth 0mm width 66.5mm %罫線
\vspace{1mm} %余白追加
{\normalsize\bf ■ クローズ {\scriptsize (速攻魔法) [ex]}} %Actionタイトル
\hfill 
{\small\bf @クイック }
  {\small\bf | } {\small\bf \$ B}

★同じスート2枚

\vspace{1mm}%余白削除
\hrule height 0.1mm depth 0mm width 66.5mm %罫線
\vspace{1mm}%余白削除
{\bf(対象)}

切札1枚を対象とする。
\vspace{1mm}%余白削除
\hrule height 0.1mm depth 0mm width 66.5mm %罫線
\vspace{1mm}%余白削除

{\bf(通常効果)}

対象の切札が表向きかつキーカードのスートと同じ場合、その切札を裏向きにする。対象の切札に対応した能力は無効になる。 

切札については「切札について」参照
%%% Action %%%
%\vspace{1zh} %余白追加
\vspace{2mm} %余白追加
\hrule height 0.5mm depth 0mm width 66.5mm %罫線
\vspace{1mm} %余白追加
{\normalsize\bf ■ 切札破壊 {\scriptsize (速攻魔法) [ex]}} %Actionタイトル
\hfill 
{\small\bf @クイック }
  {\small\bf | } {\small\bf \$ B}

★A〜K

\vspace{1mm}%余白削除
\hrule height 0.1mm depth 0mm width 66.5mm %罫線
\vspace{1mm}%余白削除
{\bf(対象)}

切札1枚を対象とする。
\vspace{1mm}%余白削除
\hrule height 0.1mm depth 0mm width 66.5mm %罫線
\vspace{1mm}%余白削除

{\bf(通常効果)}

対象の切札が表向きかつキーカードとスートと数字が同じ場合、切札を墓地に移す。対象の切札に対応した能力は無効になる。 

切札については「切札について」参照
%%% Action %%%
%\vspace{1zh} %余白追加
\vspace{2mm} %余白追加
\hrule height 0.5mm depth 0mm width 66.5mm %罫線
\vspace{1mm} %余白追加
{\normalsize\bf ■ 切札再生 {\scriptsize (誘発) [ex]}} %Actionタイトル
\hfill 
{\small\bf @クイック }


%特記事項
\vspace{1mm}%余白削除
\hrule height 0.1mm depth 0mm width 66.5mm %罫線
\vspace{1mm}%余白削除


\vspace{-1zh}%余白削除
\begin{enumerate}
\renewcommand{\labelenumi}{※}
\setlength{\leftskip}{-0.3cm}
\setlength{\itemsep}{0pt} %2. ブロック間の余白
\setlength{\parskip}{0pt} %4. 段落間余白.

\item プレイヤーはこのアクションを直接起こすことができない。

\vspace{-3mm}%余白削除
\end{enumerate}
\vspace{1mm}%余白削除
\hrule height 0.1mm depth 0mm width 66.5mm %罫線
\vspace{1mm}%余白削除
{\bf(対象)}

カード1枚を対象とする。
\vspace{1mm}%余白削除
\hrule height 0.1mm depth 0mm width 66.5mm %罫線
\vspace{1mm}%余白削除

{\bf(即時効果)}

対象のカードを手札、墓地から見つけられた場合、それを自分の切札として表向きにして置き、切札に対応した能力を発揮する。 

切札については「切札について」参照

切札に対応する能力については「アビリティリスト」参照

%\newpage %改段
%%%%%%%%%%%%%%%%%%%%%%%%%%%%%%
%%%%% AbilityList %%%%%
\begin{center}
\begin{center}
\hrule height 1mm depth 0mm width 66.5mm %罫線
\vspace{1mm}%余白削除
{\Large\bf \ruby{Ability List}{アビリティリスト}}
\vspace{1mm}%余白削除
\hrule height 0.5mm depth 0mm width 66.5mm %罫線
\end{center}
\end{center}
\vspace{-2zh}%余白削除
%%%%%%%%%%%%%%%%%%%%%%%%%%%%%%

%\tcbset{top=1mm, left=1mm, bottom=1mm, left=1mm}
\tcbset{colframe=black,coltitle=black!0!black,colbacktitle=white!0!white,colback=white!0!white,sharp corners,top=1mm, left=1mm, bottom=1mm, left=1mm,boxrule=0.2mm,toprule=0.2mm}


%%%%% Ability %%%%%
\vspace{3mm} %余白追加
\hrule height 0.5mm depth 0mm width 66.5mm %罫線
\vspace{1mm} %余白追加
{\Large\bf $\spadesuit$ A} {\normalsize\bf【革命】} %Abilityタイトル
\vspace{1mm} %余白追加

この能力が有効である時にダメージ判定アクションが効果を発揮した場合、ダメージ判定アクションの「1.兵士(アタッカー)と兵士(ブロッカー)の場合」を以下のように変更する。


\vspace{-1zh}%余白削除
\begin{enumerate}
\setlength{\leftskip}{-0.3cm}
\setlength{\parskip}{0pt} %4. 段落間余白.

\item 兵士(アタッカー)と兵士(ブロッカー)の場合、アタッカーとブロッカーで数字を比較し、大きい方を墓地に移動する。同じ場合は両方を墓地に移動する。アタッカーとブロッカーを比較した数字の差をダメージとして兵士を墓地に移した方のプレイヤーに与える。1アタッカーに対して複数ブロッカーがいる場合、ブロッカーの合計数字と比較する。
\vspace{-1zh}%余白削除
\end{enumerate}

\vspace{1mm}%余白削除

\hrule height 0.1mm depth 0mm width 66.5mm %罫線

\vspace{1mm}%余白削除



自分の兵士は以下の能力を得る。


\vspace{-1zh}%余白削除
\begin{itemize}
\setlength{\leftskip}{-0.3cm}
\setlength{\parskip}{0pt} %4. 段落間余白.

\item アタッカーかつダメージ判定アクションにてブロックされなかった場合、革命ドローアクションを起こす。
\vspace{-1zh}%余白削除
\end{itemize}

%%%%% Abilityアクション %%%%%
\begin{tcolorbox}[title={\small\bf【Action】革命ドロー}{\scriptsize (誘発)}]

{\scriptsize\bf @クイック }

\vspace{1mm} %余白追加
\hrule height 0.1mm depth 0mm width 62mm %罫線
\vspace{1mm} %余白追加

{\bf(即時効果)}

デッキの一番上からカードを1枚引き手札に加える。

\vspace{1mm} %余白追加 無いと箇条書きの場合、ギリギリになるので追加
\end{tcolorbox}

\vspace{-1zh}
  
%%%%% Abilityキャラクター %%%%%
 

%%%%% Ability %%%%%
\vspace{3mm} %余白追加
\hrule height 0.5mm depth 0mm width 66.5mm %罫線
\vspace{1mm} %余白追加
{\Large\bf $\spadesuit$ 2} {\normalsize\bf【スラッシュ】} %Abilityタイトル
\vspace{1mm} %余白追加

自分の{\normalsize $\spadesuit$} 兵士は以下の能力を得る。


\vspace{-1zh}%余白削除
\begin{itemize}
\setlength{\leftskip}{-0.3cm}
\setlength{\parskip}{0pt} %4. 段落間余白.

\item この兵士がチャージ状態でステージが空かつチャンスを持っている場合、ドライブしてスラッシュアクションを起こすことができる。
\vspace{-1zh}%余白削除
\end{itemize}

%%%%% Abilityアクション %%%%%
\begin{tcolorbox}[title={\small\bf【Action】スラッシュ}{\scriptsize (兵士起因)}]

{\scriptsize\bf @メイン }

\vspace{1mm} %余白追加  
\hrule height 0.1mm depth 0mm width 62mm %罫線
\vspace{1mm} %余白追加

{\bf(対象)}

兵士1体を対象とする。

\vspace{1mm} %余白追加
\hrule height 0.1mm depth 0mm width 62mm %罫線
\vspace{1mm} %余白追加

{\bf(即時効果)}

対象の兵士がこのアクションを起こした兵士の数字以下の場合、対象の兵士を墓地に移す。

\vspace{1mm} %余白追加 無いと箇条書きの場合、ギリギリになるので追加
\end{tcolorbox}

\vspace{-1zh}
  
%%%%% Abilityキャラクター %%%%%
 

%%%%% Ability %%%%%
\vspace{3mm} %余白追加
\hrule height 0.5mm depth 0mm width 66.5mm %罫線
\vspace{1mm} %余白追加
{\Large\bf $\spadesuit$ 3} {\normalsize\bf【ドレイン】} %Abilityタイトル
\vspace{1mm} %余白追加

ドレインアクションを起こすことができる。

%%%%% Abilityアクション %%%%%
\begin{tcolorbox}[title={\small\bf【Action】ドレイン}{\scriptsize (通常魔法)}]

{\scriptsize\bf @メイン }

\vspace{1mm} %余白追加  
\hrule height 0.1mm depth 0mm width 62mm %罫線
\vspace{1mm} %余白追加

{\bf(対象)}

対戦相手1人を対象とする。

\vspace{1mm} %余白追加
\hrule height 0.1mm depth 0mm width 62mm %罫線
\vspace{1mm} %余白追加

{\bf(通常効果)}

自分の場にいるキャラクター1体をデッキの一番下に移す。移せた場合、対象の対戦相手に2点のダメージを与える。

\vspace{1mm} %余白追加 無いと箇条書きの場合、ギリギリになるので追加
\end{tcolorbox}

\vspace{-1zh}
  
%%%%% Abilityキャラクター %%%%%
 

%%%%% Ability %%%%%
\vspace{3mm} %余白追加
\hrule height 0.5mm depth 0mm width 66.5mm %罫線
\vspace{1mm} %余白追加
{\Large\bf $\spadesuit$ 4} {\normalsize\bf【威圧】} %Abilityタイトル
\vspace{1mm} %余白追加

威圧アクションを起こすことができる。

%%%%% Abilityアクション %%%%%
\begin{tcolorbox}[title={\small\bf【Action】威圧}{\scriptsize (通常魔法)}]

{\scriptsize\bf @メイン }
  {\scriptsize\bf | \$ L }

\vspace{1mm} %余白追加
\hrule height 0.1mm depth 0mm width 62mm %罫線
\vspace{1mm} %余白追加

{\bf(通常効果)}

数字が10以下の全ての兵士を墓地に移す。

\vspace{1mm} %余白追加 無いと箇条書きの場合、ギリギリになるので追加
\end{tcolorbox}

\vspace{-1zh}
  
%%%%% Abilityキャラクター %%%%%
 

%%%%% Ability %%%%%
\vspace{3mm} %余白追加
\hrule height 0.5mm depth 0mm width 66.5mm %罫線
\vspace{1mm} %余白追加
{\Large\bf $\spadesuit$ 5} {\normalsize\bf【蘇生】} %Abilityタイトル
\vspace{1mm} %余白追加

蘇生アクションを起こすことができる。

%%%%% Abilityアクション %%%%%
\begin{tcolorbox}[title={\small\bf【Action】蘇生}{\scriptsize (通常魔法)}]

{\scriptsize\bf @メイン }
  {\scriptsize\bf | \$ BL }

\vspace{1mm} %余白追加
\hrule height 0.1mm depth 0mm width 62mm %罫線
\vspace{1mm} %余白追加

{\bf(通常効果)}

自分の墓地にある{\normalsize $\spadesuit$} 2〜10のカードを1枚選ぶ。選んだカードを兵士としてチャージ状態で場に出す。

\vspace{1mm} %余白追加 無いと箇条書きの場合、ギリギリになるので追加
\end{tcolorbox}

\vspace{-1zh}
  
%%%%% Abilityキャラクター %%%%%
 

%%%%% Ability %%%%%
\vspace{3mm} %余白追加
\hrule height 0.5mm depth 0mm width 66.5mm %罫線
\vspace{1mm} %余白追加
{\Large\bf $\spadesuit$ 7} {\normalsize\bf【瞬殺】} %Abilityタイトル
\vspace{1mm} %余白追加

瞬殺アクションを起こすことができる。

%%%%% Abilityアクション %%%%%
\begin{tcolorbox}[title={\small\bf【Action】瞬殺}{\scriptsize (速攻魔法)}]

{\scriptsize\bf @クイック }

\vspace{1mm} %余白追加  
\hrule height 0.1mm depth 0mm width 62mm %罫線
\vspace{1mm} %余白追加

{\bf(対象)}

兵士1体を対象とする。

\vspace{1mm} %余白追加
\hrule height 0.1mm depth 0mm width 62mm %罫線
\vspace{1mm} %余白追加

{\bf(即時効果)}

自分の場にいる兵士を1体以上任意の数だけ墓地に移す。この方法で墓地に移した兵士の数字の合計値が対象の兵士の数字以上の場合、対象の兵士を墓地に移す。

\vspace{1mm} %余白追加 無いと箇条書きの場合、ギリギリになるので追加
\end{tcolorbox}

\vspace{-1zh}
  
%%%%% Abilityキャラクター %%%%%
 

%%%%% Ability %%%%%
\vspace{3mm} %余白追加
\hrule height 0.5mm depth 0mm width 66.5mm %罫線
\vspace{1mm} %余白追加
{\Large\bf $\spadesuit$ J} {\normalsize\bf【リアニメーター】} %Abilityタイトル
\vspace{1mm} %余白追加

リアニメーター召喚アクションを起こすことができる。

%%%%% Abilityアクション %%%%%
\begin{tcolorbox}[title={\small\bf【Action】リアニメーター召喚}{\scriptsize (召喚)}]

{\scriptsize\bf @メイン }
  {\scriptsize\bf | \$ BL }
  {\scriptsize\bf | ★{\normalsize $\heartsuit$} 2〜10}

%特記事項
\vspace{1mm} %余白追加
\hrule height 0.1mm depth 0mm width 62mm %罫線
\vspace{1mm} %余白追加


\vspace{-1zh}%余白削除
\begin{enumerate}
\renewcommand{\labelenumi}{※}
\setlength{\leftskip}{-0.3cm}
\setlength{\itemsep}{0pt} %2. ブロック間の余白
\setlength{\parskip}{0pt} %4. 段落間余白.

\item コストを\$BDとすればタイミングをクイックとして起こすことができる。

\vspace{-3mm}%余白削除
\end{enumerate}
\vspace{-2mm} %余白削除 能力のアクションは余白が大きく開くため、追加で余白を削除
\vspace{1zh}%余白追加
\vspace{1mm} %余白追加
\hrule height 0.1mm depth 0mm width 62mm %罫線
\vspace{1mm} %余白追加

{\bf(通常効果)}

切札の{\normalsize $\spadesuit$} Jが表の場合、次を行う。


\vspace{-1zh}%余白削除
\begin{enumerate}
\setlength{\leftskip}{-0.3cm}
\setlength{\parskip}{0pt} %4. 段落間余白.

\item 墓地にあるカードを1枚選び、兵士として表向きかつチャージ状態で場に出す。

\item キーカードと切札の{\normalsize $\spadesuit$} Jをあわせてリアニメーターとしてチャージ状態で場に出す。
\vspace{-1zh}%余白削除
\end{enumerate}

\vspace{1mm} %余白追加 無いと箇条書きの場合、ギリギリになるので追加
\end{tcolorbox}

\vspace{-1zh}
  
%%%%% Abilityキャラクター %%%%%
\vspace{2mm}
\begin{tcolorbox}[title={\small\bf【Character】リアニメーター}{\scriptsize (兵士)}]

  {\scriptsize\bf ★{\normalsize $\spadesuit$} J と {\normalsize $\heartsuit$} 2〜10}

\vspace{1mm} %余白追加
\hrule height 0.1mm depth 0mm width 62mm %罫線
\vspace{1mm} %余白追加

{\bf(能力)}


\vspace{-1zh}%余白削除
\begin{itemize}
\setlength{\leftskip}{-0.3cm}
\setlength{\parskip}{0pt} %4. 段落間余白.

\item 装備アクションの対象にできない。

\item 準備(場に出たターンは、アタックアクションにて、対戦相手を攻撃する兵士(アタッカー)に指定することができない。)

\item 場から墓地もしくは手札に行く場合、対象を{\normalsize $\spadesuit$} Jとして切札再生アクションを誘発する。誘発については「誘発する場合」を参照。

\item 数字は11として扱う。

\item {\normalsize $\spadesuit$} 兵士としても{\normalsize $\heartsuit$} 兵士としても扱う。

\item 2枚で1体の兵士として扱い、墓地や手札など別の場所に移る場合2枚一緒に移す。防壁になる場合2体の防壁になる。

\item この兵士がチャージ状態でステージが空かつチャンスを持っている場合、ドライブして「リアニメイト」アクションをコストを支払わずに起こすことができる。アクションを起こした場合対象は任意に指定し、キーカードは仮想的に{\normalsize $\spadesuit$} Jと{\normalsize $\heartsuit$} Jとする。アクションを起こした後、チャンスをパスする。
\vspace{-1zh}%余白削除
\end{itemize}

\vspace{1mm} %余白追加 無いと箇条書きの場合、ギリギリになるので追加
\end{tcolorbox}

\vspace{-1zh}
 foreach end
 

%%%%% Ability %%%%%
\vspace{3mm} %余白追加
\hrule height 0.5mm depth 0mm width 66.5mm %罫線
\vspace{1mm} %余白追加
{\Large\bf $\spadesuit$ K} {\normalsize\bf【魔王】} %Abilityタイトル
\vspace{1mm} %余白追加

魔王召喚アクションを起こすことができる。

%%%%% Abilityアクション %%%%%
\begin{tcolorbox}[title={\small\bf【Action】魔王召喚}{\scriptsize (召喚)}]

{\scriptsize\bf @メイン }
  {\scriptsize\bf | \$ SS }
  {\scriptsize\bf | ★J〜K}

\vspace{1mm} %余白追加
\hrule height 0.1mm depth 0mm width 62mm %罫線
\vspace{1mm} %余白追加

{\bf(通常効果)}

切札の{\normalsize $\spadesuit$} Kが表の場合、次を行う。


\vspace{-1zh}%余白削除
\begin{enumerate}
\setlength{\leftskip}{-0.3cm}
\setlength{\parskip}{0pt} %4. 段落間余白.

\item キーカード、手札、デッキ、墓地、切札の中から{\normalsize $\clubsuit$} Kを見つける。

\item {\normalsize $\clubsuit$} Kを見つけた場合、場にいるすべての兵士を墓地に移し、切札の{\normalsize $\spadesuit$} Kと{\normalsize $\clubsuit$} Kを合わせて魔王としてチャージ状態で場に出す。

\item {\normalsize $\clubsuit$} Kが無い場合、切札の{\normalsize $\spadesuit$} Kを墓地に移す。

\item デッキを切りなおす。
\vspace{-1zh}%余白削除
\end{enumerate}

\vspace{1mm} %余白追加 無いと箇条書きの場合、ギリギリになるので追加
\end{tcolorbox}

\vspace{-1zh}
  
%%%%% Abilityキャラクター %%%%%
\vspace{2mm}
\begin{tcolorbox}[title={\small\bf【Character】魔王}{\scriptsize (兵士)}]

  {\scriptsize\bf ★{\normalsize $\spadesuit$} K と {\normalsize $\clubsuit$} K}

\vspace{1mm} %余白追加
\hrule height 0.1mm depth 0mm width 62mm %罫線
\vspace{1mm} %余白追加

{\bf(能力)}


\vspace{-1zh}%余白削除
\begin{itemize}
\setlength{\leftskip}{-0.3cm}
\setlength{\parskip}{0pt} %4. 段落間余白.

\item アクションの対象にできない。

\item 準備(このキャラクターがこのターンに場に出たカードのみで構成されている場合、アタックアクションにて対戦相手を攻撃する兵士(アタッカー)に指定することができない。)

\item 場から墓地に行く場合、世代交代アクションを2回誘発する。

\item 自分がエンドアクションを起こした場合、世代交代アクションを誘発する。

\item 数字は26として扱う。

\item {\normalsize $\clubsuit$} 兵士としても{\normalsize $\spadesuit$} 兵士としても扱う。

\item 2枚で1体の兵士として扱い、墓地や手札など別の場所に移る場合2枚一緒に移す。防壁になる場合2体の防壁になる。
\vspace{-1zh}%余白削除
\end{itemize}

\vspace{1mm} %余白追加 無いと箇条書きの場合、ギリギリになるので追加
\end{tcolorbox}

\vspace{-1zh}
 foreach end
 

%%%%% Ability %%%%%
\vspace{3mm} %余白追加
\hrule height 0.5mm depth 0mm width 66.5mm %罫線
\vspace{1mm} %余白追加
{\Large\bf $\heartsuit$ 2} {\normalsize\bf【援軍】} %Abilityタイトル
\vspace{1mm} %余白追加

援軍アクションを起こすことができる。

%%%%% Abilityアクション %%%%%
\begin{tcolorbox}[title={\small\bf【Action】援軍}{\scriptsize (通常魔法)}]

{\scriptsize\bf @メイン }
  {\scriptsize\bf | \$ BL }

%特記事項
\vspace{1mm} %余白追加
\hrule height 0.1mm depth 0mm width 62mm %罫線
\vspace{1mm} %余白追加


\vspace{-1zh}%余白削除
\begin{enumerate}
\renewcommand{\labelenumi}{※}
\setlength{\leftskip}{-0.3cm}
\setlength{\itemsep}{0pt} %2. ブロック間の余白
\setlength{\parskip}{0pt} %4. 段落間余白.

\item プレイヤーは1ターンに1回しかこのアクションを起こすことができない。

\vspace{-3mm}%余白削除
\end{enumerate}
\vspace{-2mm} %余白削除 能力のアクションは余白が大きく開くため、追加で余白を削除
\vspace{1zh}%余白追加
\vspace{1mm} %余白追加
\hrule height 0.1mm depth 0mm width 62mm %罫線
\vspace{1mm} %余白追加

{\bf(通常効果)}

デッキの一番上から2枚カードをめくり1枚を兵士としてチャージ状態で場に出し、もう1枚をデッキの一番下に移す。

\vspace{1mm} %余白追加 無いと箇条書きの場合、ギリギリになるので追加
\end{tcolorbox}

\vspace{-1zh}
  
%%%%% Abilityキャラクター %%%%%
 

%%%%% Ability %%%%%
\vspace{3mm} %余白追加
\hrule height 0.5mm depth 0mm width 66.5mm %罫線
\vspace{1mm} %余白追加
{\Large\bf $\heartsuit$ 3} {\normalsize\bf【激励】} %Abilityタイトル
\vspace{1mm} %余白追加

自分の{\normalsize $\heartsuit$} 兵士は以下の能力を得る。


\vspace{-1zh}%余白削除
\begin{itemize}
\setlength{\leftskip}{-0.3cm}
\setlength{\parskip}{0pt} %4. 段落間余白.

\item この兵士がチャージ状態かつチャンスを持っている場合、ドライブして激励アクションを起こすことができる。
\vspace{-1zh}%余白削除
\end{itemize}

%%%%% Abilityアクション %%%%%
\begin{tcolorbox}[title={\small\bf【Action】激励}{\scriptsize (兵士起因)}]

{\scriptsize\bf @クイック }

\vspace{1mm} %余白追加  
\hrule height 0.1mm depth 0mm width 62mm %罫線
\vspace{1mm} %余白追加

{\bf(対象)}

兵士1体を対象とする。

\vspace{1mm} %余白追加
\hrule height 0.1mm depth 0mm width 62mm %罫線
\vspace{1mm} %余白追加

{\bf(即時効果)}

対象の兵士にこのアクションを起こした兵士の数字をターン終了時まで加算する。

\vspace{1mm} %余白追加 無いと箇条書きの場合、ギリギリになるので追加
\end{tcolorbox}

\vspace{-1zh}
  
%%%%% Abilityキャラクター %%%%%
 

%%%%% Ability %%%%%
\vspace{3mm} %余白追加
\hrule height 0.5mm depth 0mm width 66.5mm %罫線
\vspace{1mm} %余白追加
{\Large\bf $\heartsuit$ 4} {\normalsize\bf【休戦】} %Abilityタイトル
\vspace{1mm} %余白追加

休戦アクションを起こすことができる。

%%%%% Abilityアクション %%%%%
\begin{tcolorbox}[title={\small\bf【Action】休戦}{\scriptsize (速攻魔法)}]

{\scriptsize\bf @クイック }

\vspace{1mm} %余白追加  
\hrule height 0.1mm depth 0mm width 62mm %罫線
\vspace{1mm} %余白追加

{\bf(対象)}

アタックアクションを対象とする。

\vspace{1mm} %余白追加
\hrule height 0.1mm depth 0mm width 62mm %罫線
\vspace{1mm} %余白追加

{\bf(即時効果)}


\vspace{-1zh}%余白削除
\begin{itemize}
\setlength{\leftskip}{-0.3cm}
\setlength{\parskip}{0pt} %4. 段落間余白.

\item 対象のアクションは効果を発揮しない。

\item 自分の切札にある{\normalsize $\heartsuit$} 4を裏向きにする。

\item 自分の場にいるキャラクターを全てチャージ状態にする。
\vspace{-1zh}%余白削除
\end{itemize}

\vspace{1mm} %余白追加 無いと箇条書きの場合、ギリギリになるので追加
\end{tcolorbox}

\vspace{-1zh}
  
%%%%% Abilityキャラクター %%%%%
 

%%%%% Ability %%%%%
\vspace{3mm} %余白追加
\hrule height 0.5mm depth 0mm width 66.5mm %罫線
\vspace{1mm} %余白追加
{\Large\bf $\heartsuit$ 5} {\normalsize\bf【追撃】} %Abilityタイトル
\vspace{1mm} %余白追加

追撃アクションを起こすことができる。

%%%%% Abilityアクション %%%%%
\begin{tcolorbox}[title={\small\bf【Action】追撃}{\scriptsize (通常魔法)}]

{\scriptsize\bf @メイン }
  {\scriptsize\bf | \$ BDD }

%特記事項
\vspace{1mm} %余白追加
\hrule height 0.1mm depth 0mm width 62mm %罫線
\vspace{1mm} %余白追加


\vspace{-1zh}%余白削除
\begin{enumerate}
\renewcommand{\labelenumi}{※}
\setlength{\leftskip}{-0.3cm}
\setlength{\itemsep}{0pt} %2. ブロック間の余白
\setlength{\parskip}{0pt} %4. 段落間余白.

\item プレイヤーは1ターンに1回しかこのアクションを起こすことができない。

\vspace{-3mm}%余白削除
\end{enumerate}
\vspace{-2mm} %余白削除 能力のアクションは余白が大きく開くため、追加で余白を削除
\vspace{1zh}%余白追加
\vspace{1mm} %余白追加
\hrule height 0.1mm depth 0mm width 62mm %罫線
\vspace{1mm} %余白追加

{\bf(即時効果)}


\vspace{-1zh}%余白削除
\begin{itemize}
\setlength{\leftskip}{-0.3cm}
\setlength{\parskip}{0pt} %4. 段落間余白.

\item 自分の場にいる兵士を全てチャージ状態にする。

\item アタックアクションを起こす。
\vspace{-1zh}%余白削除
\end{itemize}

\vspace{1mm} %余白追加 無いと箇条書きの場合、ギリギリになるので追加
\end{tcolorbox}

\vspace{-1zh}
  
%%%%% Abilityキャラクター %%%%%
 

%%%%% Ability %%%%%
\vspace{3mm} %余白追加
\hrule height 0.5mm depth 0mm width 66.5mm %罫線
\vspace{1mm} %余白追加
{\Large\bf $\heartsuit$ 7} {\normalsize\bf【ファントム】} %Abilityタイトル
\vspace{1mm} %余白追加


\vspace{-1zh}%余白削除
\begin{itemize}
\setlength{\leftskip}{-0.3cm}
\setlength{\parskip}{0pt} %4. 段落間余白.

\item ファントムアクションを起こすことができる。

\item 自分がエンドアクションを起こした場合、幻影アクションを誘発する。
\vspace{-1zh}%余白削除
\end{itemize}

%%%%% Abilityアクション %%%%%
\begin{tcolorbox}[title={\small\bf【Action】ファントム}{\scriptsize (通常魔法)}]

{\scriptsize\bf @メイン }
  {\scriptsize\bf | \$ B }

\vspace{1mm} %余白追加  
\hrule height 0.1mm depth 0mm width 62mm %罫線
\vspace{1mm} %余白追加

{\bf(対象)}

自分の場にいるキャラクター1体を対象とする。

\vspace{1mm} %余白追加
\hrule height 0.1mm depth 0mm width 62mm %罫線
\vspace{1mm} %余白追加

{\bf(即時効果)}

「リバース」アクションをコストを支払わずに起こす。対象は対象の自分の場にいるキャラクター、キーカードは仮想的に{\normalsize $\heartsuit$} 7と{\normalsize $\spadesuit$} 7とする。アクションを起こした後、チャンスをパスする。

\vspace{1mm} %余白追加 無いと箇条書きの場合、ギリギリになるので追加
\end{tcolorbox}

\vspace{-1zh}
\begin{tcolorbox}[title={\small\bf【Action】幻影}{\scriptsize (誘発)}]

{\scriptsize\bf @クイック }

%特記事項
\vspace{1mm} %余白追加
\hrule height 0.1mm depth 0mm width 62mm %罫線
\vspace{1mm} %余白追加


\vspace{-1zh}%余白削除
\begin{enumerate}
\renewcommand{\labelenumi}{※}
\setlength{\leftskip}{-0.3cm}
\setlength{\itemsep}{0pt} %2. ブロック間の余白
\setlength{\parskip}{0pt} %4. 段落間余白.

\item プレイヤーはこのアクションを直接起こすことができない。

\vspace{-3mm}%余白削除
\end{enumerate}
\vspace{-2mm} %余白削除 能力のアクションは余白が大きく開くため、追加で余白を削除
\vspace{1zh}%余白追加
\vspace{1mm} %余白追加
\hrule height 0.1mm depth 0mm width 62mm %罫線
\vspace{1mm} %余白追加

{\bf(通常効果)}


\vspace{-1zh}%余白削除
\begin{enumerate}
\setlength{\leftskip}{-0.3cm}
\setlength{\parskip}{0pt} %4. 段落間余白.

\item 自分の場にいる兵士を好きな数だけ選び、チャージ状態の防壁にする。兵士の時に受けた効果、能力は無くなる。兵士が複数のカードから成る場合、1枚ずつ防壁にする。

\item 自分の場にいる防壁の順番を任意の順番に並び替えてもよい。
\vspace{-1zh}%余白削除
\end{enumerate}

\vspace{1mm} %余白追加 無いと箇条書きの場合、ギリギリになるので追加
\end{tcolorbox}

\vspace{-1zh}
  
%%%%% Abilityキャラクター %%%%%
 

%%%%% Ability %%%%%
\vspace{3mm} %余白追加
\hrule height 0.5mm depth 0mm width 66.5mm %罫線
\vspace{1mm} %余白追加
{\Large\bf $\heartsuit$ 8} {\normalsize\bf【救済】} %Abilityタイトル
\vspace{1mm} %余白追加

救済アクションを起こすことができる。

%%%%% Abilityアクション %%%%%
\begin{tcolorbox}[title={\small\bf【Action】救済}{\scriptsize (通常魔法)}]

{\scriptsize\bf @メイン }
  {\scriptsize\bf | \$ B }
  {\scriptsize\bf | ★{\normalsize $\heartsuit$} A〜K}

\vspace{1mm} %余白追加
\hrule height 0.1mm depth 0mm width 62mm %罫線
\vspace{1mm} %余白追加

{\bf(通常効果)}

キーカードの数字以上の全ての兵士をオーナーのデッキの一番上にオーナーの好きな順で移す。

\vspace{1mm} %余白追加 無いと箇条書きの場合、ギリギリになるので追加
\end{tcolorbox}

\vspace{-1zh}
  
%%%%% Abilityキャラクター %%%%%
 

%%%%% Ability %%%%%
\vspace{3mm} %余白追加
\hrule height 0.5mm depth 0mm width 66.5mm %罫線
\vspace{1mm} %余白追加
{\Large\bf $\heartsuit$ J} {\normalsize\bf【チャリオット】} %Abilityタイトル
\vspace{1mm} %余白追加

チャリオット召喚アクションを起こすことができる。

%%%%% Abilityアクション %%%%%
\begin{tcolorbox}[title={\small\bf【Action】チャリオット召喚}{\scriptsize (召喚)}]

{\scriptsize\bf @メイン }
  {\scriptsize\bf | \$ BL }
  {\scriptsize\bf | ★{\normalsize $\diamondsuit$} 2〜10}

%特記事項
\vspace{1mm} %余白追加
\hrule height 0.1mm depth 0mm width 62mm %罫線
\vspace{1mm} %余白追加


\vspace{-1zh}%余白削除
\begin{enumerate}
\renewcommand{\labelenumi}{※}
\setlength{\leftskip}{-0.3cm}
\setlength{\itemsep}{0pt} %2. ブロック間の余白
\setlength{\parskip}{0pt} %4. 段落間余白.

\item コストを\$BDとすればタイミングをクイックとして起こすことができる。

\vspace{-3mm}%余白削除
\end{enumerate}
\vspace{-2mm} %余白削除 能力のアクションは余白が大きく開くため、追加で余白を削除
\vspace{1zh}%余白追加
\vspace{1mm} %余白追加
\hrule height 0.1mm depth 0mm width 62mm %罫線
\vspace{1mm} %余白追加

{\bf(通常効果)}

切札の{\normalsize $\heartsuit$} Jが表の場合、次を行う。


\vspace{-1zh}%余白削除
\begin{enumerate}
\setlength{\leftskip}{-0.3cm}
\setlength{\parskip}{0pt} %4. 段落間余白.

\item 防壁1体を選び墓地に移す。

\item キーカードと切札の{\normalsize $\heartsuit$} Jをあわせてチャリオットとしてチャージ状態で場に出す。
\vspace{-1zh}%余白削除
\end{enumerate}

\vspace{1mm} %余白追加 無いと箇条書きの場合、ギリギリになるので追加
\end{tcolorbox}

\vspace{-1zh}
  
%%%%% Abilityキャラクター %%%%%
\vspace{2mm}
\begin{tcolorbox}[title={\small\bf【Character】チャリオット}{\scriptsize (兵士)}]

  {\scriptsize\bf ★{\normalsize $\heartsuit$} J と {\normalsize $\diamondsuit$} 2〜10}

\vspace{1mm} %余白追加
\hrule height 0.1mm depth 0mm width 62mm %罫線
\vspace{1mm} %余白追加

{\bf(能力)}


\vspace{-1zh}%余白削除
\begin{itemize}
\setlength{\leftskip}{-0.3cm}
\setlength{\parskip}{0pt} %4. 段落間余白.

\item 装備アクションの対象にできない。

\item 準備(場に出たターンは、アタックアクションにて、対戦相手を攻撃する兵士(アタッカー)に指定することができない。)

\item 場から墓地もしくは手札に行く場合、対象を{\normalsize $\heartsuit$} Jとして切札再生アクションを誘発する。誘発については「誘発する場合」を参照。

\item 数字は11として扱う。

\item {\normalsize $\heartsuit$} 兵士としても{\normalsize $\diamondsuit$} 兵士としても扱う。

\item 2枚で1体の兵士として扱い、墓地や手札など別の場所に移る場合2枚一緒に移す。防壁になる場合2体の防壁になる。

\item この兵士がチャージ状態でステージが空かつチャンスを持っている場合、ドライブして「防壁破壊」アクションをコストを支払わずに起こすことができる。アクションを起こした場合対象は任意に指定し、キーカードは仮想的に{\normalsize $\heartsuit$} Jと{\normalsize $\diamondsuit$} Jとする。アクションを起こした後、チャンスをパスする。
\vspace{-1zh}%余白削除
\end{itemize}

\vspace{1mm} %余白追加 無いと箇条書きの場合、ギリギリになるので追加
\end{tcolorbox}

\vspace{-1zh}
 foreach end
 

%%%%% Ability %%%%%
\vspace{3mm} %余白追加
\hrule height 0.5mm depth 0mm width 66.5mm %罫線
\vspace{1mm} %余白追加
{\Large\bf $\heartsuit$ K} {\normalsize\bf【巨人】} %Abilityタイトル
\vspace{1mm} %余白追加

巨人召喚アクションを起こすことができる。

%%%%% Abilityアクション %%%%%
\begin{tcolorbox}[title={\small\bf【Action】巨人召喚}{\scriptsize (召喚)}]

{\scriptsize\bf @メイン }
  {\scriptsize\bf | \$ SS }
  {\scriptsize\bf | ★J〜K}

\vspace{1mm} %余白追加
\hrule height 0.1mm depth 0mm width 62mm %罫線
\vspace{1mm} %余白追加

{\bf(通常効果)}

切札の{\normalsize $\heartsuit$} Kが表の場合、次を行う。


\vspace{-1zh}%余白削除
\begin{enumerate}
\setlength{\leftskip}{-0.3cm}
\setlength{\parskip}{0pt} %4. 段落間余白.

\item キーカード、手札、デッキ、墓地、切札の中から{\normalsize $\diamondsuit$} Kを見つける。

\item {\normalsize $\diamondsuit$} Kを見つけた場合、場にいるすべての防壁を墓地に移し、切札の{\normalsize $\heartsuit$} Kと{\normalsize $\diamondsuit$} Kを合わせて巨人としてチャージ状態で場に出す。

\item {\normalsize $\diamondsuit$} Kが無い場合、切札の{\normalsize $\heartsuit$} Kを墓地に移す。

\item デッキを切りなおす。
\vspace{-1zh}%余白削除
\end{enumerate}

\vspace{1mm} %余白追加 無いと箇条書きの場合、ギリギリになるので追加
\end{tcolorbox}

\vspace{-1zh}
  
%%%%% Abilityキャラクター %%%%%
\vspace{2mm}
\begin{tcolorbox}[title={\small\bf【Character】巨人}{\scriptsize (兵士)}]

  {\scriptsize\bf ★{\normalsize $\heartsuit$} K と {\normalsize $\diamondsuit$} K}

\vspace{1mm} %余白追加
\hrule height 0.1mm depth 0mm width 62mm %罫線
\vspace{1mm} %余白追加

{\bf(能力)}


\vspace{-1zh}%余白削除
\begin{itemize}
\setlength{\leftskip}{-0.3cm}
\setlength{\parskip}{0pt} %4. 段落間余白.

\item アクションの対象にできない。

\item 準備(このキャラクターがこのターンに場に出たカードのみで構成されている場合、アタックアクションにて対戦相手を攻撃する兵士(アタッカー)に指定することができない。)

\item 場から墓地に行く場合、世代交代アクションを2回誘発する。

\item 自分のターンに可能な限りアタックアクションを起し、巨人がチャージ状態の場合、アタッカーに指定する。

\item 防壁の効果によって墓地に移されない。

\item 数字は26として扱う。

\item {\normalsize $\heartsuit$} 兵士としても{\normalsize $\diamondsuit$} 兵士としても扱う。

\item 2枚で1体の兵士として扱い、墓地や手札など別の場所に移る場合2枚一緒に移す。防壁になる場合2体の防壁になる。
\vspace{-1zh}%余白削除
\end{itemize}

\vspace{1mm} %余白追加 無いと箇条書きの場合、ギリギリになるので追加
\end{tcolorbox}

\vspace{-1zh}
 foreach end
 

%%%%% Ability %%%%%
\vspace{3mm} %余白追加
\hrule height 0.5mm depth 0mm width 66.5mm %罫線
\vspace{1mm} %余白追加
{\Large\bf $\diamondsuit$ A} {\normalsize\bf【防壁追加】} %Abilityタイトル
\vspace{1mm} %余白追加

防壁追加アクションを起こすことができる。

%%%%% Abilityアクション %%%%%
\begin{tcolorbox}[title={\small\bf【Action】防壁追加}{\scriptsize (通常魔法)}]

{\scriptsize\bf @メイン }
  {\scriptsize\bf | \$ L }

%特記事項
\vspace{1mm} %余白追加
\hrule height 0.1mm depth 0mm width 62mm %罫線
\vspace{1mm} %余白追加


\vspace{-1zh}%余白削除
\begin{enumerate}
\renewcommand{\labelenumi}{※}
\setlength{\leftskip}{-0.3cm}
\setlength{\itemsep}{0pt} %2. ブロック間の余白
\setlength{\parskip}{0pt} %4. 段落間余白.

\item プレイヤーは1ターンに1回しかこのアクションを起こすことができない。

\vspace{-3mm}%余白削除
\end{enumerate}
\vspace{-2mm} %余白削除 能力のアクションは余白が大きく開くため、追加で余白を削除
\vspace{1zh}%余白追加
\vspace{1mm} %余白追加
\hrule height 0.1mm depth 0mm width 62mm %罫線
\vspace{1mm} %余白追加

{\bf(即時効果)}

対戦相手の場にいる防壁の数が自分の場にいる防壁の数以上の場合、自分のデッキの一番上から1枚を防壁として裏向きかつチャージ状態で場に出す。

\vspace{1mm} %余白追加 無いと箇条書きの場合、ギリギリになるので追加
\end{tcolorbox}

\vspace{-1zh}
  
%%%%% Abilityキャラクター %%%%%
 

%%%%% Ability %%%%%
\vspace{3mm} %余白追加
\hrule height 0.5mm depth 0mm width 66.5mm %罫線
\vspace{1mm} %余白追加
{\Large\bf $\diamondsuit$ 3} {\normalsize\bf【交渉】} %Abilityタイトル
\vspace{1mm} %余白追加

自分の{\normalsize $\diamondsuit$} 兵士は以下の能力を得る。


\vspace{-1zh}%余白削除
\begin{itemize}
\setlength{\leftskip}{-0.3cm}
\setlength{\parskip}{0pt} %4. 段落間余白.

\item この兵士がチャージ状態かつチャンスを持っている場合、ドライブして交渉アクションを起こすことができる。
\vspace{-1zh}%余白削除
\end{itemize}

%%%%% Abilityアクション %%%%%
\begin{tcolorbox}[title={\small\bf【Action】交渉}{\scriptsize (兵士起因)}]

{\scriptsize\bf @クイック }

\vspace{1mm} %余白追加  
\hrule height 0.1mm depth 0mm width 62mm %罫線
\vspace{1mm} %余白追加

{\bf(対象)}

キャラクター1体を対象とする。

\vspace{1mm} %余白追加
\hrule height 0.1mm depth 0mm width 62mm %罫線
\vspace{1mm} %余白追加

{\bf(即時効果)}

対象のキャラクターをドライブする。

\vspace{1mm} %余白追加 無いと箇条書きの場合、ギリギリになるので追加
\end{tcolorbox}

\vspace{-1zh}
  
%%%%% Abilityキャラクター %%%%%
 

%%%%% Ability %%%%%
\vspace{3mm} %余白追加
\hrule height 0.5mm depth 0mm width 66.5mm %罫線
\vspace{1mm} %余白追加
{\Large\bf $\diamondsuit$ 4} {\normalsize\bf【相殺】} %Abilityタイトル
\vspace{1mm} %余白追加

相殺アクションを起こすことができる。

%%%%% Abilityアクション %%%%%
\begin{tcolorbox}[title={\small\bf【Action】相殺}{\scriptsize (速攻魔法)}]

{\scriptsize\bf @クイック }
  {\scriptsize\bf | \$ D }

\vspace{1mm} %余白追加  
\hrule height 0.1mm depth 0mm width 62mm %罫線
\vspace{1mm} %余白追加

{\bf(対象)}

キーカードがあるアクションを対象とする。

\vspace{1mm} %余白追加
\hrule height 0.1mm depth 0mm width 62mm %罫線
\vspace{1mm} %余白追加

{\bf(即時効果)}

対象アクションのキーカードの数字の合計値が5以上の場合、次を行う。


\vspace{-1zh}%余白削除
\begin{enumerate}
\setlength{\leftskip}{-0.3cm}
\setlength{\parskip}{0pt} %4. 段落間余白.

\item 対象アクションのキーカードの数字の合計値分ダメージを受ける。

\item 対象アクションをステージから取り除き、対象アクションのキーカードを墓地に移す。
\vspace{-1zh}%余白削除
\end{enumerate}

\vspace{1mm} %余白追加 無いと箇条書きの場合、ギリギリになるので追加
\end{tcolorbox}

\vspace{-1zh}
  
%%%%% Abilityキャラクター %%%%%
 

%%%%% Ability %%%%%
\vspace{3mm} %余白追加
\hrule height 0.5mm depth 0mm width 66.5mm %罫線
\vspace{1mm} %余白追加
{\Large\bf $\diamondsuit$ 5} {\normalsize\bf【喪失】} %Abilityタイトル
\vspace{1mm} %余白追加


\vspace{-1zh}%余白削除
\begin{itemize}
\setlength{\leftskip}{-0.3cm}
\setlength{\parskip}{0pt} %4. 段落間余白.

\item この能力が有効でありかつ、自分の場にチャージ状態の防壁が2枚以上存在する時にチャージアクションが効果を発揮した場合、以下のようにチャージアクションの効果を変更する。

\vspace{-1zh}%余白削除
\begin{itemize}
\setlength{\leftskip}{-0.3cm}
\setlength{\parskip}{0pt} %4. 段落間余白.

\item ドローアクションを起こす。
\vspace{-1zh}%余白削除
\end{itemize}
\item 自分がエンドアクションを起こした場合、疲弊アクションを誘発する。
\vspace{-1zh}%余白削除
\end{itemize}

%%%%% Abilityアクション %%%%%
\begin{tcolorbox}[title={\small\bf【Action】疲弊}{\scriptsize (誘発)}]

{\scriptsize\bf @クイック }

%特記事項
\vspace{1mm} %余白追加
\hrule height 0.1mm depth 0mm width 62mm %罫線
\vspace{1mm} %余白追加


\vspace{-1zh}%余白削除
\begin{enumerate}
\renewcommand{\labelenumi}{※}
\setlength{\leftskip}{-0.3cm}
\setlength{\itemsep}{0pt} %2. ブロック間の余白
\setlength{\parskip}{0pt} %4. 段落間余白.

\item プレイヤーはこのアクションを直接起こすことができない。

\vspace{-3mm}%余白削除
\end{enumerate}
\vspace{-2mm} %余白削除 能力のアクションは余白が大きく開くため、追加で余白を削除
\vspace{1zh}%余白追加
\vspace{1mm} %余白追加
\hrule height 0.1mm depth 0mm width 62mm %罫線
\vspace{1mm} %余白追加

{\bf(通常効果)}

自分の切札にある{\normalsize $\diamondsuit$} 5を表向きから裏向きにする。切札を裏向きにしたくない場合、2点のダメージを受ける。

\vspace{1mm} %余白追加 無いと箇条書きの場合、ギリギリになるので追加
\end{tcolorbox}

\vspace{-1zh}
  
%%%%% Abilityキャラクター %%%%%
 

%%%%% Ability %%%%%
\vspace{3mm} %余白追加
\hrule height 0.5mm depth 0mm width 66.5mm %罫線
\vspace{1mm} %余白追加
{\Large\bf $\diamondsuit$ 8} {\normalsize\bf【修繕】} %Abilityタイトル
\vspace{1mm} %余白追加

修繕アクションを起こすことができる。

%%%%% Abilityアクション %%%%%
\begin{tcolorbox}[title={\small\bf【Action】修繕}{\scriptsize (通常魔法)}]

{\scriptsize\bf @メイン }
  {\scriptsize\bf | \$ L }

\vspace{1mm} %余白追加  
\hrule height 0.1mm depth 0mm width 62mm %罫線
\vspace{1mm} %余白追加

{\bf(対象)}

自分の場にいる{\normalsize $\diamondsuit$} 2〜10のカードを含んだ兵士1体を対象とする。

\vspace{1mm} %余白追加
\hrule height 0.1mm depth 0mm width 62mm %罫線
\vspace{1mm} %余白追加

{\bf(即時効果)}

対象の兵士が装備アクションの対象とできる場合、次を行う。


\vspace{-1zh}%余白削除
\begin{enumerate}
\setlength{\leftskip}{-0.3cm}
\setlength{\parskip}{0pt} %4. 段落間余白.

\item 自分の墓地から{\normalsize $\diamondsuit$} J〜Kを1枚選ぶ。

\item 選べた場合、「装備」アクションをコストを支払わずに起こす。対象は対象の兵士、キーカードは選んだカードとする。アクションを起こした後、チャンスをパスする。

\item 選べなかった場合、対象の兵士を墓地に移す。
\vspace{-1zh}%余白削除
\end{enumerate}

\vspace{1mm} %余白追加 無いと箇条書きの場合、ギリギリになるので追加
\end{tcolorbox}

\vspace{-1zh}
  
%%%%% Abilityキャラクター %%%%%
 

%%%%% Ability %%%%%
\vspace{3mm} %余白追加
\hrule height 0.5mm depth 0mm width 66.5mm %罫線
\vspace{1mm} %余白追加
{\Large\bf $\diamondsuit$ 9} {\normalsize\bf【充足 ・休息アクションを起こすことができる。 ・自分がエンドアクションを起こした場合、充足アクションを誘発する。】} %Abilityタイトル
\vspace{1mm} %余白追加



%%%%% Abilityアクション %%%%%
\begin{tcolorbox}[title={\small\bf【Action】休息}{\scriptsize (通常魔法)}]

{\scriptsize\bf @メイン }
  {\scriptsize\bf | \$ B }

\vspace{1mm} %余白追加
\hrule height 0.1mm depth 0mm width 62mm %罫線
\vspace{1mm} %余白追加

{\bf(通常効果)}

自分の場にいる全ての兵士をチャージ状態にする。

\vspace{1mm} %余白追加 無いと箇条書きの場合、ギリギリになるので追加
\end{tcolorbox}

\vspace{-1zh}
\begin{tcolorbox}[title={\small\bf【Action】充足}{\scriptsize (誘発)}]

{\scriptsize\bf @クイック }

%特記事項
\vspace{1mm} %余白追加
\hrule height 0.1mm depth 0mm width 62mm %罫線
\vspace{1mm} %余白追加


\vspace{-1zh}%余白削除
\begin{enumerate}
\renewcommand{\labelenumi}{※}
\setlength{\leftskip}{-0.3cm}
\setlength{\itemsep}{0pt} %2. ブロック間の余白
\setlength{\parskip}{0pt} %4. 段落間余白.

\item プレイヤーはこのアクションを直接起こすことができない。

\vspace{-3mm}%余白削除
\end{enumerate}
\vspace{-2mm} %余白削除 能力のアクションは余白が大きく開くため、追加で余白を削除
\vspace{1zh}%余白追加
\vspace{1mm} %余白追加
\hrule height 0.1mm depth 0mm width 62mm %罫線
\vspace{1mm} %余白追加

{\bf(通常効果)}

自分の場にいる全てのキャラクターをチャージ状態にする。

\vspace{1mm} %余白追加 無いと箇条書きの場合、ギリギリになるので追加
\end{tcolorbox}

\vspace{-1zh}
  
%%%%% Abilityキャラクター %%%%%
 

%%%%% Ability %%%%%
\vspace{3mm} %余白追加
\hrule height 0.5mm depth 0mm width 66.5mm %罫線
\vspace{1mm} %余白追加
{\Large\bf $\diamondsuit$ J} {\normalsize\bf【策士】} %Abilityタイトル
\vspace{1mm} %余白追加

策士召喚アクションを起こすことができる。

%%%%% Abilityアクション %%%%%
\begin{tcolorbox}[title={\small\bf【Action】策士召喚}{\scriptsize (召喚)}]

{\scriptsize\bf @メイン }
  {\scriptsize\bf | \$ BL }
  {\scriptsize\bf | ★{\normalsize $\clubsuit$} 2〜10}

%特記事項
\vspace{1mm} %余白追加
\hrule height 0.1mm depth 0mm width 62mm %罫線
\vspace{1mm} %余白追加


\vspace{-1zh}%余白削除
\begin{enumerate}
\renewcommand{\labelenumi}{※}
\setlength{\leftskip}{-0.3cm}
\setlength{\itemsep}{0pt} %2. ブロック間の余白
\setlength{\parskip}{0pt} %4. 段落間余白.

\item コストを\$BDとすればタイミングをクイックとして起こすことができる。

\vspace{-3mm}%余白削除
\end{enumerate}
\vspace{-2mm} %余白削除 能力のアクションは余白が大きく開くため、追加で余白を削除
\vspace{1zh}%余白追加
\vspace{1mm} %余白追加
\hrule height 0.1mm depth 0mm width 62mm %罫線
\vspace{1mm} %余白追加

{\bf(通常効果)}

切札の{\normalsize $\diamondsuit$} Jが表の場合、次を行う。


\vspace{-1zh}%余白削除
\begin{enumerate}
\setlength{\leftskip}{-0.3cm}
\setlength{\parskip}{0pt} %4. 段落間余白.

\item ステージにあるアクションを1つ選び無効化しステージから取り除く。そのアクションは次のいずれかに該当するものとする。Xはこのアクションのキーカードの数字とする。

\vspace{-1zh}%余白削除
\begin{itemize}
\setlength{\leftskip}{-0.3cm}
\setlength{\parskip}{0pt} %4. 段落間余白.

\item キーカードが1枚かつそのキーカードの数字がX以下

\item キーカードが2枚
\vspace{-1zh}%余白削除
\end{itemize}
\item キーカードと切札の{\normalsize $\diamondsuit$} Jをあわせて策士としてチャージ状態で場に出す。
\vspace{-1zh}%余白削除
\end{enumerate}

\vspace{1mm} %余白追加 無いと箇条書きの場合、ギリギリになるので追加
\end{tcolorbox}

\vspace{-1zh}
\begin{tcolorbox}[title={\small\bf【Action】先読み}{\scriptsize (兵士起因)}]

{\scriptsize\bf @クイック }

\vspace{1mm} %余白追加
\hrule height 0.1mm depth 0mm width 62mm %罫線
\vspace{1mm} %余白追加

{\bf(即時効果)}


\vspace{-1zh}%余白削除
\begin{enumerate}
\setlength{\leftskip}{-0.3cm}
\setlength{\parskip}{0pt} %4. 段落間余白.

\item デッキの一番上からカードを2枚引き手札に加える。

\item 手札から1枚選びデッキの一番上もしくは一番下に移す。
\vspace{-1zh}%余白削除
\end{enumerate}

\vspace{1mm} %余白追加 無いと箇条書きの場合、ギリギリになるので追加
\end{tcolorbox}

\vspace{-1zh}
  
%%%%% Abilityキャラクター %%%%%
\vspace{2mm}
\begin{tcolorbox}[title={\small\bf【Character】策士}{\scriptsize (兵士)}]

  {\scriptsize\bf ★{\normalsize $\diamondsuit$} J と {\normalsize $\clubsuit$} 2〜10}

\vspace{1mm} %余白追加
\hrule height 0.1mm depth 0mm width 62mm %罫線
\vspace{1mm} %余白追加

{\bf(能力)}


\vspace{-1zh}%余白削除
\begin{itemize}
\setlength{\leftskip}{-0.3cm}
\setlength{\parskip}{0pt} %4. 段落間余白.

\item 装備アクションの対象にできない。

\item 準備(場に出たターンは、アタックアクションにて、対戦相手を攻撃する兵士(アタッカー)に指定することができない。)

\item 場から墓地もしくは手札に行く場合、対象を{\normalsize $\diamondsuit$} Jとして切札再生アクションを誘発する。誘発については「誘発する場合」を参照。

\item 数字は11として扱う。

\item {\normalsize $\diamondsuit$} 兵士としても{\normalsize $\clubsuit$} 兵士としても扱う。

\item 2枚で1体の兵士として扱い、墓地や手札など別の場所に移る場合2枚一緒に移す。防壁になる場合2体の防壁になる。

\item この兵士がチャージ状態かつチャンスを持っている場合、ドライブして先読みアクションを起こすことができる。
\vspace{-1zh}%余白削除
\end{itemize}

\vspace{1mm} %余白追加 無いと箇条書きの場合、ギリギリになるので追加
\end{tcolorbox}

\vspace{-1zh}
 foreach end
 

%%%%% Ability %%%%%
\vspace{3mm} %余白追加
\hrule height 0.5mm depth 0mm width 66.5mm %罫線
\vspace{1mm} %余白追加
{\Large\bf $\clubsuit$ 4} {\normalsize\bf【スイッチ】} %Abilityタイトル
\vspace{1mm} %余白追加

自分の{\normalsize $\clubsuit$} 兵士は以下の能力を得る。


\vspace{-1zh}%余白削除
\begin{itemize}
\setlength{\leftskip}{-0.3cm}
\setlength{\parskip}{0pt} %4. 段落間余白.

\item この兵士がチャージ状態かつチャンスを持っている場合、ドライブしてスイッチアクションを起こすことができる。
\vspace{-1zh}%余白削除
\end{itemize}

%%%%% Abilityアクション %%%%%
\begin{tcolorbox}[title={\small\bf【Action】スイッチ}{\scriptsize (兵士起因)}]

{\scriptsize\bf @クイック }

\vspace{1mm} %余白追加
\hrule height 0.1mm depth 0mm width 62mm %罫線
\vspace{1mm} %余白追加

{\bf(即時効果)}


\vspace{-1zh}%余白削除
\begin{enumerate}
\setlength{\leftskip}{-0.3cm}
\setlength{\parskip}{0pt} %4. 段落間余白.

\item 手札からカードを1枚選び、兵士としてチャージ状態で場に出す。

\item このアクションを起こした兵士を手札に戻す。
\vspace{-1zh}%余白削除
\end{enumerate}

\vspace{1mm} %余白追加 無いと箇条書きの場合、ギリギリになるので追加
\end{tcolorbox}

\vspace{-1zh}
  
%%%%% Abilityキャラクター %%%%%
 

%%%%% Ability %%%%%
\vspace{3mm} %余白追加
\hrule height 0.5mm depth 0mm width 66.5mm %罫線
\vspace{1mm} %余白追加
{\Large\bf $\clubsuit$ 5} {\normalsize\bf【内乱】} %Abilityタイトル
\vspace{1mm} %余白追加


\vspace{-1zh}%余白削除
\begin{itemize}
\setlength{\leftskip}{-0.3cm}
\setlength{\parskip}{0pt} %4. 段落間余白.

\item 内乱アクションを起こすことができる。

\item 対戦相手が手札を捨てた場合、暴動アクションを誘発する。
\vspace{-1zh}%余白削除
\end{itemize}

%%%%% Abilityアクション %%%%%
\begin{tcolorbox}[title={\small\bf【Action】内乱}{\scriptsize (通常魔法)}]

{\scriptsize\bf @メイン }
  {\scriptsize\bf | \$ B }
  {\scriptsize\bf | ★{\normalsize $\clubsuit$} A〜K}

%特記事項
\vspace{1mm} %余白追加
\hrule height 0.1mm depth 0mm width 62mm %罫線
\vspace{1mm} %余白追加


\vspace{-1zh}%余白削除
\begin{enumerate}
\renewcommand{\labelenumi}{※}
\setlength{\leftskip}{-0.3cm}
\setlength{\itemsep}{0pt} %2. ブロック間の余白
\setlength{\parskip}{0pt} %4. 段落間余白.

\item プレイヤーは1ターンに1回しかこのアクションを起こすことができない。

\vspace{-3mm}%余白削除
\end{enumerate}
\vspace{-2mm} %余白削除 能力のアクションは余白が大きく開くため、追加で余白を削除
\vspace{1zh}%余白追加
\vspace{1mm} %余白追加  
\hrule height 0.1mm depth 0mm width 62mm %罫線
\vspace{1mm} %余白追加

{\bf(対象)}

対戦相手1人を対象とする。

\vspace{1mm} %余白追加
\hrule height 0.1mm depth 0mm width 62mm %罫線
\vspace{1mm} %余白追加

{\bf(通常効果)}

対象の対戦相手の手札の枚数が自分の手札の枚数以上の場合、次を行う。


\vspace{-1zh}%余白削除
\begin{itemize}
\setlength{\leftskip}{-0.3cm}
\setlength{\parskip}{0pt} %4. 段落間余白.

\item 対象の対戦相手は手札を2枚捨てる。
\vspace{-1zh}%余白削除
\end{itemize}

\vspace{1mm} %余白追加 無いと箇条書きの場合、ギリギリになるので追加
\end{tcolorbox}

\vspace{-1zh}
\begin{tcolorbox}[title={\small\bf【Action】暴動}{\scriptsize (誘発)}]

{\scriptsize\bf @クイック }

%特記事項
\vspace{1mm} %余白追加
\hrule height 0.1mm depth 0mm width 62mm %罫線
\vspace{1mm} %余白追加


\vspace{-1zh}%余白削除
\begin{enumerate}
\renewcommand{\labelenumi}{※}
\setlength{\leftskip}{-0.3cm}
\setlength{\itemsep}{0pt} %2. ブロック間の余白
\setlength{\parskip}{0pt} %4. 段落間余白.

\item プレイヤーはこのアクションを直接起こすことができない。

\vspace{-3mm}%余白削除
\end{enumerate}
\vspace{-2mm} %余白削除 能力のアクションは余白が大きく開くため、追加で余白を削除
\vspace{1zh}%余白追加
\vspace{1mm} %余白追加
\hrule height 0.1mm depth 0mm width 62mm %罫線
\vspace{1mm} %余白追加

{\bf(即時効果)}

このアクションを誘発した時に捨てられた手札の枚数×2点のダメージを、手札を捨てたプレイヤーに与える。

\vspace{1mm} %余白追加 無いと箇条書きの場合、ギリギリになるので追加
\end{tcolorbox}

\vspace{-1zh}
  
%%%%% Abilityキャラクター %%%%%
 

%%%%% Ability %%%%%
\vspace{3mm} %余白追加
\hrule height 0.5mm depth 0mm width 66.5mm %罫線
\vspace{1mm} %余白追加
{\Large\bf $\clubsuit$ 7} {\normalsize\bf【手札補充】} %Abilityタイトル
\vspace{1mm} %余白追加

自分がエンドアクションを起こした場合、手札補充アクションを誘発する。

%%%%% Abilityアクション %%%%%
\begin{tcolorbox}[title={\small\bf【Action】手札補充}{\scriptsize (誘発)}]

{\scriptsize\bf @クイック }

%特記事項
\vspace{1mm} %余白追加
\hrule height 0.1mm depth 0mm width 62mm %罫線
\vspace{1mm} %余白追加


\vspace{-1zh}%余白削除
\begin{enumerate}
\renewcommand{\labelenumi}{※}
\setlength{\leftskip}{-0.3cm}
\setlength{\itemsep}{0pt} %2. ブロック間の余白
\setlength{\parskip}{0pt} %4. 段落間余白.

\item プレイヤーはこのアクションを直接起こすことができない。

\vspace{-3mm}%余白削除
\end{enumerate}
\vspace{-2mm} %余白削除 能力のアクションは余白が大きく開くため、追加で余白を削除
\vspace{1zh}%余白追加
\vspace{1mm} %余白追加
\hrule height 0.1mm depth 0mm width 62mm %罫線
\vspace{1mm} %余白追加

{\bf(通常効果)}

自分の手札が2枚以下の場合、2枚カードを引き、3枚の場合1枚カードを引く。

\vspace{1mm} %余白追加 無いと箇条書きの場合、ギリギリになるので追加
\end{tcolorbox}

\vspace{-1zh}
  
%%%%% Abilityキャラクター %%%%%
 

%%%%% Ability %%%%%
\vspace{3mm} %余白追加
\hrule height 0.5mm depth 0mm width 66.5mm %罫線
\vspace{1mm} %余白追加
{\Large\bf $\clubsuit$ 8} {\normalsize\bf【リセット】} %Abilityタイトル
\vspace{1mm} %余白追加


\vspace{-1zh}%余白削除
\begin{itemize}
\setlength{\leftskip}{-0.3cm}
\setlength{\parskip}{0pt} %4. 段落間余白.

\item この能力が有効になった時に前触れアクションを誘発する。

\item リセットアクションを起こすことができる。
\vspace{-1zh}%余白削除
\end{itemize}

%%%%% Abilityアクション %%%%%
\begin{tcolorbox}[title={\small\bf【Action】前触れ}{\scriptsize (誘発)}]

{\scriptsize\bf @クイック }

%特記事項
\vspace{1mm} %余白追加
\hrule height 0.1mm depth 0mm width 62mm %罫線
\vspace{1mm} %余白追加


\vspace{-1zh}%余白削除
\begin{enumerate}
\renewcommand{\labelenumi}{※}
\setlength{\leftskip}{-0.3cm}
\setlength{\itemsep}{0pt} %2. ブロック間の余白
\setlength{\parskip}{0pt} %4. 段落間余白.

\item プレイヤーはこのアクションを直接起こすことができない。

\vspace{-3mm}%余白削除
\end{enumerate}
\vspace{-2mm} %余白削除 能力のアクションは余白が大きく開くため、追加で余白を削除
\vspace{1zh}%余白追加
\vspace{1mm} %余白追加
\hrule height 0.1mm depth 0mm width 62mm %罫線
\vspace{1mm} %余白追加

{\bf(即時効果)}

8点のダメージを受ける。

\vspace{1mm} %余白追加 無いと箇条書きの場合、ギリギリになるので追加
\end{tcolorbox}

\vspace{-1zh}
\begin{tcolorbox}[title={\small\bf【Action】リセット}{\scriptsize (通常魔法)}]

{\scriptsize\bf @メイン }
  {\scriptsize\bf | \$ L }
  {\scriptsize\bf | ★{\normalsize $\clubsuit$} A〜K}

\vspace{1mm} %余白追加
\hrule height 0.1mm depth 0mm width 62mm %罫線
\vspace{1mm} %余白追加

{\bf(通常効果)}


\vspace{-1zh}%余白削除
\begin{itemize}
\setlength{\leftskip}{-0.3cm}
\setlength{\parskip}{0pt} %4. 段落間余白.

\item 全ての防壁と兵士を墓地に移す。

\item 切札リセットアクションを誘発する。
\vspace{-1zh}%余白削除
\end{itemize}

\vspace{1mm} %余白追加 無いと箇条書きの場合、ギリギリになるので追加
\end{tcolorbox}

\vspace{-1zh}
\begin{tcolorbox}[title={\small\bf【Action】切札リセット}{\scriptsize (誘発)}]

{\scriptsize\bf @クイック }

%特記事項
\vspace{1mm} %余白追加
\hrule height 0.1mm depth 0mm width 62mm %罫線
\vspace{1mm} %余白追加


\vspace{-1zh}%余白削除
\begin{enumerate}
\renewcommand{\labelenumi}{※}
\setlength{\leftskip}{-0.3cm}
\setlength{\itemsep}{0pt} %2. ブロック間の余白
\setlength{\parskip}{0pt} %4. 段落間余白.

\item プレイヤーはこのアクションを直接起こすことができない。

\vspace{-3mm}%余白削除
\end{enumerate}
\vspace{-2mm} %余白削除 能力のアクションは余白が大きく開くため、追加で余白を削除
\vspace{1zh}%余白追加
\vspace{1mm} %余白追加
\hrule height 0.1mm depth 0mm width 62mm %罫線
\vspace{1mm} %余白追加

{\bf(即時効果)}

全ての切札を裏向きにする。

\vspace{1mm} %余白追加 無いと箇条書きの場合、ギリギリになるので追加
\end{tcolorbox}

\vspace{-1zh}
  
%%%%% Abilityキャラクター %%%%%
 

%%%%% Ability %%%%%
\vspace{3mm} %余白追加
\hrule height 0.5mm depth 0mm width 66.5mm %罫線
\vspace{1mm} %余白追加
{\Large\bf $\clubsuit$ 9} {\normalsize\bf【要塞】} %Abilityタイトル
\vspace{1mm} %余白追加

対戦相手がキーカードに{\normalsize $\spadesuit$} を含むアクションの効果を発揮する直前に自分の場にキャラクターがいる場合、その効果によってあなたはダメージを受けない。

%%%%% Abilityアクション %%%%%
 
%%%%% Abilityキャラクター %%%%%
 

%%%%% Ability %%%%%
\vspace{3mm} %余白追加
\hrule height 0.5mm depth 0mm width 66.5mm %罫線
\vspace{1mm} %余白追加
{\Large\bf $\clubsuit$ 10} {\normalsize\bf【悪あがき】} %Abilityタイトル
\vspace{1mm} %余白追加

悪あがきアクションを起こすことができる。

%%%%% Abilityアクション %%%%%
\begin{tcolorbox}[title={\small\bf【Action】悪あがき}{\scriptsize (速攻魔法)}]

{\scriptsize\bf @クイック }

\vspace{1mm} %余白追加  
\hrule height 0.1mm depth 0mm width 62mm %罫線
\vspace{1mm} %余白追加

{\bf(対象)}

キーカードがあるアクションを対象とする。

\vspace{1mm} %余白追加
\hrule height 0.1mm depth 0mm width 62mm %罫線
\vspace{1mm} %余白追加

{\bf(即時効果)}


\vspace{-1zh}%余白削除
\begin{itemize}
\setlength{\leftskip}{-0.3cm}
\setlength{\parskip}{0pt} %4. 段落間余白.

\item 対象アクションをステージから取り除き、対象アクションのキーカードを墓地に移す。

\item 手札を全て捨てる。手札がない場合、自分の切札{\normalsize $\clubsuit$} 10を墓地に移す。
\vspace{-1zh}%余白削除
\end{itemize}

\vspace{1mm} %余白追加 無いと箇条書きの場合、ギリギリになるので追加
\end{tcolorbox}

\vspace{-1zh}
  
%%%%% Abilityキャラクター %%%%%
 

%%%%% Ability %%%%%
\vspace{3mm} %余白追加
\hrule height 0.5mm depth 0mm width 66.5mm %罫線
\vspace{1mm} %余白追加
{\Large\bf $\clubsuit$ J} {\normalsize\bf【騎士】} %Abilityタイトル
\vspace{1mm} %余白追加

騎士召喚アクションを起こすことができる。

%%%%% Abilityアクション %%%%%
\begin{tcolorbox}[title={\small\bf【Action】騎士召喚}{\scriptsize (召喚)}]

{\scriptsize\bf @メイン }
  {\scriptsize\bf | \$ BL }
  {\scriptsize\bf | ★{\normalsize $\spadesuit$} 2〜10}

%特記事項
\vspace{1mm} %余白追加
\hrule height 0.1mm depth 0mm width 62mm %罫線
\vspace{1mm} %余白追加


\vspace{-1zh}%余白削除
\begin{enumerate}
\renewcommand{\labelenumi}{※}
\setlength{\leftskip}{-0.3cm}
\setlength{\itemsep}{0pt} %2. ブロック間の余白
\setlength{\parskip}{0pt} %4. 段落間余白.

\item コストを\$BDとすればタイミングをクイックとして起こすことができる。

\vspace{-3mm}%余白削除
\end{enumerate}
\vspace{-2mm} %余白削除 能力のアクションは余白が大きく開くため、追加で余白を削除
\vspace{1zh}%余白追加
\vspace{1mm} %余白追加
\hrule height 0.1mm depth 0mm width 62mm %罫線
\vspace{1mm} %余白追加

{\bf(通常効果)}

切札の{\normalsize $\clubsuit$} Jが表の場合、次を行う。


\vspace{-1zh}%余白削除
\begin{enumerate}
\setlength{\leftskip}{-0.3cm}
\setlength{\parskip}{0pt} %4. 段落間余白.

\item 対戦相手は自分の場にある兵士を1枚選び墓地に移す。

\item キーカードと切札の{\normalsize $\clubsuit$} Jをあわせて騎士としてチャージ状態で場に出す。
\vspace{-1zh}%余白削除
\end{enumerate}

\vspace{1mm} %余白追加 無いと箇条書きの場合、ギリギリになるので追加
\end{tcolorbox}

\vspace{-1zh}
\begin{tcolorbox}[title={\small\bf【Action】薙払}{\scriptsize (兵士起因)}]

{\scriptsize\bf @クイック }

\vspace{1mm} %余白追加  
\hrule height 0.1mm depth 0mm width 62mm %罫線
\vspace{1mm} %余白追加

{\bf(対象)}

対戦相手1人を対象とする。

\vspace{1mm} %余白追加
\hrule height 0.1mm depth 0mm width 62mm %罫線
\vspace{1mm} %余白追加

{\bf(即時効果)}

対象の対戦相手は自分の場にいるキャラクターを1体墓地に移す。

\vspace{1mm} %余白追加 無いと箇条書きの場合、ギリギリになるので追加
\end{tcolorbox}

\vspace{-1zh}
  
%%%%% Abilityキャラクター %%%%%
\vspace{2mm}
\begin{tcolorbox}[title={\small\bf【Character】騎士}{\scriptsize (兵士)}]

  {\scriptsize\bf ★{\normalsize $\clubsuit$} J と {\normalsize $\spadesuit$} 2〜10}

\vspace{1mm} %余白追加
\hrule height 0.1mm depth 0mm width 62mm %罫線
\vspace{1mm} %余白追加

{\bf(能力)}


\vspace{-1zh}%余白削除
\begin{itemize}
\setlength{\leftskip}{-0.3cm}
\setlength{\parskip}{0pt} %4. 段落間余白.

\item 装備アクションの対象にできない。

\item 準備(場に出たターンは、アタックアクションにて、対戦相手を攻撃する兵士(アタッカー)に指定することができない。)

\item 場から墓地もしくは手札に行く場合、対象を{\normalsize $\clubsuit$} Jとして切札再生アクションを誘発する。誘発については「誘発する場合」を参照。

\item 数字は11として扱う。

\item {\normalsize $\spadesuit$} 兵士としても{\normalsize $\clubsuit$} 兵士としても扱う。

\item 2枚で1体の兵士として扱い、墓地や手札など別の場所に移る場合2枚一緒に移す。防壁になる場合2体の防壁になる。

\item この兵士がチャージ状態かつチャンスを持っている場合、ドライブして薙払アクションを起こすことができる。
\vspace{-1zh}%余白削除
\end{itemize}

\vspace{1mm} %余白追加 無いと箇条書きの場合、ギリギリになるので追加
\end{tcolorbox}

\vspace{-1zh}
 foreach end
 
 
%% 奥付 %%
\thispagestyle{empty}
\vspace*{\stretch{1}}
\begin{flushright}
\begin{minipage}{0.6\hsize}
\begin{description}
  \item{誌名:}BlackPoker ExtraList\\ \hspace{3pt} 第五版 extra v5.30.0
  \item{発行:}2019/06/16
\end{description}
\end{minipage}
\end{flushright}

\begin{flushright}
\copyright 2013 BlackPoker
\end{flushright}

\end{document}