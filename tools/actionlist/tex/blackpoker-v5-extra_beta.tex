\documentclass[twocolumn,a5paper,papersize,10pt]{jarticle}
\usepackage{bxpapersize}
\usepackage{booktabs}
\usepackage{tabularx}
\usepackage[dvipdfmx]{graphicx} %画像読み込み設定
\usepackage{here} %画像を好きな位置に出力
\usepackage[absolute,overlay]{textpos}%座標指定

\usepackage{newtxtext,newtxmath} %timeフォント設定
\usepackage{titlesec}%タイトルの文字サイズ変更
\usepackage{setspace} % setspaceパッケージのインクルード

\usepackage{tcolorbox} % 枠囲み
\usepackage{okumacro} % ルビ

%\columnseprule=0.1pt %段組み罫線
\setlength{\columnsep}{0.5cm} %段組みの幅


\titleformat*{\section}{\large\bfseries} %sectionの文字サイズ
\titleformat*{\subsection}{\normalsize\bfseries} %subsectionの文字サイズ
\titleformat*{\subsubsection}{\small\bfseries} %subsubsectionの文字サイズ

\setlength{\hoffset}{-2cm}
\setlength{\voffset}{-4cm}
\setlength{\marginparsep}{0pt}
\setlength{\marginparwidth}{0pt}
\setlength{\headheight}{5pt}
\setlength{\textheight}{19cm}
\setlength{\textwidth}{13.8cm}

\setlength\intextsep{0pt} %図の余白をなくす
\setlength\textfloatsep{0pt} %図の余白をなくす

\setlength\floatsep{0pt} %図と図の間の余白
\setlength\textfloatsep{0pt} %本文と図の間の余白
\setlength\intextsep{0pt} %本文中の図の余白
\setlength\abovecaptionskip{0pt} %図とキャプションの間の余白


\setstretch{1} % ページ全体の行間を設定
\parindent = 0pt %インデントを無効化

\title{\empty}
\author{\empty}
\date{\empty}

%%%%%%    TEXT START    %%%%%%
\begin{document}


% ロゴ出力
% この場合は (230pt, 100pt) の位置に 0.4\linewidth の幅のブロックができる.
\begin{textblock*}{0.4\linewidth}(55pt, 145pt)
    \centering
    \includegraphics[width=1.2cm]{blackpoker_logo.pdf}
\end{textblock*}

%%%%% QRコード第三版用 %%%%%
\begin{textblock*}{0.4\linewidth}(200pt, 135pt)
    \centering
    \includegraphics[width=1.7cm,keepaspectratio]{qr_blackpoker-support_v5-ex.pdf}
\end{textblock*}
\begin{textblock*}{0.4\linewidth}(200pt, 130pt)
    \centering
    \textcolor{black}{Web版}
\end{textblock*}
%%%%% QRコード第三版用 %%%%%

\section*{\textrm{\Large BlackPoker}}
\vspace{-1zh}%余白削除
\noindent

\begin{center}
第五版 extra
ex5.30.0

{\scriptsize 2019/06/XX}

\end{center}

\scriptsize%本文の文字サイズを小さく設定
\renewcommand{\labelitemi}{・}%箇条書きのラベルを変更

{\quad}extraとは、BlackPokerの遊び方の一つです。プレイヤーは切札を設置してからゲームを開始します。
切札については 切札についてを参照して下さい。extraではプロ版のアクション、キャラクターに加え切札を操作するアクションとそれに対応する切札の能力を使うことが出来ます。

{\quad}本誌はextraで使用できるアクションと切札を記載しています。

\vspace{2mm}%余白削除
\hrule height 0.5mm depth 0mm width 66.5mm %罫線
\vspace{-3zh}%余白削除
\subsection*{(補足)切札について}
%\vspace{-2mm}%余白削除
%\subsubsection*{切札について}
\vspace{-1zh}%余白削除
\begin{itemize}
\setlength{\leftskip}{-0.3cm}%箇条書きを左詰め
%\setlength{\itemsep}{0pt}      %2. ブロック間の余白
\setlength{\parskip}{0pt}      %4. 段落間余白.
%\setlength{\itemindent}{-10pt}   %5. 最初のインデント
%\setlength{\labelsep}{15pt}     %6. item と文字の間

\item 対戦前に裏向きで2枚まで切札を置くことができる。
\item 切札はデッキと直角に交わるようにデッキの下に置く。
\item 切札を表にするときはスートと数字が見えるようにし、対応する能力の名称を言う。
\item デッキが0枚になった場合、切札が残っていても敗北する。
\end{itemize}
※詳しくは、公式ルール参照。
\vspace{-1zh}%余白削除

%%%%%%%%%%%%%%%%%%%%%%%%%%%%%%
%%%%% ActionList %%%%%
\begin{center}
\begin{center}
\hrule height 1mm depth 0mm width 66.5mm %罫線
\vspace{1mm}%余白削除
{\Large\bf \ruby{Action List}{アクションリスト}}
\vspace{1mm}%余白削除
\hrule height 0.5mm depth 0mm width 66.5mm %罫線
\end{center}
\end{center}
\vspace{-1zh}%余白削除
%%%%%%%%%%%%%%%%%%%%%%%%%%%%%%

%%% Action %%%
%\vspace{1zh} %余白追加
\vspace{2mm} %余白追加
\hrule height 0.5mm depth 0mm width 66.5mm %罫線
\vspace{1mm} %余白追加
{\normalsize\bf ■ オープン {\scriptsize (通常魔法) [ex]}} %Actionタイトル
\hfill 
{\small\bf @メイン }
  {\small\bf | } {\small\bf \$ BBD}


%特記事項
\vspace{1mm}%余白削除
\hrule height 0.1mm depth 0mm width 66.5mm %罫線
\vspace{1mm}%余白削除


\vspace{-1zh}%余白削除
\begin{enumerate}
\renewcommand{\labelenumi}{※}
\setlength{\leftskip}{-0.3cm}
\setlength{\itemsep}{0pt} %2. ブロック間の余白
\setlength{\parskip}{0pt} %4. 段落間余白.

\item コストを\$DDとすればタイミングをクイックとして起こすことができる。

\vspace{-3mm}%余白削除
\end{enumerate}
\vspace{1mm}%余白削除
\hrule height 0.1mm depth 0mm width 66.5mm %罫線
\vspace{1mm}%余白削除

{\bf(即時効果)}

自分の切札を1枚表にし、切札に対応した能力を発揮する。 

切札については「切札について」参照

切札に対応する能力については「アビリティリスト」参照
%%% Action %%%
%\vspace{1zh} %余白追加
\vspace{2mm} %余白追加
\hrule height 0.5mm depth 0mm width 66.5mm %罫線
\vspace{1mm} %余白追加
{\normalsize\bf ■ クローズ {\scriptsize (速攻魔法) [ex]}} %Actionタイトル
\hfill 
{\small\bf @クイック }
  {\small\bf | } {\small\bf \$ B}

★同じスート2枚

\vspace{1mm}%余白削除
\hrule height 0.1mm depth 0mm width 66.5mm %罫線
\vspace{1mm}%余白削除
{\bf(対象)}

切札1枚を対象とする。
\vspace{1mm}%余白削除
\hrule height 0.1mm depth 0mm width 66.5mm %罫線
\vspace{1mm}%余白削除

{\bf(通常効果)}

対象の切札が表向きかつキーカードのスートと同じ場合、その切札を裏向きにする。対象の切札に対応した能力は無効になる。 

切札については「切札について」参照
%%% Action %%%
%\vspace{1zh} %余白追加
\vspace{2mm} %余白追加
\hrule height 0.5mm depth 0mm width 66.5mm %罫線
\vspace{1mm} %余白追加
{\normalsize\bf ■ 切札破壊 {\scriptsize (速攻魔法) [ex]}} %Actionタイトル
\hfill 
{\small\bf @クイック }
  {\small\bf | } {\small\bf \$ B}

★A〜K

\vspace{1mm}%余白削除
\hrule height 0.1mm depth 0mm width 66.5mm %罫線
\vspace{1mm}%余白削除
{\bf(対象)}

切札1枚を対象とする。
\vspace{1mm}%余白削除
\hrule height 0.1mm depth 0mm width 66.5mm %罫線
\vspace{1mm}%余白削除

{\bf(通常効果)}

対象の切札が表向きかつキーカードとスートと数字が同じ場合、切札を墓地に移す。対象の切札に対応した能力は無効になる。 

切札については「切札について」参照
%%% Action %%%
%\vspace{1zh} %余白追加
\vspace{2mm} %余白追加
\hrule height 0.5mm depth 0mm width 66.5mm %罫線
\vspace{1mm} %余白追加
{\normalsize\bf ■ 切札再生 {\scriptsize (誘発) [ex]}} %Actionタイトル
\hfill 
{\small\bf @クイック }


%特記事項
\vspace{1mm}%余白削除
\hrule height 0.1mm depth 0mm width 66.5mm %罫線
\vspace{1mm}%余白削除


\vspace{-1zh}%余白削除
\begin{enumerate}
\renewcommand{\labelenumi}{※}
\setlength{\leftskip}{-0.3cm}
\setlength{\itemsep}{0pt} %2. ブロック間の余白
\setlength{\parskip}{0pt} %4. 段落間余白.

\item プレイヤーはこのアクションを直接起こすことができない。

\vspace{-3mm}%余白削除
\end{enumerate}
\vspace{1mm}%余白削除
\hrule height 0.1mm depth 0mm width 66.5mm %罫線
\vspace{1mm}%余白削除
{\bf(対象)}

カード1枚を対象とする。
\vspace{1mm}%余白削除
\hrule height 0.1mm depth 0mm width 66.5mm %罫線
\vspace{1mm}%余白削除

{\bf(即時効果)}

対象のカードを手札、墓地から見つけられた場合、それを自分の切札として表向きにして置き、切札に対応した能力を発揮する。 

切札については「切札について」参照

切札に対応する能力については「アビリティリスト」参照

%\newpage %改段
%%%%%%%%%%%%%%%%%%%%%%%%%%%%%%
%%%%% AbilityList %%%%%
\begin{center}
\begin{center}
\hrule height 1mm depth 0mm width 66.5mm %罫線
\vspace{1mm}%余白削除
{\Large\bf \ruby{Ability List}{アビリティリスト}}
\vspace{1mm}%余白削除
\hrule height 0.5mm depth 0mm width 66.5mm %罫線
\end{center}
\end{center}
\vspace{-2zh}%余白削除
%%%%%%%%%%%%%%%%%%%%%%%%%%%%%%

%\tcbset{top=1mm, left=1mm, bottom=1mm, left=1mm}
\tcbset{colframe=black,coltitle=black!0!black,colbacktitle=white!0!white,colback=white!0!white,sharp corners,top=1mm, left=1mm, bottom=1mm, left=1mm,boxrule=0.2mm,toprule=0.2mm}


%%%%% Ability %%%%%
\vspace{3mm} %余白追加
\hrule height 0.5mm depth 0mm width 66.5mm %罫線
\vspace{1mm} %余白追加
{\Large\bf $\spadesuit$ 3} {\normalsize\bf【ドレイン】} %Abilityタイトル
\vspace{1mm} %余白追加

ドレインアクションを起こすことができる。

\begin{tcolorbox}[title={\small\bf【Action】ドレイン}{\scriptsize (通常魔法)}]

{\scriptsize\bf @メイン }

\vspace{1mm} %余白追加  
\hrule height 0.1mm depth 0mm width 62mm %罫線
\vspace{1mm} %余白追加

{\bf(対象)}

対戦相手1人を対象とする。

\vspace{1mm} %余白追加
\hrule height 0.1mm depth 0mm width 62mm %罫線
\vspace{1mm} %余白追加

{\bf(通常効果)}

自分の場にいるキャラクター1体をデッキの一番下に移す。移せた場合、対象の対戦相手に2点のダメージを与える。

\vspace{1mm} %余白追加 無いと箇条書きの場合、ギリギリになるので追加
\end{tcolorbox}

\vspace{-1zh}

 
 
 
 
 

%%%%% Ability %%%%%
\vspace{3mm} %余白追加
\hrule height 0.5mm depth 0mm width 66.5mm %罫線
\vspace{1mm} %余白追加
{\Large\bf $\spadesuit$ 7} {\normalsize\bf【瞬殺】} %Abilityタイトル
\vspace{1mm} %余白追加

瞬殺アクションを起こすことができる。

\begin{tcolorbox}[title={\small\bf【Action】瞬殺}{\scriptsize (速攻魔法)}]

{\scriptsize\bf @クイック }

\vspace{1mm} %余白追加  
\hrule height 0.1mm depth 0mm width 62mm %罫線
\vspace{1mm} %余白追加

{\bf(対象)}

兵士1体を対象とする。

\vspace{1mm} %余白追加
\hrule height 0.1mm depth 0mm width 62mm %罫線
\vspace{1mm} %余白追加

{\bf(即時効果)}

自分の場にいる兵士を1体以上任意の数だけ墓地に移す。この方法で墓地に移した兵士の数字の合計値が対象の兵士の数字以上の場合、対象の兵士を墓地に移す。

\vspace{1mm} %余白追加 無いと箇条書きの場合、ギリギリになるので追加
\end{tcolorbox}

\vspace{-1zh}

 
 
 
 
 

%%%%% Ability %%%%%
\vspace{3mm} %余白追加
\hrule height 0.5mm depth 0mm width 66.5mm %罫線
\vspace{1mm} %余白追加
{\Large\bf $\heartsuit$ 4} {\normalsize\bf【休戦】} %Abilityタイトル
\vspace{1mm} %余白追加

休戦アクションを起こすことができる。

\begin{tcolorbox}[title={\small\bf【Action】休戦}{\scriptsize (速攻魔法)}]

{\scriptsize\bf @クイック }

\vspace{1mm} %余白追加  
\hrule height 0.1mm depth 0mm width 62mm %罫線
\vspace{1mm} %余白追加

{\bf(対象)}

アタックアクションを対象とする。

\vspace{1mm} %余白追加
\hrule height 0.1mm depth 0mm width 62mm %罫線
\vspace{1mm} %余白追加

{\bf(即時効果)}


\vspace{-1zh}%余白削除
\begin{itemize}
\setlength{\leftskip}{-0.3cm}
\setlength{\parskip}{0pt} %4. 段落間余白.

\item 対象のアクションは効果を発揮しない。

\item 自分の切札にある{\normalsize $\heartsuit$} 4を裏向きにする。 

\item 自分の場にいるキャラクターを全てチャージ状態にする。
\vspace{-1zh}%余白削除
\end{itemize}

\vspace{1mm} %余白追加 無いと箇条書きの場合、ギリギリになるので追加
\end{tcolorbox}

\vspace{-1zh}

 
 
 
 
 

%%%%% Ability %%%%%
\vspace{3mm} %余白追加
\hrule height 0.5mm depth 0mm width 66.5mm %罫線
\vspace{1mm} %余白追加
{\Large\bf $\heartsuit$ 5} {\normalsize\bf【追撃】} %Abilityタイトル
\vspace{1mm} %余白追加

追撃アクションを起こすことができる。

\begin{tcolorbox}[title={\small\bf【Action】追撃}{\scriptsize (通常魔法)}]

{\scriptsize\bf @メイン }
  {\scriptsize\bf | \$ BDD }

%特記事項
\vspace{1mm} %余白追加
\hrule height 0.1mm depth 0mm width 62mm %罫線
\vspace{1mm} %余白追加


\vspace{-1zh}%余白削除
\begin{enumerate}
\renewcommand{\labelenumi}{※}
\setlength{\leftskip}{-0.3cm}
\setlength{\itemsep}{0pt} %2. ブロック間の余白
\setlength{\parskip}{0pt} %4. 段落間余白.

\item プレイヤーは1ターンに1回しかこのアクションを起こすことができない。

\vspace{-3mm}%余白削除
\end{enumerate}
\vspace{-2mm} %余白削除 能力のアクションは余白が大きく開くため、追加で余白を削除
\vspace{1zh}%余白追加
\vspace{1mm} %余白追加
\hrule height 0.1mm depth 0mm width 62mm %罫線
\vspace{1mm} %余白追加

{\bf(即時効果)}


\vspace{-1zh}%余白削除
\begin{itemize}
\setlength{\leftskip}{-0.3cm}
\setlength{\parskip}{0pt} %4. 段落間余白.

\item 自分の場にいる兵士を全てチャージ状態にする。

\item アタックアクションを起こす。
\vspace{-1zh}%余白削除
\end{itemize}

\vspace{1mm} %余白追加 無いと箇条書きの場合、ギリギリになるので追加
\end{tcolorbox}

\vspace{-1zh}

 
 
 
 
 

%%%%% Ability %%%%%
\vspace{3mm} %余白追加
\hrule height 0.5mm depth 0mm width 66.5mm %罫線
\vspace{1mm} %余白追加
{\Large\bf $\diamondsuit$ 4} {\normalsize\bf【相殺】} %Abilityタイトル
\vspace{1mm} %余白追加

相殺アクションを起こすことができる。

\begin{tcolorbox}[title={\small\bf【Action】相殺}{\scriptsize (速攻魔法)}]

{\scriptsize\bf @クイック }
  {\scriptsize\bf | \$ D }

\vspace{1mm} %余白追加  
\hrule height 0.1mm depth 0mm width 62mm %罫線
\vspace{1mm} %余白追加

{\bf(対象)}

キーカードがあるアクションを対象とする。

\vspace{1mm} %余白追加
\hrule height 0.1mm depth 0mm width 62mm %罫線
\vspace{1mm} %余白追加

{\bf(即時効果)}

対象アクションのキーカードの数字の合計値が5以上の場合以下を行う。


\vspace{-1zh}%余白削除
\begin{enumerate}
\setlength{\leftskip}{-0.3cm}
\setlength{\parskip}{0pt} %4. 段落間余白.

\item 対象アクションのキーカードの数字の合計値分ダメージを受ける。

\item 対象アクションをステージから取り除き、対象アクションのキーカードを墓地に移す。
\vspace{-1zh}%余白削除
\end{enumerate}

\vspace{1mm} %余白追加 無いと箇条書きの場合、ギリギリになるので追加
\end{tcolorbox}

\vspace{-1zh}

 
 
 
 
 

%%%%% Ability %%%%%
\vspace{3mm} %余白追加
\hrule height 0.5mm depth 0mm width 66.5mm %罫線
\vspace{1mm} %余白追加
{\Large\bf $\diamondsuit$ 5} {\normalsize\bf【喪失】} %Abilityタイトル
\vspace{1mm} %余白追加


\vspace{-1zh}%余白削除
\begin{itemize}
\setlength{\leftskip}{-0.3cm}
\setlength{\parskip}{0pt} %4. 段落間余白.

\item この能力が有効でありかつ、自分の場にチャージ状態の防壁が2枚存在する時にチャージアクションが効果を発揮した場合、以下のようにチャージアクションの効果を変更する。

\vspace{-1zh}%余白削除
\begin{itemize}
\setlength{\leftskip}{-0.3cm}
\setlength{\parskip}{0pt} %4. 段落間余白.

\item ドローアクションを起こす。
\vspace{-1zh}%余白削除
\end{itemize}
\item 自分がエンドアクションを起こした場合、疲弊アクションを誘発する。
\vspace{-1zh}%余白削除
\end{itemize}

\begin{tcolorbox}[title={\small\bf【Action】疲弊}{\scriptsize (誘発)}]

{\scriptsize\bf @クイック }

%特記事項
\vspace{1mm} %余白追加
\hrule height 0.1mm depth 0mm width 62mm %罫線
\vspace{1mm} %余白追加


\vspace{-1zh}%余白削除
\begin{enumerate}
\renewcommand{\labelenumi}{※}
\setlength{\leftskip}{-0.3cm}
\setlength{\itemsep}{0pt} %2. ブロック間の余白
\setlength{\parskip}{0pt} %4. 段落間余白.

\item プレイヤーはこのアクションを直接起こすことができない。

\vspace{-3mm}%余白削除
\end{enumerate}
\vspace{-2mm} %余白削除 能力のアクションは余白が大きく開くため、追加で余白を削除
\vspace{1zh}%余白追加
\vspace{1mm} %余白追加
\hrule height 0.1mm depth 0mm width 62mm %罫線
\vspace{1mm} %余白追加

{\bf(通常効果)}

以下のいずれかを行う。


\vspace{-1zh}%余白削除
\begin{itemize}
\setlength{\leftskip}{-0.3cm}
\setlength{\parskip}{0pt} %4. 段落間余白.

\item 2点ダメージを受ける。

\item 自分の切札にある{\normalsize $\diamondsuit$} 5を裏向きにする。
\vspace{-1zh}%余白削除
\end{itemize}

\vspace{1mm} %余白追加 無いと箇条書きの場合、ギリギリになるので追加
\end{tcolorbox}

\vspace{-1zh}

 
 
 
 
 

%%%%% Ability %%%%%
\vspace{3mm} %余白追加
\hrule height 0.5mm depth 0mm width 66.5mm %罫線
\vspace{1mm} %余白追加
{\Large\bf $\clubsuit$ 8} {\normalsize\bf【リセット】} %Abilityタイトル
\vspace{1mm} %余白追加


\vspace{-1zh}%余白削除
\begin{itemize}
\setlength{\leftskip}{-0.3cm}
\setlength{\parskip}{0pt} %4. 段落間余白.

\item この能力が有効になった時に前触れアクションを誘発する。

\item リセットアクションを起こすことができる。
\vspace{-1zh}%余白削除
\end{itemize}

\begin{tcolorbox}[title={\small\bf【Action】前触れ}{\scriptsize (誘発)}]

{\scriptsize\bf @クイック }

%特記事項
\vspace{1mm} %余白追加
\hrule height 0.1mm depth 0mm width 62mm %罫線
\vspace{1mm} %余白追加


\vspace{-1zh}%余白削除
\begin{enumerate}
\renewcommand{\labelenumi}{※}
\setlength{\leftskip}{-0.3cm}
\setlength{\itemsep}{0pt} %2. ブロック間の余白
\setlength{\parskip}{0pt} %4. 段落間余白.

\item プレイヤーはこのアクションを直接起こすことができない。

\vspace{-3mm}%余白削除
\end{enumerate}
\vspace{-2mm} %余白削除 能力のアクションは余白が大きく開くため、追加で余白を削除
\vspace{1zh}%余白追加
\vspace{1mm} %余白追加
\hrule height 0.1mm depth 0mm width 62mm %罫線
\vspace{1mm} %余白追加

{\bf(即時効果)}

8点のダメージを受ける。

\vspace{1mm} %余白追加 無いと箇条書きの場合、ギリギリになるので追加
\end{tcolorbox}

\vspace{-1zh}

 
 
\begin{tcolorbox}[title={\small\bf【Action】リセット}{\scriptsize (通常魔法)}]

{\scriptsize\bf @メイン }
  {\scriptsize\bf | \$ L }
  {\scriptsize\bf | ★{\normalsize $\clubsuit$} A〜K}

\vspace{1mm} %余白追加
\hrule height 0.1mm depth 0mm width 62mm %罫線
\vspace{1mm} %余白追加

{\bf(通常効果)}


\vspace{-1zh}%余白削除
\begin{itemize}
\setlength{\leftskip}{-0.3cm}
\setlength{\parskip}{0pt} %4. 段落間余白.

\item 全ての防壁と兵士を墓地に移す。

\item 切札リセットアクションを誘発する。
\vspace{-1zh}%余白削除
\end{itemize}

\vspace{1mm} %余白追加 無いと箇条書きの場合、ギリギリになるので追加
\end{tcolorbox}

\vspace{-1zh}

 
\begin{tcolorbox}[title={\small\bf【Action】切札リセット}{\scriptsize (誘発)}]

{\scriptsize\bf @クイック }

%特記事項
\vspace{1mm} %余白追加
\hrule height 0.1mm depth 0mm width 62mm %罫線
\vspace{1mm} %余白追加


\vspace{-1zh}%余白削除
\begin{enumerate}
\renewcommand{\labelenumi}{※}
\setlength{\leftskip}{-0.3cm}
\setlength{\itemsep}{0pt} %2. ブロック間の余白
\setlength{\parskip}{0pt} %4. 段落間余白.

\item プレイヤーはこのアクションを直接起こすことができない。

\vspace{-3mm}%余白削除
\end{enumerate}
\vspace{-2mm} %余白削除 能力のアクションは余白が大きく開くため、追加で余白を削除
\vspace{1zh}%余白追加
\vspace{1mm} %余白追加
\hrule height 0.1mm depth 0mm width 62mm %罫線
\vspace{1mm} %余白追加

{\bf(即時効果)}

全ての切札を裏向きにする。

\vspace{1mm} %余白追加 無いと箇条書きの場合、ギリギリになるので追加
\end{tcolorbox}

\vspace{-1zh}

 
 

%%%%% Ability %%%%%
\vspace{3mm} %余白追加
\hrule height 0.5mm depth 0mm width 66.5mm %罫線
\vspace{1mm} %余白追加
{\Large\bf $\clubsuit$ 10} {\normalsize\bf【悪あがき】} %Abilityタイトル
\vspace{1mm} %余白追加

悪あがきアクションを起こすことができる。

\begin{tcolorbox}[title={\small\bf【Action】悪あがき}{\scriptsize (速攻魔法)}]

{\scriptsize\bf @クイック }

\vspace{1mm} %余白追加  
\hrule height 0.1mm depth 0mm width 62mm %罫線
\vspace{1mm} %余白追加

{\bf(対象)}

キーカードがあるアクションを対象とする。

\vspace{1mm} %余白追加
\hrule height 0.1mm depth 0mm width 62mm %罫線
\vspace{1mm} %余白追加

{\bf(即時効果)}


\vspace{-1zh}%余白削除
\begin{itemize}
\setlength{\leftskip}{-0.3cm}
\setlength{\parskip}{0pt} %4. 段落間余白.

\item 対象アクションをステージから取り除き、対象アクションのキーカードを墓地に移す。

\item 手札を全て捨てる。手札がない場合、自分の切札{\normalsize $\clubsuit$} 10を墓地に移す。
\vspace{-1zh}%余白削除
\end{itemize}

\vspace{1mm} %余白追加 無いと箇条書きの場合、ギリギリになるので追加
\end{tcolorbox}

\vspace{-1zh}

 
 
 
 
 
 
%% 奥付 %%
\thispagestyle{empty}
\vspace*{\stretch{1}}
\begin{flushright}
\begin{minipage}{0.6\hsize}
\begin{description}
  \item{誌名:}BlackPoker ExtraList\\ \hspace{3pt} 第五版 extra v5.30.0
  \item{発行:}2019/06/XX
\end{description}
\end{minipage}
\end{flushright}

\begin{flushright}
\copyright 2013 BlackPoker
\end{flushright}

\end{document}