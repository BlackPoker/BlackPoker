\documentclass[twocolumn,a5paper,papersize,10pt]{jarticle}
\usepackage{bxpapersize}
\usepackage{booktabs}
\usepackage{tabularx}
\usepackage[dvipdfmx]{graphicx} %画像読み込み設定
\usepackage{here} %画像を好きな位置に出力
\usepackage[absolute,overlay]{textpos}%座標指定

\usepackage{newtxtext,newtxmath} %timeフォント設定
\usepackage{titlesec}%タイトルの文字サイズ変更
\usepackage{setspace} % setspaceパッケージのインクルード

\usepackage{tcolorbox} % 枠囲み
\usepackage{okumacro} % ルビ

%\columnseprule=0.1pt %段組み罫線
\setlength{\columnsep}{0.5cm} %段組みの幅


\titleformat*{\section}{\large\bfseries} %sectionの文字サイズ
\titleformat*{\subsection}{\normalsize\bfseries} %subsectionの文字サイズ
\titleformat*{\subsubsection}{\small\bfseries} %subsubsectionの文字サイズ

\setlength{\hoffset}{-2cm}
\setlength{\voffset}{-4cm}
\setlength{\marginparsep}{0pt}
\setlength{\marginparwidth}{0pt}
\setlength{\headheight}{5pt}
\setlength{\textheight}{19cm}
\setlength{\textwidth}{13.8cm}

\setlength\intextsep{0pt} %図の余白をなくす
\setlength\textfloatsep{0pt} %図の余白をなくす

\setlength\floatsep{0pt} %図と図の間の余白
\setlength\textfloatsep{0pt} %本文と図の間の余白
\setlength\intextsep{0pt} %本文中の図の余白
\setlength\abovecaptionskip{0pt} %図とキャプションの間の余白


\setstretch{1} % ページ全体の行間を設定
\parindent = 0pt %インデントを無効化

\title{\empty}
\author{\empty}
\date{\empty}

%%%%%%    TEXT START    %%%%%%
\begin{document}


% ロゴ出力
% この場合は (230pt, 100pt) の位置に 0.4\linewidth の幅のブロックができる.
\begin{textblock*}{0.4\linewidth}(55pt, 145pt)
    \centering
    \includegraphics[width=1.2cm]{blackpoker_logo.pdf}
\end{textblock*}

%%%%% QRコード第三版用 %%%%%
\begin{textblock*}{0.4\linewidth}(200pt, 135pt)
    \centering
    \includegraphics[width=1.7cm,keepaspectratio]{${data.qr}}
\end{textblock*}
\begin{textblock*}{0.4\linewidth}(200pt, 130pt)
    \centering
    \textcolor{black}{Web版}
\end{textblock*}
%%%%% QRコード第三版用 %%%%%

\section*{\textrm{\Large BlackPoker}}
\vspace{-1zh}%余白削除
\noindent

\begin{center}
${data.ver}
v${data.v}

#if(${data.newOnly} != "")
{\scriptsize ${data.lastupdate} 追加分のみ} 
#else
{\scriptsize ${data.lastupdate}} 
#end
\end{center}

\scriptsize%本文の文字サイズを小さく設定
\renewcommand{\labelitemi}{・}%箇条書きのラベルを変更

{\quad}extraとは、BlackPokerの遊び方の一つです。プレイヤーは切札を設置してからゲームを開始します。
切札については 切札についてを参照して下さい。extraではスタンダード版のアクション、キャラクターに加え切札を操作するアクションとそれに対応する切札の能力を使うことが出来ます。

{\quad}本誌はextraで使用できるアクションと切札を記載しています。

\vspace{2mm}%余白削除
\hrule height 0.5mm depth 0mm width 66.5mm %罫線
\vspace{-3zh}%余白削除
\subsection*{(補足)切札について}
%\vspace{-2mm}%余白削除
%\subsubsection*{切札について}
\vspace{-1zh}%余白削除
\begin{itemize}
\setlength{\leftskip}{-0.3cm}%箇条書きを左詰め
%\setlength{\itemsep}{0pt}      %2. ブロック間の余白
\setlength{\parskip}{0pt}      %4. 段落間余白.
%\setlength{\itemindent}{-10pt}   %5. 最初のインデント
%\setlength{\labelsep}{15pt}     %6. item と文字の間

\item 対戦前に裏向きで2枚まで切札を置くことができる。
\item 切札はデッキと直角に交わるようにデッキの下に置く。
\item 切札を表にするときはスートと数字が見えるようにし、対応する能力の名称を言う。
\item デッキが0枚になった場合、切札が残っていても敗北する。
\end{itemize}
※詳しくは、公式ルール参照。
\vspace{-1zh}%余白削除

%%%%%%%%%%%%%%%%%%%%%%%%%%%%%%
%%%%% ActionList %%%%%
\begin{center}
\begin{center}
\hrule height 1mm depth 0mm width 66.5mm %罫線
\vspace{1mm}%余白削除
{\Large\bf \ruby{Action List}{アクションリスト}}
\vspace{1mm}%余白削除
\hrule height 0.5mm depth 0mm width 66.5mm %罫線
\end{center}
\end{center}
\vspace{-1zh}%余白削除
%%%%%%%%%%%%%%%%%%%%%%%%%%%%%%

#foreach( $row in ${list1})
#if(${row.std} != "")
%%% Action %%%
%\vspace{1zh} %余白追加
\vspace{2mm} %余白追加
\hrule height 0.5mm depth 0mm width 66.5mm %罫線
\vspace{1mm} %余白追加
{\normalsize\bf ■ ${row.actName} {\scriptsize (${row.actType}) [ex]}} %Actionタイトル
\hfill 
#if(${row.actTime} != "")
{\small\bf @${row.actTime} }
#end
#if(${row.actCost} != "")
  {\small\bf | } {\small\bf \$ ${row.actCost}}
#end

#if(${row.actKey.trim()} != "")
★${texFn.cnv(${row.actKey})}
#end

#if(${row.actNote.trim()} != "")
%特記事項
\vspace{1mm}%余白削除
\hrule height 0.1mm depth 0mm width 66.5mm %罫線
\vspace{1mm}%余白削除

${texFn.cnv(${row.actNote})}
#end
#if(${row.actTarget.trim()} != "")
\vspace{1mm}%余白削除
\hrule height 0.1mm depth 0mm width 66.5mm %罫線
\vspace{1mm}%余白削除
{\bf(対象)}

${texFn.cnv(${row.actTarget})}
#end
#if(${row.actEffect.trim()} != "")
\vspace{1mm}%余白削除
\hrule height 0.1mm depth 0mm width 66.5mm %罫線
\vspace{1mm}%余白削除

{\bf(効果)}

${texFn.cnv(${row.actEffect})}
#end
#end
#end

%\newpage %改段
%%%%%%%%%%%%%%%%%%%%%%%%%%%%%%
%%%%% AbilityList %%%%%
\begin{center}
\begin{center}
\hrule height 1mm depth 0mm width 66.5mm %罫線
\vspace{1mm}%余白削除
{\Large\bf \ruby{Ability List}{アビリティリスト}}
\vspace{1mm}%余白削除
\hrule height 0.5mm depth 0mm width 66.5mm %罫線
\end{center}
\end{center}
\vspace{-2zh}%余白削除
%%%%%%%%%%%%%%%%%%%%%%%%%%%%%%

%\tcbset{top=1mm, left=1mm, bottom=1mm, left=1mm}
\tcbset{colframe=black,coltitle=black!0!black,colbacktitle=white!0!white,colback=white!0!white,sharp corners,top=1mm, left=1mm, bottom=1mm, left=1mm,boxrule=0.2mm,toprule=0.2mm}

## Abilityループ
#foreach( $row in ${list})
#if(${row.output.trim()} != "")

%%%%% Ability %%%%%
\vspace{3mm} %余白追加
\hrule height 0.5mm depth 0mm width 66.5mm %罫線
\vspace{1mm} %余白追加
{\Large\bf ${texFn.cnvSimpleAlphabetSuit(${row.suit})} ${row.num}} {\normalsize\bf【${row.name}】} %Abilityタイトル
\vspace{1mm} %余白追加

${texFn.cnv(${row.ability})}

## Abilityアクション
#if((${row.output.trim()} != "")&&(${row.actName} != ""))
\begin{tcolorbox}[title={\small\bf【Action】${row.actName}}{\scriptsize (${row.actType})}]

#if(${row.actTime} != "")
{\scriptsize\bf @${row.actTime} }
#end
#if(${row.actCost} != "")
  {\scriptsize\bf | \$ ${row.actCost} }
#end
#if(${row.actKey.trim()} != "")
  {\scriptsize\bf | ★${texFn.cnv(${row.actKey})}}
#end

#if(${row.actNote.trim()} != "")
%特記事項
\vspace{1mm} %余白追加
\hrule height 0.1mm depth 0mm width 62mm %罫線
\vspace{1mm} %余白追加

${texFn.cnv(${row.actNote})}
\vspace{-2mm} %余白削除 能力のアクションは余白が大きく開くため、追加で余白を削除
\vspace{1zh}%余白追加
#end
#if(${row.actTarget.trim()} != "")
\vspace{1mm} %余白追加  
\hrule height 0.1mm depth 0mm width 62mm %罫線
\vspace{1mm} %余白追加

{\bf(対象)}

${texFn.cnv(${row.actTarget})}

#end
#if(${row.actEffect.trim()} != "")
\vspace{1mm} %余白追加
\hrule height 0.1mm depth 0mm width 62mm %罫線
\vspace{1mm} %余白追加

{\bf(効果)}

${texFn.cnv(${row.actEffect})}
#end

\vspace{1mm} %余白追加 無いと箇条書きの場合、ギリギリになるので追加
\end{tcolorbox}

\vspace{-1zh}

#end ##Abilityアクション end 

## Abilityキャラクター 
#if((${row.output.trim()} != "")&&(${row.charName} != ""))
\vspace{2mm}
\begin{tcolorbox}[title={\small\bf【Character】${row.charName}}{\scriptsize (${row.charType})}]

#if(${row.actKey.trim()} != "")
  {\scriptsize\bf ★${texFn.cnv(${row.charKey})}}
#end

#if(${row.charNote.trim()} != "")
%特記事項
\vspace{1mm} %余白追加
\hrule height 0.1mm depth 0mm width 62mm %罫線
\vspace{1mm} %余白追加

${texFn.cnv(${row.actNote})}
\vspace{1zh}%余白追加
#end
#if(${row.charAbility.trim()} != "")
\vspace{1mm} %余白追加
\hrule height 0.1mm depth 0mm width 62mm %罫線
\vspace{1mm} %余白追加

{\bf(能力)}

${texFn.cnv(${row.charAbility})}
#end

\vspace{1mm} %余白追加 無いと箇条書きの場合、ギリギリになるので追加
\end{tcolorbox}

\vspace{-1zh}

#end ##Abilityキャラクター end 

#if((${row.output.trim()} != "")&&(${row.actName1} != ""))
\begin{tcolorbox}[title={\small\bf【Action】${row.actName1}}{\scriptsize (${row.actType1})}]

#if(${row.actTime1} != "")
{\scriptsize\bf @${row.actTime1} }
#end
#if(${row.actCost1} != "")
  {\scriptsize\bf | \$ ${row.actCost1} }
#end
#if(${row.actKey1.trim()} != "")
  {\scriptsize\bf | ★${texFn.cnv(${row.actKey1})}}
#end

#if(${row.actNote1.trim()} != "")
%特記事項
\vspace{1mm} %余白追加
\hrule height 0.1mm depth 0mm width 62mm %罫線
\vspace{1mm} %余白追加

${texFn.cnv(${row.actNote1})}
\vspace{-2mm} %余白削除 能力のアクションは余白が大きく開くため、追加で余白を削除
\vspace{1zh}%余白追加
#end
#if(${row.actTarget1.trim()} != "")
\vspace{1mm} %余白追加
\hrule height 0.1mm depth 0mm width 62mm %罫線
\vspace{1mm} %余白追加

{\bf(対象)}

${texFn.cnv(${row.actTarget1})}

#end
#if(${row.actEffect1.trim()} != "")
\vspace{1mm} %余白追加
\hrule height 0.1mm depth 0mm width 62mm %罫線
\vspace{1mm} %余白追加

{\bf(効果)}

${texFn.cnv(${row.actEffect1})}
#end

\vspace{1mm} %余白追加 無いと箇条書きの場合、ギリギリになるので追加
\end{tcolorbox}

\vspace{-1zh}

#end ##Abilityアクション1 end 

#end
#end ##Abilityループ end

%% 奥付 %%
\thispagestyle{empty}
\vspace*{\stretch{1}}
\begin{flushright}
\begin{minipage}{0.6\hsize}
\begin{description}
  \item{誌名:}BlackPoker ExtraList\\ \hspace{3pt} ${data.ver} v${data.v}
  \item{発行:}${data.lastupdate}
\end{description}
\end{minipage}
\end{flushright}

\begin{flushright}
\copyright 2013 BlackPoker
\end{flushright}

\end{document}