\documentclass[twocolumn,a5paper,papersize,10pt]{jarticle}
\usepackage{bxpapersize}
\usepackage{booktabs}
\usepackage{tabularx}
\usepackage[dvipdfmx]{graphicx} %画像読み込み設定
\usepackage{here} %画像を好きな位置に出力
\usepackage[absolute,overlay]{textpos}%座標指定

\usepackage{newtxtext,newtxmath} %timeフォント設定
\usepackage{titlesec}%タイトルの文字サイズ変更
\usepackage{setspace} % setspaceパッケージのインクルード

\usepackage{tcolorbox} % 枠囲み
\usepackage{okumacro} % ルビ

%\columnseprule=0.1pt %段組み罫線
\setlength{\columnsep}{0.5cm} %段組みの幅


\titleformat*{\section}{\small\bfseries} %sectionの文字サイズ
\titleformat*{\subsection}{\normalsize\bfseries} %subsectionの文字サイズ
\titleformat*{\subsubsection}{\scriptsize\bfseries} %subsubsectionの文字サイズ

\setlength{\hoffset}{-2cm}
\setlength{\voffset}{-4cm}
\setlength{\marginparsep}{0pt}
\setlength{\marginparwidth}{0pt}
\setlength{\headheight}{5pt}
\setlength{\textheight}{19cm}
\setlength{\textwidth}{13.8cm}

\setlength\intextsep{0pt} %図の余白をなくす
\setlength\textfloatsep{0pt} %図の余白をなくす

\setlength\floatsep{0pt} %図と図の間の余白
\setlength\textfloatsep{0pt} %本文と図の間の余白
\setlength\intextsep{0pt} %本文中の図の余白
\setlength\abovecaptionskip{0pt} %図とキャプションの間の余白


\setstretch{1} % ページ全体の行間を設定
\parindent = 0pt %インデントを無効化

\title{\empty}
\author{\empty}
\date{\empty}

%%%%%%    TEXT START    %%%%%%
\begin{document}

% ロゴ出力
% この場合は (230pt, 100pt) の位置に 0.4\linewidth の幅のブロックができる.
\begin{textblock*}{0.4\linewidth}(55pt, 145pt)
    \centering
    \includegraphics[width=1.2cm]{blackpoker_logo.pdf}
\end{textblock*}

%%%%% QRコード第三版用 %%%%%
\begin{textblock*}{0.4\linewidth}(150pt, 30pt)
    \centering
    \includegraphics[width=6.5cm]{qr_v3-std.pdf}
\end{textblock*}
\begin{textblock*}{0.4\linewidth}(205pt, 134pt)
    \centering
    \textcolor{black}{Web版}
\end{textblock*}
%%%%% QRコード第三版用 %%%%%

\section*{\textrm{\Large BlackPoker}}
\vspace{-1zh}%余白削除
\noindent

\begin{center}
第三版


{\scriptsize ${data.lastupdate}}
\end{center}

\scriptsize%本文の文字サイズを小さく設定
\renewcommand{\labelitemi}{・}%箇条書きのラベルを変更

{\quad}本誌はライト版、スタンダード版のActionList,CharacterListです。
ライト版、スタンダード版で使用できる
アクション、キャラクターが異なるため、記載されている表記により使い分けて下さい。
\vspace{-1zh}%余白削除
\begin{tabbing}
 \hspace{2mm} \= \hspace{3mm} \= \hspace{1mm} \= hspace{15mm} \kill
\> lite \> : \>ライト版で使えます。\\
\> std \> : \>スタンダード版で使えます。\\
\end{tabbing}

\vspace{-2zh}%余白削除
\hrule height 0.5mm depth 0mm width 66.5mm %罫線
\vspace{-3zh}%余白削除
\subsection*{(補足)記号の意味}
\vspace{-1zh}%余白削除
{\small @:タイミング, 
\$:コスト, 
★:キーカード}

\vspace{1mm}%余白削除
\hrule height 0.1mm depth 0mm width 66.5mm %罫線
\vspace{-3zh}%余白削除

\subsection*{(補足)コストの意味}
\vspace{-1zh}%余白削除
\begin{small}
\begin{tabbing}
 \hspace{2mm} \= \hspace{2mm} \= \hspace{1mm} \= hspace{15mm} \kill
\> B \> : \>防壁をドライブする \\
\> L \> : \>1点ダメージを受ける \\
\> D \> : \>手札を1枚捨てる \\
\> S \> : \>キャラクター1体を墓地に移す \\
\end{tabbing}
\end{small}

\vspace{-3zh}%余白削除
\hrule height 0.1mm depth 0mm width 66.5mm %罫線
\vspace{-3zh}%余白削除

\subsection*{(補足)防壁の置き方}
\vspace{-1zh}%余白削除
\begin{itemize}
\setlength{\leftskip}{-0.3cm}%箇条書きを左詰め
%\setlength{\itemsep}{0pt}      %2. ブロック間の余白
\setlength{\parskip}{0pt}      %4. 段落間余白.
%\setlength{\itemindent}{-10pt}   %5. 最初のインデント
%\setlength{\labelsep}{15pt}     %6. item と文字の間

\item 防壁を置く時はデッキ側に詰めて置く。
\item 防壁の左右の入れ替えは行わない。
\end{itemize}

\vspace{-1zh}%余白削除
\hrule height 0.1mm depth 0mm width 66.5mm %罫線
\vspace{-3zh}%余白削除

\subsection*{(補足)誘発する場合}
\vspace{-1zh}%余白削除
アクションを誘発する場合、以下のようにアクションをステージに乗せる。
\vspace{-1zh}%余白削除
\begin{itemize}
\setlength{\leftskip}{-0.3cm}
%\setlength{\parindent}{-10zw}
%\setlength{\itemsep}{0pt}      %2. ブロック間の余白
\setlength{\parskip}{0pt}      %4. 段落間余白.
%\setlength{\itemindent}{-10pt}   %5. 最初のインデント
%\setlength{\labelsep}{15pt}     %6. item と文字の間

\item アクションを起こした後または、アクションが効果を発揮した後に誘発したアクションをステージの一番上に乗せる。
\item 誘発したアクションが複数ある場合、このターンを開始したプレイヤーから好きな順で誘発したアクションをステージに乗せる。
\end{itemize}
\vspace{-1zh}%余白削除


%%%%%%%%%%%%%%%%%%%%%%%%%%%%%%
%%%%% ActionList %%%%%
\begin{center}
\begin{center}
\hrule height 1mm depth 0mm width 66.5mm %罫線
\vspace{1mm}%余白削除
{\Large\bf \ruby{Action List}{アクションリスト}}
\vspace{1mm}%余白削除
\hrule height 0.5mm depth 0mm width 66.5mm %罫線
\end{center}
\end{center}
\vspace{-1zh}%余白削除
%%%%%%%%%%%%%%%%%%%%%%%%%%%%%%

#foreach( $nestList in $listlist)

%%% 大項目 %%%
\tcbset{colframe=black,coltitle=black!0!black,coltext=white!0!white,colbacktitle=white!0!white,colback=black!0!black,sharp corners,top=0mm, left=0mm, bottom=0mm, right=0mm,boxrule=0mm,toprule=0mm,valign=center,halign=center}
\begin{tcolorbox}
{\scriptsize\bf ${nestList[0].type}}
\end{tcolorbox}

#foreach( $row in ${nestList})
#if((${arg0} == "std") || ((${arg0} == "lite") && (${row.lite.trim()} != "")))
%%% Action %%%
%\vspace{1zh} %余白追加
\vspace{2mm} %余白追加
\hrule height 0.5mm depth 0mm width 66.5mm %罫線
\vspace{1mm} %余白追加
#if((${row.lite.trim()} != "") && (${row.std.trim()} != ""))
{\normalsize\bf ■ ${row.actName} {\scriptsize (${row.actType}) [lite, std]}} %Actionタイトル
#elseif(${row.std.trim()} != "")
{\normalsize\bf ■ ${row.actName} {\scriptsize (${row.actType}) [std]}} %Actionタイトル
#end
\hfill 
#if(${row.actTime} != "")
{\small\bf @${row.actTime} }
#end
#if(${row.actCost} != "")
  {\small\bf | } {\small\bf \$ ${row.actCost}}
#end

#if(${row.actKey.trim()} != "")
★${texFn.cnv(${row.actKey})}
#end

#if(${row.actNote.trim()} != "")
%特記事項
\vspace{1mm}%余白削除
\hrule height 0.1mm depth 0mm width 66.5mm %罫線
\vspace{1mm}%余白削除

${texFn.cnv(${row.actNote})}
#end
#if(${row.actTarget.trim()} != "")
\vspace{1mm}%余白削除
\hrule height 0.1mm depth 0mm width 66.5mm %罫線
\vspace{1mm}%余白削除
{\bf(対象)}

${texFn.cnv(${row.actTarget})}
#end
#if(${row.actEffect.trim()} != "")
\vspace{1mm}%余白削除
\hrule height 0.1mm depth 0mm width 66.5mm %罫線
\vspace{1mm}%余白削除

{\bf(効果)}

${texFn.cnv(${row.actEffect})}
#end
#end
#end

#end

%%%%%%%%%%%%%%%%%%%%%%%%%%%%%%
%%%%% CharacterList %%%%%
\begin{center}
\begin{center}
\hrule height 1mm depth 0mm width 66.5mm %罫線
\vspace{1mm}%余白削除
{\Large\bf \ruby{CharacterList}{キャラクターリスト}}
\vspace{1mm}%余白削除
\hrule height 0.5mm depth 0mm width 66.5mm %罫線
\end{center}
\end{center}
\vspace{-1zh}%余白削除
%%%%%%%%%%%%%%%%%%%%%%%%%%%%%%

#foreach( $nestList in $list1list)

%%% 大項目 %%%
\tcbset{colframe=black,coltitle=black!0!black,coltext=white!0!white,colbacktitle=white!0!white,colback=black!0!black,sharp corners,top=0mm, left=0mm, bottom=0mm, right=0mm,boxrule=0mm,toprule=0mm,valign=center,halign=center}
\begin{tcolorbox}
{\scriptsize\bf ${nestList[0].type}}
\end{tcolorbox}


#foreach( $row in ${nestList})
#if((${arg0} == "std") || ((${arg0} == "lite") && (${row.lite.trim()} != "")))
%%% Character %%%
%\vspace{1zh} %余白追加
\vspace{2mm} %余白追加
\hrule height 0.5mm depth 0mm width 66.5mm %罫線
\vspace{1mm} %余白追加
#if((${row.lite.trim()} != "") && (${row.std.trim()} != ""))
{\normalsize\bf ■ ${row.charName} {\scriptsize (${row.type}) [lite, std]}} %Characterタイトル
#elseif(${row.std.trim()} != "")
{\normalsize\bf ■ ${row.charName} {\scriptsize (${row.type}) [std]}} %Characterタイトル
#end
\hfill 
#if(${row.charKey.trim()} != "")
{\small\bf ★${texFn.cnv(${row.charKey})} }
#end

#if(${row.charNote.trim()} != "")
%特記事項
\vspace{1mm}%余白削除
\hrule height 0.1mm depth 0mm width 66.5mm %罫線
\vspace{1mm}%余白削除

${texFn.cnv(${row.charNote})}
#end
#if(${row.charAbility.trim()} != "")
\vspace{1mm}%余白削除
\hrule height 0.1mm depth 0mm width 66.5mm %罫線
\vspace{1mm}%余白削除

{\bf(能力)}

${texFn.cnv(${row.charAbility})}
#end
#end
#end

#end

\thispagestyle{empty}
\vspace*{\stretch{1}}
\begin{flushright}
\begin{minipage}{0.6\hsize}
\begin{description}
  \item{誌名:}BlackPoker Support Paper 第三版
  \item{発行:}${data.lastupdate}
\end{description}
\end{minipage}
\end{flushright}

\begin{flushright}
\copyright 2013 BlackPoker
\end{flushright}

\end{document}
